\chapter{GEANT simulations}

\section{Motivation}

\EliasComm{Maybe move this into methods.}

In the following chapters various arrays of scintillator arrays (in this case caesium-iodide doped with thallium, CsI(Tl)) are emplyed to measure the spectrum of gamma radiation from various sources (bremsstrahlung, linear and non-linear inverse Compton scattering and hard betatron radiation).

In the soft and hard X-ray regime spectra can be deduced by comparing the transmission of the radiation through various materials with characteristic K-absorption edges or relying on crystal reflections matching the Bragg-condition or a combination of both.
For these conditions a good knowledge of the material properties and the positioning of the crystal and cameras is essential.
The energy-dependent transmission through the material is related to an interplay of different mechanisms (Compton scattering, Rayleigh scattering, photoelectric effect, pair production and so on). The sum of these cross sections can be combined and found in specific repositories.

At higher X-ray energies entering the gamma regime $\hbar \omega \rightarrow 1\,\mathrm{MeV}$ K-edges to effectively distinguish spectral bands become rarer and most materials become more transparent, as a consequence the resolving a full spectrum becomes challenging. At the same time pair production becomes possible and more likely with increasing energy.
A detector interacting with such high energetic gamma-radiation is now faced with energy deposition by the radiation itself and secondary particles and radiation. An extended detector can mitigate this effect by forcing pair production in an earlier part of the detector and performing an again well resolved measurement in a second part of the detector. The complex interplay of pair production and energy deposition from secondary sources requires numerical solutions calculating the cross-sections for various processes dynamically throughout the interaction with the detector. This is where GEANT or other Monte Carlo based simulation codes come into play.

At the few MeV level the spectrum of the generated electron-positron pairs scales directly with the energy of the gamma radiation interacting with the matter. Measuring the spectrum of these particles, for instance with a magnetic electron spectrometer, and comparing the results with a simulation result can then be used to deduce a gamma spectrum.

The spectrum of the pairs, however, becomes less sensitive to the energy of the gamma-radiation at higher (tens of MeV) energies. An alternative method to measure such a hard spectrum is tracking the energy deposition of radiation and secondary sources throughout an array of scintillator material until it decays.
This method has the advantage of being variable and viable for the few MeV level and much harder radiation.

Simulations linked to this technique are outlined in this work and are then applied to experimental data in the following chapters to measure radiation from different materials in a range of hundreds of keV to hundred MeV (over three orders of magnitude).


\section{Simulation Code and Functionality}

The at CERN developed GEANT code is a Monte Carlo code incorporating most known and well measured cross-sections for fundamental particle/light-matter interactions.

The specific simulation code for GEANT used is strongly based on work by Jason Cole (Imperial College) who in turn benefitted from input by Kristjan Poder (formerly Imperial College, now at DESY in Hamburg).

In this case the code was used to simulate the energy deposited in an array of scintillator crystals by individual X-ray photons or in other word to simulate the response of the detector. It was assumed that multi-photon processes and photon-photon interactions would be negligible and hence tracking individual photons through the detector and simply adding the respective detector response would be sufficient.
It was also assumed that the energy deposited within the scintillator crystals is linearly related to the light yield of the scintillator, i.e. the number of fluorescence photons emitted, and independent of the process of energy deposition. Above a certain threshold this appears to be true (REF).

The code was used to simulate the response of various configurations of scintillator crystals and materials used in experiment, and also to simulate the bremsstrahlung spectrum from electrons interacting with slabs of material and the respective response of detectors to this secondary radiation.


\section{Energy Deposition in one scintillation layer}

In some experiments an additional and separate array of crystals was posititoned before the main detector to measure the profile of the gamma ray burst.
This gives information about the divergence of the beam but also in some way the number of photons and energy contained in the burst of radiation.
How well this scales with the actual total energy is subject to the energy range and the processes involved and was to be investigated as well.

The profile is also an indicator the the intensity of the interaction in non-linear ICS and carries the imprint of the interacting electron beam within it.
It is hence a useful diagnostic to not only confirm an interaction took place but also to characterise the conditions at the interaction.

When planning to use such a detector as measurement of the total energy of the radiation one has to keep in mind that this is not a calorimeter in the gamma regime as radiation is bound to be transmitted through the stack. However, it is expected that the energy deposited per photon will vary and of course the number of photons will leave their trace on this detector.


The detector in the simulation:
the detector assembled in the simulation is based on a CsI stack used in some experiments and provided by the Helmholtz Institut in Jena.
It consists of 45 x 45 CsI(Tl) crystals of face dimension 1 mm x 1 mm and a thickness of 10 mm. The crystals are spaced by a 0.2 mm layer of TiO2 and the frontside has another layer of TiO2 of 0.5 mm which keeps the entire stack together.

The stack was defined as such in the simulation and it can be seen in figure ADD HERE.

\begin{figure}
\centering
\includegraphics[width=.3\columnwidth]{ScreenshotGEANT_JenaStack.png}
\caption{Simulated stack.}
\end{figure}

The front side of the detector is XX m away from the source of the radiation, similarly to experiment conditions, hence being in general able to cover plus minus NUMBER divergence from a point source if centred along the same axis.

Now individual photons with ranging energy from 10 keV to 500 MeV are launched from the source point and the energy deposited in the crystals is being recorded. Each energy step is simulated 10.000 to 100.000 times and then averaged to obtain an average energy deposited per photon of this energy.

The results from these simulation can be seen in figure ADD HERE. Photon energies up to ENERGY are almost completely absorbed, backscattered and rarely transmitted. At around 200 keV suddenly a large amount of radiation is being absorbed. This indicates a material characteristic absorption edge.
This means that whilst for the majority of the enegry ranges the CsI profile will give a good indication as energy times flux is monotonic increasing and somewhat linear for the two regimes of interest. In the regime of hundreds of keV we have to consider that a lot of energy will be absorped in the front layer and we have to check how this behaves further downstream in the next detector.

\begin{figure}
\centering
\includegraphics[width=.5\columnwidth]{Edep_JenaStack.pdf}\includegraphics[width=.5\columnwidth]{Edep_JenaStack_rel.pdf}
\caption{Simulation results: energy deposited per photon in a profile stack.}
\end{figure}

If we look at this in terms of relative energy deposited per photon, so in other words energy deposited over energy of the photon, this does not look like a sharp edge but merely a steady decrease in energy deposited relative to the total photon energy. Scattering cross sections typically decrease with increase in energy and this goes along the same lines.

\section{Energy Deposition in an extended scintillator array}

The scintillator stack used in the first results chapter is described in these publications (REFS).
It is a 33x47 crystal array of crystals with face dimension 5 mm x 5 mm and length 10 mm. The crystals are spaced by aluminium (Thickness) and a plate is holding the crystals in place on their face. The holes are circular and cover some part of the face. In Poder publication this stack was used as a profile measurement. In this work the array is arranged such that the radiation penetrates further into the array of crystals (see figure). The radiation is expected to decay more and more while propagating deeper into the stack. The front plate is covered by several mms of steel which can also be replaced by a plastic layer.

The stack was defined as described in the simulation and looks like this in the viewer (ADD FIGURE).


The simulation range as before.

The response of the detector is being shown in FIGURES.
As one can see the deposited energy falls off exponentially with the crystal depth as one expects from Beer's Law.
At higher photon energies a peak becomes apparent. This is due to the production of electron-positron pairs in a QED shower which then in turn deposit energy or scatter generating secondary radiation themselves which then decays exponentially as well.

The peak of this distribution moves further into the stack with increasing energy. This might be due to more energy being available to the pairs, more pairs being produced per photon and so on.

The energy conversion is of order.... percent.


\section{Energy Deposition in an extended scintillator array in two directions}

Similarly as before an array of crystals continuing in the penetration direction of the radiation is being modelled. This time the orientation of the stacks is changing from layer to layer. In addition the stack is longer in the propagation direction and spaced by plastic instead of aluminium and lacks a thick steel plate as cover. As radiation is in these situations very collimated the detector is elongated to enable resolving the full decay process, also at higher energies.
The lower Z spacers and front plate also is supposed to help make use of this extra length and space out the decay.

\begin{figure}
\centering
\includegraphics[width=.5\columnwidth]{ScreenshotGEANT_DualAxis_Side.png}
\includegraphics[width=.5\columnwidth]{ScreenshotGEANT_DualAxisStack.png}
\caption{Simulated stack: dual axis}
\end{figure}


\section{Comparison of scintillator array responses}

The described detectors (dual axis, single axis and profile) were combined in different ways and the cover plate for one of the detectors was changed to see the change in response.
This is using the simulation data as described in the previous sections but comparing the response to each other.

RAL stack with steel and plastic and gamma profile in front.
Dual axis array. Dependence of peak from materials in between, mainly front plate. Shifts depending on that.

As expected adding more material or higher density materials forces earlier attenuation and pair production (as the cross-section increases).
Adding a profile detector has the same effect as adding a steel plate. The plastic spacers seem to have the effect that the peak propagates slower within the stack even though it starts quite deep in already. The main effect is carried by the dense scintillator material.


\begin{figure}
\centering
\includegraphics[width=.5\columnwidth]{EdepMax_Cases.pdf}
\caption{Comparison simulation results.}
\end{figure}


\section{Angular dependence of responses}

The along the transverse direction integrated response remains unchanged for a source merely translated in a direction.
Does it change within a realistically expected range of angles?

Running simulations with sources and angles. Does the response curves for one energy change with change of angle?

If not that means that we can take slices of the response to measure the differential spectrum.

The 2D array of crystals allows to also obtain an angular response from the detector which could be used to deduce the angular spectrum of the radiation. Even a monoenergetic beam will break up in secondary radiation, scatter and shower particles into the detector resulting in a certain spatial extent of the response. The width of that response is an upper limit to the divergence of the radiation, but a thin scintillator screen is a better indicator due to the scattering and pair showers. 

One screen, however, is unable to tell us anything about an angular structure and spectrum. Does the spectrometer react differently to a global angular pointing of the beam? these simulations indicate the opposite. The realistic angular changes are of order of mrads which is smaller than a crystal over the size of the stack, meaning even at an angle the response is fairly limited to the same crystal rows. Integrated the response looks almost identical.


\section{Combining responses to spectra}

The simulations so far have assumed individual photons at a defined energy. The response functions and the energy deposition was averaged over the number of photons (10.000 to 100.000 per energy).

In an experiment, however, perfectly monoenergetic photon sources are not to be found. Even deeply linear inverse Compton scattering undergoes a broadening from various sources. To restore the response of a realistic spectrum the simulated responses are being added up and weighted according to the shape of the spectrum. This assumes that photon-photon interactions or multi-photon interactions are negligible and that each photon travelling unperturbed and isolated for all practical purposes. This adding up of spectra can mathematically easily be performed by applying a matrix multiplication: the matrix contains the responses of a range of photon energies and multiplying it with a vector containing the weighting for each photon energy results in the detector response of a full spectrum.

This means that in principle a backwards transformation should be possible to invert the matrix and immediately gain a spectrum from experimental data from this process. Previous work, however, indicated that this process does not perform well due to noise in the experiment meaning the problem is ill-posed and the inversion proces does not converge. Instead a spectrum is assumed an a range of parameters used to vary this spectrum. The experimental data and the range of responses can then be compared using a least-square fit. This technique requires an input spectrum and a good knowledge of the spectrum expected. Especially in a transient regime where spectral shapes are expected to change this does not perform well.

\section{Model-independent unfolding algorithm}

To overcome the assumption of a well-known shape we have to resort to an algorithm based on Bayes-Theorem to find a spectrum independent of any presumption. In the case of some results to follow later for linear inverse Compton scattering the electron spectrum is expected to remain unperturbed while the laser pulse is strongly defocused and is also not expected to vary its intensity much. Under these conditions this method can be fielded and compared to the linear theory.

\section{Bremsstrahlung simulations}

One potential application is measuring hard bremsstrahlung from a source. Bremsstrahlung is very well understood as a process and is only mildly dependent on the shape of the electron spectrum, although large energy fluctuations will make a difference. This also turns out to be a good way to calibrate a detector and in experiment spot differences in crystal efficiency, flaws in the imaging system and so on.

In this simulation monoenergetic electrons were launched at a converter foil at distance...
The material of the converter foil and the thickness was varied. The electrons are after their interaction dispersed by a magnetic field such that they do not interact with the detector. 

Measuring conversion of electrons into bremsstrahlung (efficiency) and response for experiment factors.
