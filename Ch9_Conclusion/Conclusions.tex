\chapter{Conclusion}
\label{Chap:Conclusion}

\section{Summary of Results}

The experimental results presented in this Thesis demonstrated the capability of laser wakefield acceleration (LWFA) to produce highly energetic radiation from few to hundreds of MeV photon energies, and their application in the context of fundamental studies of quantum electrodynamics (QED):

\begin{itemize}
\item In Chapter \ref{Chap:linICS}, relativistic electron beams with energies up to $1.3\,\mathrm{GeV}$ were collided with laser pulses at $a_0 \sim 0.3$, producing over 100s of shots radiation from linear inverse Compton scattering of variable spectral shape in the range of 10s of MeV photon energies. 
A procedure to align and overlap the laser pulse and electron beam and maintain the alignment over extended periods of time was developed. Linear inverse Compton scattering was also considered as non-invasive electron beam diagnostics.

\item In Chapter \ref{Chap:RR15}, relativistic electron beams of energy $\epsilon \approx 550\,\mathrm{MeV}$ were collided with a highly intense laser pulse of intensity $a_0 \approx 10$, producing broadband synchrotron-like radiation with a critical energy $\epsilon_{crit} > 35\,\mathrm{MeV}$ from non-linear inverse Compton scattering. The high energy of the emitted photons lead to a measurable energy loss in the electron spectra due to radiation reaction.


\item In Chapter \ref{Chap:BW}, electron beams were propagated through a bismuth target to produce a highly-energetic gamma-ray beam from bremsstrahlung. The yield, collimation and noise of the bremsstrahlung source were optimised for its application in a photon-photon collider experiment with the aim to produce electron-positron pairs from the linear Breit-Wheeler process. The second photon source was a keV X-ray source from a burn-through foil heated by a high-intensity laser. The variations in performance of both high-energy photon sources and the implications for measuring Breit-Wheeler pairs in this and future experiments were discussed.
\end{itemize}

\section{Future Measurements of Radiation Reaction}


\subsection{Motivation and Science Case}

Phenomena to look into

\begin{itemize}
\item Precision measurement of radiation reaction force
\item Breakdown of classical description
\item Breakdown of LCFA quantum description
\item Pair production through non-linear BW
\item Narozhny conjecture \cite{Ritus1972_NAROZHNY,Narozhny1980_NAROZHNY,Fedotov2017_NAROZHNY} to be measured by Yakimenko (PRL 2019, Blackburn 2019, Baumann arxiv 2019
\end{itemize}

\EliasComm{A plot of interesting phenomena on a $a_0$ and electron energy map (mark where LWFA, crystals and so on are).}

\subsection{Measurement routes}


\EliasComm{What is the distinction of these different approaches?}

\EliasComm{Maybe make map of nonlinearity $a_0$ or equivalent in crystals and beam energy}

Three main branches measuring radiation reaction:
\begin{itemize}
\item LWFA and high-intensity lasers
\item Conventional accelerators and high-intensity lasers
\item Positrons in aligned crystal lattices
\end{itemize}

Overview of LWFA projects:
\begin{itemize}
\item ELI pillars
\item CoReLS (Gwangju, Korea)
\item Gemini/EPAC
\item APOLLON
\item SULF
\item HERCULES
\item Nebraska
\item Munich (FOR or Cala)
\end{itemize}

Overview of conventional accelerator projects:
\begin{itemize}
\item LUXE (XFEL Hamburg)
\item SFQED (FACET-II, SLAC)
\item Munich (FOR or Cala)
\end{itemize}


Overview of crystal measurements:
\begin{itemize}
\item CERN
\end{itemize}


Beam-beam measurements also for photon-photon collisions (also Narozhny conjecture)


\section{Experimental Setup for Future Radiation Reaction Measurements using LWFA}


\subsection{Motivation}

\subsubsection{Achieved so far - Summary}


\begin{itemize}
\item managed to get some measurements of radiation reaction but low confidence
\item would want to make precision measurement to distinguish the models, using other observables
\item want to interact at even higher intensities to produce pairs
\end{itemize}

What we have done so far
\begin{itemize}
\item shown we can overlap beams in space and time
\item shown overlap over 100s of shots at low intensity
\item shown that this can be used as beam diagnostics
\item have a technique to go from there to high-intensity collisions
\item shown we were able to perform high-intensity collisions
\item were able to determine statistical confidence by measuring the gamma-ray spectrum and the electron spectrum simultaneously radiation reaction
\item identified parameters which are most suitable for high-precision studies 
\end{itemize} \cite{Arran2019_RR_PPCF,Arran2019_RR_SPIE}.


\subsubsection{Where to go from here}


We might be able to reach a regime where QED effects become relevant as the `quantumness' parameter $\eta$, indicating how much the interaction is dominated by QED effects \cite{Bell2008_PairsEta,Blackburn2014_QRR}, approaches values close to $1$. In this regime we expect gamma rays to be produced early in the interaction that can then combine with the copious amounts of laser photons available to produce electron-positron pairs \cite{Blackburn2017_pairs}. At the same time, this work in conjunction with \cite{Arran2019_RR_PPCF,Arran2019_RR_SPIE} indicates that by improving the electron beam quality differences in models can be measured even at comparable laser and electron energies, giving future experiments at Gemini or comparable laser facilities the opportunity to bridge the gap between these first measurements of radiation reaction and the future projects being worked on at the next generation experiments.




\begin{sidewaysfigure}
\centering
\includegraphics[width=1.0\columnwidth]{RR20_ExpSetup_V12Sep.png}
\caption{Future experiment layout incorporating lessons learned.}
\label{Concl:Fig:FutureExp}
\end{sidewaysfigure}

What could then be possible in realistic terms (calculate eta)
\begin{itemize}
\item Use beam as achieved, maybe assume we can boost it to 2 GeV
\item Assume we can get $a_0 = 10$ safely, what can be done with 25
\item what is the real peak intensity?
\item table for different energies and eta
\end{itemize}


A proposed experiment layout based on the experiments described in the previous Chapters is shown in Figure \ref{Concl:Fig:FutureExp} and is outlined in more detail in the next sections, describing the requirements and ideas with the reference to the corresponding chapter where this diagnostic or technique was fielded and shown to be useful.



\begin{itemize}
\item for each area make a table to summarise what has been fielded and what is proposed as new
\end{itemize}


\subsection{Main Objectives}

\begin{itemize}
\item Successful high-intensity collisions $N>100$ at $a_0 \sim 10$
\item Precision study of radiation reaction phenomena 
\item Very high-intensity collisions at $a_0 > 20$
\end{itemize}


\subsection{Laser Wakefield Accelerator}


\begin{itemize}
\item Focusing geometry: long focusing $f/40$ 
\item polarisation: at Gemini given by folding mirrors but ideally cross-polarised to scattering beam
\item Targetry and injection: gas jet (diagnostics access and damaging properties, stability can be improved, less important at high intensities, narrow-energy spread beams through shock injection, could fix timing with shock injection
\item Diagnostic for laser
\item Diagnostic for plasmas
\end{itemize}

As now firmly established for wakefield experiments at Gemini, we propose an f/40 long-focusing geometry to drive a wakefield accelerator. We emphasise that we aim to deliver the maximum possible energy on-target which means increasing the beam size to its maximum on the folding mirrors to lower the fluence. On next-generation facilities this might be a lesser problem as experimental areas might provide enough space to avoid folding the beam. The folding geometry also fixes the polarisation of the laser to horizontal.

The gas target would be a gas jet, as concluded in Chapter \ref{Chap:RR15}, as it allows debris-free high-intensity interaction and open access for diagnostics, which is more favourable than the potential increases shot-to-shot stability of a gas cell target as used in Chapter \ref{Chap:BW}. It has been shown in Chapter \ref{Chap:linICS} that we can shoot suitably high above the nozzle ($up to 14\,\mathrm{mm}$) to avoid damage to the nozzle, whilst still encountering a sufficient density profile and still produce suitable electron beams.

Introducing a shock front on purpose in Chapter \ref{Chap:linICS} produced comparable electron beams to beams measured in \ref{Chap:RR15}. In Chapter \ref{Chap:RR15} the beams exhibited a distinct spectral feature at 550 MeV. This was reproduced in Chapter \ref{Chap:linICS}, but reaching higher energies up to 1.3 GeV. The beams were of high charge and electrons were measured consistently over 100s of shots. In some cases, shock injection also showed the potential to produce high-charge narrow-energy-spread beams at the 1 GeV level at few percent energy spread. The reproducibility of these beams has to be improved and is ongoing research. However, the potential for these beams is favourable especially to distinguish models.
A shock injected beam is also expected to have a fixed injection point and hence constant relative timing to the driver beam at the interaction.

We hope to explore whether we can increase the final energy of the electron beam whilst maintaining the low energy spread by increasing the length of the gas jet and increasing the driver beam energy. 
\vspace{\baselineskip}

The gas jet will be diagnosed by a shadowgraphy and interferometry setup to determine the density profile, but also to identify interesting features in the channel and use TIMING OVERLAP FRINGES PROBE. This gives visual confirmation of the transverse position and overlap and timing.
This has been successfully demonstrated in Chapter XX. The spatial resolution of the diagnostics could be improved by increasing the magnification which would increase the accuracy of the setup. High magnification top view and also for the transverse probe. This could be achieved by splitting off the transverse beam after the interaction and using a microscope objective. For the top view a second camera with objective can be used.

In addition to a high magnification, the alignment and temporal overlap could benefit from an ultrashort probe to resolve the position of the bubble and the laser pulse accurately. This requires high level of control over the system and might for the next campaign add too much complexity (REF SAVERT).

The self-emission of the plasma is also imaged from two axes. This helps to monitor potential drifts in the alignment.
HIGH MAG FOR FUTURE CAMPAIGNS.


\begin{figure}
\centering
\includegraphics[width=1.0\columnwidth]{GasJetCloseUp_Combo.pdf}
\caption[Close-up sketch of gas jet and diagnostics to characterise the plasma.]{Close-up sketch of gas jet and diagnostics to characterise the plasma. A transverse probe is used to obtain a shadowgram and an interferometry image. The self-emission of the plasma is imaged from the side and the top.}
\end{figure}


High mag fast probe \cite{Buck2011_BUBBLE,Savert2015_BUBBLE,Siminos2016_FASTSHADOW}

Multiple probes, chirped pulse as temporally resolved probe (will add a lot of complexity)

Low energy spread beams (shock injection, but also \cite{Wang2013_GEV_FIUDICIAL} and see Chapter \ref{Chap:linICS})




We are able to produce a high-charge electron beam with the potential to be a narrow-energy spread beam (shock injection, Chapter \ref{Chap:linICS}) with fixed injection point. 




\begin{table}[h]
\centering
\begin{tabular}{l|l|l|l|c}
Diagnostic & Observable & Setup & Comment & Fielded?\\ \hline \hline
Shadowgraphy & Channel & Probe & Leakage, Timing & \cmark \\
Shadowgraphy II & Channel & Probe & Leakage, Timing & \cmark/\xmark \\
Shadowgraphy HM & Channel & Probe & Leakage, Timing & \xmark \\
Shadowgraphy FASTHM & Channel & Probe & Leakage, Timing, Fast & \xmark \\
Interferometry & Density & Probe & Leakage, Timing & \cmark \\
Self-Emission & Channel & Probe & Top View & \cmark \\
Self-Emission HM & Channel & Probe & Top View & \xmark \\
\end{tabular}
\caption[Overview of plasma diagnostics.]{Overview of plasma diagnostics.}
\end{table}



\subsection{Scattering Beam}


\begin{itemize}
\item Geometry (hole or not)
\item Beam propagation
\item Optics and noise consideration
\item Polarisation
\item ellipticity 
\item estimate $\eta$
\end{itemize}

As fielded successfully in Chapters \ref{Chap:linICS} and \ref{Chap:RR15}, we propose the use of an f/2 OAP which allows tight focusing to a spot of similar dimensions as the electron beam at the exit of the accelerator. We also propose to keep the head-on geometry as used so far which requires a central hole in the parabola. 
Loss in energy for the hole is very similar to loss when going off-axis. Focus close to the target (will be explained later). We hope to use the maximum energy possible which from the standpoint of optics not be a problem. This would allow us reaching intensities at focus in excess of $a_0 = 20$ at 45 fs pulse duration, which combined with a 1.3 GeV electron beam brings us to CHI ETA XX NUMBER and if we manage to increase the electron beam energy to 2 GeV to CHI ETA XX NUMBER.

Due to the short focal length the OAP has to be protected again against scattered light from the driver beam defocused in the plasma. This is done by covering a ring around the hole with plastic.
Since we decided on a gas jet the level of debris expected is low, but a pellicle (AR thin foil) is still recommended. The beam dump for the defocusing scattering beam should be positioned far away to reduce the fluence and the debris generated. A second parabola could be used to recollimate the beam, transport it out of the chamber as diagnostics, but since the laser pulse propagates through plasma at different intensities and focal points, we can not confidently predict the properties of the beam, where it might focus and produce damage, so a simple beam block facing away from any sensitive optics is preferred.

Depending on the predicted divergence of produced pairs in very intense interactions, the central hole might not be large enough and we have to consider going off-axis or to increase the hole size, which currently is anyways effectively larger due to the plastic cover.

An adaptive optic and wavefront measurement is important as well, optimisation using focal spot camera again.

For the intensity measurements that depend on the polarisation of the scattering beam. It would be useful to implement a waveplate on a rotation mount, one for linear polarisation and one for circular polarisation.

The polarisation of the beam is ideally also diagnosed as verification. Such a diagnostic was recently built and fielded by Matthew Streeter (Imperial College) at Gemini.

As implemented as standard at Gemini, we require an adaptive optic and a wavefront sensor to optimise the spot, along with a focal spot camera at the interaction point. Similarly we benefit from on-shot laser energy measurements.

The fluctuation of the laser, expected to be small due to the short focal length, also have to be characterised.

\subsection{Spatial and temporal Overlap}


\begin{itemize}
\item photo diode (also in vacuum to avoid pump-down procedures?)
\item  spatial interferometry at TCC \cite{Cole2018_RR}
\item  spectral interferometry at TCC \cite{Corvan2016_TIMING}
\item  on-shot timing, as done at Gemini \cite{Shalloo_GEMINIDRIFT}, upstairs, downstairs, analysis pending (staging)
\item on-shot pointing reference (done on staging)
\item  Shock injection to fix laser-electron offset and short bunches, calculate fluctuations from shock fluctuations.
\item  Finding alignment using raster scan as described in Chapter \ref{Chap:linICS}
\item  outside plasma, other effects like fringes or plasma afterglow \cite{Scherkl2019_PLASMAAFTERGLOW} to consider? only mJ laser required but match and remove driver laser, tune via shuttering?
\item opto-electric (calibrate laser-electron offset, which then can be used in conjunction with spectral interferometry on-shot, find delay via some shuttering)? \cite{Cavalieri2005_EOS,Yan2000_EOS} GaP sample?
\item another ICS source (estimate brightness required)
\item fast transverse probe (shadowgraphy)
\item fast transverse probe also gives polarogram?
\item probe or top view with high mag?
\item improvements to prism setup and actual overlap (wire at end of gas jet, if shooting very far away?)
\end{itemize}


\begin{table}[h]
\centering
\begin{tabular}{l|l|l|l|c}
Diagnostic & Accuracy & Setup & Mode & Fielded?\\ \hline \hline
Photodetector & ps & Fast diode and Oscilloscope & LP & \cmark \\
Spatial Interferometry & fs & Prism and camera & LP & \cmark \\
Spectral interferometry & ps & grating and camera/spectrometer & LP/FP & \cmark \\
Transverse Probe & ps/fs & Shadowgraphy/Scatter & FP & \cmark/\xmark\\
Opto-electric & fs & Crystal and Probe & FP & \xmark
\end{tabular}
\caption[Overview of timing diagnostics.]{Overview of timing diagnostics.}
\end{table}

Estimate laser-electron jitter:

\begin{itemize}
\item for self-injection using front velocity or dephasing length?
\item fluctuation for measured variation in shock position assuming fixed injection point
\item fix the ofset
\end{itemize}


\subsection{Electron beam diagnostics}




\begin{table}[h]
\centering
\begin{tabular}{l|l|l|l|c}
Diagnostic & Observable & Setup & Comment & Fielded?\\ \hline \hline
Beam Profile & Profile & Lanex & Undispersed, scatter & \cmark \\
Spectrometer I & Energy & Magnet, Lanex & Dispersed, scatter & \cmark \\
Spectrometer II & Energy & Magnet, Lanex & Dispersed, scatter & \cmark \\
Fiudicials & Reference & Wires & Scatter, block, noise & \cmark/\xmark
\end{tabular}
\caption[Overview of electron beam diagnostics.]{Overview of electron beam diagnostics.}
\end{table}

\subsubsection{Electron beam profile}

Since we will be attempting high-intensity interactions at small laser focus, we will need to characterise the pointing and divergence fluctuations. This is necessary to estimate the collision likelihood but also to estimate the error on an intensity measurement based on the divergence profile.

An simple and effective method is a Lanex profile screen in the undispersed beam as used frequently on experiments. This, however, was unsuitable in the experimental setup proposed as the laser pulse is not disposed of by a tape and the backside is damaged, which produces debris and also bremsstrahlung which is unsuitable during collisions. In the process of colliding beams, we can then move to diagnosing the properties of the beam using linear ICS as investigated in Chapter \ref{Chap:linICS} which provides a non-invasive technique that can be sustained without material damage over hundreds of shots.

\subsubsection{Electron spectrometer}

The electron energy will be measured as previously using a magnetic spectrometer setup. As in Chapter \ref{Chap:linICS} this can be done just outside the vacuum window which allows easy access to the screen for absolute calibration and imaging.

A thin window (kevlar kapton) as used in Chapter \ref{Chap:linICS} helps to reduce the scattering and producing secondaries. 

Since we have seen in Chapter \ref{Chap:linICS} that the electron beam underlies significant jitter in pointing and the an accurate knowledge of the energy is important for the measurement, we propose to install a second Lanex screen to account for global pointing fluctuations \cite{Soloviev2011_TWOSCREEN}, as done in Chapter \ref{Chap:linICS} but not explicitly used in the analysis so far. We also consider using fiudicials as spatial references \cite{Wang2013_GEV_FIUDICIAL}, but the available drift distances, scattering in screens and additional noise are reasons against this. Since we are also interested in measuring changes in energy spread, we propose a separate camera imaging a region of interest at high resolution. This was done in Chapter \ref{Chap:linICS}, but due to insufficient filtering the images were saturated.

\begin{figure}
\centering
\includegraphics[width=0.9\columnwidth]{Ebeam_Diag.png}
\caption{Magnet spectrometer setup with two Lanex screens.}
\end{figure}

Since we have the capability to produce micron sized electron beams and interact with a large amount of them, we do not require large dynamic range and can use standard Lanex. However, more sensitive Lanex (Biomax) can again be used to measure an on-axis X-ray signature as on-shot pointing reference.
The second screen in Chapter \ref{Chap:linICS} was in line with the gamma profile screen which also provides a laser reference. This preferably should e done again.

The energy measurement is cut off at 220 MeV due to the aperture of the magnet. It might be interesting to measure the full spectrum, which could be achieved by placing a Lanex screen inside onto the yoke. Scintillators lose efficiency in magnets, so this might provide only a low signal and will also be hard to calibrate in absolute numbers.


We have shown that a statistical characterisation of the electron beam can be done and that these are normally (randomly) distributed after removing correlations.
We were able to measure energy loss from radiation reaction and that the statistical analysis helped estimating the likelihood.

\subsection{Radiation diagnostics}

\subsubsection{Gamma Profile}

 \cite{HarShemesh2012_INTENSITY,Yan2017_ICS} NO REFERENCE FROM T BLACKBURN YET)

\begin{itemize}
\item scintillating screen
\item spatial diagnostic
\item intensity diagnostic
\item calculate field of view required for $a_0 = 25$ interactions
\item spatial resolution required?
\end{itemize}


Estimate required spatial dimensions of the gamma profile screen as follow:

\begin{itemize}
\item the radiation emitted is $\theta \sim a_0/\gamma$ for a half angle or variance $\sigma = a_0/4\gamma$
\item this adding up with an initial electron beam divergence of $theta_e$
\item and a typical beam pointing fluctuation of $\sigma_e$
\item assume the extreme case of $a_0 = 25$ and $1$ GeV
\item this gives $12.7 mrad$, is this full or half angle?
\item initial divergence around 3 mrad
\item pointing around 1 mrad
\item need around 17 mrad
\item ignore source size
\item if full angle then at same plane as this time (2.31 m), we are at 39.3 mm
\item if double then suddenly 87.8 mm, so need larger stack
\item at the same time spatial resolution as is now is important
\item if single angle then one stack (well centred), otherwise need double sized and would prefer to have full view
\item since we have seen signals of order 10 mm already, probably more like 90 mm in total
\item stacking different stacks is not ideal due to different responses
\item need to consider window sizes for these kind of measurements and noise production at the hole?
\end{itemize}

\subsubsection{Gamma Spectrometer}


\begin{figure}
\centering
\includegraphics[width=0.9\columnwidth]{GammaDets.png}
\caption{Gamma detectors as proposed for a future campaign. Gamma profile and (dual-axis) gamma spectrometer as used before to measure profile and intensity. A new, not yet fielded, addition would be the converter spectrometer separating produced pairs and Compton electrons using a magnet and measuring these to increase the sensitivity of the spectrometer setup.}
\end{figure}

\begin{itemize}
\item gamma spectrometer to estimate spectrum
\item dual-axis could be useful and fielded
\item want to improve resolution, so maybe combine with converter
\item has to be balanced with gamma profile position, calculate dispersion needed, noise from aperture and so on
\end{itemize}


\subsubsection{X-ray Crystal Spectrometer}

In Chapter \ref{Chap:BW}, we used a crystal spectrometer setup in vacuum to measure radiation at around 1.5 keV produced from a burn-through foil heated by a laser. In the context of radiation reaction, lower energy X-rays are also of interest and the capability to measure those is of advantage. Derivations in the locally constant field approximation (LCFA) \cite{Ritus1985_QRR} and models beyond deviate in particular at lower energies \cite{DiPiazza2018_LCFA}. In LCFA low energy X-rays are emitted at larger divergences, so measuring low energy X-rays at high divergences is a suitable test of the approximation. Including a polarisation dependence will help distinguishing signal. Due to the low energy it is necessary to measure the X-rays in vacuum, but at the same time we have to stay clear of the laser axis to avoid laser damage and to interfere with the electron beam and the gamma radiation.

\begin{figure}
\centering
\includegraphics[width=0.9\columnwidth]{LCFACam_CloseUp.png}
\caption{Measuring divergent X-ray components with a crystal spectrometer setup as used in Chapter \ref{Chap:BW}. A measurement of this radiation can help to investigate the validity of the locally constant field approximation (LCFA) in these interactions. The positioning of the crystal is challenging as it has to stay clear of the electron, positron and main gamma beam, and also needs to be protected from laser damage.}
\end{figure}


\begin{figure}
\centering
\includegraphics[width=0.5\columnwidth]{LCFA_DiPiazza.png}
\caption{Radiation emitted in an interaction of a 10 GeV electron beam XXX REF NUM XX with a laser pulse. The blue and red curve have been calculated XX. The dotted curve indicates the calculated spectrum in the locally constant field approximation (LCFA). Reprinted Figure 3 with permission from \cite{DiPiazza2018_LCFA}. Copyright (2019) by the American Physical Society.}
\end{figure}

\begin{figure}
\centering
\includegraphics[width=0.9\columnwidth]{GammaDiag_Overview.pdf}
\caption{Overview. Need to work this over and also include somewhere whether we already fielded diagnostics or not.}
\end{figure}

\begin{itemize}
\item motivation LCFA
\item angular distribution, calculate feasibility based on laser and gamma-ray divergence
\item polarisation dependence
\item what kind of angle or crystal would we require?
\item shielding or signal-to-noise?
\item inside chamber?
\end{itemize}



\subsection{Single Particle Diagnostics}

In high-intensity interactions with sufficiently energetic electrons the gamma rays produced in nonlinear ICS can merge with multiple laser photons to produce pairs from nonlinear BW and other pair production processes. The pairs are produced, similarly as in Chapter \ref{Chap:BW} in the direction of the electron beam, and due to the low number of particles (ESTIMATE) we require a detector that is able to measure down to the single particle level. In Chapter \ref{Chap:BW} we presented such a setup. A magnetic chicane is separating the right energy band of particles. Shielding reduces the noise from stray particles onto the detectors and allow sensitivity to single particle level. Large aperture magnets help reducing the chance of producing pairs from scattered radiation and particles close to the axis, which then make it to the detector plane through the chicanes. We have to investigate if an on-axis f/2 holey parabola makes any sense or will produce too much noise (FOVs). Tracking layers help to discriminate against noise and identify particle species. For this we used silicon-based, fast TimePix3 detectors. Absolute calibration of single particle detectors enables identification of correct particles. This has been shown in Chapter \ref{Chap:BW}.
The magnets were also tested for electron beams using an identical set of magnets, so the fields are nicely mapped.
We have also shown that we can use the Bethe-Heitler process to calibrate our chicane system, by inserting a high-Z material in the path of gamma-rays to calibrate the effective yield we can measure (tungsten blob).

\begin{figure}
\centering
\includegraphics[width=0.9\columnwidth]{MagneticChicane.png}
\caption{Sketch of the beam transport for electrons and positrons consisting of two permanent large aperture dipole magnets as used in Chapter \ref{Chap:BW}. Electrons and positrons are dispersed horizontally. The positrons are redirected onto a set of single particle detectors by a second dipole magnet.}
\end{figure}


\begin{figure}
\centering
\includegraphics[width=0.9\columnwidth]{SPDDiag_Overview.pdf}
\caption{Overview. Need to work this over and also include somewhere whether we already fielded diagnostics or not.}
\end{figure}



We fielded a magnetic chicane to isolate positrons in Chapter \ref{Chap:BW} and tracked particles for noise discrimination using TimePix3 detectors and measured particles on a single-particle counting CsI array. This shows us the shielding and detector capabilities enable us to measure single particles produced in an event and that we are able to distinguish those from noise. This would be of interest in a very intense interaction where in a second step non-linear BW. This gives us a threshold for intensity.
We were able to calibrate these diagnostics by converting gamma radiation into pairs by inserting converter materials in the beam path.


\begin{itemize}
\item when required? motivate energies and intensities (Tom B)
\item fielded before
\item symmetric setup
\item calibration with main e-beam
\item calibration with tungsten blob
\item why require a magnetic chicane?
\item noise levels
\item detectors: SPD
\item tracking layers
\item coincidence measurement possible?
\end{itemize}

\subsection{Detailed Experimental Measurements}

\begin{itemize}
\item provide each sub-goal with a table that shows what is required, or optional, what is feasible and fielded and what is new (and where fielded Chap)
\end{itemize}
Objective, conditions required, observables, improvements needed or optional, benefit for overall experiment?

\subsubsection{On-shot interaction intensity}

Measuring the interaction intensity of electrons and laser pulse.

Requires high-intensity interactions.

Requires low-intensity characterisation of the electron beam.

Requires gamma profile screen of sufficient field of view and spatial resolution.

Requires camera imaging the screen, here 14-bit EMCCD (Andor iXon).

This can help to constrain the experiment conditions and narrows down the model.

Optional rotating polarisation and changing from linear to circular.

Also ideally measurement of polarisation.

Varying intensity and electron beam intensity.

\cite{HarShemesh2012_INTENSITY,Yan2017_ICS}

REF TOM BLACKBURN MODEL INDEPENDENT A0

\subsubsection{High confidence No RR-RR Comparison}

See Chapter \ref{Chap:RR15}

\subsubsection{Precision Measurement of RR at high-intensities (quantum corrections)}

\cite{Blackburn2014_QRR,Ridgers2017_QRR}

\cite{Arran2019_RR_PPCF,Arran2019_RR_SPIE}


\subsubsection{Very high-intensity collisions with pair production}

\cite{Blackburn2017_pairs}

\subsubsection{Very high-intensity collisions with spin polarisation measurement}

\cite{SokolovTernov1964_POL}
\cite{DelSorbo2017_SPIN,Seipt2018_SPIN}

\subsubsection{Spectral Measurement increased accuracy}

Detector as in \cite{Cole2018_RR,Behm2018_Gamma}
combined with \cite{Corvan2014_Gamma}
as proposed in \cite{Lisi2018_Gamma}.

\subsubsection{Double-differential Gamma spectrum}





\subsubsection{LCFA low X-ray measurement}

\cite{Ritus1985_QRR}
\cite{DiPiazza2018_LCFA}



\begin{figure}
\centering
\includegraphics[width=1.0\columnwidth]{CARRAN_PLACEHOLDER.png}

\includegraphics[width=0.5\columnwidth]{CARRAN_Nmin.png}\includegraphics[width=0.5\columnwidth]{CARRAN_SingleShotConfidence.png}

\caption{Chris Arran's simulations including a dot for our best electron beams and show that these would be suitable for radiation reaction. Maybe just take picture from paper and reference (ask for permission).}
\end{figure}

\section{Final Remarks}