
\chapter{Conclusion}


... LWFA is versatile radiation source (betatron, linear ICS, non-linear ICS, bremsstrahlung)

\EliasComm{Make a plot of gamma radiation spectra to show the bandwidth and spectral shapes.}

... showed application of gamma spectrometer for measuring radiation in the range of few 100's keV to 100's MeV (betatron, linear ICS, nonlinear ICS, bremsstrahlung)

... shown gamma ray signal and background characterisation can be used to identify successful collisions in an ICS setup (linear and nonlinear ICS)

... shown control over laser and electron overlap over course of 100's of shots (linear ICS, nonlinear 4)

... retrieval requires shape input and hence detailed knowledge of the physics (all results)

... showed that smaller variations in the gamma spectrum will not be picked up (see for instance Chapter BW)

... shown that calibration procedure was optimised using PTFE converters (linear ICS)

... adding another diagnostic (converter target, pairs) could constrain this (see Chapter BW)

... shown that better knowledge of exact interaction conditions would constrain measurements more

... shown that a profile screen can be combined with a spectrometer we can look at the ellipticity of the source. This can in the future be interesting to measure the intensity at the interaction (model-independent a0 measurement, otherwise UMSTADTER REF IF NO REFERENCE FROM T BLACKBURN YET)

... shown that by using a dual-axis detector we can follow the decay of the gamma signal and measure its ellipticity. This might help constrain models more (DEPENDS IF TOM PUBLISHED SOMETHING AT THAT POINT).

... in BW campaign we fielded the magnetic chicane, particle tracking from TimePix and single particle detection using a CsI-camera setup. In high intensity non-linear ICS interactions we can produce pairs and the results show we are able to measure events down to the single-particle level (BW Chapter)

... in addition electron spectra were investigated and how their properties affect RR measurements. Other observables are explored in C ARRAN REFS.

... in Chapter linICS investigated shock injection and found narrow energy spread beams (REF PENDING). Shock injection also bears the option of keeping timing stable

... demonstrated overlap and timing possible, in linICS attempted to measure on-shot timing

... combining the spectral measurement with the electron statistical analysis and other observables we can do better measurements of RR, even at existing sub-PW facilities


\begin{figure}[h]
\centering
\includegraphics[width=0.8\columnwidth]{RR20_ExpSetup.png}
\caption{Future experiment layout incorporating lessons learned.}
\label{Concl:Fig:FutureExp}
\end{figure}

... all set for another attempt, for instance using a setup as shown in Figure \ref{Concl:Fig:FutureExp}.

\EliasComm{Make plot (Blender) of a future experiment setup including all the new diagnostics and indicating role of components. Interesting measurements (draw up new experiment plan basically, same as before plus another gamma detector).}

\subsubsection{Interaction Point (TCC)}

... f/40 laser at maximum energy possible

... f/2 (holey) at edge of gas target at maximum intensity possible, tight spot

... gas jet target to allow debris-free high-intensity interaction

... blade in gas jet for shock injection to produce high charge beams with narrow energy spread

... shock injection to fix timing

... transverse probe to see shock and diagnose density

... transverse probe to have visual confirmation of timing (relative, use probe) and overlap, high magnification for overlap

... top view to have visual confirmation of transverse position (high mag)

... spatial interferometry (prism)


\subsubsection{post-TCC}

... converter targets for calibrations

... 1.6 mm PTFE for narrow emission and on-shot measurement of electron spectra

... bismuth for efficient conversion

... thick lead to allow for full saturation

... magnet spectrometer (here use wide apertured horizontal)

... thin kevlar-kapton window

... first lanex screen high sensitivity right at the window to minimise scattering

... covering axis to get some kind of reference

... magnetic field to separate particles generated in window/beam block etc. Has to disperse vertically

... second lanex further down in one plane with gamma profile to allow for reference. double screen to increase energy resolution

... gamma profile at the front of CsI spectrometer (or needed as converter?). Needed for a0 measurement

... dual-axis CsI spectrometer to measure spectrum and potentially 2D spectral measurement

... screens above and below stack to capture converted particles

... magnetic chicane for single particles in high intensity interaction

... TimePix for tracking and background measurement

... single particle CsI measurement

\subsubsection{Laser diagnostics}

... on-shot spectral interferometry timing

... on shot far-field pointing

... on-shot energy measurement

... control over polarisation and measuring diagnostic

... focal spot cameras