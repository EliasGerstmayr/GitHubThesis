
\chapter{Conclusion}

\section{Summary of previous Results}

... LWFA is versatile radiation source (betatron, linear ICS, non-linear ICS, bremsstrahlung)

\EliasComm{Make a plot of gamma radiation spectra to show the bandwidth}

... showed application of gamma spectrometer for measuring radiation in the range of few 100's keV to 100's MeV

... shown gamma ray signal and background characterisation can be used to identify successful collisions in an ICS setup

... shown control over laser and electron overlap over course of 100's of shots

... retrieval requires shape input and hence detailed knowledge of the physics

... showed that smaller variations in the spectrum will not be picked up (see for instance Chapter BW)

... adding another diagnostic (converter target, pairs) could constrain this (see Chapter BW)

... shown that better knowledge of exact interaction conditions would constrain measurements more

... shown that a profile screen can be combined with a spectrometer we can look at the ellipticity of the source (model-independent a0 measurement)

... shown that by using a dual-axis detector we can follow the decay of the gamma signal and measure its ellipticity. This might help constrain models more.

... in BW campaign we fielded the magnetic chicane, particle tracking from TimePix and single particle detection using a CsI-camera setup. In high intensity non-linear ICS interactions we can produce pairs and the results show we are able to measure events down to the single-particle level.

... combining the spectral measurement with the electron statistical analysis and other observables we can do better measurements of RR, even at existing sub-PW facilities


\begin{figure}[h]
\centering
\includegraphics[width=0.8\columnwidth]{RR20_ExpSetup.png}
\caption{Future experiment layout incorporating lessons learned.}
\end{figure}

... all set for another attempt

\EliasComm{Make plot (Blender) of a future experiment setup including all the new diagnostics and indicating role of components. Interesting measurements (draw up new experiment plan basically).}

