
\chapter{Conclusion}

\section{Summary of Results}

... LWFA is versatile radiation source (betatron, linear ICS, non-linear ICS, bremsstrahlung)

\EliasComm{Make a plot of gamma radiation spectra to show the bandwidth and spectral shapes.}

... showed application of gamma spectrometer for measuring radiation in the range of few 100's keV to 100's MeV (betatron, linear ICS, nonlinear ICS, bremsstrahlung)

... shown gamma ray signal and background characterisation can be used to identify successful collisions in an ICS setup (linear and nonlinear ICS)

... shown control over laser and electron overlap over course of 100's of shots (linear ICS, nonlinear 4)

... retrieval requires shape input and hence detailed knowledge of the physics (all results)

... showed that smaller variations in the gamma spectrum will not be picked up (see for instance Chapter BW)

... shown that calibration procedure was optimised using PTFE converters (linear ICS)

... adding another diagnostic (converter target, pairs) could constrain this (see Chapter BW)

... shown that better knowledge of exact interaction conditions would constrain measurements more

... shown that a profile screen can be combined with a spectrometer we can look at the ellipticity of the source. This can in the future be interesting to measure the intensity at the interaction (model-independent a0 measurement, otherwise UMSTADTER REF IF NO REFERENCE FROM T BLACKBURN YET)

... shown that by using a dual-axis detector we can follow the decay of the gamma signal and measure its ellipticity. This might help constrain models more (DEPENDS IF TOM PUBLISHED SOMETHING AT THAT POINT).

... in BW campaign we fielded the magnetic chicane, particle tracking from TimePix and single particle detection using a CsI-camera setup. In high intensity non-linear ICS interactions we can produce pairs and the results show we are able to measure events down to the single-particle level (BW Chapter)

... in addition electron spectra were investigated and how their properties affect RR measurements. Other observables are explored in C ARRAN REFS.

... in Chapter linICS investigated shock injection and found narrow energy spread beams (REF PENDING). Shock injection also bears the option of keeping timing stable

... demonstrated overlap and timing possible, in linICS attempted to measure on-shot timing

... combining the spectral measurement with the electron statistical analysis and other observables we can do better measurements of RR, even at existing sub-PW facilities



\section{Experimental Setup for Future RR campaign}

... reaching a regime where QED effects become relevant as the `quantumness' parameter $\eta$, indicating how much the interaction is dominated by QED effects \cite{Bell2008_PairsEta,Blackburn2014_QRR}, approaches values close to $1$. In this regime we expect gamma rays to be produced early in the interaction that can then combine with the copious amounts of laser photons available to produce electron-positron pairs \cite{Blackburn2017_pairs}.
... At the same time, this work in conjunction with \cite{Arran2019_RR_PPCF,Arran2019_RR_SPIE} indicates that by improving the electron beam quality differences in models can be measured even at comparable laser and electron energies, giving future experiments at Gemini or comparable laser facilities the opportunity to bridge the gap between these first measurements of radiation reaction and the future projects being worked on at the next generation experiments.

Future experiments at conventional facilities SFQED (SLAC), LUXE (Hamburg), Cala (Munich) but also at high intensity laser facilities at even more non-linear conditions at CORELS (Gwangju), ELI pillars, Munich (FOR), maybe Gemini again, EPAC, APOLLON will be able to take precision measurements.		


\begin{figure}[h]
\centering
\includegraphics[width=0.8\columnwidth]{RR20_ExpSetup.png}
\caption{Future experiment layout incorporating lessons learned.}
\label{Concl:Fig:FutureExp}
\end{figure}





... all set for another attempt, for instance using a setup as shown in Figure \ref{Concl:Fig:FutureExp}.

\EliasComm{Make plot (Blender) of a future experiment setup including all the new diagnostics and indicating role of components. Interesting measurements (draw up new experiment plan basically, same as before plus another gamma detector). Somewhat like a proposal for a new experiment? Challenges and experience answers?}


\subsection{Wakefield driver beam}

Use again an f/40 geometry with a wakefield driver at maximum energy possible.

In Chapter XX used XX J and managed to get beams. Shock injection is less dependent on exact pulse evolution, so more robust in that sense.
Use adaptive optic. More intensity might be able to drive further and increase energy. 


\subsection{Scattering beam}

As before use an f/2 OAP as this allows tight focusing to a spot of similar dimensions as the electron beam. Shorter focal lengths will not work with incoming beam etc. Off-axis can be done but might render our alignment and raster method not as suitable anymore.

Holey parabola will require again protection of the optics, also against debris and scattered light.

Loss in energy for the hole is very similar to loss when going off-axis. Focus close to the target (will be explained later).
Maximum energy as possible, at Gemini at 45 fs XX J, was not possible before.

Adaptive optic and wavefront measurement.

Waveplate to change polarisation would be useful to investigate dependence on polarisation but also for intensity measurements.

Beam dump.

\subsection{Laser diagnostics}

On-shot far field pointing to be able to distinguish collisions and probability on basis of on-shot data. Also identify drifts in the alignment. Commissioned for the first time properly in Chapter XX.
On-shot energy measurements are also important but also done already.

We require on-shot timing diagnostics, ideally spectral interferometry. This has been fielded in the setup described in Chapter XX, but needs confirmation of validity (upstairs and downstairs), analysis pending (OR WAS THIS STAGING)?.

Focal spot cameras to optimise the spots, wavefront measurement important as well.

Polarisation diagnostic as well important to confirm.



\subsection{Interaction Point (TCC)}

The gas target would be a gas jet, as concluded in Chapter \ref{Chap:RR15}, as it allows debris-free high-intensity interaction and open access for diagnostics, which is more favourable than the potential increases shot-to-shot stability of a gas cell target. It has been shown in Chapter \ref{Chap:linICS} that we can shoot very high above the nozzle and still produce suitable electron beams.

Introducing a shock front on purpose in Chapter \ref{Chap:linICS} produced comparable electron beams as in \ref{Chap:RR15}. The beams were of high charge and electrons were measured consistently over 100s of shots. Shock injection also showed the potential to produce high-charge narrow-energy-spread beams at the 1 GeV level at few percent energy spread. The reproducibility of these beams has to be improved and is ongoing research. However, the potential for these beams is favourable especially to distinguish models.
A shock injected beam is also expected to have a fixed injection point and hence constant relative timing to the driver beam at the interaction.

Potential to maybe extend the length of the gas jet to increase the energy. Although depends on mechanism for the short bunches, might be straight shock and not dephasing.

The gas jet will be diagnosed with a shadowgraphy and interferometry setup to determine the density profile, but also to identify interesting features in the channel and use TIMING OVERLAP FRINGES PROBE. This gives visual confirmation of the transverse position and overlap and timing.
This has been successfully demonstrated in Chapter XX.
HIGH MAG FOR FUTURE CAMPAIGNS.

ULTRASHORT PROBE FOR FUTURE TO MONITOR OVERLAP? (REF SAVERT)

The self-emission of the plasma is also imaged from two axes. This helps to monitor potential drifts in the alignment.
HIGH MAG FOR FUTURE CAMPAIGNS.


RELATIVE TIMING? 
Top view is used to position the prism along with the high mag etc. The angle of the prism needs to be fixed properly. Is there a way to improve overlap? We got the on-shot spectral interferometry working. Is there a way to gate some gas reaction?


\begin{figure}
\centering
\includegraphics[width=1.0\columnwidth]{GasJetCloseUp_Combo.pdf}
\caption[Close-up sketch of gas jet and diagnostics to characterise the plasma.]{Close-up sketch of gas jet and diagnostics to characterise the plasma. A transverse probe is used to obtain a shadowgram and an interferometry image. The self-emission of the plasma is imaged from the side and the top.}
\end{figure}


\subsection{post-TCC}

... converter targets for calibrations

... 1.6 mm PTFE for narrow emission and on-shot measurement of electron spectra

... bismuth for efficient conversion

... thick lead to allow for full saturation

... tungsten blob target for calibration of positron detectors

... bremsstrahlung targets have been investigated in detail in Chapter \ref{Chap:BW} along with the positron calibration and the magnetic chicane

\subsection{Particle diagnostic setup}


... magnet spectrometer (here use wide apertured horizontal to avoid production of pairs, although holey f/2 has small aperture)

... thin kevlar-kapton window to reduce secondary production and scattering of the main beam

... first lanex screen high sensitivity right at the window to minimise scattering

... covering axis to get some kind of reference

... magnetic field to separate particles generated in window/beam block etc. Has to disperse vertically

... second lanex further down in one plane with gamma profile to allow for reference. double screen to increase energy resolution

... magnetic chicane for single particles in high intensity interaction

... TimePix for tracking and background measurement

... single particle CsI measurement

\subsection{Gamma diagnostics}



... gamma profile at the front of CsI spectrometer (or needed as converter?). Needed for a0 measurement

... dual-axis CsI spectrometer to measure spectrum and potentially 2D spectral measurement

... screens above and below stack to capture converted particles




\subsection{Designing a full setup}

... Only RR setup (do the BW setup by itself in the discussion of that chapter)

\subsection{Procedure}

.. raster scan and analysis of data?

... or is this too specific?

\subsection{Measurement goals}

\subsubsection{Intensity Measurement using ellipse}
\subsubsection{Polarisation dependency Intensity}
\subsubsection{High confidence No RR-RR Comparison}
\subsubsection{Precision Measurement of RR at high-intensities }
\subsubsection{Very high-intensity collisions with pair production}
\subsubsection{Very high-intensity collisions with spin polarisation measurement}
\subsubsection{Spectral Measurement increased accuracy}
\subsubsection{Double-differential Gamma spectrum}


... measurement goal and requirements/challenges?

... a0 measurements and polarisation dependency

... discriminating radiation reaction and no radiation reaction to 5 sigma (using a stable charge beam and at 1.2 GeV at least)

... estimate number of shots based on previous estimate (using NO RR RR COMPARISON)

... discriminate semi-classical and stochastic measurement by look at change in energy width (requires narrow energy spread and stability, see C ARRAN RESULTS)

\EliasComm{Add figure here like shown in EAAC talk including my plot on no RR RR.}

... measuring pair production: need 2 GeV at least, so need longer target, and high intensity so can not use cell.

\EliasComm{Check if polarisation measurement possible.}

\EliasComm{Check if pair production realistic.}

\EliasComm{Intensity measurement.}

\EliasComm{Accurate spectral measurement using double-differential measurement (tungsten wire on a stage?)}

\EliasComm{Characterisation of the phase space, beam profile and fluctuations. Check if a scan in timing can give us a scan through betatron oscillations.}

\EliasComm{Precision measurement of radiation reaction including energy spread cooling and energy loss.}

\begin{figure}
\centering
\includegraphics[width=1.0\columnwidth]{CARRAN_PLACEHOLDER.png}

\includegraphics[width=0.5\columnwidth]{CARRAN_Nmin.png}\includegraphics[width=0.5\columnwidth]{CARRAN_SingleShotConfidence.png}

\caption{Chris Arran's simulations including a dot for our best electron beams and show that these would be suitable for radiation reaction. Maybe just take picture from paper and reference (ask for permission).}
\end{figure}

\section{Final Remarks}