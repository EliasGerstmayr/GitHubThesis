\chapter{Conclusion and Outlook}
\label{Chap:Conclusion_and_Outlook}

\section{Summary of Results}
\label{Chap:Conclusion}

The experimental results presented in this work demonstrate the capability of laser wakefield acceleration (LWFA) to produce highly energetic radiation from few to hundreds of MeV photon energies, and their application in the context of fundamental studies of quantum electrodynamics (QED):

\begin{itemize}
\item \textbf{Chapter \ref{Chap:linICS}} described how relativistic electron beams with energies up to $1.3\,\mathrm{GeV}$ were collided with a laser pulse at $a_0 \sim 0.2 - 1$, producing radiation from \textbf{linear inverse Compton scattering} of variable spectral shape in the range of 10s of MeV photon energies. The radiation yield and spectrum were used to diagnose the properties of the electron and the laser beam at the interaction. Linear inverse Compton scattering promises to become a useful \textbf{beam diagnostics and alignment tool} for future studies of radiation reaction.

\item \textbf{Chapter \ref{Chap:RR15}} outlined how relativistic electron beams of energy $\epsilon \approx 550\,\mathrm{MeV}$ were collided with a highly intense laser pulse of intensity $a_0 \approx 10$, producing broadband synchrotron-like radiation with a critical energy $\epsilon_{crit} > 35\,\mathrm{MeV}$ from \textbf{non-linear inverse Compton scattering}. The high energy of the emitted photons lead to a measurable energy loss in the electron spectra due to \textbf{radiation reaction}.


\item \textbf{Chapter \ref{Chap:BW}} reported how electron beams were propagated through a solid target to produce an energetic gamma-ray beam from \textbf{bremsstrahlung} reaching photon energies of several hundreds of MeV. The yield and collimation of the source were optimised for its application in a photon-photon collider experiment with the aim to produce electron-positron pairs from the \textbf{linear Breit-Wheeler process}. The second photon source was a $~$keV X-ray source from a burn-through foil heated by a second high-intensity laser.
\end{itemize}

\section{Discussion}
\label{Conclusion:Discussion}

\begin{itemize}
\item LWFA are still developing accelerators, in a few decades reached energies significantly, at similar rate as conventional accelerators (Livingston)
\item This has also been driven by the development of high-intensity lasers
\item At the same major improvements to the beam quality (see different injection mechanisms).
\item There are still major challenges to reproducibility, efficiency, repetition rate, energy spread and so on
\item diagnostic challenges as well to characterise the exact beam chirping, pulse duration, size on a on-shot basis
\item a full plasma accelerator for colliders is still in the far distance, but feasible as booster technology
\item traditional colliders for high precision measurement require different technologies, but LWFA community contributed important expertise
\item eventually very little alternative to plasma technology (dielectric...), high-field, as superconducting struggle to overcome gradient and scale/cost not feasible
\item but already now LWFA are feasible tool to investigate phenomena in the emerging field of strong-field QED (Challenges \cite{Samarin2017_RR}
\item all-optical setups for studies of radiation reaction (see Chapters\addnum{} and \cite{Cole2018_RR,Poder2018_RR}) and photon-photon colliders (see Chapter\addnum{} and \cite{Pike2014_BW})
\item competing with state-of-the-art particle accelerators (XFELs! SLAC, LUXE)
\item the co-location, intrinsic synchronisation, suitable energies, source size, beam width make LWFA a suited tool with unique properties 
\item whilst other properties are improved in the process, mitigation strategies, e.g. on-shot energy measurements
\item eventually need to decouple the uncertainties of the tool and the physics
\item current state of beams are, however, already feasible (C ARRAN papers)
\item also allows probing interplay of collective and strong-field quantum effects, HEDP!
\end{itemize}

\subsubsection{Challenges and Requirements}
Major challenges to go from here for LWFA to play a significant role:

\begin{itemize}
\item Laser characterisation
\begin{itemize}
\item intensity and wave front measurements at full power
\end{itemize}
\item Laser performance
\begin{itemize}
\item increased stability, repetition rates and efficiencies
\item new technologies to go forward
\end{itemize}
\item Electron beam characterisation
\item Electron beam quality
\end{itemize}

In general, also valid for conventional accelerators:
\begin{itemize}
\item Radiation characterisation (gamma radiation)
\item Improved theoretical description
\item spatio-temporal synchronisation
\end{itemize}

\subsubsection{New Physics}


\section{Outlook: Future Measurements of Radiation Reaction using LWFA}
\label{Chap:Outlook}

\subsection{Motivation}

\subsubsection{Achieved so far - Summary}

In Chapter \ref{Chap:RR15} we presented results from a measurement of radiation reaction (see also \cite{Cole2018_RR,Poder2018_RR}) which indicated first deviations from classical behaviour. The low number of shots due to the challenge that is overlapping the electron beam and the laser pulse in time, require a precision measurement to clearly distinguish models at high precision. In Chapter \ref{Chap:linICS} we shown methods to achieve overlap systematically over many shots, such that is within reach at a future experiment that could look at the models in a suitable regime \cite{Arran2019_RR_PPCF,Arran2019_RR_SPIE}. The experiment described in Chapter \ref{Chap:BW}, on the other hand, demonstrated our capabilities to successfully realise a photon-photon collider and to select single particles generated in interactions. This work in conjunction with \cite{Arran2019_RR_PPCF,Arran2019_RR_SPIE} indicates that by improving the electron beam quality differences in models can be measured even at comparable laser and electron energies, giving future experiments at Gemini or comparable laser facilities the opportunity to bridge the gap between these first measurements of radiation reaction and the future projects being worked on at the next generation experiments.
In addition to precision measurements of radiation reaction, higher energies and intensities will enable access to regimes and qualitatively different phenomena, which we will discuss in the following.
We will then discuss the different components of an experimental setup and suitable diagnostics.

\subsubsection{Where to go from here}

\begin{itemize}
\item what kind of interesting phenomena are there when increasing energy and intensity?
\item what kind of diagnostics and measurements would that require (which then leads over to the sections)?
\end{itemize}


This will facilitate extensive studies of Compton scattering, in particular the transition from the classical \cite{Yan2017_ICS} and the perturbative quantum \cite{Bula1996_RR} to the non-perturbative strong-field regime \cite{Bula1996_RR,TaPhuoc2012_ICS,Chen2013_ICS,Powers2014_ICS,Sarri2014_ICS,Khrennikov2015_ICS,Mackenroth2013_nlCompton}. As electrons in the central region of the laser focus will emit, on average, multiple hard gamma photons, also strong-field radiation reaction is accessible \cite{Wistisen2018_RR,Thomas2012_LL,Blackburn2014_QRR,Dinu2016_QRR,Harvey2017_QUENCHING,Vranic2014_RR} and precision measurements become feasible \cite{Bell2008_PairsEta,Blackburn2014_QRR}. 

\begin{equation}
\eta \propto \gamma a_0 \sim \gamma \sqrt{I_L}
\end{equation}

Notably, the first observation of tunnelling electron-positron pair production will be feasible \cite{Bell2008_PairsEta,Hu2010_TRIDENT,Ilderton2011_TRIDENT}, which is qualitatively different from multi-photon pair production observed by SLAC E-144 \cite{Burke1997_RR}. This is another term for the Schwinger pair production or vacuum breakdown. Reaching supercritical field strenghts will be a challenge \cite{Blackburn2019_SUPER}. We will also see pair production from cascades which are important in the astrophysical context \cite{Blackburn2017_pairs}.
With the right diagnostic we will also be able to investigate the breakdown of the local constant crossed field approximation (LCFA) \cite{Ritus1985_QRR,DiPiazza2018_LCFA}.
At increasing intensities we are also able to provide multi-step QED cascades \addref{} and stimulate prolific pair production for studies of QED plasmas (dense electron positron plasmas) \addref{}.

Intense interactions allow a polarisation of the electron beam in a fast Sokolov-Ternov effect \cite{SokolovTernov1964_POL}\cite{DelSorbo2017_SPIN,Seipt2018_SPIN} which bears promise for future linear colliders and producing spin-polarised measurements.

At extremely high intensities and energies in the future, potentially requiring the use of conventional accelerators with PW class lasers, we might be able to access the strong-field frontier or the strong-coupling regime of SFQED which requires XX\addnum{}, Narozhny conjecture \cite{Ritus1972_NAROZHNY,Narozhny1980_NAROZHNY,Fedotov2017_NAROZHNY}, which bears interesting physics beyond the description of current QED but will be hard to reach  \cite{Yakimenko2019_NONPERTURB}, \cite{Blackburn2019_SUPER}, \cite{Baumann2019_NONPERTURB}.
\EliasComm{In the proposal the energies were high and the intensity low, maybe check what the conditions are here, lower energy but higher intensity.}

Using the radiation produced in these intense interactions and combining those with an intense laser will give access to photon-photon interactions and analogue phenomena treating the vacuum as a nonlinear medium. This includes vacuum birefringence and dichroism, as well as nonlinear and linear Breit-Wheeler process, allowing the comparison of virtuality (see ATLAS experiments).
Other measurements include probing physics beyond the Standard Model (e.g. sleptons and axions), and the Hawking-Unruh effect \cite{Schutzhold2006_UnruhEffect,Chen1999_UnruhEffect}.

\begin{sidewaysfigure}
\centering
\includegraphics[width=1.0\columnwidth]{RR20_ExpSetup_V12Sep.png}
\caption{Future experiment layout incorporating lessons learned.}
\label{Concl:Fig:FutureExp}
\end{sidewaysfigure}

\EliasComm{A plot of interesting phenomena on a $a_0$ and electron energy map (mark where LWFA, crystals and so on are).}

All these developments require a superb control over the beam properties, electron and laser, and innovation and precise diagnostics for beam and radiation parameters, in particular challenging for radiation in the GeV range.


\subsubsection{Measurement routes and current projects}

There are different approaches to measuring strong-field phenomena.

\EliasComm{What is the distinction of these different approaches?}

\EliasComm{Maybe make map of nonlinearity $a_0$ or equivalent in crystals and beam energy. Is there a reason why crystals can't reach certain $a_0$ equivalent?}

Three main branches measuring radiation reaction and high-field effects currently being pursued:
\begin{itemize}
\item LWFA and high-intensity lasers \cite{Danson2019_PWLASERS}: ELI pillars \cite{Gales2018_ELINP,Weber2017_ELIBeamlines}, BELLA \cite{Leemans2013_BELLA,Toth2017_BELLA}, CoReLS (Gwangju, Korea), Gemini/EPAC \cite{EPAC_Website,Hooker2006_Gemini,Hooker2008_Gemini}, APOLLON \cite{Zou2015_Apollon}, SULF \cite{Cartlidge2018_SEL,Li2017_SULF}, HERCULES (ZEUS) \cite{Yanovsky2008_HERCULES,Maksimchuk2019_ZEUS}, Nebraska, Munich (FOR or Cala) \cite{Kessel2018_PWMPQ}
\item Conventional accelerators and high-intensity lasers:LUXE (XFEL Hamburg) \cite{Burkart2019_LUXE,Abramowicz2019_LUXE}, SFQED (FACET-II, SLAC) \cite{SFQEDOverview2019}, SULF CHINA XFEL
\item Positrons in aligned crystal lattices:  CERN \cite{Wistisen2018_RR,Wistisen2019_RR}
\item in future beam-beam interactions  \cite{Yakimenko2019_NONPERTURB}
\end{itemize}

For laser interactions three geometries for high-field (where particles are constrained and not ponderomotively lost) \cite{Blackburn2019_RRReview}
\begin{itemize}
\item laser and electron colliding pulses
\item solid targets
\item standing waves
\end{itemize}
 

In the following we will outline requirements to different parts of an experimental setup considering only the colliding pulse geometry. See also Guillermo's paper\addref.

\subsection{Laser Wakefield Accelerator}


\begin{itemize}
\item Focusing geometry: long focusing $f/40$ or longer
\item give example of maximum energy, show low energy spread is possible
\item Targetry and injection: gas jet (diagnostics access and damaging properties, stability can be improved, less important at high intensities, narrow-energy spread beams through shock injection, could fix timing with shock injection
\item for now not considering focusing optics like a capillary as size was much larger 
\end{itemize}



The gas target would be a gas jet, as concluded in Chapter \ref{Chap:RR15}, as it allows debris-free high-intensity interaction and open access for diagnostics, which is more favourable than the potential increases shot-to-shot stability of a gas cell target as used in Chapter \ref{Chap:BW}. It has been shown in Chapter \ref{Chap:linICS} that we can shoot suitably high above the nozzle ($up to 14\,\mathrm{mm}$) to avoid damage to the nozzle, whilst still encountering a sufficient density profile and still produce suitable electron beams.

Introducing a shock front on purpose in Chapter \ref{Chap:linICS} produced comparable electron beams to beams measured in \ref{Chap:RR15}. In Chapter \ref{Chap:RR15} the beams exhibited a distinct spectral feature at 550 MeV. This was reproduced in Chapter \ref{Chap:linICS}, but reaching higher energies up to 1.3 GeV. The beams were of high charge and electrons were measured consistently over 100s of shots. In some cases, shock injection also showed the potential to produce high-charge narrow-energy-spread beams at the 1 GeV level at few percent energy spread. The reproducibility of these beams has to be improved and is ongoing research. However, the potential for these beams is favourable especially to distinguish models.
A shock injected beam is also expected to have a fixed injection point and hence constant relative timing to the driver beam at the interaction.

We hope to explore whether we can increase the final energy of the electron beam whilst maintaining the low energy spread by increasing the length of the gas jet and increasing the driver beam energy. 
\vspace{\baselineskip}



Low energy spread beams (shock injection, but also \cite{Wang2013_GEV_FIUDICIAL} and see Chapter \ref{Chap:linICS})




We are able to produce a high-charge electron beam with the potential to be a narrow-energy spread beam (shock injection, Chapter \ref{Chap:linICS}) with fixed injection point. 



\iffalse
\begin{table}[h]
\centering
\begin{tabular}{l|l|l|l|c}
Diagnostic & Observable & Setup & Comment & Fielded?\\ \hline \hline
Shadowgraphy & Channel & Probe & Leakage, Timing & \cmark \\
Shadowgraphy II & Channel & Probe & Leakage, Timing & \cmark/\xmark \\
Shadowgraphy HM & Channel & Probe & Leakage, Timing & \xmark \\
Shadowgraphy FASTHM & Channel & Probe & Leakage, Timing, Fast & \xmark \\
Interferometry & Density & Probe & Leakage, Timing & \cmark \\
Self-Emission & Channel & Probe & Top View & \cmark \\
Self-Emission HM & Channel & Probe & Top View & \xmark \\
\end{tabular}
\caption[Overview of plasma diagnostics.]{Overview of plasma diagnostics.}
\end{table}
\fi

Diagnosing the plasma


\begin{figure}
\centering
\includegraphics[width=1.0\columnwidth]{GasJetCloseUp_Combo.pdf}
\caption[Close-up sketch of gas jet and diagnostics to characterise the plasma.]{Close-up sketch of gas jet and diagnostics to characterise the plasma. A transverse probe is used to obtain a shadowgram and an interferometry image. The self-emission of the plasma is imaged from the side and the top.}
\end{figure}



\subsection{Scattering Geometry}

Simplistically the highest $\eta$ is preferable which is maximised at the highest electron energies, laser intensities and in a head-on collision.
After considering the electron source we are now discussing the laser intensities and the collision angle.
\begin{equation}
\eta = \frac{(1-\cos \theta) \gamma E_L}{E_S} \propto (1-\cos\theta) \sqrt{I_L}
\end{equation}

\subsubsection{Short focal length}

In this work an f/2 off-axis parabola was used to focus the laser tightly to interact with the electron beam. The tighter the focusing, the higher the intensities that can be achieved eventually. On the other hand, longer focal length optics maintain a high intensity over a wider range (larger $z_R$) which facilitate overlap.
At the exit of the wakefield accelerator the electron beam is of $\sim\microns$ size. A focal spot of slightly larger size than that would be preferable (at least) such that the electron beam acts as a probe, otherwise it is harder to model an interaction. As a result, smaller focuses than that are for fundamental studies maybe not suitable. Also it is hard to reach diffraction limit (micron).

\subsubsection{Beam geometry}
\begin{figure}
\centering
\includegraphics[width=.5\columnwidth]{DoubleF2Setup.png}\includegraphics[width=.5\columnwidth]{HoleyF2Setup.png}
\caption{Scattering geometries using either a head-on geometry (right) or an off-axis parabola at an off-angle (left). Here we also use a second parabola to re-collimate the light.}
\end{figure}

$\eta$ is maximised in a head-on collision. As a result, it is favourable to place the OAP as close as possible to the laser axis. One can then either place the OAP directly onto the axis, which then requires a hole in the optics, or place it close enough leaving it at an angle. A hole in the beam results in loss of energy and the head-on collision bears risks of damaging the optic and, more importantly, the laser upstream. It also has to be large enough to avoid any interaction with radiation as it could produce secondary particles (noise) or could in general affect the measurement of particles and radiation. The laser after the interaction is then rapidly diverging through the plasma such that the gas target has to be far enough from the axis or very robust. A gas jet is preferable. Otherwise we produce debris which again can damage optics in the area, especially at these high intensities at future facilities.
On the other hand, on-axis not only maximised $\eta$, it is also easier to align and the overlap is longer and easier to maintain. The prolonged overlap of the laser pulse and the electron beam is particularly crucial to induce cascades.

The off-axis angle, on the other hand, allows more flexibility of the target (height) and the full energy is used. Potential diffraction effects are also gone at cost of the reduced angle. The beam can also be recollimated by another OAP after the interaction to diagnose the beam\addref{}. The alignment is slightly more complicated and the overlap is reduced. Whilst this is not the ideal configuration to study cascades, it is being proposed that an oblique angle of incidence is required to reach critical or even supercritical energies and to avoid losses through radiation and particle production \cite{Blackburn2019_SUPER}.

In summary, both geometries (see Figure XX\addnum) have their advantages and disadvantages, and it might be advisable to switch between those for different research questions, in particular reaching high effective $\eta$ (supercritical) versus cascades and long overlap.


\subsubsection{Polarisation}

A useful control to have over the scattering beam is the polarisation of the laser pulse. Detailed studies of nonlinear Compton scattering are polarisation dependent and so are radiation reaction effects as a result. In addition, the polarisation is important in order to enable on-shot intensity measurements  \cite{HarShemesh2012_INTENSITY,Yan2017_ICS,Blackburn2019_ModelIndependentLaser} and to check that these are real effects (and not electron divergence for instance). Consequently, a suitable polarisation diagnostic along with waveplates are required. Finally, it has been proposed that a right degree of elliptical polarisation can be used to polarise the electron beam in an interaction and to spatially separate the spin-polarised components\addref.


\subsubsection{Laser diagnostics}

Other laser pulse diagnostics and controls that are in particular useful, are the control over temporal and spatial shape of the laser pulse. We also need to measure these on-shot as a result. The spatial wave front and its control using an adaptive optic and on-shot wave front measurement, for instance, is important to model and monitor the interaction appropriately. At the same time it provides a powerful tool to manipulate the interaction. One example would be to use an astigmatic focal spot to interaction with parts of the electron beam to enable an on-shot measurement of the unperturbed intensity (see Baird\addref).
The control over the temporal shape, on the other hand, is important for the interaction as it determines when high values of $\eta$ are achieved within the interaction. It can then be used to stimulate radiation early for cascades or to mitigate early radiation losses in order to achieve a high peak $\eta$. The control is given by a dazzler, whereas a FROG measurement could be used on-shot, for instance using the laser chain and the post-interaction laser pulse in the off-axis scenario.



\subsection{Spatial and temporal Overlap and alignment}


\subsubsection{Laser-laser diagnostics}

To overlap both laser pulses we have demonstrated in the previous Chapters a multi-step approach, using different diagnostics for various timing accuracies.
We propose to use photo diodes for a rough picosecond timing, now also using a vacuum compatible setup as pump-down and let-up procedures affect the timing and alignment (see Section\addnum). Spatial \cite{Cole2018_RR} or spectral interferometry \cite{Corvan2016_TIMING} can then again be used to adjust for the fine femtosecond scale timing.

In addition to these diagnostics for the timing it will be important to monitor the timing throughout the experiment to maintain alignment and synchronisation over long time (see Section\addnum) \cite{Shalloo_GEMINIDRIFT}. For this purpose spectral interferometry \cite{Corvan2016_TIMING} with transmitted laser light can be used, matched to the TCC timing, similarly as the far field and near field pointing reference.

Moreover it would be useful to also monitor the post-interaction light to establish the actual timing offset of the wakefield driver beam at the end.
This is only possible when using the off-axis geometry as the laser is strongly dispersed in the plasma in the other case.

In Chapter \ref{Chap:linICS} we demonstrated that a transverse optical probe can resolve the channels of both laser pulses. A high magnification shadowgraphy and fast shadowgraphy as in \cite{Buck2011_BUBBLE,Savert2015_BUBBLE,Siminos2016_FASTSHADOW} could be used from two orthogonal axes (see Figure\addnum) to monitor the spatial overlap (and to correct for it) and also for the change in timing. However, especially the fast characteristic requires a lot of work and adds complexity.

\subsubsection{Laser-electron diagnostics}

\begin{itemize}
\item  Shock injection to fix laser-electron offset and short bunches, calculate fluctuations from shock fluctuations.
\item  Finding alignment using raster scan as described in Chapter \ref{Chap:linICS}
\item  outside plasma, other effects like fringes or plasma afterglow \cite{Scherkl2019_PLASMAAFTERGLOW} to consider? only mJ laser required but match and remove driver laser, tune via shuttering?
\item opto-electric (calibrate laser-electron offset, which then can be used in conjunction with spectral interferometry on-shot, find delay via some shuttering)? \cite{Cavalieri2005_EOS,Yan2000_EOS} GaP sample?
\item another ICS source (estimate brightness required)
\end{itemize}

\begin{table}[h]
\centering
\begin{tabular}{l|l|l|l|c}
Diagnostic & Accuracy & Setup & Mode & Fielded?\\ \hline \hline
Photodetector & ps & Fast diode and Oscilloscope & LP & \cmark \\
Spatial Interferometry & fs & Prism and camera & LP & \cmark \\
Spectral interferometry & ps & grating and camera/spectrometer & LP/FP & \cmark \\
Transverse Probe & ps/fs & Shadowgraphy/Scatter & FP & \cmark/\xmark\\
Opto-electric & fs & Crystal and Probe & FP & \xmark
\end{tabular}
\caption[Overview of timing diagnostics.]{Overview of timing diagnostics.}
\end{table}


\begin{figure}
\centering
\includegraphics[width=0.5\columnwidth]{CrossProbe.png}
\caption{Crossed probe beam at high magnification and temporal resolution for overlap.}
\end{figure}


\begin{itemize}
\item for self-injection using front velocity or dephasing length?
\item fluctuation for measured variation in shock position assuming fixed injection point
\item fix the offset
\end{itemize}


\subsubsection{Automated alignment}

Bayesian inference?

\subsection{Electron beam diagnostics}



\begin{itemize}
\item electron beam measurement
\item ICS for profile measurement, ellipticity, etc.
\item double screen spectrometer, fiudicials
\item polarisation measurement with ICS?
\item high resolution/different dynamic range/high mag
\item overlap radiation and energy axis for calibration and energy reference
\item statistical analysis
\item absolute calibration, in a permanent setup ideally
\end{itemize}

The electron energy will be measured as previously using a magnetic spectrometer setup. As in Chapter \ref{Chap:linICS} this can be done just outside the vacuum window which allows easy access to the screen for absolute calibration and imaging.

A thin window (kevlar kapton) as used in Chapter \ref{Chap:linICS} helps to reduce the scattering and producing secondaries. 

Since we have seen in Chapter \ref{Chap:linICS} that the electron beam underlies significant jitter in pointing and the an accurate knowledge of the energy is important for the measurement, we propose to install a second Lanex screen to account for global pointing fluctuations \cite{Soloviev2011_TWOSCREEN}, as done in Chapter \ref{Chap:linICS} but not explicitly used in the analysis so far. We also consider using fiudicials as spatial references \cite{Wang2013_GEV_FIUDICIAL}, but the available drift distances, scattering in screens and additional noise are reasons against this. Since we are also interested in measuring changes in energy spread, we propose a separate camera imaging a region of interest at high resolution. This was done in Chapter \ref{Chap:linICS}, but due to insufficient filtering the images were saturated.

\begin{figure}
\centering
\includegraphics[width=0.9\columnwidth]{Ebeam_Diag.png}
\caption{Magnet spectrometer setup with two Lanex screens.}
\end{figure}



\subsection{Radiation diagnostics}

\subsubsection{Gamma Profile}

 \cite{HarShemesh2012_INTENSITY,Yan2017_ICS,Blackburn2019_ModelIndependentLaser}

\begin{itemize}
\item scintillating screen
\item spatial diagnostic
\item intensity diagnostic
\item calculate field of view required for $a_0 = 25$ interactions
\item spatial resolution required?
\item the larger the less feasible a pixellated array becomes, but if very bright can use different material (DRZ, Lanex) 
\end{itemize}

\subsubsection{Gamma Spectrometer}


\begin{figure}
\centering
\includegraphics[width=0.9\columnwidth]{GmmaDets_DoubleScreen.png}
\caption{Gamma detectors as proposed for a future campaign. Gamma profile and (dual-axis) gamma spectrometer as used before to measure profile and intensity. A new, not yet fielded, addition would be the converter spectrometer separating produced pairs and Compton electrons using a magnet and measuring these to increase the sensitivity of the spectrometer setup.}
\end{figure}

\begin{itemize}
\item gamma spectrometer to estimate spectrum
\item dual-axis could be useful or shorter crystals such that it effectively sees things from two side all the time?
\item 2D screens followed by mirror, but complicated and requires a lot of space
\item to improve spectral retrieval require second measurements
\item converter target and magnet required
\item has to be balanced with gamma profile position, calculate dispersion needed, noise from aperture and so on
\end{itemize}

Detector as in \cite{Cole2018_RR,Behm2018_Gamma}
combined with \cite{Corvan2014_Gamma}
as proposed in \cite{Lisi2018_Gamma}.

\subsubsection{X-ray Crystal Spectrometer}

In Chapter \ref{Chap:BW}, we used a crystal spectrometer setup in vacuum to measure radiation at around 1.5 keV produced from a burn-through foil heated by a laser. In the context of radiation reaction, lower energy X-rays are also of interest and the capability to measure those is of advantage. Derivations in the locally constant field approximation (LCFA) \cite{Ritus1985_QRR} and models beyond deviate in particular at lower energies \cite{DiPiazza2018_LCFA}. In LCFA low energy X-rays are emitted at larger divergences, so measuring low energy X-rays at high divergences is a suitable test of the approximation. Including a polarisation dependence will help distinguishing signal. Due to the low energy it is necessary to measure the X-rays in vacuum, but at the same time we have to stay clear of the laser axis to avoid laser damage and to interfere with the electron beam and the gamma radiation.

\begin{figure}
\centering
\includegraphics[width=0.9\columnwidth]{LCFACam_CloseUp.png}
\caption{Measuring divergent X-ray components with a crystal spectrometer setup as used in Chapter \ref{Chap:BW}. A measurement of this radiation can help to investigate the validity of the locally constant field approximation (LCFA) in these interactions. The positioning of the crystal is challenging as it has to stay clear of the electron, positron and main gamma beam, and also needs to be protected from laser damage.}
\end{figure}


\begin{figure}
\centering
\includegraphics[width=0.5\columnwidth]{LCFA_DiPiazza.png}
\caption{Radiation emitted in an interaction of a 10 GeV electron beam XXX REF NUM XX with a laser pulse. The blue and red curve have been calculated XX. The dotted curve indicates the calculated spectrum in the locally constant field approximation (LCFA). Reprinted Figure 3 with permission from \cite{DiPiazza2018_LCFA}. Copyright (2019) by the American Physical Society.}
\end{figure}

\begin{itemize}
\item motivation LCFA
\item angular distribution, calculate feasibility based on laser and gamma-ray divergence
\item polarisation dependence
\item what kind of angle or crystal would we require?
\item shielding or signal-to-noise?
\item inside chamber
\item is radiation polarisation dependent, i.e. is the crystal orientation important?
\item silicon detectors instead of cameras as at XFELs.
\end{itemize}



\subsection{Single Particle Diagnostics}

In high-intensity interactions with sufficiently energetic electrons the gamma rays produced in nonlinear ICS can merge with multiple laser photons to produce pairs from nonlinear BW and other pair production processes. The pairs are produced, similarly as in Chapter \ref{Chap:BW} in the direction of the electron beam, and due to the low number of particles (ESTIMATE) we require a detector that is able to measure down to the single particle level. In Chapter \ref{Chap:BW} we presented such a setup. A magnetic chicane is separating the right energy band of particles. Shielding reduces the noise from stray particles onto the detectors and allow sensitivity to single particle level. Large aperture magnets help reducing the chance of producing pairs from scattered radiation and particles close to the axis, which then make it to the detector plane through the chicanes. We have to investigate if an on-axis f/2 holey parabola makes any sense or will produce too much noise (FOVs). Tracking layers help to discriminate against noise and identify particle species. For this we used silicon-based, fast TimePix3 detectors. Absolute calibration of single particle detectors enables identification of correct particles. This has been shown in Chapter \ref{Chap:BW}.
The magnets were also tested for electron beams using an identical set of magnets, so the fields are nicely mapped.
We have also shown that we can use the Bethe-Heitler process to calibrate our chicane system, by inserting a high-Z material in the path of gamma-rays to calibrate the effective yield we can measure (tungsten blob).

\begin{figure}
\centering
\includegraphics[width=0.9\columnwidth]{MagneticChicane.png}
\caption{Sketch of the beam transport for electrons and positrons consisting of two permanent large aperture dipole magnets as used in Chapter \ref{Chap:BW}. Electrons and positrons are dispersed horizontally. The positrons are redirected onto a set of single particle detectors by a second dipole magnet.}
\end{figure}




We fielded a magnetic chicane to isolate positrons in Chapter \ref{Chap:BW} and tracked particles for noise discrimination using TimePix3 detectors and measured particles on a single-particle counting CsI array. This shows us the shielding and detector capabilities enable us to measure single particles produced in an event and that we are able to distinguish those from noise. This would be of interest in a very intense interaction where in a second step non-linear BW. This gives us a threshold for intensity.
We were able to calibrate these diagnostics by converting gamma radiation into pairs by inserting converter materials in the beam path.


\begin{itemize}
\item when required? motivate energies and intensities (Tom B)
\item fielded before
\item symmetric setup
\item calibration with main e-beam
\item calibration with tungsten blob
\item why require a magnetic chicane?
\item noise levels
\item detectors: SPD
\item tracking layers
\item coincidence measurement possible?
\item real calorimeter
\end{itemize}
