\chapter{Summary of Results}
\label{Chap:Conclusion}

The experimental results presented in this Thesis demonstrated the capability of laser wakefield acceleration (LWFA) to produce highly energetic radiation from few to hundreds of MeV photon energies, and their application in the context of fundamental studies of quantum electrodynamics (QED):

\begin{itemize}
\item In Chapter \ref{Chap:linICS}, relativistic electron beams with energies up to $1.3\,\mathrm{GeV}$ were collided with laser pulses at $a_0 \sim 0.3$, producing over 100s of shots radiation from linear inverse Compton scattering of variable spectral shape in the range of 10s of MeV photon energies. 
A procedure to align and overlap the laser pulse and electron beam and maintain the alignment over extended periods of time was developed. Linear inverse Compton scattering was also considered as non-invasive electron beam diagnostics.

\item In Chapter \ref{Chap:RR15}, relativistic electron beams of energy $\epsilon \approx 550\,\mathrm{MeV}$ were collided with a highly intense laser pulse of intensity $a_0 \approx 10$, producing broadband synchrotron-like radiation with a critical energy $\epsilon_{crit} > 35\,\mathrm{MeV}$ from non-linear inverse Compton scattering. The high energy of the emitted photons lead to a measurable energy loss in the electron spectra due to radiation reaction.


\item In Chapter \ref{Chap:BW}, electron beams were propagated through a bismuth target to produce a highly-energetic gamma-ray beam from bremsstrahlung. The yield, collimation and noise of the bremsstrahlung source were optimised for its application in a photon-photon collider experiment with the aim to produce electron-positron pairs from the linear Breit-Wheeler process. The second photon source was a keV X-ray source from a burn-through foil heated by a high-intensity laser. The variations in performance of both high-energy photon sources and the implications for measuring Breit-Wheeler pairs in this and future experiments were discussed.
\end{itemize}
