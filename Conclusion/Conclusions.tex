\chapter{Conclusion and Outlook}
\label{Chap:Conclusion_and_Outlook}

\section{Summary of Results}
\label{Chap:Conclusion}

The experimental results presented in this work demonstrate the capability of laser wakefield acceleration (LWFA) to produce highly energetic radiation from few to hundreds of MeV photon energies, and their application in the context of fundamental studies of quantum electrodynamics (QED):

\begin{itemize}
\item \textbf{Chapter \ref{Chap:linICS}} described how relativistic electron beams with energies up to $1.3\,\mathrm{GeV}$ were collided with a laser pulse at $a_0 \sim 0.2 - 1$, producing radiation from \textbf{linear inverse Compton scattering} of variable spectral shape in the range of 10s of MeV photon energies. The radiation yield and spectrum were used to diagnose the properties of the electron and the laser beam at the interaction. Linear inverse Compton scattering promises to become a useful \textbf{beam diagnostics and alignment tool} for future studies of radiation reaction.

\item \textbf{Chapter \ref{Chap:RR15}} outlined how relativistic electron beams of energy $\epsilon \approx 550\,\mathrm{MeV}$ were collided with a highly intense laser pulse of intensity $a_0 \approx 10$, producing broadband synchrotron-like radiation with a critical energy $\epsilon_{crit} > 35\,\mathrm{MeV}$ from \textbf{non-linear inverse Compton scattering}. The high energy of the emitted photons lead to a measurable energy loss in the electron spectra due to \textbf{radiation reaction}.


\item \textbf{Chapter \ref{Chap:BW}} reported how electron beams were propagated through a solid target to produce an energetic gamma-ray beam from \textbf{bremsstrahlung} reaching photon energies of several hundreds of MeV. The yield and collimation of the source were optimised for its application in a photon-photon collider experiment with the aim to produce electron-positron pairs from the \textbf{linear Breit-Wheeler process}. The second photon source was a $~$keV X-ray source from a burn-through foil heated by a second high-intensity laser.
\end{itemize}

\section{Discussion and Outlook}
\label{Conclusion:Discussion}

The experimental results presented in this work demonstrate that an all-optical colliding pulse geometry using a laser wakefield accelerator (LWFA) is suited to investigate fundamental (strong-field) QED (SFQED) phenomena (see Chapters \ref{Chap:RR15} and \ref{Chap:BW}), and can contribute significantly to this emerging research field \cite{Pike2014_BW,Cole2018_RR,Poder2018_RR}.
Whilst this bears exciting research opportunities, especially at the next generation of multi-PW laser systems \cite{Gales2018_ELINP,Weber2017_ELIBeamlines,Zou2015_Apollon,Li2017_SULF,Shen2018_SULF,Cartlidge2018_SEL,Toth2017_BELLA,Bashinov2014_XCELS,Kessel2018_PWMPQ,EPAC_Website,Maksimchuk2019_ZEUS,Yanovsky2008_HERCULES}, but demands a high performance of the laser system and accelerator, and good control over both, as well as the development of suitable diagnostics \cite{Samarin2017_RR} to facilitate precision measurements with statistical significance and to be competitive (or complementary) with mature conventional accelerator facilities investigating SFQED phenomena \cite{Burkart2019_LUXE,Abramowicz2019_LUXE,SFQEDOverview2019,Hogan2016_FACETII}.
\vspace{\baselineskip}

Since the first reported measurement of quasi-monoenergetic electron beams in 2004 \cite{Mangles2004_MONO,Faure2004_MONO,Geddes2004_MONO}, the maximum electron energies have increased by a factor of 100 \cite{Gonsalves2019_GEV} -- a rapid increase that aligns with the rate of energy gain observed historically in conventional accelerator technology \cite{LivingstonPlot}, e.g. see Figure \ref{Introduction:Figs:CERN_Livingston} \cite{Panofsky1997_Livingston} in Section \ref{Introduction:Sec:PartAccelerators}. This progress in the field of LWFA has over the past decades been strongly driven by the enhanced control over high-intensity lasers and the exponentially rising peak intensities. Along with an increase of the maximum electron energies there have been significant improvements to the overall beam quality \cite{Osterhoff2008_CELL} and an enhanced control over the final properties of the accelerated bunch \cite{Banerjee2013_TunableLWFA}, also related to a better understanding of injection mechanisms \cite{Buck2013_SHOCK}, the evolution of the bubble \cite{Guillaume_REPHASING}, and the tailoring of both. By now LWFA electron beams and the energetic radiation they can generate are suited for various applications \cite{Albert2016_APP}, including the study of fundamental QED phenomena discussed here \cite{Pike2014_BW,Cole2018_RR,Poder2018_RR}.

However, LWFA is still an actively evolving accelerator technology that has not yet reached a level of maturity comparable to those of accelerators based on radio-frequency (RF) cavities: 
current challenges are, for instance, the stability (or shot-to-shot reproducibility) of the accelerator and the production of low emittance and energy spread beams, for instance to be used in FELs \cite{Nakajima1996_FEL}. Another key challenge is to enable an operation of the accelerator at high repetition rate and efficiency \cite{Danson2019_PWLASERS}. 
There are also diagnostic challenges to characterise the pulse duration, beam size and the chirp of the beam accurately in an experiment  \cite{Downer2018_DiagnosticReview}.
This indicates that a particle collider fully based on plasma technology is still far in the future as traditional particle accelerators are employed for precision measurements. 
Next generations of lepton colliders, e.g. CLIC \cite{CLIC2012} or ILC \cite{ILC2013}, are hence still based on superconducting RF cavities. Nonetheless, the increasing scale and cost of RF-based accelerators, and their stagnating field gradients, will eventually require in some form the use of high-field technology such as plasma-based systems. 
For now, the wakefield community provides crucial expertise to enhance existing conventional accelerator facilities with beam-driven wakefield technology \cite{Caldwell2009_ProtonPWFA} and their use as energy boosters/afterburners \cite{Lee2002_Afterburner} as essential component in hybrid schemes \cite{Adli2013_PWFACollider} is investigated.
\vspace{\baselineskip}

The following sections will briefly recapitulate the research opportunities in SFQED for all-optical colliding-pulse setups using LWFA, and discuss more specifically challenges regarding diagnostics, and the required performance of the accelerator and laser system when attempting a measurement.


\subsection{Research Opportunities}
\label{Concl:Sec:ResearchOpportunities}

All-optical colliding-pulse setups where an LWFA electron beam is collided with a counter-propagating intense laser pulse are uniquely suited for studies of inverse Compton scattering, radiation reaction and other high-field phenomena (see Chapter \ref{Chap:RR15} and \cite{Cole2018_RR,Poder2018_RR}): 
in LWFA setups an intense laser system and a relativistic electron accelerator are co-located, and both are intrinsically synchronised to each other. 
Moreover, the small size of the electron beam at the end of the accelerator \cite{Weingartner2012_BUNCH} and its short bunch duration \cite{Lundh2011_BUNCH}, allows for a full overlap with a tightly focused (and hence very intense) laser pulse and hence an interaction with almost all of the electrons at a comparable intensity \cite{Harvey2016_ICSFOCUS}.
Since relativistic electron beams and intense laser pulses can produce a wide range of energetic radiation, this setup can also be converted in a pure photon-photon collider (see Chapter \ref{Chap:BW} and \cite{Pike2014_BW}).
As a result, LWFA and next generation high-intensity multi-PW laser systems \cite{Danson2019_PWLASERS,Gales2018_ELINP,Weber2017_ELIBeamlines,Zou2015_Apollon,Li2017_SULF,Shen2018_SULF,Cartlidge2018_SEL,Toth2017_BELLA,Bashinov2014_XCELS,Kessel2018_PWMPQ,EPAC_Website,Maksimchuk2019_ZEUS,Yanovsky2008_HERCULES} are in this context in direct competition with state-of-the-art particle accelerators like FACET-II at SLAC, LUXE at the EuXFEL \cite{Burkart2019_LUXE,Abramowicz2019_LUXE,SFQEDOverview2019,Hogan2016_FACETII}, and studies using aligned crystals at CERN \cite{Wistisen2018_RR,Wistisen2019_RR}.
\vspace{\baselineskip}

Colliding-pulse geometries with relativistic particles and highly intense laser will enable a detailed investigation of, for instance, Compton scattering (from the classical \cite{Yan2017_ICS} to the perturbative quantum description \cite{Bula1996_RR} and in the non-perturbative strong-field regime \cite{Bula1996_RR,TaPhuoc2012_ICS,Chen2013_ICS,Powers2014_ICS,Sarri2014_ICS,Khrennikov2015_ICS,Mackenroth2013_nlCompton}). Approaching a quantum nonlinearity parameter, $\eta$, of unity we can probe the transition from classical to quantum radiation reaction \cite{Wistisen2018_RR,Thomas2012_LL,Blackburn2014_QRR,Dinu2016_QRR,Vranic2014_RR} and perform precision measurements of (quantum) radiation reaction \cite{Arran2019_RR_PPCF,Arran2019_RR_SPIE}, e.g. the manifestations of stochasticity \cite{Niel2018_RRStochastic} through energy straggling \cite{Shen1972_STRAGGLING} and radiation quenching \cite{Harvey2017_QUENCHING} visible in the emission spectrum and the post-interaction electron spectrum \cite{Ridgers2017_QRR}. The emission might also result in a spin-polarisation of the electron beam \cite{DelSorbo2017_SPIN,Seipt2018_SPIN}\cite{SokolovTernov1964_POL}. 

In this regime intense interactions also give rise to (multi-step) cascades and showers \cite{Blackburn2017_pairs,Bulanov2013_Cascade}, which are particularly relevant in the astrophysical context. At $\eta \sim 1$ we reach the regime of nonperturbative strong-field QED \cite{Yakimenko2019_NONPERTURB,Blackburn2019_SUPER,Baumann2019_NONPERTURB}, where quantum effects dominate. Here QED cascades can produce dense pair plasmas \cite{Ridgers2012_DENSE,DoE2015_FES}. 

Suitable radiation diagnostics will enable measuring the formation length of the photon emission via breakdown of the local-constant field approximation \cite{Ritus1985_QRR,DiPiazza2018_LCFA,Ilderton2019_QEDPerturbBreakdown} and it will allow scrutinising the range of applicability of state-of-the-art numerical methods \cite{DiPiazza2019_LCFANUM,Elkina2011_QEDcascadesNUM,Ridgers2014_QEDNUM,Gonoskov2015_ReviewQEDNUM}.
\vspace{\baselineskip}

They also enable the investigation of photon processes such as two \cite{Pike2014_BW,Drebot2017_BW_ICS} and multi-photon pair production \cite{Burke1997_RR}, at $\eta \sim 1$ vacuum pair production \cite{Bell2008_PairsEta,Hu2010_TRIDENT,Ilderton2011_TRIDENT}, and probing the quantum fluctuations and the vacuum itself through nonlinear properties of the vacuum at these extreme fields such as vacuum birefringence  \cite{King2016_VB,Nakamiya2017_VB}, vacuum dichroism \cite{Bragin2017_VBVD}, and vacuum recollision \cite{Meuren2015_HERecollision,Kuchiev2007_Recollision}. Photon-photon interactions are also a potential route for probing physics beyond the Standard Model, e.g. sleptons \cite{Beresford2019_PhPh_SLEPTONS} and axions \cite{Knapen2017_PhPh_AXIONLIKE,Baldenegro2018_PhPh_AXIONLIKE}.
\vspace{\baselineskip}

At more extreme conditions the Hawking-Unruh effect \cite{Schutzhold2006_UnruhEffect,Chen1999_UnruhEffect} and the conjectured strong-coupling regime of QED, the Ritus-Narozhny conjecture \cite{Ritus1972_NAROZHNY,Narozhny1980_NAROZHNY,Fedotov2017_NAROZHNY}, become accessible.

\subsection{Experimental Considerations}
\label{Concl:Sec:ChallengesReq}

In order to make useful measurements of new phenomena in the SFQED regime we have to have an accurate understanding of the conditions at the interaction, e.g. electron spectra and laser properties, and measure the interaction products or post-interaction spectra accurately. The study of SFQED phenomena will require an improved performance or stability of the source and suitably precise diagnostics for laser pulses, particles and energetic radiation. Since laser wakefield accelerators suffer from `intrinsic' shot-to-shot fluctuations, the development of on-shot diagnostics are more pressing than in conventional accelerators. An improved performance of the laser system will also feed into a better performance of the electron source. Overall we also expect that theoretical and numerical techniques mature as more experimental results become available, and we will not discuss these explicitly.

\subsubsection{Scattering geometry}

We first consider the scattering geometry in the colliding-pulse setup.
Simplistically the highest $\eta$ is preferable in an interaction as it produces a stronger response of the system. $\eta$ increases with the electron energy, $\gamma$, and the laser intensity$\sqrt{I_L}$:
\begin{equation}
\eta = \frac{(1-\cos \theta) \gamma E_L}{E_S} \propto (1-\cos\theta) \sqrt{I_L}.
\end{equation}
Consequently, it is preferable to focus the laser pulse as tightly to increase the intensity. At the exit of the wakefield accelerator the electron beam is of $\sim\microns$ size. A focal spot of slightly larger size than that would be preferable (at least) such that the electron beam acts as a probe, otherwise different parts of the electron beam experience different intensities and the wavefront of the laser pulse has to be considered in the modelling process. On the other hand, longer focal length optics maintain a high intensity over a wider range and larger $z_R$ facilitate overlap between the laser pulse and the electron bunch.


\begin{figure}
\centering
\includegraphics[width=.5\columnwidth]{DoubleF2Setup.png}\includegraphics[width=.5\columnwidth]{HoleyF2Setup.png}
\caption{Scattering geometries using either a head-on geometry (right) or an off-axis parabola at an off-angle (left). Here we also use a second parabola to re-collimate the light.}
\label{Conclusion:Figs:ScatteringOAPs}
\end{figure}

At a fixed electron energy and laser intensity $\eta$ is maximised in a head-on collision, i.e. $\theta = \pi$, such that it is favourable to place the focusing optic as close as possible to the laser axis.
One can then either place the OAP directly onto the axis (see Figure \ref{Conclusion:Figs:ScatteringOAPs} (right)), which then requires a hole in the optics, or place it close enough leaving it at an angle (see Figure \ref{Conclusion:Figs:ScatteringOAPs} (left)). 

A hole in the beam results in loss of energy and the head-on collision bears risks of damaging the optic and, more importantly, the laser upstream. It also has to be large enough to avoid any interaction with radiation as it could produce secondary particles (noise) or could in general affect the measurement of particles and radiation. The laser after the interaction is then rapidly diverging through the plasma such that the gas target has to be far enough from the axis or very robust. A gas jet is preferable. Otherwise we produce debris which again can damage optics in the area, especially at these high intensities at future facilities.
On the other hand, on-axis not only maximised $\eta$, it is also easier to align and the overlap is longer and easier to maintain. The prolonged overlap of the laser pulse and the electron beam is particularly crucial to induce cascades. The polarisation is also direct. Larger beam sizes reduce the impact of the hole (might be able to use a holey mirror beforehand and use this as probe?). Very risky at high intensities.

The off-axis angle, on the other hand, allows more flexibility of the target (height) and the full energy is used. Potential diffraction effects are also gone at cost of the reduced angle. The beam can also be recollimated by another OAP after the interaction to diagnose the beam. The alignment is slightly more complicated and the overlap is reduced. Whilst this is not the ideal configuration to study cascades, especially if the conditions are at the threshold, it is being proposed that an oblique angle of incidence is required to reach critical or even supercritical energies and to avoid losses through radiation and particle production \cite{Blackburn2019_SUPER}. This is also less risky for future generations where the lasers are so intense that a perfect counterpropagating scenario is too dangerous, especially at high repetition rates.

As demonstrated in Chapter \ref{Chap:BW} a second interaction point is suited for photon-photon collisions. Considering the precarious situation where we do not want to produce secondary particles from interactions with matter, we use an OAP off-axis. We see that this can be done either directly by a laser to probe multiphoton pair processes, high-field effects, or use the laser to produce higher energetic radiation which is suited for two-photon pair production and some scatter experiments (energy scaling).

In the experiment described in Chapters \ref{Chap:RR15} the trade-off between energy loss and angle are almost equivalent, such that the benefit of the easy alignment was chosen. At higher intensities this might change in favour of an off-axis arrangement.
In summary, both geometries (see Figure \ref{Conclusion:Figs:ScatteringOAPs}) have their advantages and disadvantages, and it might be advisable to switch between those for different research questions, in particular reaching high effective $\eta$ (supercritical) versus cascades and long overlap, in particular at lower intensities.

\subsubsection{High-intensity lasers}

\noindent\textbf{\sffamily{}Performance.}\ To reach new physics the performance of high-intensity lasers and their access has to be improved \cite{Danson2019_PWLASERS}. One is the intensity, which is steadily increasing at the moment, but new technologies are required to go beyond this, especially if we will not use conventional large scale accelerators with high energies, we require high intensities. The other is an increased stability, repetition rates and efficiencies which make laser systems more feasible to operate like particle accelerators. Higher repetition rates and stability also, similarly as for accelerators, increase the discovery reach. Additional control over the energy and the spatio-temporal distribution (adaptive optics, pulse shaping in time) is also required for different applications, e.g. to reach supercritical values a slow-rising pulse is not suitable, whereas cascades might benefit from those \cite{Blackburn2019_SUPER}. Long endurance of optics (damage threshold), which might be facilitated or say mitigated by having designated target areas that match the long focal lengths. Debris in high repetition rate experiments (rely on gases). Also want to have control over the polarisation for polarisation dependent measurements and intensity measurements.
\vspace{\baselineskip}

\noindent\textbf{\sffamily{}Diagnostics.}\ In order to probe new physics we have to decouple the uncertainties of the laser beam from the model or theoretical uncertainties. A major challenge for high-intensity lasers is the characterisation of the full power high-intensity laser pulse on-shot. Typically the laser pulse is characterised in a low-power configuration (attenuated) and is then scaled up by the attenuation. However, at ultrahigh intensities this analogue might not be true anymore. There are multi-shot methods that propose a full wave front measurement of transmitted beams \cite{TERMITES}.
The key challenge will be measuring the intensity at the focus. There have been multiple proposals how to do this using gas jets, either relying on ionisation rates \cite{Ciappina2019_DiagPRA,Ciappina2019_Diag} or the ponderomotive effect onto the relativistic electrons \cite{Mackenroth2019_PonderomotiveInt}. Especially in the context of inverse Compton scattering there have been proposals that are also suitable on-shot, e.g. by considering the yield of the radiation in the linear case (limit $a_0 \sim 0$) in Chapter \ref{Chap:linICS}, the contribution of higher harmonic radiation in the approach of $a_0 \rightarrow 1$ \cite{Babzien2006_ICS_2ndHarmonic} or the ellipticity of the emitted gamma signal in the highly nonlinear regime \cite{HarShemesh2012_INTENSITY,Yan2017_ICS,Blackburn2019_ModelIndependentLaser}. In the future when radiation reaction is well characterised, these counterpropagating experiments might themselves become a tool to characterise a laser pulse by passing them through specially tailored electron beams.


\subsubsection{Electron beam and synchronisation}

\noindent\textbf{\sffamily{}Performance.}\ Electron beam quality needs to be improved, but at the same time LWFA electron beams seem to have a bad reputation despite significant improvements in the beam quality over the past decade or so. For results in this frontier field to be accepted as in particle physics terms (discovery etc., statistical significant) and considering the complexity of setups, the quality has to be improved. Key areas are reproducibility and stability, energy spread. The maximum energy is also of interest but many precision studies are already possible at the GeV to few GeV scale that is achieved in experiments. Stable beams increase the discovery reach (as for accelerators) as fewer shots are required to reach discovery significance (see for instance Section \ref{Chap:RR:Sec:StabilityStatisticsModel} and \cite{Arran2019_RR_PPCF,Arran2019_RR_SPIE}). Promising routes lie in density tailoring and in stabilising the laser pulse. Shock injection has been demonstrated to be suited to produce high-quality low-energy-spread beams, and will also localise the injection point to fix the laser-electron offset. Gas jets will also be the target of choice to avoid damage and debris.
\vspace{\baselineskip}



\noindent\textbf{\sffamily{}Diagnostics.}\ Considering the intrinsic fluctuations of the electron beam we have to characterise the source either statistically or on-shot. This requires a high resolution of the electron energy, e.g. using multiple screens \cite{Soloviev2011_TWOSCREEN}, fiudicials \cite{Wang2013_GEV_FIUDICIAL} and deconvolution techniques to account for pointing and divergence/source size. The pointing of the electron beam is also important as well as its accurate charge. In the context of the intensity measurements relying on the ellipticity of the electron beam the ellipticity (and source size) is an important quantity that can alter the inferred intensity. It can be measured with a beam profile monitor but we have demonstrated a technique based on linear ICS that could non-invasive do this as well. Eventually we would want to measure this on-shot during high-intensity interactions. For this purpose we would have to use multiple beam optics to redirect the radiation and the beam first. Proposals for setups include plasma lenses (active, discharge), but here we might lose the benefit of the micron sized source. Potentially there is an avenue of finding a correlation between the beam ellipticity and the laser properties. More diagnostics will help solving this problem. A thorough characterisation of the laser pulse and its peak intensity (relate this to on-shot wavefront measurement \cite{TERMITES}) should mitigate the issue.

A more challenging measurement is the chirp of the electron beam and its bunch duration. These factors, together with the beam size, can influence the resulting radiation reaction, post-interaction spectra due to changes in intensity from focusing. It could be that short bunches are preferable to characterise radiation reaction in a detailed study and to then use longer bunches in a classical radiation reaction regime to actually measure the duration and chirp. In conventional accelerators electro-optical sampling and transverse deflecting structures (TDS) are used, e.g. LOLA. These are not easy to implement in a laser-plasma beam line and would reduce conventional beam optics first. There might be laser-based alternatives (see \cite{Downer2018_DiagnosticReview}). Also transition radiation or magneto-optic sampling. However, do not want any material in the beam path, so crystal based technologies are suited as they are non-invasive.

Especially in the context of high intensity interactions and potential spin-polarisation we also have to develop the tools to measure the spin polarisation. In conventional accelerators this is being done as well with linear Compton scattering, such that this would require a separated interaction point to avoid overlap of the radiation with the high-intensity signal (magnetic chicane). This also requires larger experimental target areas and high precision measurements of the charge and the gamma ray yield to distinguish different rates. 


\subsubsection{Spatio-temporal synchronisation}

Another major challenge is the systematic overlap of the laser pulse and the electron bunch, or in a second interaction point the light sources.
In all-optical setups we can synchronise the laser pulses to each other \cite{Shalloo_GEMINIDRIFT}\cite{Cole2018_RR}\cite{Corvan2016_TIMING} but there is an offset between the laser pulse and the electron bunch (see Section\addnum). At the same time there is a temporal jitter and potentially change in the injection point. This can be mitigated by a well controlled density perturbation (shock injection). In Section\addnum{} we have shown that we can potentially use the interaction and radiation yield itself to characterise the offset. We also motivated that this could be used to automate the alignment procedure and to account for jitter and drifts in the system.

Methods to synchronise instead the electron beam and the laser pulse are based on EOS \cite{Cavalieri2005_EOS,Yan2000_EOS} but the short pulse durations are challenging and have to be explored. Other methods that synchronise electron bunches and laser are, for instance, plasma afterglow \cite{Scherkl2019_PLASMAAFTERGLOW} or fast shadowgraphy \cite{Buck2011_BUBBLE,Savert2015_BUBBLE,Siminos2016_FASTSHADOW}.
Ultimately it seems using the injection itself and some EOS would be the way forward. Whilst there is feedback in terms of radiation we can use for electron-laser interactions, it is more challenging in the light-light case as we only expect strong interactions when both are well overlapped. One approach would be scatter the electron bunch at the second interaction point and then activate the dispersion afterwards. By temporally linking both scattering laser pulses one can then monitor even with dispersion if there are drifts (another time spectral interferometry). Current all-optical setups use commonly one to two intense main beams (driver and scatterer or reflected) and an optical probe beam. This proposed geometry requires many more laser pulses and is accordingly more complex. Automation, i.e. quantitative on-shot evaluation of data, becomes necessary, but is feasible already at 5 Hz laser operation \cite{Streeter2018_TEMPFEEDBACK}. Machine learning using single-shot diagnostic and then infer without these on normal shots has been fielded at SLAC\addref. 

\subsubsection{Radiation diagnostics}

\begin{figure}
\centering
\includegraphics[width=.6\columnwidth]{GmmaDets_DoubleScreen.png}
\caption{An example of a combined gamma spectrometer setup.}
\end{figure}

The retrieval of highly energetic gamma radiation is crucial to infer the conditions at the interaction, e.g. of the electron beam and the laser pulse, but is also very relevant for photon-photon interactions (see Chap\addnum). Highly energetic radiation from hundreds of MeV to several GeV scale are expected, with divergent lower energy X-rays/Gamma rays are important to measure the breakdown of LCFA, i.e. allow measuring the formation length.

The measurement of full spectra beyond the few tens of MeV range becomes increasingly difficult due to the flat response curve which is dominated by lower energy photons. Using an array of scintillating crystals, for instance, we have seen in this work that we require a good knowledge of the spectral shape to start with \cite{Cole2018_RR,Behm2018_Gamma}. Compton scattering is more defined and distinct (see cross sections) at the lower level such that \cite{Corvan2014_Gamma} becomes less sensitive. Pair production becomes more dominant and the cross section is almost flat above few tens of MeV photon energies. 

It has been proposed to combine both diagnostics in a pair spectrometer with scintillator array which can be used to find a spectrum as demonstrated in \cite{Lisi2018_Gamma} and proposed in \cite{Abramowicz2019_LUXE}. By measuring the spectra of the Compton scattered and pair produced electrons and positrons (coincidence measurement!) and the response of the scintillator array simultaneously we can infer the spectrum more accurately. 

By using multiple screens and a converter target, maybe a thin wire, we also gain access to a double differential spectrum. 
A full converter and multiple screens and a deconvolution algorithm could give access to a 3-dimensional spectrum. 
However, at the hundreds of MeV and GeV-scale this still has to be demonstrated, and requires a high sensitivity and efficient shielding as the flat cross section means changes are small. 
We also have to find a suitable method for an absolute calibration of these diagnostics and a test source that is suitable and well understood, depositing energy in a same manner. 
Here we also have to consider the degrading performance of detectors over time, especially if used under a lot of radiation.


\begin{sidewaysfigure}
\centering
\includegraphics[width=1.0\columnwidth]{RR20_ExpSetup_V12Sep.png}
\caption{Future experiment layout incorporating lessons learned.}
\label{Concl:Fig:FutureExp}
\end{sidewaysfigure}



\subsection{Concluding remarks}

LWFA has been demonstrated to be a suitable tool to investigate strong-field phenomena in the emerging field of SFQED. Improvements and detector development are necessary to render the prospect of precision measurements feasible and to increase the discovery reach. Key developments that will facilitate these studies are the stability of the electron and laser source and a set of precise diagnostics that characterise the laser pulse, the electron bunch and secondary radiation and particles, ideally on-shot. This is necessary to decouple experimental and diagnostic uncertainties of model uncertainties. Eventually once these phenomena are well characterised, fundamental processes like radiation reaction and pair production could act themselves as diagnostic, e.g., for the intense laser pulse.

This might require some shifts in how LWFA experiments are operated, moving towards more automation and maybe applying machine learning (genetic algorithms, neural networks, Bayesian) to optimise and maintain parameters at high repetition rates. It will then become a useful tool as in conventional accelerators to increase the discovery reach.

It would also be beneficial to move to more fixed setups as done partially now in the community with at least semi-permanent setups that allow more complex setups and diagnostics. The discussion indicated a wide range of diagnostics will be required (see also Chapters, especially BW). On the other hand, this might eventually indicate that components of the experiment will become more and more separated as in conventional facilities where the accelerator and the interaction point and the diagnostics are often well separated, and beam optics (focusing, beam transport) are used frequently. Despite there is probably resistance as this goes away from the mantra of compact or table-top accelerators, the field gradients and the replacement of conventional to plasma optics is still there and it is still of much much smaller order of magnitude. This might be required specifically for these experiments. Other applications might still remain very compact (imaging/XANES applications).
 
 