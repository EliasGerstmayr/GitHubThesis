\chapter{Conclusion and Outlook}
\label{Chap:Conclusion_and_Outlook}

\section{Summary of Results}
\label{Chap:Conclusion}

The experimental results presented in this work demonstrate the capability of laser wakefield acceleration (LWFA) to produce highly energetic radiation from few to hundreds of MeV photon energies, and their application in the context of fundamental studies of quantum electrodynamics (QED):

\begin{itemize}
\item \textbf{Chapter \ref{Chap:linICS}} described how relativistic electron beams with energies up to $1.3\,\mathrm{GeV}$ were collided with a laser pulse at $a_0 \sim 0.2 - 1$, producing radiation from \textbf{linear inverse Compton scattering} of variable spectral shape in the range of 10s of MeV photon energies. The radiation yield and spectrum were used to diagnose the properties of the electron and the laser beam at the interaction. Linear inverse Compton scattering promises to become a useful \textbf{beam diagnostics and alignment tool} for future studies of radiation reaction.

\item \textbf{Chapter \ref{Chap:RR15}} outlined how relativistic electron beams of energy $\epsilon \approx 550\,\mathrm{MeV}$ were collided with a highly intense laser pulse of intensity $a_0 \approx 10$, producing broadband synchrotron-like radiation with a critical energy $\epsilon_{crit} > 35\,\mathrm{MeV}$ from \textbf{non-linear inverse Compton scattering}. The high energy of the emitted photons lead to a measurable energy loss in the electron spectra due to \textbf{radiation reaction}.


\item \textbf{Chapter \ref{Chap:BW}} reported how electron beams were propagated through a solid target to produce an energetic gamma-ray beam from \textbf{bremsstrahlung} reaching photon energies of several hundreds of MeV. The yield and collimation of the source were optimised for its application in a photon-photon collider experiment with the aim to produce electron-positron pairs from the \textbf{linear Breit-Wheeler process}. The second photon source was a $\sim$keV X-ray source from a burn-through foil heated by a second high-intensity laser.
\end{itemize}

\section{Discussion and Outlook}
\label{Conclusion:Discussion}

The experimental results presented in this work demonstrate that an all-optical colliding-pulse geometry using a laser wakefield accelerator (LWFA) is suited to investigate fundamental (strong-field) QED (SFQED) phenomena (see Chapters \ref{Chap:RR15} and \ref{Chap:BW}), and can contribute significantly to this emerging research field \cite{Pike2014_BW,Cole2018_RR,Poder2018_RR}.
Whilst this bears exciting research opportunities, especially at the next generation of multi-PW laser systems \cite{Gales2018_ELINP,Weber2017_ELIBeamlines,Zou2015_Apollon,Li2017_SULF,Shen2018_SULF,Cartlidge2018_SEL,Toth2017_BELLA,Bashinov2014_XCELS,Kessel2018_PWMPQ,EPAC_Website,Maksimchuk2019_ZEUS,Yanovsky2008_HERCULES}, the results have also shown the need for a better control over the laser system and the accelerator \cite{Samarin2017_RR}, and motivate the development of suitable diagnostics to facilitate precision measurements with statistical significance \cite{Arran2019_RR_PPCF,Arran2019_RR_SPIE}.
\vspace{\baselineskip}

Since the first reported measurement of quasi-monoenergetic electron beams in 2004 \cite{Mangles2004_MONO,Faure2004_MONO,Geddes2004_MONO}, maximum electron energies have increased by a factor of 100 \cite{Gonsalves2019_GEV} -- a rapid increase that aligns with the rate of energy gain observed historically in conventional accelerator technology \cite{LivingstonPlot}, e.g. see Figure \ref{Introduction:Figs:CERN_Livingston} \cite{Panofsky1997_Livingston} in Section \ref{Introduction:Sec:PartAccelerators}. The progress in the field of LWFA has over the past decades been strongly driven by the enhanced control over high-intensity lasers and the exponentially rising peak intensities \cite{Esarey2009_LPA_Review}. Along with an increase of the maximum electron energies there have been significant improvements to the overall beam quality \cite{Osterhoff2008_CELL} and an enhanced control over the final properties of the accelerated bunch \cite{Banerjee2013_TunableLWFA}, which can be attributed to a better understanding of the evolution of the bubble \cite{Guillaume_REPHASING}, injection mechanisms \cite{Buck2013_SHOCK}, and the ability to tailor both. LWFA electron beams and the energetic radiation they generate are suited for a wide range of applications \cite{Albert2016_APP}, including the study of fundamental QED phenomena discussed here \cite{Pike2014_BW,Cole2018_RR,Poder2018_RR}.

However, LWFA is still an actively evolving accelerator technology that has not yet reached a level of maturity comparable to those of accelerators based on radio-frequency (RF) cavities: 
current challenges are, for instance, the stability or shot-to-shot reproducibility of the accelerator (see also Sections \ref{Chap:linICS:sec:Espec}, \ref{Chap:RR:Sec:StabilityStatisticsModel} and \ref{BW:sec:Espec}), and the production of low emittance and low energy spread beams, for instance to be used in free-electron lasers \cite{Nakajima1996_FEL}. Another key challenge is to operate the accelerator at a high repetition rate and efficiency \cite{Danson2019_PWLASERS}, and to fully diagnose the accelerator and the electron beam it produces \cite{Downer2018_DiagnosticReview}.
This indicates that a particle collider solely based on plasma technology still lies further in the future as the prerequisite for precision measurements is the full characterisation of the probe.
The designs for the proposed next generations of lepton colliders, e.g. CLIC \cite{CLIC2012} or ILC \cite{ILC2013}, are hence still based on superconducting RF cavities, but the increasing scale and cost of RF-based accelerators eventually require the use of high-field technology as found in plasmas. 
For now, the wakefield community provides crucial impulses and expertise to enhance existing conventional accelerator facilities with beam-driven wakefield technology \cite{Caldwell2009_ProtonPWFA}, e.g. as final amplification stage (PWFA afterburner) \cite{Lee2002_Afterburner}, and explores the potential of hybrid collider schemes \cite{Adli2013_PWFACollider}.
\vspace{\baselineskip}

The following sections will briefly recapitulate research opportunities in SFQED that can be explored in all-optical colliding-pulse setups using LWFA, and will, based on the results presented in this work (Chapters \ref{Chap:linICS}, \ref{Chap:RR15} and \ref{Chap:BW}), discuss specific experimental considerations for the accelerator, scattering geometries and diagnostics that will facilitate measurements in the future.


\subsection{Research Opportunities}
\label{Concl:Sec:ResearchOpportunities}

All-optical colliding-pulse setups where an LWFA electron beam collides with a counter-propagating intense laser pulse are uniquely suited for studies of inverse Compton scattering, radiation reaction and other high-field phenomena (see Chapter \ref{Chap:RR15} and \cite{Cole2018_RR,Poder2018_RR}): 
in LWFA setups an intense laser system and a relativistic electron accelerator are co-located, and both are intrinsically synchronised to each other. 
Moreover, the small size of the electron beam at the end of the accelerator \cite{Weingartner2012_BUNCH} and its short bunch duration \cite{Lundh2011_BUNCH} allow a full overlap with the tightly focused (and hence very intense) laser pulse such that the interaction occurs with almost all of the electrons at a comparable intensity \cite{Harvey2016_ICSFOCUS}.
Since relativistic electron beams and intense laser pulses can produce a wide range of energetic radiation, this setup can also be converted into a real photon-photon collider (see Chapter \ref{Chap:BW} and \cite{Pike2014_BW}).
As a result, LWFA and the next generation multi-PW laser systems \cite{Danson2019_PWLASERS,Gales2018_ELINP,Weber2017_ELIBeamlines,Zou2015_Apollon,Li2017_SULF,Shen2018_SULF,Cartlidge2018_SEL,Toth2017_BELLA,Bashinov2014_XCELS,Kessel2018_PWMPQ,EPAC_Website,Maksimchuk2019_ZEUS,Yanovsky2008_HERCULES} are in this context in direct competition with state-of-the-art particle accelerators \cite{Burkart2019_LUXE,Abramowicz2019_LUXE,SFQEDOverview2019,Hogan2016_FACETII,DiPiazza2017_CrystalRR,Artru1994_CHANNELING,Katkov1998_CHANNELING,Wistisen2018_RR,Wistisen2019_RR,Beresford2019_PhPh_SLEPTONS,Knapen2017_PhPh_AXIONLIKE,Baldenegro2018_PhPh_AXIONLIKE}.
\vspace{\baselineskip}

Colliding-pulse geometries with relativistic particles and highly intense laser pulses will enable a detailed investigation of Compton scattering in different regimes, from the classical \cite{Yan2017_ICS} to the perturbative quantum description \cite{Bula1996_RR} and in the non-perturbative strong-field regime \cite{Bula1996_RR,TaPhuoc2012_ICS,Chen2013_ICS,Powers2014_ICS,Sarri2014_ICS,Khrennikov2015_ICS,Mackenroth2013_nlCompton}. Approaching a quantum nonlinearity parameter, $\eta$, of unity we can measure quantum radiation reaction \cite{Ridgers2017_QRR,Ritus1985_QRR,Wistisen2018_RR,Thomas2012_LL,Blackburn2014_QRR,Dinu2016_QRR,Vranic2014_RR} and perform statistically significant precision measurements \cite{Arran2019_RR_PPCF,Arran2019_RR_SPIE} of the transition from a regime where classical descriptions hold to a regime where quantum effects dominate. This includes the manifestations of the stochastic quantum nature of radiation reaction \cite{Niel2018_RRStochastic} such as energy straggling \cite{Shen1972_STRAGGLING} and radiation quenching \cite{Harvey2017_QUENCHING} visible in the emission spectrum and the post-interaction electron spectrum \cite{Ridgers2017_QRR}. The emission might also result in a spin-polarisation of the electron beam \cite{DelSorbo2017_SPIN,Seipt2018_SPIN,Li2019_singleshot_beampol} through the Sokolov-Ternov effect \cite{SokolovTernov1964_POL}. 

In this regime intense interactions also give rise to (multi-step) cascades and showers \cite{Blackburn2017_pairs,Bulanov2013_Cascade}, which are particularly relevant in the astrophysical context. At $\eta \sim 1$ we reach the regime of nonperturbative strong-field QED \cite{Yakimenko2019_NONPERTURB,Blackburn2019_SUPER,Baumann2019_NONPERTURB}, where quantum effects dominate. Here QED cascades can produce dense pair plasmas \cite{Ridgers2012_DENSE}, which are of interest for high-energy density studies \cite{DoE2015_FES}. 

Suitable radiation diagnostics will enable measuring the formation length of the photon emission via breakdown of the local-constant field approximation (LCFA) \cite{Ritus1985_QRR,DiPiazza2018_LCFA,Ilderton2019_QEDPerturbBreakdown} and it will allow scrutinising the range of applicability of state-of-the-art numerical methods \cite{DiPiazza2019_LCFANUM,Elkina2011_QEDcascadesNUM,Ridgers2014_QEDNUM,Gonoskov2015_ReviewQEDNUM}.
\vspace{\baselineskip}

This geometry also enables the investigation of pair production processes such as two (linear) \cite{Pike2014_BW,Drebot2017_BW_ICS} and multi-photon (nonlinear) Breit-Wheeler pair production \cite{Burke1997_RR}, and at $\eta \sim 1$ vacuum pair production (vacuum breakdown) \cite{Bell2008_PairsEta,Hu2010_TRIDENT,Ilderton2011_TRIDENT}. In these extreme fields it also facilitates probing the quantum fluctuations and the vacuum acts as nonlinear medium manifested in phenomena like vacuum birefringence  \cite{King2016_VB,Nakamiya2017_VB}, vacuum dichroism \cite{Bragin2017_VBVD}, and vacuum recollision processes \cite{Meuren2015_HERecollision,Kuchiev2007_Recollision}. Photon-photon interactions are also a potential route for probing physics beyond the Standard Model, e.g. sleptons \cite{Beresford2019_PhPh_SLEPTONS} and axions \cite{Knapen2017_PhPh_AXIONLIKE,Baldenegro2018_PhPh_AXIONLIKE}.

At more extreme conditions the Hawking-Unruh effect \cite{Schutzhold2006_UnruhEffect,Chen1999_UnruhEffect} and the conjectured strong-coupling regime of QED, the Ritus-Narozhny conjecture \cite{Ritus1972_NAROZHNY,Narozhny1980_NAROZHNY,Fedotov2017_NAROZHNY}, become accessible.

\subsection{Experimental Considerations}
\label{Concl:Sec:ChallengesReq}

A prerequisite for useful measurements of new phenomena in the SFQED regime is the detailed knowledge of the conditions at the interaction, e.g. electron spectra and laser properties, the capability to diagnose the reaction products accurately, and to maintain conditions consistently such that the results gain statistical significance. 
The results presented in this work (Chapters \ref{Chap:linICS}, \ref{Chap:RR15}, \ref{Chap:BW}) can help to identify specific aspects of the experimental setup or technique that can be addressed in order to design an experiment that fulfils these criteria.

\subsubsection{Laser Wakefield Accelerator}

In a colliding-pulse setup the performance and choice of the targetry of the laser wakefield accelerator is crucial (see Sections \ref{Methods:Sec:GasTargets} and \ref{Chap:RR:Sec:StabilityStatisticsModel}). The ideal accelerator has to provide over long periods of time consistently a stable but also tunable electron beam, where low-energy spread, high charge and high maximum electron energies are examples of favourable properties \cite{Arran2019_RR_PPCF,Arran2019_RR_SPIE}. 

Gas cells have in some instances demonstrated to produce electron beams of higher quality than gas jets at the same laser system \cite{Osterhoff2008_CELL}, but gas jets offer wide access for diagnostics (see Figure \ref{Concl:Figs:GasJetDiags}) and are easier to align, also avoiding potential damage from the wakefield driver beam. Moreover, it enables the use of shock injection which has demonstrated to produce high-quality electron beams of high charge and low energy spread with great tunability and stability \cite{Schmid2010_SHOCK,Buck2013_SHOCK,Swanson2017_SHOCK,Tsai2018_SHOCK}. 
Electron beams with these properties are in particular suited for precision studies of radiation reaction (see for instance Section \ref{Chap:RR:Sec:StabilityStatisticsModel} and \cite{Ridgers2017_QRR,Arran2019_RR_PPCF,Arran2019_RR_SPIE}). 

Gas jets facilitate a debris-free interaction close to the accelerator where the electron beam is small such that the entire bunch can interact with the intense laser pulse. Gas cells, on the other hand, risk damage if an interaction occurs close to it, which could result in a degradation of its performance. Moreover, the production of debris and scattered light could in turn damage optics in the vicinity. To use a gas cell sustainably interactions would need to take place further away from the target where the electron beam is larger. In this scenario a plasma lens could be used to focus the electron beam onto a second interaction point.
As a consequence, for now gas jets appear to be the most suitable targets for colliding-pulse experiments.
 

\begin{figure}
\centering
\includegraphics[width=1.0\columnwidth]{GasJetCloseUp_Combo.pdf}
\caption[Sketch of a gas jet and diagnostics that characterise the plasma.]{Sketch of a gas jet and example images from diagnostics that characterise the plasma: An optical transverse probe was used to obtain a shadowgram and an interferometry image. The self-emission of the plasma was imaged from the side and the top.}
\label{Concl:Figs:GasJetDiags}
\end{figure}


\subsubsection{Electron diagnostics}

As the properties of electron beam from a LWFA vary from shot-to-shot, it is necessary to measure its properties on-shot or characterise its fluctuations statistically (see Sections \ref{Chap:RR:Sec:CharacElec} and \ref{Chap:RR:Sec:StabilityStatisticsModel}).
In a typical LWFA setup the energy of the electron beam is measured in a magnetic spectrometer (see Section \ref{Chap:Methods:Sec:Espec}), where multiple screens \cite{Soloviev2011_TWOSCREEN} and spatial references \cite{Wang2013_GEV_FIUDICIAL} can be used to account for beam pointing and divergence to improve the energy resolution.
In a colliding-pulse setup this allows an on-shot measurement of the \textit{post}-interaction electron spectrum and a statistical characterisation of the \textit{pre}-interaction spectrum when the scattering beam is not used. Given a sufficiently populated reference dataset the statistical approach can yield significant results but requires more shots depending on the stability of the source and the strength of the interaction (see Sections \ref{Chap:RR:Sec:CharacElec} and \ref{Chap:RR:Sec:StabilityStatisticsModel} or \cite{Cole2018_RR,Arran2019_RR_PPCF,Arran2019_RR_SPIE}). Ideally we would also measure the on-shot pre-interaction spectrum, which is, for instance, discussed in \cite{Baird2019_RR}.
\vspace{\baselineskip}

Observables that are in particular important in the context of colliding-pulse setups are the pointing variations of the electron beam, to estimate overlap, and its divergence or ellipticity, in the context of on-shot intensity measurements. These are commonly measured with beam profile screens, e.g. Lanex, in the undispersed electron beam. 
This approach is, however, not suitable for on-shot measurements in colliding-pulse setups as any material in the beam path produces energetic radiation from bremsstrahlung which affects the spectral retrieval of the Compton signal (see Section \ref{Chap:Methods:subsec:GammaSpec}). Such a screen would also be prone to damage from the intense laser pulse. Moreover, the response of a Lanex beam profile screen is dominated by low-energy electrons \cite{Glinec2006_Lanex} which are commonly abundant in wakefield accelerators.

Section \ref{Chap:linICS:Sec:ICS_EbeamDiag} outlines how inverse Compton scattering can be used as beam profile monitor to measure beam properties such as the ellipticity of the electron beam, its pointing and to resolve spatial features from the emitted gamma radiation. Due to the dependence of the energy of the emitted radiation on the electron energy this technique is also less sensitive to low-energy ($\sim$ MeV) electrons (see Section \ref{Chap:linICS:Sec:EdepResponse}). At low laser intensities this enables a non-invasive on-shot diagnostic of the electron beam profile. At high intensities the scattering process becomes nonlinear and the emission profiles becomes elongated in the polarisation axis (see Section \ref{Theory:Sec:NonlinearICS}). At high intensities and electron energies radiation reaction becomes significant (see Section \ref{Theory:Sec:RadiationReaction}). Consequently, this is a suitable for a statistical characterisation of the pre-interaction beam profile. Its more specific use on-shot at high interactions is discussed in the following section.
\vspace{\baselineskip}

A more challenging measurement is the chirp of the electron beam and its bunch duration. These factors, together with the beam size, can influence the shape of the post-interaction spectrum similarly as changes in intensity and focusing effects, and, if not considered, could lead to wrong inferred condition at the interaction and deviations from radiation reaction models. 
These properties can be measured using electro-optical sampling, transverse deflecting structures or transition radiation \cite{Downer2018_DiagnosticReview}.  Once radiation reaction is well characterised it might, on the other hand, be possible to use inverse Compton scattering and radiation reaction in reverse as diagnostics for both properties by considering energy loss and stochastic effects \cite{Ridgers2017_QRR}. For now shorter electron bunches, for instance narrow-energy spread beams injected from shock injection, and short laser pulse durations could be used to avoid a significant impact \cite{Harvey2016_ICSFOCUS}.
\vspace{\baselineskip}

Finally, at high intensities electron beams can be spin-polarised \cite{DelSorbo2017_SPIN,Seipt2018_SPIN,Li2019_singleshot_beampol}, similarly as in beam storage rings through the Sokolov-Ternov effect \cite{SokolovTernov1964_POL}. In conventional accelerators the spin-polarisation of beams is measured using Compton scattering \cite{Baylac2002_POL}. Since an intense laser-electron interaction already produces a large gamma-ray signal, this would require diverting the electron beam, then performing the scattering and divert the electron beam again to separate two gamma signals and the electron beam from the radiation. The scattering laser could be a low-intensity laser pulse (see Section \ref{Chap:linICS:Sec:Nphotons}). Suitable diagnostics to measure gamma radiation are outlined in Section \ref{Concl:Sec:ExpCons:Radiation}. On the other hand, an appropriate choice of polarisation for the scattering beam results in a `splitting' of the electron beam which could be measured on a magnetic spectrometer and the scattered radiation \cite{Li2019_singleshot_beampol}.


\subsubsection{Scattering geometry and choice of focusing optics}

Simplistically it is preferable to achieve the highest possible value of the quantum nonlinearity parameter, $\eta$, in an interaction as a higher value gives access to a wider range of phenomena and also produces a stronger response of the system. A stronger response in turn facilitates statistically significant measurements requiring fewer data points. $\eta$ increases with the electron energy, $\gamma$, and the laser intensity, $I_L$:
\begin{equation}
\boxed{\eta = \frac{(1-\cos \theta) \gamma E_L}{E_S} \propto (1-\cos\theta) \gamma \sqrt{I_L}.}
\end{equation}
Neglecting the properties of the electron beam as they were discussed already, it is becomes evident that it is preferable to focus the laser pulse tightly to increase the peak intensity. On the other hand, at the exit of the wakefield accelerator the electron beam is of $\sim\microns$ size: the optimum spot size is then at least of similar magnitude or even slightly larger such that the electron beam acts as a real probe and is not affected by wavefront and intensity variations that would then need to be considered in the modelling process as well. At the same time very short focal lengths place the optic into the vicinity of the interaction and risk damage to it. 
Longer focal length optics maintain a high intensity over a wider distance and larger $z_R$ facilitate overlap between the laser pulse and the electron bunch, but also result in larger absolute pointing fluctuations. In all experiments discussed in this work an $f/2$ off-axis parabola was used with a Rayleigh length of $z_R \sim 8\microns$.

\begin{figure}
\centering
\includegraphics[width=.5\columnwidth]{DoubleF2Setup.png}\includegraphics[width=.5\columnwidth]{HoleyF2Setup.png}
\caption[Scattering geometries using either a head-on geometry or an off-axis parabola at an off-angle]{Scattering geometries using either a head-on geometry (right) or an off-axis parabola at an off-angle (left). Here we also use a second parabola to re-collimate the light. The wakefield driver beam enters from the left and is incident onto the gas jet (blue).}
\label{Conclusion:Figs:ScatteringOAPs}
\end{figure}

At a fixed electron energy and laser intensity, $\eta$ is maximised in a head-on collision, i.e. $\theta = \pi$, such that it is desirable to place the focusing optic as close as possible to the laser axis.
One can then either place the OAP directly onto the axis [see Figure \ref{Conclusion:Figs:ScatteringOAPs} (right) and Chapters \ref{Chap:linICS} and \ref{Chap:RR15}], which then requires a hole in the optics, or place it close but off-axis at an angle [see Figure \ref{Conclusion:Figs:ScatteringOAPs} (left) and Chapter \ref{Chap:BW}]. 

The hole in the optic has to be large enough to avoid any interaction with particles and radiation as it would affect measurements downstream. In particular, it has to be large enough to avoid being hit by the wakefield driver beam, also considering that the laser might drift or defocus in the plasma, as this could damage the optics and, more importantly, the laser chain upstream. However, the larger the hole, the more energy is lost in the process. Larger beam optics that might be used at multi-PW facilities could reduce the relative impact of the hole (assuming a flat-top beam profile) and make the relative angle more relevant, but the high intensities increase the risk of damage to the laser. 

On the other hand, the position on-axis maximises $\eta$ through $\theta$, and also allows easier alignment of the two laser pulses and facilitates longer overlap over a wider spatial range. In the experiments presented in Chapters \ref{Chap:linICS} and \ref{Chap:RR15} the loss of energy through the hole and potential loss of $\eta$ due to an angle were of a similar value ($\sim 5\%$), such that the on-axis position was chosen for the ease of alignment.
Consistent overlap was demonstrated in Chapter \ref{Chap:linICS} and also enabled the alignment procedure based on spatial correlation (laser raster scan) outlined in Section \ref{Chap:linICS:Sec:RasterScan}, which would only be applicable in a limited fashion in an off-angle geometry.

The extended temporal overlap of the laser pulse and the electron beam might also be beneficial to induce cascades \cite{Blackburn2017_pairs}. 
Moreover, in this head-on geometry the polarisation vector of the laser pulse is always orthogonal to the electron axis, whereas in an off-axis geometry this would only be true in one axis and the projected polarisation vector would be smaller in the other axis.
\begin{figure}
\centering
\includegraphics[width=.5\columnwidth]{F2LinearBW.png}\includegraphics[width=.5\columnwidth]{F2NLinearBW.png}
\caption[Scattering geometries for photon-photon colliders using an off-axis parabola at an oblique angle.]{Scattering geometries for photon-photon collisions using an off-axis parabola at an oblique angle either heating an X-ray target with an intense laser (red) for a measurement of the linear Breit-Wheeler process (left) or colliding the laser pulse directly with the gamma-ray beam (green) to produce pairs from the nonlinear Breit-Wheeler process (right).}
\label{Conclusion:Figs:ScatteringOAPs_BW}
\end{figure}

When placing the parabola off-axis, on the other hand, the full energy of the beam is used, and potential diffraction effects caused by the hole are avoided, but at cost of the reduced scattering angle. This geometry reduces the risk of damage to the laser system, and also enables the use of a second OAP to recollimate the beam and transport it to appropriate laser diagnostics \cite{Abramowicz2019_LUXE,SFQEDOverview2019} [see Figure \ref{Conclusion:Figs:ScatteringOAPs} (left)]. This would also reduce the production of potential debris around the interaction region as the laser pulse is terminated far away from it.

As a result of the angle, the relative alignment is slightly more involved, and the overlap of the electron bunch and the scattering pulse is reduced. Whilst this is not the ideal configuration to study cascades, especially if the experimental conditions are at the threshold to study these phenomena, it is being proposed that an oblique angle of incidence might even be \textit{required} to achieve critical or even supercritical $\eta$ and to avoid losses through radiation and particle production before reaching the peak $\eta$ \cite{Blackburn2019_SUPER}. 
This geometry is also well suited for photon-photon collisions at a second interaction point as shown in Chapter \ref{Chap:BW}. Here any material close to the axis is a potential source of noise that should be avoided. In a photon-photon collider the off-axis geometry would also allows switching between a direct collision using the scattering laser, e.g. to study the nonlinear Breit-Wheeler process, and a laser-heated X-ray target as discussed in Chapter \ref{Chap:BW} for a measurement of the linear Breit-Wheeler process (see Figure \ref{Conclusion:Figs:ScatteringOAPs_BW}).

\begin{table}[h]
\centering
{\rowcolors{2}{white}{lightgray!50}
\begin{tabular}{|c|l|l|}
\hline
\textbf{Category} & \textbf{On-axis (with hole)} & \textbf{Off-axis} \\ \hline \hline
$\eta$ (+) & Optimum angle $\theta$ & Full laser energy\\
$\eta$ (--) & Energy loss (hole) & Reduced angle $\theta$\\
\hline
Overlap & Maximised & Reduced\\
\hline
Phenomena & Radiation Reaction & Supercritical $\eta$\\
 & Showers and Cascades & Photon-photon coll.\\
\hline
Main Risks & Laser/Optic Damage & No Overlap\\ 
\hline
\end{tabular}
}
\caption{Comparison of scattering geometries under different aspects.}
\label{Concl:Table:Comparison_ScatterGeometries}
\end{table}

In summary, both geometries (see Figure \ref{Conclusion:Figs:ScatteringOAPs}) are associated with a trade-off of specific advantages and disadvantages (see Table \ref{Concl:Table:Comparison_ScatterGeometries}), and it might be beneficial to switch between configurations for different research questions, in particular for reaching high effective $\eta$ (supercritical) and photon-photon interactions in the on-axis, and cascades in the off-axis geometry. At high intensities of multi-PW laser facilities the risk of damage to optics and the laser chain upstream might be too significant to implement a head-on geometry.

\subsubsection{High-intensity laser diagnostics and laser controls}

In order to probe new physics the uncertainties on the laser beam properties have to be decoupled from the model or theoretical uncertainties. A major challenge for high-intensity lasers is the characterisation of the actual full power (not attenuated) high-intensity laser focus on-shot. Typically the laser pulse is characterised in a low-power configuration (attenuated) and it is assumed that the properties of the low-power and full-power laser focus are comparable such that both can be related by simply scaling the total energy contained in the laser pulse accordingly. However, at ultrahigh intensities this assumption might not be true anymore. There are multi-shot methods that propose a full wave front measurement of transmitted beams \cite{TERMITES}, but the key challenge remains the measurement of the intensity \textit{at the focus}. There have been multiple proposals how to do this using gas jets, relying on ionisation rates \cite{Ciappina2019_DiagPRA,Ciappina2019_Diag} and ponderomotive effects onto the relativistic electrons \cite{Mackenroth2019_PonderomotiveInt}. 

Especially in the context of inverse Compton scattering there have been proposals that are also suitable on-shot, e.g. by considering the yield of the radiation in the linear case (limit $a_0 \sim 0$) in Chapter \ref{Chap:linICS}, the contribution of higher harmonic radiation in the approach of $a_0 \rightarrow 1$ (at moderate intensities) \cite{Babzien2006_ICS_2ndHarmonic} or the ellipticity of the emitted gamma signal in the highly nonlinear regime \cite{HarShemesh2012_INTENSITY,Yan2017_ICS,Blackburn2019_ModelIndependentLaser}. This can be measured using the setup described in Chapter \ref{Chap:linICS}, but needs to decoupled from the electron beam properties.
In the future when radiation reaction is well characterised, these counterpropagating experiments might themselves become a tool to characterise the intense laser pulse by passing them through specially tailored electron beams. The rate of electron positron pairs produced in the process is particularly sensitive (exponential) to the laser intensity and is suited for a precise measurement of the laser intensity.
\vspace{\baselineskip}

In addition to the capability to characterise the laser pulse and the intense focus, we also require a good control over the properties of the laser pulse for certain measurements. For instance, changing the polarisation either from linear to circular or rotating the polarisation axis, e.g. using waveplates, is of interest for on-shot intensity measurements (observing the elongated axis) and to measure polarisation-dependent cross sections.
The control over the spatio-temporal shape and energy distribution might be required to achieve supercritical values of $\eta$ \cite{Blackburn2019_SUPER} and to stimulate emissions early in an interaction to study cascades. 
Finally, the capability to operate at high repetition rates and stability increases the discovery reach, similarly as the luminosity in conventional accelerators.


\subsubsection{Spatio-temporal synchronisation}

\begin{figure}
\centering
\includegraphics[width=0.5\columnwidth]{CrossProbe.png}\includegraphics[trim={4.5cm 0 5cm 0}, clip, width=.5\columnwidth]{linICS_Raster_WaistSize_V3_annotated.png}
\caption[Monitoring spatio-temporal overlap: crossed transverse probes and laser raster scan.]{Monitoring spatio-temporal overlap. Left: Two optical transverse probes (red) are crossing perpendicular to each other through a gas jet (blue) and are used as shadowgraphy (planes). Right: Sketch of a laser raster scan (see Figure \ref{linICS:Figs:RasterTemporalTranslationIdea} in Section \ref{Chap:linICS:Sec:RasterScan} for more details).}
\label{Concl:Figs:SpatioTemp_CrossedProbe_Raster}
\end{figure}

One of the key challenges in colliding-pulse experiments is the spatio-temporal overlap of the electron bunch and the laser pulse, or the light sources in case of photon-photon interactions, respectively. Two laser pulses can be synchronised in an attenuated low-power mode, e.g. using a combination of photodiodes and interferometry techniques (see Section \ref{Methods:Sec:DualBeamTiming}). For on-shot full power shots there are three main issues to consider (see Section \ref{Chap:RR:Sec:Intensity}): first, the alignment and the time of arrival of the laser pulses is varying from shot-to-shot (jitter). Second, the spatial and temporal alignment might be drifting over time such that the electron bunch and the laser pulse do not remain overlapped indefinitely (see Section \ref{Methods:Sec:TimingStability}). Third, there is an offset between the wakefield driver beam and the electron bunch it accelerates (see Section \ref{Methods:Sec:Beam_Laser_Timing}), which might also vary from shot-to-shot due to fluctuating injection points.
\vspace{\baselineskip}

A spectral interferometry setup measures the relative delay between two laser pulses \cite{Shalloo_GEMINIDRIFT,Corvan2016_TIMING} and could monitor said \textit{temporal} jitter and drifts, for instance by combining components of the beams that are transmitted through optics or the post-interaction laser pulses.

In the experiment described in Chapter \ref{Chap:linICS} the optical transverse probe (shadowgraphy) and the recombination light of the plasma channel were useful diagnostics to monitor, on the other hand, the \textit{spatial} overlap and drifts in the alignment, but at a limited spatial and temporal resolution.
Two transverse optical probes with high magnification crossing through the plasma perpendicular with respect to each other could resolve jitter and drifts in the spatial position of the plasma channel [see Figure \ref{Concl:Figs:SpatioTemp_CrossedProbe_Raster} (left)].  A \textit{fast} shadowgraphy \cite{Buck2011_BUBBLE,Savert2015_BUBBLE,Siminos2016_FASTSHADOW} could, on the other hand, enable a more accurate localisation of the bubble (instead of an expanded channel), but would require significant efforts to maintain and would add complexity to an already complex multi-beam setup. The pointing variations of the scattering beam, on the other hand, could be characterised by a far-field diagnostic.
\vspace{\baselineskip}

The offset between the electron bunch and the wakefield driver beam depends on the density of the plasma, its length, and the point at which the electrons are injected into the bubble. Due to the complex interplay of nonlinear effects in the plasma the injection point might vary from shot to shot (bubble evolution). It is desirable to reduce the variability of the injection point, e.g. through shock injection, and to measure the offset and relative jitter, e.g. from a shot-to-shot shift of the shock position, between the electron bunch and the laser pulse to find a systematic route to consistent overlap between the scattering laser pulse and the electron beam.

Chapter \ref{Chap:linICS} demonstrated that inverse Compton scattering itself can be used as tool for this purpose:
Section \ref{Chap:linICS:Sec:RasterScan} described a spatial-correlation technique that can be used to align the electron beam and the scattering laser pulse over multiple shots and that could systematically facilitate high-intensity interactions [see Figure \ref{Concl:Figs:SpatioTemp_CrossedProbe_Raster} (right)].

Section \ref{Chap:linICS:Sec:YieldBeamSize}, on the other hand, outlined that the radiation yield measured in a low-intensity interaction can be used to infer the intensity at the interaction on a single shot and, as a consequence, effectively characterises the \textit{temporal jitter} between the scattering laser and the electron bunch. By comparing this with a measurement of the jitter between the two laser pulses, e.g. using spectral interferometry, one could also infer the jitter between the wakefield driver and the electron bunch it accelerates. 

A statistical measurement of the temporal jitter and offset is useful but especially at high intensities where there are significant model uncertainties an independent on-shot measurement is required.
Diagnostics that measure the relative timing between an electron bunch and a laser pulse are, for instance, electro-optical sampling (EOS) \cite{Cavalieri2005_EOS,Yan2000_EOS} and magneto-optic sampling \cite{Downer2018_DiagnosticReview}, which have been demonstrated at the femtosecond level accuracy.

Considering the rising number of laser pulses and diagnostics in such a setup and the potential danger of spatio-temporal drifts and misalignment (see Section \ref{Theory:Sec:AddConsiderationsTiming}), it might be useful to automate some parts of the alignment process according to selected quantities \cite{Streeter2018_TEMPFEEDBACK} and to evaluate data immediately `on the fly'. Multi-shot diagnostics, statistical characterisations and automation benefit from high repetition rates. The two techniques introduced in Sections \ref{Chap:linICS:Sec:RasterScan} and \ref{Chap:linICS:Sec:YieldBeamSize} are well suited to be automated with machine learning, for instance Bayesian inference which can reduce the number of shots required to optimise parameters in a multi-dimensional scan. Quantitative measurements of alignment drifts could then also be corrected in real-time and increase the number of successful collisions and ultimately the discovery reach. 

\subsubsection{Radiation diagnostics}
\label{Concl:Sec:ExpCons:Radiation}

\begin{figure}
\centering
\includegraphics[width=.5\columnwidth]{GammaDetDouble.pdf}\includegraphics[width=.5\columnwidth]{LCFACam_CloseUp.pdf}
\caption[Measuring high-energy radiation: combined gamma converter and scintillator array design, and crystal spectrometer.]{Measuring high-energy radiation: Left: Gamma spectrometer design combining a scintillator array as used in this work with a gamma converter spectrometer. The gamma rays (green) pass through a vacuum window or converter target (orange) and produce electron-positron pairs. A magnet (grey) disperses them and the particles are measured on detector screens. The gamma-ray beam propagates into the scintillator stack (yellow) where its profile and spectrum are measured. Right: A crystal (black) is used to disperse divergent low-energy X-rays onto the CCD of a X-ray camera (blue).}
\label{Concl:Figs:GammaDetDouble_LCFACam}
\end{figure}

The spectral retrieval of highly energetic gamma radiation is in this context a crucial capability to infer the conditions at the interaction, e.g. of the electron beam and the laser pulse (see Chapters \ref{Chap:linICS} and \ref{Chap:RR15}), and to estimate expected cross sections for photon-photon interactions (see Chapter \ref{Chap:BW}). In the study of strong-field effects we anticipate a wide bandwidth of energetic radiation ranging from keV to several GeV photon energies. 

X-ray energies in the tens of keV range are routinely measured in LWFA experiments, for instance, using direct or indirect detection X-ray cameras with filter packs \cite{WoodThesis}. The setup described in Chapter \ref{Chap:BW}, on the other hand, included a crystal spectrometer setup, which could be employed to measure the formation length of a photon in the breakdown of the local-constant field approximation (LCFA) in the halo of the main emission cone \cite{Ritus1985_QRR,DiPiazza2018_LCFA,Ilderton2019_QEDPerturbBreakdown} [see Figure \ref{Concl:Figs:GammaDetDouble_LCFACam} (right)].

An accurate spectral retrieval over a large bandwidth and particularly at high photon energies is, however, challenging. This work demonstrated that the retrieval of a gamma spectrum from a spectrometer on scintillating crystals relies on a good knowledge of the spectral shape as the response of the detector is insensitive to small changes to it (see Sections \ref{Chap:Methods:subsec:GammaSpec}, \ref{linICS:Chap:Sec:SpectralRetrieval} and \ref{BW:sec:GammaCharac}).
An alternative spectrometer setup uses the spectrum of Compton scattered electrons, and at higher energies electron-positron pairs, from gamma rays passing through a converter foil to infer the spectrum \cite{Corvan2014_Gamma}. However, as the pair production cross sections plateaus at few tens of MeV photon energies this diagnostic also loses sensitivity with increasing photon energies.

A potential solution sees both diagnostics, a scintillator array and a pair spectrometer, combined to infer the spectrum more accurately from the two independent measurements \cite{Lisi2018_Gamma,Abramowicz2019_LUXE} [see Figure \ref{Concl:Figs:GammaDetDouble_LCFACam} (left)]. 
In this configuration a thin converter target, for instance a wire, and multiple detector screens measuring the spectrum of the pairs could give access to a double differential gamma spectrum or even a triple differential spectrum with suitable deconvolution methods. 
However, such a spectrometer still has to be fielded experimentally at hundreds of MeV to GeV-scale photon energies and requires a detailed analysis of signal-to-noise ratios and shielding concepts.
\vspace{\baselineskip}

A crucial development area towards the measurement of absolute cross sections is the absolute calibration of radiation detectors. In this work and commonly done in the LWFA community radioactive sources (here Na-22 and Cs-137, emitting particles or radiation at an energy of around $0.5 - 1$ MeV) were used to calibrate the energy deposition in the scintillator arrays. Whilst good agreement of the calibration with expected radiation yields was demonstrated in Section \ref{Chap:linICS:Sec:EdepResponse}, the fundamentally different response of the detector at higher photon energies motivates the use of a light source that provides photons of comparable energy. Linear inverse Compton scattering and bremsstrahlung are well understood phenomena and reach similar energies (see Sections \ref{Chap:Methods:subsec:GammaSpec} and \ref{Chap:linICS:Sec:EdepResponse}), and are hence suitable candidates for this purpose.

\begin{sidewaysfigure}
\centering
\includegraphics[width=0.8\columnwidth]{ConclFullSetup2.pdf}
\caption[Sketch of an experimental setup to measure strong-field QED phenomena incorporating some of the ideas and concepts discussed in this work and this section.]{Sketch of an experimental setup to measure strong-field QED phenomena incorporating some of the ideas and concepts discussed in this work and this section. From left to right: An $f/40$ off-axis parabola (OAP) focuses an intense laser pulse (red) onto the edge of a gas jet (blue) to accelerate electrons (blue) via LWFA. The flow of the gas jet is perturbed by a blade to induce shock injection. The plasma is diagnosed by two orthogonal probe beams (inset). A pair of $f/2$ OAPs is used for radiation reaction studies: the first OAP focuses the second laser pulse onto the LWFA electron beam to produce energetic gamma-rays (green) from inverse Compton scattering (ICS), the second OAP recollimates it and transports it away to diagnostics. In a photon-photon (PhPh) setup the OAP pair can be replaced by a bremsstrahlung converter (grey). A dipole magnet (grey) disperses the electron beam to separate it from the gamma-ray cone. Another $f/2$ OAP is either used to collide with the gamma-ray beam directly (PhPh, nonlinear Breit-Wheeler) or heats an X-ray target (orange, PhPh, linear Breit-Wheeler), which is then characterised by a crystal spectrometer and a pinhole camera (turquoise). In an ICS setup this laser pulse scatters off the electron beam at low intensity to act as beam profile and polarisation measurement. Electrons and positrons are dispersed by another dipole magnet: electrons pass through the vacuum window (orange) and multiple Lanex screens. The positrons are redirected by another dipole magnet (magnetic chicane) and through multiple tracking screens (brown) onto a calorimeter (gold). The gamma-rays from both interaction points pass through the vacuum window where they produce electron-positron pairs. Magnets disperse these onto Lanex screens, while the gamma-ray cones are incident onto sets of gamma profile and crystal spectrometers. A set of crystal spectrometers are, on the other hand, measuring the divergent low-energy X-rays predicted by the breakdown of the locally constant field approximation. Components such as collimators and laser diagnostics were omitted due to space constraints.}
\label{Concl:Fig:FutureExp}
\end{sidewaysfigure}

\clearpage
\section{Concluding remarks}

Strong-field QED is a novel research field that bears ample research opportunities into fundamental phenomena and novel diagnostics. Laser wakefield accelerators are a powerful tool to explore high-field phenomena in colliding-pulse geometries where electric field strengths are greatly enhanced in the rest frame of relativistic electrons and can realise real photon-photon colliders.
\vspace{\baselineskip}

The stringent requirements of these studies on laser systems, accelerators and diagnostics are expected to drive innovation, but will also make experimental setups more complex such that purpose-built facilities and designated target areas will be needed to facilitate precision measurements. High repetition rates and automated alignment procedures, e.g. guided by machine learning, might become a necessity to cope sustainably with the large number of diagnostics in real time and to achieve a similar standard for `discoveries' as particle accelerators.
\vspace{\baselineskip}

The overlapping interests of the conventional accelerator, and the high-intensity laser and laser-plasma community might also signal the opportunity for a closer alignment of the fields beyond PWFA research, which could stimulate a fruitful exchange between the disciplines and establish plasma-based technology further into the mainstream of accelerator science.
 
 