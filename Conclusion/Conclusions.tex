\chapter{Conclusion and Outlook}
\label{Chap:Conclusion_and_Outlook}

\section{Summary of Results}
\label{Chap:Conclusion}

The experimental results presented in this work demonstrate the capability of laser wakefield acceleration (LWFA) to produce highly energetic radiation from few to hundreds of MeV photon energies, and their application in the context of fundamental studies of quantum electrodynamics (QED):

\begin{itemize}
\item \textbf{Chapter \ref{Chap:linICS}} described how relativistic electron beams with energies up to $1.3\,\mathrm{GeV}$ were collided with a laser pulse at $a_0 \sim 0.2 - 1$, producing radiation from \textbf{linear inverse Compton scattering} of variable spectral shape in the range of 10s of MeV photon energies. The radiation yield and spectrum were used to diagnose the properties of the electron and the laser beam at the interaction. Linear inverse Compton scattering promises to become a useful \textbf{beam diagnostics and alignment tool} for future studies of radiation reaction.

\item \textbf{Chapter \ref{Chap:RR15}} outlined how relativistic electron beams of energy $\epsilon \approx 550\,\mathrm{MeV}$ were collided with a highly intense laser pulse of intensity $a_0 \approx 10$, producing broadband synchrotron-like radiation with a critical energy $\epsilon_{crit} > 35\,\mathrm{MeV}$ from \textbf{non-linear inverse Compton scattering}. The high energy of the emitted photons lead to a measurable energy loss in the electron spectra due to \textbf{radiation reaction}.


\item \textbf{Chapter \ref{Chap:BW}} reported how electron beams were propagated through a solid target to produce an energetic gamma-ray beam from \textbf{bremsstrahlung} reaching photon energies of several hundreds of MeV. The yield and collimation of the source were optimised for its application in a photon-photon collider experiment with the aim to produce electron-positron pairs from the \textbf{linear Breit-Wheeler process}. The second photon source was a $~$keV X-ray source from a burn-through foil heated by a second high-intensity laser.
\end{itemize}

\section{Discussion and Outlook}
\label{Conclusion:Discussion}

The experimental results presented in this work demonstrate that an all-optical colliding-pulse geometry using a laser wakefield accelerator (LWFA) is suited to investigate fundamental (strong-field) QED (SFQED) phenomena (see Chapters \ref{Chap:RR15} and \ref{Chap:BW}), and can contribute significantly to this emerging research field \cite{Pike2014_BW,Cole2018_RR,Poder2018_RR}.
Whilst this bears exciting research opportunities, especially at the next generation of multi-PW laser systems \cite{Gales2018_ELINP,Weber2017_ELIBeamlines,Zou2015_Apollon,Li2017_SULF,Shen2018_SULF,Cartlidge2018_SEL,Toth2017_BELLA,Bashinov2014_XCELS,Kessel2018_PWMPQ,EPAC_Website,Maksimchuk2019_ZEUS,Yanovsky2008_HERCULES}, the results have also shown the need for a better performance of the laser system and the accelerator, and good control over both \cite{Samarin2017_RR}. They also motivate the development of suitable diagnostics to facilitate precision measurements with statistical significance \cite{Arran2019_RR_PPCF,Arran2019_RR_SPIE} and to be competitive (or complementary) with mature conventional accelerator facilities investigating SFQED phenomena \cite{Burkart2019_LUXE,Abramowicz2019_LUXE,SFQEDOverview2019,Hogan2016_FACETII}.
\vspace{\baselineskip}

Since the first reported measurement of quasi-monoenergetic electron beams in 2004 \cite{Mangles2004_MONO,Faure2004_MONO,Geddes2004_MONO}, the maximum electron energies have increased by a factor of 100 \cite{Gonsalves2019_GEV} -- a rapid increase that aligns with the rate of energy gain observed historically in conventional accelerator technology \cite{LivingstonPlot}, e.g. see Figure \ref{Introduction:Figs:CERN_Livingston} \cite{Panofsky1997_Livingston} in Section \ref{Introduction:Sec:PartAccelerators}. This progress in the field of LWFA has over the past decades been strongly driven by the enhanced control over high-intensity lasers and the exponentially rising peak intensities. Along with an increase of the maximum electron energies there have been significant improvements to the overall beam quality \cite{Osterhoff2008_CELL} and an enhanced control over the final properties of the accelerated bunch \cite{Banerjee2013_TunableLWFA}, also related to a better understanding of injection mechanisms \cite{Buck2013_SHOCK}, the evolution of the bubble \cite{Guillaume_REPHASING}, and the tailoring of both. By now LWFA electron beams and the energetic radiation they can generate are suited for various applications \cite{Albert2016_APP}, including the study of fundamental QED phenomena discussed here \cite{Pike2014_BW,Cole2018_RR,Poder2018_RR}.

However, LWFA is still an actively evolving accelerator technology that has not yet reached a level of maturity comparable to those of accelerators based on radio-frequency (RF) cavities: 
current challenges are, for instance, the stability (or shot-to-shot reproducibility) of the accelerator (see also Sections \ref{Chap:linICS:sec:Espec}, \ref{Chap:RR:Sec:StabilityStatisticsModel} and \ref{BW:sec:Espec}) and the production of low emittance and energy spread beams, for instance to be used in FELs \cite{Nakajima1996_FEL}. Another key challenge is to enable an operation of the accelerator at high repetition rate and efficiency \cite{Danson2019_PWLASERS}. 
There are also diagnostic challenges to characterise the pulse duration, beam size and the chirp of the beam accurately in an experiment  \cite{Downer2018_DiagnosticReview}.
This indicates that a particle collider fully based on plasma technology is still further in the future as traditional particle accelerators are employed for precision measurements. 
Next generations of lepton colliders, e.g. CLIC \cite{CLIC2012} or ILC \cite{ILC2013}, are hence still based on superconducting RF cavities. Nonetheless, the increasing scale and cost of RF-based accelerators, and their stagnating field gradients, will eventually require in some form the use of high-field technology such as plasma-based systems. 
For now, the wakefield community provides crucial expertise to enhance existing conventional accelerator facilities with beam-driven wakefield technology \cite{Caldwell2009_ProtonPWFA} and their use as final amplification stage (PWFA afterburner) \cite{Lee2002_Afterburner} as essential component in hybrid schemes \cite{Adli2013_PWFACollider} is investigated.
\vspace{\baselineskip}

The following sections will briefly recapitulate the research opportunities in SFQED for all-optical colliding-pulse setups using LWFA, and discuss more specifically considerations regarding diagnostics, and the required performance of the accelerator and laser system when attempting a measurement based on the results presented on the results presented in this work (Chapters \ref{Chap:linICS}, \ref{Chap:RR15} and \ref{Chap:BW}).


\subsection{Research Opportunities}
\label{Concl:Sec:ResearchOpportunities}

All-optical colliding-pulse setups where an LWFA electron beam is collided with a counter-propagating intense laser pulse are uniquely suited for studies of inverse Compton scattering, radiation reaction and other high-field phenomena (see Chapter \ref{Chap:RR15} and \cite{Cole2018_RR,Poder2018_RR}): 
in LWFA setups an intense laser system and a relativistic electron accelerator are co-located, and both are intrinsically synchronised to each other. 
Moreover, the small size of the electron beam at the end of the accelerator \cite{Weingartner2012_BUNCH} and its short bunch duration \cite{Lundh2011_BUNCH}, allows for a full overlap with a tightly focused (and hence very intense) laser pulse and hence an interaction with almost all of the electrons at a comparable intensity \cite{Harvey2016_ICSFOCUS}.
Since relativistic electron beams and intense laser pulses can produce a wide range of energetic radiation, this setup can also be converted in a pure photon-photon collider (see Chapter \ref{Chap:BW} and \cite{Pike2014_BW}).
As a result, LWFA and next generation high-intensity multi-PW laser systems \cite{Danson2019_PWLASERS,Gales2018_ELINP,Weber2017_ELIBeamlines,Zou2015_Apollon,Li2017_SULF,Shen2018_SULF,Cartlidge2018_SEL,Toth2017_BELLA,Bashinov2014_XCELS,Kessel2018_PWMPQ,EPAC_Website,Maksimchuk2019_ZEUS,Yanovsky2008_HERCULES} are in this context in direct competition with state-of-the-art particle accelerators like FACET-II at SLAC, LUXE at the EuXFEL \cite{Burkart2019_LUXE,Abramowicz2019_LUXE,SFQEDOverview2019,Hogan2016_FACETII}, and studies using aligned crystals at CERN \cite{Wistisen2018_RR,Wistisen2019_RR}.
\vspace{\baselineskip}

Colliding-pulse geometries with relativistic particles and highly intense laser will enable a detailed investigation of, for instance, Compton scattering (from the classical \cite{Yan2017_ICS} to the perturbative quantum description \cite{Bula1996_RR} and in the non-perturbative strong-field regime \cite{Bula1996_RR,TaPhuoc2012_ICS,Chen2013_ICS,Powers2014_ICS,Sarri2014_ICS,Khrennikov2015_ICS,Mackenroth2013_nlCompton}). Approaching a quantum nonlinearity parameter, $\eta$, of unity we can probe the transition from classical to quantum radiation reaction \cite{Wistisen2018_RR,Thomas2012_LL,Blackburn2014_QRR,Dinu2016_QRR,Vranic2014_RR} and perform precision measurements of (quantum) radiation reaction \cite{Arran2019_RR_PPCF,Arran2019_RR_SPIE}, e.g. the manifestations of stochasticity \cite{Niel2018_RRStochastic} through energy straggling \cite{Shen1972_STRAGGLING} and radiation quenching \cite{Harvey2017_QUENCHING} visible in the emission spectrum and the post-interaction electron spectrum \cite{Ridgers2017_QRR}. The emission might also result in a spin-polarisation of the electron beam \cite{DelSorbo2017_SPIN,Seipt2018_SPIN}\cite{SokolovTernov1964_POL}. 

In this regime intense interactions also give rise to (multi-step) cascades and showers \cite{Blackburn2017_pairs,Bulanov2013_Cascade}, which are particularly relevant in the astrophysical context. At $\eta \sim 1$ we reach the regime of nonperturbative strong-field QED \cite{Yakimenko2019_NONPERTURB,Blackburn2019_SUPER,Baumann2019_NONPERTURB}, where quantum effects dominate. Here QED cascades can produce dense pair plasmas \cite{Ridgers2012_DENSE,DoE2015_FES}. 

Suitable radiation diagnostics will enable measuring the formation length of the photon emission via breakdown of the local-constant field approximation \cite{Ritus1985_QRR,DiPiazza2018_LCFA,Ilderton2019_QEDPerturbBreakdown} and it will allow scrutinising the range of applicability of state-of-the-art numerical methods \cite{DiPiazza2019_LCFANUM,Elkina2011_QEDcascadesNUM,Ridgers2014_QEDNUM,Gonoskov2015_ReviewQEDNUM}.
\vspace{\baselineskip}

They also enable the investigation of photon processes such as two \cite{Pike2014_BW,Drebot2017_BW_ICS} and multi-photon pair production \cite{Burke1997_RR}, at $\eta \sim 1$ vacuum pair production \cite{Bell2008_PairsEta,Hu2010_TRIDENT,Ilderton2011_TRIDENT}, and probing the quantum fluctuations and the vacuum itself through nonlinear properties of the vacuum at these extreme fields such as vacuum birefringence  \cite{King2016_VB,Nakamiya2017_VB}, vacuum dichroism \cite{Bragin2017_VBVD}, and vacuum recollision \cite{Meuren2015_HERecollision,Kuchiev2007_Recollision}. Photon-photon interactions are also a potential route for probing physics beyond the Standard Model, e.g. sleptons \cite{Beresford2019_PhPh_SLEPTONS} and axions \cite{Knapen2017_PhPh_AXIONLIKE,Baldenegro2018_PhPh_AXIONLIKE}.
\vspace{\baselineskip}

At more extreme conditions the Hawking-Unruh effect \cite{Schutzhold2006_UnruhEffect,Chen1999_UnruhEffect} and the conjectured strong-coupling regime of QED, the Ritus-Narozhny conjecture \cite{Ritus1972_NAROZHNY,Narozhny1980_NAROZHNY,Fedotov2017_NAROZHNY}, become accessible.

\subsection{Experimental Considerations}
\label{Concl:Sec:ChallengesReq}

In order to make useful measurements of new phenomena in the SFQED regime we have to have an accurate understanding of the conditions at the interaction, e.g. electron spectra and laser properties, and measure the interaction products or post-interaction spectra accurately. The results presented in this work (Chapters \ref{Chap:linICS}, \ref{Chap:RR15}, \ref{Chap:BW}) indicated that we require an improved performance or stability of the source and suitably precise diagnostics for laser pulses, particles and energetic radiation for this purpose. Since laser wakefield accelerators suffer from `intrinsic' shot-to-shot fluctuations, the development of on-shot diagnostics are more pressing than in conventional accelerators. An improved performance of the laser system will also feed into a better performance of the electron source. 

We will discuss the different parts of the experiment, e.g. the accelerator, radiation diagnostics and so on, and identify potential requirements and solutions.
Overall we also expect that theoretical and numerical techniques mature as more experimental results become available, and we will not discuss these explicitly.

\subsubsection{Laser Wakefield Accelerator}


\begin{itemize}
\item Focusing geometry: long focusing $f/40$ or longer
\item give example of maximum energy, show low energy spread is possible
\item Targetry and injection: gas jet (diagnostics access and damaging properties, stability can be improved, less important at high intensities, narrow-energy spread beams through shock injection, could fix timing with shock injection
\item for now not considering focusing optics like a capillary as size was much larger 
\item high repetition rates and debris
\end{itemize}

\begin{figure}
\centering
\includegraphics[width=1.0\columnwidth]{GasJetCloseUp_Combo.pdf}
\caption[Close-up sketch of gas jet and diagnostics to characterise the plasma.]{Close-up sketch of gas jet and diagnostics to characterise the plasma. A transverse probe is used to obtain a shadowgram and an interferometry image. The self-emission of the plasma is imaged from the side and the top.}
\end{figure}

The gas target would be a gas jet, as concluded in Chapter \ref{Chap:RR15}, as it allows debris-free high-intensity interaction and open access for diagnostics, which is more favourable than the potential increases shot-to-shot stability of a gas cell target as used in Chapter \ref{Chap:BW}. It has been shown in Chapter \ref{Chap:linICS} that we can shoot suitably high above the nozzle ($up to 14\,\mathrm{mm}$) to avoid damage to the nozzle, whilst still encountering a sufficient density profile and still produce suitable electron beams.

Introducing a shock front on purpose in Chapter \ref{Chap:linICS} produced comparable electron beams to beams measured in \ref{Chap:RR15}. In Chapter \ref{Chap:RR15} the beams exhibited a distinct spectral feature at 550 MeV. This was reproduced in Chapter \ref{Chap:linICS}, but reaching higher energies up to 1.3 GeV. The beams were of high charge and electrons were measured consistently over 100s of shots. In some cases, shock injection also showed the potential to produce high-charge narrow-energy-spread beams at the 1 GeV level at few percent energy spread. The reproducibility of these beams has to be improved and is ongoing research. However, the potential for these beams is favourable especially to distinguish models.
A shock injected beam is also expected to have a fixed injection point and hence constant relative timing to the driver beam at the interaction.

We hope to explore whether we can increase the final energy of the electron beam whilst maintaining the low energy spread by increasing the length of the gas jet and increasing the driver beam energy. 
\vspace{\baselineskip}



Low energy spread beams (shock injection, but also \cite{Wang2013_GEV_FIUDICIAL} and see Chapter \ref{Chap:linICS})


Diagnosing the plasma


DURABLE OPTICS OR LONG FOCAL LENGTH AREAS.


Key areas are reproducibility and stability, energy spread. The maximum energy is also of interest but many precision studies are already possible at the GeV to few GeV scale that is achieved in experiments. Stable beams increase the discovery reach (as for accelerators) as fewer shots are required to reach discovery significance (see for instance Section \ref{Chap:RR:Sec:StabilityStatisticsModel} and \cite{Arran2019_RR_PPCF,Arran2019_RR_SPIE}). Promising routes lie in density tailoring and in stabilising the laser pulse. Shock injection has been demonstrated to be suited to produce high-quality low-energy-spread beams, and will also localise the injection point to fix the laser-electron offset. Gas jets will also be the target of choice to avoid damage and debris. Low energy spread is good for quantum radiation reaction. Also shock injected beams can produce mono-energetic beams without dark current (compare to quasi-monoenergetic beams).

\subsubsection{Electron diagnostics}

Considering the intrinsic fluctuations of the electron beam we have to characterise the source either statistically or on-shot. This requires a high resolution of the electron energy, e.g. using multiple screens \cite{Soloviev2011_TWOSCREEN}, fiudicials \cite{Wang2013_GEV_FIUDICIAL} and deconvolution techniques to account for pointing and divergence/source size. The pointing of the electron beam is also important as well as its accurate charge. In the context of the intensity measurements relying on the ellipticity of the electron beam the ellipticity (and source size) is an important quantity that can alter the inferred intensity. It can be measured with a beam profile monitor but we have demonstrated a technique based on linear ICS that could non-invasive do this as well (see Section \ref{Chap:linICS:Sec:ICS_EbeamDiag}). Eventually we would want to measure this on-shot during high-intensity interactions. For this purpose we would have to use multiple beam optics to redirect the radiation and the beam first. Proposals for setups include plasma lenses (active, discharge), but here we might lose the benefit of the micron sized source. Potentially there is an avenue of finding a correlation between the beam ellipticity and the laser properties. More diagnostics will help solving this problem.

A more challenging measurement is the chirp of the electron beam and its bunch duration. These factors, together with the beam size, can influence the resulting radiation reaction, post-interaction spectra due to changes in intensity from focusing. It could be that short bunches are preferable to characterise radiation reaction in a detailed study and to then use longer bunches in a classical radiation reaction regime to actually measure the duration and chirp. In conventional accelerators electro-optical sampling and transverse deflecting structures (TDS) are used, e.g. LOLA. Alternatives that have been used in LWFA are transition radiation and magento-optic sampling (see \cite{Downer2018_DiagnosticReview}). However, in these experiments we do not want to place any objects into the beam path (damage and secondary particles/radiation).

Especially in the context of high intensity interactions and potential spin-polarisation we also have to develop the tools to measure the spin polarisation. In conventional accelerators this is being done as well with linear Compton scattering, such that this would require a separated interaction point to avoid overlap of the radiation with the high-intensity signal (magnetic chicane). This also requires larger experimental target areas and high precision measurements of the charge and the gamma ray yield to distinguish different rates.  We discussed in Section \ref{Chap:linICS:Sec:Nphotons} that low-energy compact lasers will be sufficient to produce a sufficient signal. Use a long focal length or Fabry-Perot cavity.



\subsubsection{Scattering geometry}

We now consider the scattering geometry in the colliding-pulse setup.
Simplistically the highest $\eta$ is preferable in an interaction as it produces a stronger response of the system and gives access to more phenomena. $\eta$ increases with the electron energy, $\gamma$, and the laser intensity, $\sqrt{I_L}$:
\begin{equation}
\eta = \frac{(1-\cos \theta) \gamma E_L}{E_S} \propto (1-\cos\theta) \sqrt{I_L}.
\end{equation}
Consequently, assuming we already optimised the electron beam, it is preferable to focus the laser pulse tightly to increase the peak intensity. At the exit of the wakefield accelerator the electron beam is of $\sim\microns$ size. A focal spot of slightly larger size than that would be preferable (at least) such that the electron beam acts as a probe, otherwise different parts of the electron beam experience different intensities and the wavefront of the laser pulse has to be considered in the modelling process. Very short focal lengths place the optic into the vicinity of the interaction and could result in damage. On the other hand, longer focal length optics maintain a high intensity over a wider range and larger $z_R$ facilitate overlap between the laser pulse and the electron bunch. At the same time pointing fluctuations are larger.
In this work an $f/2$ off-axis parabola (OAP) was used in all cases which appears to be a good choice.

\begin{figure}
\centering
\includegraphics[width=.5\columnwidth]{DoubleF2Setup.png}\includegraphics[width=.5\columnwidth]{HoleyF2Setup.png}
\caption{Scattering geometries using either a head-on geometry (right) or an off-axis parabola at an off-angle (left). Here we also use a second parabola to re-collimate the light.}
\label{Conclusion:Figs:ScatteringOAPs}
\end{figure}

At a fixed electron energy and laser intensity $\eta$ is maximised in a head-on collision, i.e. $\theta = \pi$, such that it is favourable to place the focusing optic as close as possible to the laser axis.
One can then either place the OAP directly onto the axis (see Figure \ref{Conclusion:Figs:ScatteringOAPs} (right) and Chapters \ref{Chap:linICS} and \ref{Chap:RR15}), which then requires a hole in the optics, or place it close enough leaving it at an angle (see Figure \ref{Conclusion:Figs:ScatteringOAPs} (left) and Chapter \ref{Chap:BW}). 

A hole in the beam results in loss of energy and the head-on collision bears risks of damaging the optic and, more importantly, the laser upstream. It also has to be large enough to avoid any interaction with radiation as it could produce secondary particles (noise) or could in general affect the measurement of particles and radiation. The laser after the interaction is then rapidly diverging through the plasma such that the gas target has to be far enough from the axis or very robust. 
On the other hand, on-axis not only maximised $\eta$, it is also easier to align and the overlap is longer and easier to maintain. In the experiments presented in Chapters \ref{Chap:linICS} and \ref{Chap:RR15} the loss of energy and potential loss of $\eta$ due to an angle were about the same, such that this method was chosen for the ease of alignment.
Persistent overlap was demonstrated in Chapter \ref{Chap:linICS} and in particular in the alignment procedure (raster scan) outlined in Section \ref{Chap:linICS:Sec:RasterScan}.
The prolonged overlap of the laser pulse and the electron beam might also be beneficial to induce cascades \addref. In this geometry the polarisation vector of the laser pulse is also always orthogonal to the electron axis, whereas an off-axis geometry this would only be true in one axis. Larger beam sizes reduce the impact of the hole (might be able to use a holey mirror beforehand and use this as probe?), but this scenario might become very risky at multi-PW facilities and prone to damage.

When placing the parabola off-axis, on the other hand, the full energy is used. Potential diffraction effects are also gone at cost of the reduced angle. The beam can also be recollimated by another OAP after the interaction to diagnose the beam (see Figure \ref{Conclusion:Figs:ScatteringOAPs} (left), which might be beneficial at future facilities where lasers are very intense and might produce debris. The alignment is slightly more complicated and the overlap is reduced. Whilst this is not the ideal configuration to study cascades, especially if the conditions are at the threshold, it is being proposed that an oblique angle of incidence is required to reach critical or even supercritical energies and to avoid losses through radiation and particle production \cite{Blackburn2019_SUPER}. This is also less risky for future generations where the lasers are so intense that a perfect counterpropagating scenario is too dangerous, especially at high repetition rates. This geometry is also well suited for photon-photon collisions as shown in Chapter \ref{Chap:BW}. Here the OAP comprises a potential source of secondary particles and emissions and is better located off-axis. We see that this can be done either directly by a laser to probe multiphoton pair processes, high-field effects, or use the laser to produce higher energetic radiation which is suited for two-photon pair production and some scatter experiments (energy scaling). In the off-axis geometry this can be easy changed between.

In summary, both geometries (see Figure \ref{Conclusion:Figs:ScatteringOAPs}) have their advantages and disadvantages, and it might be advisable to switch between those for different research questions, in particular reaching high effective $\eta$ (supercritical) versus cascades and long overlap. At high intensities and for photon-photon interactions the off-axis geometry appears to be more sensible.

\subsubsection{High-intensity laser diagnostics and control}


In order to probe new physics we have to decouple the uncertainties of the laser beam from the model or theoretical uncertainties. A major challenge for high-intensity lasers is the characterisation of the full power high-intensity laser pulse on-shot. Typically the laser pulse is characterised in a low-power configuration (attenuated) and is then scaled up by the attenuation. However, at ultrahigh intensities this analogue might not be true anymore. There are multi-shot methods that propose a full wave front measurement of transmitted beams \cite{TERMITES}.
The key challenge will be measuring the intensity at the focus. There have been multiple proposals how to do this using gas jets, either relying on ionisation rates \cite{Ciappina2019_DiagPRA,Ciappina2019_Diag} or the ponderomotive effect onto the relativistic electrons \cite{Mackenroth2019_PonderomotiveInt}. 

Especially in the context of inverse Compton scattering there have been proposals that are also suitable on-shot, e.g. by considering the yield of the radiation in the linear case (limit $a_0 \sim 0$) in Chapter \ref{Chap:linICS}, the contribution of higher harmonic radiation in the approach of $a_0 \rightarrow 1$ \cite{Babzien2006_ICS_2ndHarmonic} or the ellipticity of the emitted gamma signal in the highly nonlinear regime \cite{HarShemesh2012_INTENSITY,Yan2017_ICS,Blackburn2019_ModelIndependentLaser}. This can be measured using the setup described in Chapter \ref{Chap:linICS}, but need to decouple from electron beam properties.
In the future when radiation reaction is well characterised, these counterpropagating experiments might themselves become a tool to characterise a laser pulse by passing them through specially tailored electron beams.


Also want to have control over the polarisation for polarisation dependent measurements and intensity measurements.
Higher repetition rates and stability also, similarly as for accelerators, increase the discovery reach. Additional control over the energy and the spatio-temporal distribution (adaptive optics, pulse shaping in time) is also required for different applications, e.g. to reach supercritical values a slow-rising pulse is not suitable, whereas cascades might benefit from those \cite{Blackburn2019_SUPER}.


\subsubsection{Spatio-temporal synchronisation}

\begin{figure}
\centering
\includegraphics[width=0.5\columnwidth]{CrossProbe.png}\includegraphics[trim={4.5cm 0 5cm 0}, clip, width=.5\columnwidth]{linICS_Raster_WaistSize_V3_annotated.png}
\caption[Monitoring spatio-temporal overlap: crossed transverse probes and laser raster scan.]{Monitoring spatio-temporal overlap. Left: Two optical transverse probes (red) are crossing perpendicular to each other through a gas jet (blue) and are used as shadowgraphy (planes). Right: Sketch of a laser raster scan (see Figure \ref{linICS:Figs:RasterTemporalTranslationIdea} in Section \ref{Chap:linICS:Sec:RasterScan} for more details).}
\label{Concl:Figs:SpatioTemp_CrossedProbe_Raster}
\end{figure}

One of the key challenges of colliding-pulse experiments is the overlap of the electron bunch and the laser pulse, or the light sources in case of photon-photon interactions, respectively.
In all-optical setups the laser pulses can be synchronised to each other using a combination of diodes and interferometry techniques (see Section \ref{Methods:Sec:DualBeamTiming} and \cite{Cole2018_RR}).

The results in Chapter \ref{Chap:RR15} suggest that after achieving spatio-temporal overlap there are three main issues to consider (see Section \ref{Chap:RR:Sec:Intensity}): first, the alignment and timing is varying from shot-to-shot (jitter). Second, spatial and temporal alignment might be drifting over time such that the electron bunch and the laser pulse do not remaining overlapped indefinitely (see Section \ref{Methods:Sec:TimingStability}). Third, there is an offset between the wakefield driver beam and the electron bunch it accelerates (see Section \ref{Methods:Sec:Beam_Laser_Timing}), which might also vary from shot-to-shot.
\vspace{\baselineskip}

Temporal jitter and drifts between the laser pulses can be monitored by a spectral interferometry which can measure the relative delay between the two quantitatively on full-power shots using transmitted beams at the end (use double parabola configuration) or leakage beams \cite{Shalloo_GEMINIDRIFT,Corvan2016_TIMING} after matching them to the absolute timing measurement.

In Chapter \ref{Chap:linICS} we found that the optical transverse probe (shadowgraphy) and the emission light are useful to monitor the spatial overlap, but the limited spatial and temporal resolution restricted the applicability. We propose to use two transverse optical probes with high magnification at the end of the accelerator to detect if the channel moves significantly over long time periodes (see Figure \ref{Concl:Figs:SpatioTemp_CrossedProbe_Raster} (left)), whereas the positioning of the scattering beam is expected to remain fairly stable (or use a simple pointing/far field diagnostic). A fast shadowgraphy \cite{Buck2011_BUBBLE,Savert2015_BUBBLE,Siminos2016_FASTSHADOW} could, on the other hand, enable a more accurate localisation of the bubble (instead of an expanded channel), but requires a significant effort to maintain the temporal compression and is, especially in the framework of a very complex multi-beam setup, less desirable.
\vspace{\baselineskip}

The offset between the electron bunch and the wakefield driver beam depends on the density of the plasma, its length, and the point at which the electrons are injected into the bubble. The complex interplay of nonlinear effects in the plasma the injection point might also vary from shot-to-shot. It is desirable to reduce the uncertainty on the injection point, e.g. through shock injection, and to measure the offset and relative jitter between the electron bunch and the laser pulse to find a systematic route to consistent overlap between the scattering laser pulse and the electron beam.

Examples of laser-electron timing diagnostics are electro-optical sampling (EOS) \cite{Cavalieri2005_EOS,Yan2000_EOS} and magneto-optic sampling \cite{Downer2018_DiagnosticReview}, which require a crystal and a laser which could be synchronised. This might be a technical challenge but has been demonstrated on a femtosecond level.
Plasma afterglow \cite{Scherkl2019_PLASMAAFTERGLOW} and fast shadowgraphy \cite{Buck2011_BUBBLE,Savert2015_BUBBLE,Siminos2016_FASTSHADOW} are mainly of interest in the context of PWFA.
In Chapter \ref{Chap:linICS} we demonstrated techniques based on inverse Compton scattering itself:
In Section \ref{Chap:linICS:Sec:RasterScan} we demonstrated a spatial-correlation technique that could systematically enable overlap (see Figure \ref{Concl:Figs:SpatioTemp_CrossedProbe_Raster} (right)).
We shown in Section \ref{Chap:linICS:Sec:YieldBeamSize} that we can use the radiation yield measured in an interaction to characterise the interaction at lower intensities and use this to statistically characterise the temporal laser-electron jitter.
Especially at higher intensities where there are model uncertainties it might be useful to constrain the conditions by performing some kind of measurement but this represents a useful tool to achieve high-intensity interactions systematically. This is particularly true for photon-photon colliders, where we could first perform a statistical measurement of the jitter using ICS, but would not be able to infer quantities on-shot without EOS or similar.

Especially multi-shot diagnostics and statistical characterisations will benefit from high repetition rates. Considering the increasingly large number of laser pulses and diagnostics, the volatile setup and the need to operate at high repetition rates, it might be useful to automate some parts of the alignment process according to selected quantities \cite{Streeter2018_TEMPFEEDBACK} and evaluate data immediately. In particular the laser raster scan and the yield are suitable techniques that could be automated with machine learning, for instance Bayesian inference. Quantitative measurements of alignment drifts could then also be corrected in real-time and increase the number of successful collisions and the discovery reach. 

\subsubsection{Radiation diagnostics}

\begin{figure}
\centering
\includegraphics[width=.5\columnwidth]{GammaDetDouble.pdf}\includegraphics[width=.5\columnwidth]{LCFACam_CloseUp.png}
\caption[Measuring high-energy radiation: combined gamma converter and scintillator array design, and crystal spectrometer.]{Measuring high-energy radiation: Left: Gamma spectrometer design combining a scintillator array as used in this work with a gamma converter spectrometer. The gamma rays (green) pass through a vacuum window or converter target (orange) and produce electron-positron pairs. A magnet (grey) disperses them and the particles are measured on detector screens. The gamma-ray beam propagates into the scintillator stack (yellow) where its profile and spectrum are measured. Right: A crystal (black) is used to disperse divergent low-energy X-rays onto the CCD of a X-ray camera (blue).}
\label{Concl:Figs:GammaDetDouble_LCFACam}
\end{figure}

The retrieval of highly energetic gamma radiation is a crucial tool to infer the conditions at the interaction, e.g. of the electron beam and the laser pulse (see Chapters \ref{Chap:linICS} and \ref{Chap:RR15}), and for photon-photon interactions (see Chapter \ref{Chap:BW}). In the study of strong-field effects we anticipate energetic radiation of hundreds of MeV to several GeV photon energies, but we have seen in this work (see Sections \ref{linICS:Chap:Sec:SpectralRetrieval} and \ref{BW:sec:GammaCharac}) that the accurate measurement of full spectra can be challenging.
The measurement of full spectra beyond the few tens of MeV range becomes increasingly difficult due to the flat response curve which is dominated by lower energy photons. The retrieval of the spectrum from an array of scintillating crystals as used here (see also Section \ref{Chap:Methods:subsec:GammaSpec}) we have seen in this work that we require a good knowledge of the spectral shape to start with \cite{Cole2018_RR,Behm2018_Gamma}. 

An alternative method in this regime is the measurement of Compton scattered electrons when the radiation passes through a converter foil \cite{Corvan2014_Gamma}. The cross sections are very sensitive up to tens of MeV where pair production becomes dominant instead, which, however, quickly reaches a flat response.

It has been proposed to combine both diagnostics in a pair spectrometer with scintillator array which can be used to find a spectrum as demonstrated in \cite{Lisi2018_Gamma} and proposed in \cite{Abramowicz2019_LUXE} (see Figure \ref{Concl:Figs:GammaDetDouble_LCFACam} (left)). By measuring the spectra of the Compton scattered and pair produced electrons and positrons in coincidence and simultaneously the response of the scintillator array we can infer the spectrum more accurately. 

By using multiple screens and a converter target, maybe a thin wire, we also gain access to a double differential spectrum. 
A full converter and multiple screens and a deconvolution algorithm could give access to a 3-dimensional spectrum. 
However, at the hundreds of MeV and GeV-scale this still has to be demonstrated, and requires a high sensitivity and efficient shielding as the flat cross section means changes are small. 
\vspace{\baselineskip}

We also demonstrated that the absolute energy deposition is important and that indicated that simple calibration with a radioactive source is not sufficient as the energy range and response is very different.
We also have to find a suitable method for an absolute calibration of these diagnostics and a test source that is suitable and well understood, depositing energy in a same manner. 
Here we also have to consider the degrading performance of detectors over time, especially if used under a lot of radiation.
\vspace{\baselineskip}

Radiation with divergent lower energy X-rays/Gamma rays are important to measure the breakdown of LCFA, i.e. allow measuring the formation length.
See Figure \ref{Concl:Figs:GammaDetDouble_LCFACam} (right). We can use a crystal setup for lower energies as in Chapter \ref{Chap:BW} or to increase the range of energies we can use a bent crystal. This works up to a the 100 keV level or so.

\section{Concluding remarks}

LWFA has been demonstrated to be a suitable tool to investigate strong-field phenomena in the emerging field of SFQED. Improvements and detector development are necessary to render the prospect of precision measurements feasible and to increase the discovery reach. Key developments that will facilitate these studies are the stability of the electron and laser source and a set of precise diagnostics that characterise the laser pulse, the electron bunch and secondary radiation and particles, ideally on-shot. This is necessary to decouple experimental and diagnostic uncertainties of model uncertainties. Eventually once these phenomena are well characterised, fundamental processes like radiation reaction and pair production could act themselves as diagnostic, e.g., for the intense laser pulse.

This might require some shifts in how LWFA experiments are operated, moving towards more automation and maybe applying machine learning (genetic algorithms, neural networks, Bayesian) to optimise and maintain parameters at high repetition rates. It will then become a useful tool as in conventional accelerators to increase the discovery reach.

It would also be beneficial to move to more fixed setups as done partially now in the community with at least semi-permanent setups that allow more complex setups and diagnostics. The discussion indicated a wide range of diagnostics will be required (see also Chapters, especially BW). On the other hand, this might eventually indicate that components of the experiment will become more and more separated as in conventional facilities where the accelerator and the interaction point and the diagnostics are often well separated, and beam optics (focusing, beam transport) are used frequently. Despite there is probably resistance as this goes away from the mantra of compact or table-top accelerators, the field gradients and the replacement of conventional to plasma optics is still there and it is still of much much smaller order of magnitude. This might be required specifically for these experiments. Other applications might still remain very compact (imaging/XANES applications).
 
 
\begin{sidewaysfigure}
\centering
\includegraphics[width=1.0\columnwidth]{FullSetup_V6Dec.png}
\caption{Sketch of an experimental setup incorporating some of the ideas discussed in this section.}
\label{Concl:Fig:FutureExp}
\end{sidewaysfigure}