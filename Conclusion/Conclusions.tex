\chapter{Conclusion and Outlook}
\label{Chap:Conclusion_and_Outlook}

\section{Summary of Results}
\label{Chap:Conclusion}

The experimental results presented in this work demonstrate the capability of laser wakefield acceleration (LWFA) to produce highly energetic radiation from few to hundreds of MeV photon energies, and their application in the context of fundamental studies of quantum electrodynamics (QED):

\begin{itemize}
\item \textbf{Chapter \ref{Chap:linICS}} described how relativistic electron beams with energies up to $1.3\,\mathrm{GeV}$ were collided with a laser pulse at $a_0 \sim 0.2 - 1$, producing radiation from \textbf{linear inverse Compton scattering} of variable spectral shape in the range of 10s of MeV photon energies. The radiation yield and spectrum were used to diagnose the properties of the electron and the laser beam at the interaction. Linear inverse Compton scattering promises to become a useful \textbf{beam diagnostics and alignment tool} for future studies of radiation reaction.

\item \textbf{Chapter \ref{Chap:RR15}} outlined how relativistic electron beams of energy $\epsilon \approx 550\,\mathrm{MeV}$ were collided with a highly intense laser pulse of intensity $a_0 \approx 10$, producing broadband synchrotron-like radiation with a critical energy $\epsilon_{crit} > 35\,\mathrm{MeV}$ from \textbf{non-linear inverse Compton scattering}. The high energy of the emitted photons lead to a measurable energy loss in the electron spectra due to \textbf{radiation reaction}.


\item \textbf{Chapter \ref{Chap:BW}} reported how electron beams were propagated through a solid target to produce an energetic gamma-ray beam from \textbf{bremsstrahlung} reaching photon energies of several hundreds of MeV. The yield and collimation of the source were optimised for its application in a photon-photon collider experiment with the aim to produce electron-positron pairs from the \textbf{linear Breit-Wheeler process}. The second photon source was a $~$keV X-ray source from a burn-through foil heated by a second high-intensity laser.
\end{itemize}

\section{Discussion and Outlook}
\label{Conclusion:Discussion}


The work presented in this thesis presents a proof-of-principle of the two main experimental geometries to be used in colliding pulse geometries to investigate SFQED. 
This work shows that LWFA can contribute significantly to this emerging field and are a capable technology that can advance knowledge at the frontier, in direct competition with significantly more mature conventional accelerator technology. 
This bears exciting opportunities but comes with significant challenges with respect to diagnostics and technology that have to be overcome to secure this community a significant role in the process.
The research opportunities (qualitatively new physics), and the challenges and requirements on diagnostics and the accelerator/laser will be discussed in the following.


\subsection{Research Opportunities for LWFA}

Laser wakefield acceleration (LWFA) is still a technology very much in active development. Since the first measurement of quasi-monoenergetic electron beams in 2004, the maximum electron energies have increased by a factor of 100, a rapid increase aligning with previous trends observed in conventional accelerator technology (see Livingston plot). The developments in the field of LWFA are strongly driven by the increased intensity and control of high-intensity lasers which have seen an exponential increase in focused intensities over the past decades. Along with an increase in the maximum electron energies there have been major improvements to the beam quality and control via different injection mechanisms, producing a wide range of spectral shapes. The electron beams and the radiation they are capable of producing have been by now also applied to investigate other physical research questions and have demonstrated their use.

However, there are still major challenges the field faces to improve the reproducibility, energy spread and emittance of beams, and to improve the efficiency and repetition rate of high-intensity lasers and the accelerator. There are also diagnostic challenges to characterise the pulse duration, beam size and the chirp of the beam accurately in an experiment. This indicates that a particle collider fully based on plasma technology is still far in the future as traditional particle accelerators are employed for precision measurements. Next generations of lepton colliders, e.g. CLIC and ILC, are hence still based on superconducting RF cavities. Nonetheless, the increasing scale and cost of RF-based accelerators, and their stagnating field gradients, will eventually require the use of high-field technology like plasma-based systems. For now, the wakefield community provides crucial expertise to enhance existing conventional accelerator facilities with beam-driven wakefield technology and their use as energy boosters/afterburners is investigated.
\vspace{\baselineskip}

However, LWFA is already now a feasible tool for certain applications. Most notably, it has been demonstrated to be a powerful tool to investigate phenomena in the emerging field of strong-field QED. All-optical colliding-pulse setups where an LWFA electron beam is collided with a counter-propagating intense laser pulse are uniquely suited for studies of inverse Compton scattering and radiation reaction (see Chapter \ref{Chap:RR15} and \cite{Cole2018_RR,Poder2018_RR}). The co-location of an intense laser system and a relativistic electron accelerator, and the intrinsic synchronisation between the two provides LWFA with an edge. Moreover, the small size of the electron beam at the end of the accelerator and its short duration in time, allows for a full overlap of a tightly focused (and hence very intense) laser pulse with the electron beam at a comparable intensity without background.
Since electron beams and lasers can produce a wide range of energetic radiation, this setup can also be converted in a pure photon-photon collider (see Chapter \ref{Chap:BW} and \cite{Pike2014_BW}).
As a result, LWFA and next generation high-intensity multi-PW laser systems \cite{Danson2019_PWLASERS,Gales2018_ELINP,Weber2017_ELIBeamlines,Zou2015_Apollon,Li2017_SULF,Shen2018_SULF,Cartlidge2018_SEL,Toth2017_BELLA,Bashinov2014_XCELS,Kessel2018_PWMPQ,EPAC_Website,Maksimchuk2019_ZEUS,Yanovsky2008_HERCULES} are in this context in direct competition with state-of-the-art particle accelerators like FACET-II at SLAC and LUXE at the EuXFEL \cite{Burkart2019_LUXE,Abramowicz2019_LUXE,SFQEDOverview2019,Hogan2016_FACETII} and aligned crystal studies \cite{Wistisen2018_RR,Wistisen2019_RR}.

Colliding pulse geometries with ultrarelativistic particles and highly intense laser will enable the study of, for instance, Compton scattering (from classical \cite{Yan2017_ICS} to the perturbative quantum \cite{Bula1996_RR} and the non-perturbative strong-field regime \cite{Bula1996_RR,TaPhuoc2012_ICS,Chen2013_ICS,Powers2014_ICS,Sarri2014_ICS,Khrennikov2015_ICS,Mackenroth2013_nlCompton}), precision measurements of (quantum) radiation reaction \cite{Wistisen2018_RR,Thomas2012_LL,Blackburn2014_QRR,Dinu2016_QRR,Harvey2017_QUENCHING,Vranic2014_RR}, spin polarisation \cite{DelSorbo2017_SPIN,Seipt2018_SPIN}\cite{SokolovTernov1964_POL}, (multi-step) cascades and showers \cite{Blackburn2017_pairs}, and gives access to the regime of nonperturbative strong-field QED \cite{Yakimenko2019_NONPERTURB}, \cite{Blackburn2019_SUPER}, \cite{Baumann2019_NONPERTURB}. Important developments in theory beyond LCFA \cite{Ritus1985_QRR,DiPiazza2018_LCFA} and numerical tools will be driven.

They also enable the investigation of photon processes such as two \cite{Pike2014_BW,Drebot2017_BW_ICS} and multi-photon pair production \cite{Burke1997_RR}, vacuum pair production \cite{Bell2008_PairsEta,Hu2010_TRIDENT,Ilderton2011_TRIDENT}, and of the quantum fluctuations and the vacuum itself in vacuum birefringence, vacuum dichroism, vacuum recollision. This also allows probing physics beyond the standard model.

In the future they might also allow measuring the Hawking-Unruh effect \cite{Schutzhold2006_UnruhEffect,Chen1999_UnruhEffect} and reaching the conjectured strong-coupling regime of QED, the Ritus-Narozhny conjecture \cite{Ritus1972_NAROZHNY,Narozhny1980_NAROZHNY,Fedotov2017_NAROZHNY}
\vspace{\baselineskip} 
 

As conventional accelerator communities and LWFA/high-intensity laser communities become more intertwined now with common research goals, it is expected that the exchange of technology and knowledge will let both communities thrive and grow together. 
All-optical setups and LWFA will in the process improve the control and face the challenges \cite{Samarin2017_RR}, but if it will be able to decouple the uncertainties of the accelerator/laser, the probe, and the new physics it is meant to probe, already current laser systems and LWFA parameters are suited for precision studies of radiation reaction \cite{Arran2019_RR_PPCF,Arran2019_RR_SPIE}. In particular, high-intensity laser systems will more naturally also be able to probe the complex interplay of collective and strong-field quantum effects in a new regime of high-energy density physics (HEDP).

\subsection{Challenges and Requirements}

These experiments at the frontier of strong-field QED have demanding specifications on the performance of lasers and electron sources and diagnostics.
The requirements are in general two-fold: one is the improvement of the performance, the second is the development of more precise and capable diagnostics.
Not all diagnostics have to provide quantities on-shot but due to the variability of both lasers and LWFA properties this is more required or urgent than in conventional accelerators. However, by improving the stability of both this can also be done off-shot.
Improving the laser capabilities also feeds into the performance of the wakefield accelerator and the scattering beam (interaction).
Due to the intrinsic fluctuations of laser wakefield accelerators (are they intrinsic?) the calls for on-shot diagnostics, which are commonly harder to realise especially in these challenging regimes.

beam and laser is challenging and requires better diagnostics, more stable electron beam and properties, better characterisation. Ellipticity measurements showed that this is necessary too. Also intensity at interaction can be significant and there can be drifts in time. Most experiments were at their limits due to damage thresholds and unreliable laser technology. Designated areas that match the specs of the laser system and allow for ample diagnostics access. Also these experiments will require multiple interaction points, magnets and ample shielding which requires distance at high energies too. This goes against the traditional table-top theme but is the way to achieve still in a much more compact scenario research goals (see BELLA). Last experiments were also assembled in weeks time instead of years. More complex experiments. High repetition rates and more automation as in conventional accelerators. See that more data and improved readout (see Matt Streeter's work).

\subsubsection{Scattering geometry}
\textbf{Experiment geometries for phenomena} \cite{Blackburn2019_SUPER}


Simplistically the highest $\eta$ is preferable which is maximised at the highest electron energies, laser intensities and in a head-on collision.
After considering the electron source we are now discussing the laser intensities and the collision angle.
\begin{equation}
\eta = \frac{(1-\cos \theta) \gamma E_L}{E_S} \propto (1-\cos\theta) \sqrt{I_L}
\end{equation}


\noindent\textbf{\sffamily{}Short focal length.}\ In this work an f/2 off-axis parabola was used to focus the laser tightly to interact with the electron beam. The tighter the focusing, the higher the intensities that can be achieved eventually. On the other hand, longer focal length optics maintain a high intensity over a wider range (larger $z_R$) which facilitate overlap.
At the exit of the wakefield accelerator the electron beam is of $\sim\microns$ size. A focal spot of slightly larger size than that would be preferable (at least) such that the electron beam acts as a probe, otherwise it is harder to model an interaction. As a result, smaller focuses than that are for fundamental studies maybe not suitable. Also it is hard to reach diffraction limit (micron).
\vspace{\baselineskip}

\begin{figure}
\centering
\includegraphics[width=.5\columnwidth]{DoubleF2Setup.png}\includegraphics[width=.5\columnwidth]{HoleyF2Setup.png}
\caption{Scattering geometries using either a head-on geometry (right) or an off-axis parabola at an off-angle (left). Here we also use a second parabola to re-collimate the light.}
\end{figure}

\noindent\textbf{\sffamily{}Beam geometry.}\ $\eta$ is maximised in a head-on collision. As a result, it is favourable to place the OAP as close as possible to the laser axis. One can then either place the OAP directly onto the axis, which then requires a hole in the optics, or place it close enough leaving it at an angle. A hole in the beam results in loss of energy and the head-on collision bears risks of damaging the optic and, more importantly, the laser upstream. It also has to be large enough to avoid any interaction with radiation as it could produce secondary particles (noise) or could in general affect the measurement of particles and radiation. The laser after the interaction is then rapidly diverging through the plasma such that the gas target has to be far enough from the axis or very robust. A gas jet is preferable. Otherwise we produce debris which again can damage optics in the area, especially at these high intensities at future facilities.
On the other hand, on-axis not only maximised $\eta$, it is also easier to align and the overlap is longer and easier to maintain. The prolonged overlap of the laser pulse and the electron beam is particularly crucial to induce cascades. The polarisation is also direct. Larger beam sizes reduce the impact of the hole (might be able to use a holey mirror beforehand and use this as probe?).

The off-axis angle, on the other hand, allows more flexibility of the target (height) and the full energy is used. Potential diffraction effects are also gone at cost of the reduced angle. The beam can also be recollimated by another OAP after the interaction to diagnose the beam\addref{}. The alignment is slightly more complicated and the overlap is reduced. Whilst this is not the ideal configuration to study cascades, it is being proposed that an oblique angle of incidence is required to reach critical or even supercritical energies and to avoid losses through radiation and particle production \cite{Blackburn2019_SUPER}. This is also less risky for future generations where the lasers are so intense that a perfect counterpropagating scenario is too dangerous, especially at high repetition rates.

In summary, both geometries (see Figure XX\addnum) have their advantages and disadvantages, and it might be advisable to switch between those for different research questions, in particular reaching high effective $\eta$ (supercritical) versus cascades and long overlap.
\vspace{\baselineskip}

\noindent\textbf{\sffamily{}Secondary interaction point.}\ As demonstrated in Chapter \ref{Chap:BW} a second interaction point is suited for photon-photon collisions. Considering the precarious situation where we do not want to produce secondary particles from interactions with matter, we use an OAP off-axis. We see that this can be done either directly by a laser to probe multiphoton pair processes, high-field effects, or use the laser to produce higher energetic radiation which is suited for two-photon pair production and some scatter experiments (energy scaling).

\subsubsection{High-intensity lasers}

\noindent\textbf{\sffamily{}Performance.}\ To reach new physics the performance of high-intensity lasers and their access has to be improved \cite{Danson2019_PWLASERS}. One is the intensity, which is steadily increasing at the moment, but new technologies are required to go beyond this, especially if we will not use conventional large scale accelerators with high energies, we require high intensities. The other is an increased stability, repetition rates and efficiencies which make laser systems more feasible to operate like particle accelerators. Higher repetition rates and stability also, similarly as for accelerators, increase the discovery reach. Additional control over the energy and the spatio-temporal distribution (adaptive optics, pulse shaping in time) is also required for different applications, e.g. to reach supercritical values a slow-rising pulse is not suitable, whereas cascades might benefit from those \cite{Blackburn2019_SUPER}. Long endurance of optics (damage threshold), which might be facilitated or say mitigated by having designated target areas that match the long focal lengths. Debris in high repetition rate experiments (rely on gases). Also want to have control over the polarisation for polarisation dependent measurements and intensity measurements.
\vspace{\baselineskip}

\noindent\textbf{\sffamily{}Diagnostics.}\ In order to probe new physics we have to decouple the uncertainties of the laser beam from the model or theoretical uncertainties. A major challenge for high-intensity lasers is the characterisation of the full power high-intensity laser pulse on-shot. Typically the laser pulse is characterised in a low-power configuration (attenuated) and is then scaled up by the attenuation. However, at ultrahigh intensities this analogue might not be true anymore. There are multi-shot methods that propose a full wave front measurement of transmitted beams \cite{TERMITES}.
The key challenge will be measuring the intensity at the focus. There have been multiple proposals how to do this using gas jets, either relying on ionisation rates \cite{Ciappina2019_DiagPRA,Ciappina2019_Diag} or the ponderomotive effect onto the relativistic electrons \cite{Mackenroth2019_PonderomotiveInt}. Especially in the context of inverse Compton scattering there have been proposals that are also suitable on-shot, e.g. by considering the yield of the radiation in the linear case (limit $a_0 \sim 0$) in Chapter \ref{Chap:linICS}, the contribution of higher harmonic radiation in the approach of $a_0 \rightarrow 1$ \cite{Babzien2006_ICS_2ndHarmonic} or the ellipticity of the emitted gamma signal in the highly nonlinear regime \cite{HarShemesh2012_INTENSITY,Yan2017_ICS,Blackburn2019_ModelIndependentLaser}. In the future when radiation reaction is well characterised, these counterpropagating experiments might themselves become a tool to characterise a laser pulse by passing them through specially tailored electron beams.


\subsubsection{Electron beam and synchronisation}

\noindent\textbf{\sffamily{}Performance.}\ Electron beam quality needs to be improved, but at the same time LWFA electron beams seem to have a bad reputation despite significant improvements in the beam quality over the past decade or so. For results in this frontier field to be accepted as in particle physics terms (discovery etc., statistical significant) and considering the complexity of setups, the quality has to be improved. Key areas are reproducibility and stability, energy spread. The maximum energy is also of interest but many precision studies are already possible at the GeV to few GeV scale that is achieved in experiments. Stable beams increase the discovery reach (as for accelerators) as fewer shots are required to reach discovery significance (see for instance Section\addnum{} and REF ARRAN\addref). Promising routes lie in density tailoring and in stabilising the laser pulse. Shock injection has been demonstrated to be suited to produce high-quality low-energy-spread beams, and will also localise the injection point to fix the laser-electron offset. Gas jets will also be the target of choice to avoid damage and debris.
\vspace{\baselineskip}



\noindent\textbf{\sffamily{}Diagnostics.}\ Considering the intrinsic fluctuations of the electron beam we have to characterise the source either statistically or on-shot. This requires a high resolution of the electron energy, e.g. using multiple screens \cite{Soloviev2011_TWOSCREEN}, fiudicials \cite{Wang2013_GEV_FIUDICIAL} and deconvolution techniques to account for pointing and divergence/source size. The pointing of the electron beam is also important as well as its accurate charge. In the context of the intensity measurements relying on the ellipticity of the electron beam the ellipticity (and source size) is an important quantity that can alter the inferred intensity. It can be measured with a beam profile monitor but we have demonstrated a technique based on linear ICS that could non-invasive do this as well. Eventually we would want to measure this on-shot during high-intensity interactions. For this purpose we would have to use multiple beam optics to redirect the radiation and the beam first. Proposals for setups include plasma lenses (active, discharge), but here we might lose the benefit of the micron sized source. Potentially there is an avenue of finding a correlation between the beam ellipticity and the laser properties. More diagnostics will help solving this problem. A thorough characterisation of the laser pulse and its peak intensity (relate this to on-shot wavefront measurement) should mitigate the issue.

A more challenging measurement is the chirp of the electron beam and its bunch duration. These factors, together with the beam size, can influence the resulting radiation reaction, post-interaction spectra due to changes in intensity from focusing. It could be that short bunches are preferable to characterise radiation reaction in a detailed study and to then use longer bunches in a classical radiation reaction regime to actually measure the duration and chirp. In conventional accelerators electro-optical sampling and transverse deflecting structures (TDS) are used, e.g. LOLA. These are not easy to implement in a laser-plasma beam line and would reduce conventional beam optics first. There might be laser-based alternatives (see Downer). Also transition radiation or magneto-optic sampling. However, do not want any material in the beam path, so crystal based technologies are suited as they are non-invasive.

Especially in the context of high intensity interactions and potential spin-polarisation we also have to develop the tools to measure the spin polarisation. In conventional accelerators this is being done as well with linear Compton scattering, such that this would require a separated interaction point to avoid overlap of the radiation with the high-intensity signal (magnetic chicane). This also requires larger experimental target areas and high precision measurements of the charge and the gamma ray yield to distinguish different rates. 


\subsubsection{Spatio-temporal synchronisation}

Another major challenge is the systematic overlap of the laser pulse and the electron bunch, or in a second interaction point the light sources.
In all-optical setups we can synchronise the laser pulses to each other \cite{Shalloo_GEMINIDRIFT}\cite{Cole2018_RR}\cite{Corvan2016_TIMING} but there is an offset between the laser pulse and the electron bunch (see Section\addnum). At the same time there is a temporal jitter and potentially change in the injection point. This can be mitigated by a well controlled density perturbation (shock injection). In Section\addnum{} we have shown that we can potentially use the interaction and radiation yield itself to characterise the offset. We also motivated that this could be used to automate the alignment procedure and to account for jitter and drifts in the system.

Methods to synchronise instead the electron beam and the laser pulse are based on EOS \cite{Cavalieri2005_EOS,Yan2000_EOS} but the short pulse durations are challenging and have to be explored. Other methods that synchronise electron bunches and laser are, for instance, plasma afterglow \cite{Scherkl2019_PLASMAAFTERGLOW} or fast shadowgraphy\cite{Buck2011_BUBBLE,Savert2015_BUBBLE,Siminos2016_FASTSHADOW}.
Ultimately it seems using the injection itself and some EOS would be the way forward. Whilst there is feedback in terms of radiation we can use for electron-laser interactions, it is more challenging in the light-light case as we only expect strong interactions when both are well overlapped. One approach would be scatter the electron bunch at the second interaction point and then activate the dispersion afterwards. By temporally linking both scattering laser pulses one can then monitor even with dispersion if there are drifts (another time spectral interferometry). Current all-optical setups use commonly one to two intense main beams (driver and scatterer or reflected) and an optical probe beam. This proposed geometry requires many more laser pulses and is accordingly more complex. Automation, i.e. quantitative on-shot evaluation of data, becomes necessary, but is feasible already at 5 Hz laser operation (Matt\addref). Machine learning using single-shot diagnostic and then infer without these on normal shots has been fielded at SLAC\addref. 

\subsubsection{Radiation diagnostics}

\begin{figure}
\centering
\includegraphics[width=.6\columnwidth]{GmmaDets_DoubleScreen.png}
\caption{An example of a combined gamma spectrometer setup.}
\end{figure}


\noindent\textbf{\sffamily{}Diagnostics.}\ The characterisation of highly energetic gamma radiation is a crucial diagnostic into the interaction of the electron beam and the laser pulse but also for future measurements of vacuum birefringence and dichroism. The measurement of full spectra beyond the few tens of MeV range becomes increasingly difficult due to the flat response curve which is dominated by lower energy photons. Using an array of scintillating crystals, for instance, we have seen in this work that we require a good knowledge of the spectral shape to start with \cite{Cole2018_RR,Behm2018_Gamma}. Compton scattering is more defined and distinct (see cross sections) at the lower level such that \cite{Corvan2014_Gamma} becomes less sensitive. Pair production becomes more dominant and the cross section is almost flat above few tens of MeV photon energies. It has been proposed to combine both diagnostics in a pair spectrometer with scintillator array which can be used to find a spectrum as demonstrated in \cite{Lisi2018_Gamma} and proposed in \cite{Abramowicz2019_LUXE}. By measuring the spectra of the Compton scattered and pair produced electrons and positrons (coincidence measurement!) and the response of the scintillator array simultaneously we can infer the spectrum more accurately. By using multiple screens and a converter target, maybe a thin wire, we also gain access to a double differential spectrum. A full converter and multiple screens and a deconvolution algorithm could give access to a 3-dimensional spectrum. However, at the hundreds of MeV and GeV-scale this still has to be demonstrated, and requires a high sensitivity and efficient shielding as the flat cross section means changes are small. We also have to find a suitable method for an absolute calibration of these diagnostics and a test source that is suitable and well understood, depositing energy in a same manner. Here we also have to consider the degrading performance of detectors over time, especially if used under a lot of radiation.

A second radiation topic is the lower energy and highly divergent radiation predicted in the LCFA. Large-area detectors, e.g. silicon chips with crystals, are required to measure these simultaneously, such that here the overlap with XFEL detector and particle detector groups becomes closer.

\subsubsection{Theoretical and numerical tools}

Improved theoretical description beyond LCFA and confirmation of numerical tools (currently based on semi-classical models).

\begin{sidewaysfigure}
\centering
\includegraphics[width=1.0\columnwidth]{RR20_ExpSetup_V12Sep.png}
\caption{Future experiment layout incorporating lessons learned.}
\label{Concl:Fig:FutureExp}
\end{sidewaysfigure}



\subsubsection{Concluding remarks}

LWFA has been demonstrated to be a suitable tool to investigate strong-field phenomena in the emerging field of SFQED. Improvements and detector development are necessary to allow for feasible precision measurements and to compete with conventional accelerator facilities. Key developments are the stability of the electron and laser source and the diagnostics characterising the laser, electron and radiation. This might require a shift in thinking from compact to larger setups with proper beam lines, and designated facilities that have semi-permanent setups that allow for the complexity of setups and diagnostics. This is already being observed in the community. Automation and machine learning will similar as in conventional facilities become useful tools as high repetition rates will increase the discovery reach. All in all indication that LWFA is slowly heading towards a setup where things are separated (accelerator, interaction point, focusing optics) as they are in conventional accelerators. Despite there is probably resistance as this goes away from the mantra of compact or table-top accelerators, the field gradients and the replacement of conventional to plasma optics is still there and it is still of much much smaller order of magnitude.
An example of a full setup is shown in Figure\addnum. Also more automation and less manual work as done on laser systems, now that technology has matured. This includes Bayesian or simple other methods.
 