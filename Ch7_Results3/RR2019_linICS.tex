\chapter{Linear Inverse Compton Scattering at Astra Gemini}

\section{Motivation}

... what is radiation of variable bandwidth in the 10 MeV range good for?

... overlapping in time and space helps for studies of nonlinearity

\section{Experimental Setup}

This experiment was performed at the Astra Gemini facility in early 2019 using both laser arms of the dual laser beam facility.
A sketch of the relevant components of the setup are shown in Figure \ref{LinICS:Fig:SetupBlend}.
\vspace{\baselineskip}




\begin{figure}
\centering
\includegraphics[width=0.8\columnwidth]{GeminiShock2019_ExpBlend_V1_Aug17.png}
\caption{Sketch of experiment setup.}
\label{LinICS:Fig:SetupBlend}
\end{figure}


\subsubsection{South beam and target}

The first laser pulse is focused with an f/40 off-axis parabola onto the edge of a 15 mm conical supersonic helium gas jet. The measured pulse duration was $61\,\mathrm{fs}$ \textsc{fwhm} with an average energy on-target of $12.5 \pm 0.2\,\mathrm{J}$, reaching a peak power of $195\,\mathrm{TW}$. The size of the focal spot was around $48.6 \times 39.2\,\mathrm{\mu m}$ \textsc{fwhm}, giving a peak normalised vector potential of $a_0 = 1.88$ in vacuum. The laser is polarised in the horizontal plane.
\vspace{\baselineskip}

The exit of the gas jet was positioned $\sim 14\,\mathrm{mm}$ below the beam axis. A steel razor blade is introduced into the gas flow $\sim 4\,\mathrm{mm}$ above the jet at $- 32.4\pm 0.3^\circ$ angle in vertical direction relative to the laser axis to produce a shock front. The electron density without the shock feature was $1.5 \times 10^{18}\,\mathrm{cm}^{-3}$.

The gas target was characterised on-shot by a transverse probe used as shadowgraphy and coupled to a Mach-Zehnder interferometry to determine the density. The beam was sourced from the transmission of the main beam through the first mirror, then re-collimated using an f/7 OAP and a lens to size XXX.
The pathways are matched and a delay slide gives independent control over the timing.

The gas target and recombination light of the plasma channel are also imaged by a Canon DSLR camera and a CCD chip from the top (Top View).

\subsubsection{North beam}

The second laser pulse is focused down tightly by an f/2 off-axis parabola (OAP) at the opposite of the gas jet in a geometry for head-on collisions. The f/2 OAP has a 21-mm-diameter hole in the centre to allow the electrons and radiation from ICS to propagate through. A round piece of plastic with radius $28\,\mathrm{mm}$ is also fitted around the hole to protect the optic from scattered light from the first laser. This reduces the on-target intensity by 16 degrees assuming a perfect top-hat laser profile. To protect the optic from potential debris a thin plastic layer with anti-reflective coating with a hole is attached to the OAP mount. The minimum spot size reached was about XX NUMBER, which translates into a peak normalised vector potential of $a_0$ ~ NUMBER. The laser is linearly polarised in the vertical plane. The beam can be defocused to larger spot sizes which makes an interaction less intense.

\subsubsection{Two-beam timing}

Both beams were timed to each other using spatial interferometry. A 90-degree knife-edge prism was inserted at the interaction point, deflecting both beams collinearly onto the CCD chip of a camera.

Two sets of timing measurements relying on spectral interferometry are also performed on-shot using transmitted beams, one in the laser area upstairs (before the beams enter the target area), and a second set within the chamber.

The in-chamber transmitted beams are also used as pointing references.

\subsubsection{Particle and Radiation diagnostics}

The radiation and the electrons propagate through the hole in the f/2 OAP into a large aperture ($\sim 10\,\mathrm{cm} \times 60\,\mathrm{cm}$ ) permanent dipole magnet\footnote{designed by Dominik Hollatz (Jena)} of integrated magnet field $\int B(x) \mathrm{d}x = 0.35\,\mathrm{Tm}$. A linear translation stage carrying high-Z metal converter targets can be driven into the beam path to intercept the electron beam and to produce gamma radiation from bremsstrahlung. Targets were tungsten, bismuth on PTFE XX CHECK THESE BITS.
The electrons are dispersed spectrally in the horizontal plane and leave the vacuum chamber through a two-layer vacuum window\footnote{designed and tested by Mechanical Engineering CLF, in particular Daniel Treverrow.} of dimensions $580 \,\mathrm{mm} \times 70\,\mathrm{mm}$. The inside layer is $25\,\mathrm{\mu m}$ of kapton with $375\,\mathrm{\mu m}$ of kevlar to provide additional mechanical stability and a fibre layer that holds as failsafe.
A scintillating LANEX screen placed just after the window measures the spectrum with very little scattering from the window due to the short drift distance. A second LANEX screen measures the spectrum about 1 m further downstream and can be used to account for pointing of the beam.
The electron spectrum is cut off at around 200 MeV as lower energetic electrons hit the yoke of the magnet and do not make out of the magnet.
Both screens are imaged by Andor Neo cameras with a third camera looking at a smaller part of the first screen.
\vspace{\baselineskip}

The radiation propagates straight through the aperture of the magnet, an $120\,\mathrm{\mu m}$ aluminium beam block and the kevlar-kapton window. Bright radiation is also captured by the first LANEX screen as it extends beyond the axis. A scintillating stack of caesium-iodide (CsI) crystals is measuring the profile of the radiation at the plane of the second LANEX screen (two on top of each other? EXACT DIMENSIONS DESY AND JENA). Another long stack of crystals can be used to retrieve the spectrum of the radiation (DUAL AXIS DIMENSIONS), both imaged by Andor iXon cameras. Both diagnostics are described in more detail in the Methods section and their GEANT modelling in the simulations chapter.

\section{Characterising Electron Spectra}

... Variety of electron spectra in terms of maximum energy and energy spread

... different injection mechanisms as well (have some interactions with self-injection or ionisation injection)

\section{Linear ICS and Nonlinearity}

... Use some of the examples and calculate the response according to linear ICS

... show linear ICS equation

... also given in Theory.

\begin{equation}
E_{ph}' = 2\gamma^2 (1-\cos\theta) E_{ph}
\end{equation}

... Discuss the variety of those 

... Now for a range of $a_0$ show what the impact of nonlinearity would be in terms of broadening, redshift, harmonic

... maybe for one example add a plot of linear, weakly and moderately non-linear (based on redshift and harmonics)


\section{Spectral Retrieval for linear ICS}

... show the profile screen and gamma spec raw data

... link this to Methods and briefly say a few things on the simulations and the fitting

... how to decide on successful collisions? Is that something too early to mention in this Chapter?

... use the linear spectra for the examples to fit to the response

... try to perturb with nonlinearity and see whether it finds better, give estimate of $a_0$

... energy deposition in one line? For linear ICS this should be fairly straightforward.


\section{Using ICS as electron diagnostics}

... ICS is acting as laser wire and gives pointing reference on each shot

... this is something Dominik Hollatz (Jena) is working on

... screen covering the axis with lanex

... second screen on plane with CsI profile screen to see reference/laser axis

...in linear regime divergence is 1/gamma from the scattering process and will be in direction of the electron beam (so more reliable than a betatron reference)



\section{Conclusion}