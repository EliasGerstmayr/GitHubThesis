\chapter{Linear Inverse Compton Scattering at Astra Gemini}
\label{Chap:linICS}
\section{Motivation}

... what is radiation of variable bandwidth in the 10 MeV range good for?

... overlapping in time and space helps for studies of nonlinearity

... being used as laser wire

... imaging capabilities

... testing timing etc. for other experiment

\section{Chapter Outline}

... Experimental Setup

... characterise electron beam and properties

... correlate yield and Espec energy
... estimate number of photons one would produce

... spectral retrieval

... use ICS as e-beam diagnostic: 

... correlate pointing in vertical and show good agreement

... same for divergence

... show horizontal pointing should also be reliable and indicate the errors this would introduce for the error

... show some examples of the beam profile and using this an non-intrusive method and show examples

... features visible on spectrometer screen (narrow energy)

... slight tilt

... oscillation in one axis

... do intensities match up?

... how do the intensities affect the measurement?

\section{Experimental Setup}

This experiment was performed at the Astra Gemini facility in early 2019 using both laser arms of the dual laser beam facility.
A sketch of the relevant components of the setup are shown in Figure \ref{LinICS:Fig:SetupBlend}.

\begin{figure}
\centering
\includegraphics[width=0.8\columnwidth]{GeminiShock2019_ExpBlend_V1_Aug20_annotated.png}
\caption{Sketch of experiment setup.}
\label{LinICS:Fig:SetupBlend}
\end{figure}


\subsubsection{Laser wakefield accelerator}

The driver beam for the wakefield accelerator is focused with an f/40 off-axis parabola (OAP) onto the edge of a 15 mm conical supersonic helium gas jet. The measured pulse duration was $61\,\mathrm{fs}$ \textsc{fwhm} with an average energy on-target of $12.5 \pm 0.2\,\mathrm{J}$, reaching a peak power of $195\,\mathrm{TW}$. The size of the focal spot measures $48.6 \times 39.2\,\mathrm{\mu m}$ \textsc{fwhm}, amounting to a peak normalised vector potential of $a_0 = 1.88$ in vacuum. The laser is polarised in the horizontal plane.
\vspace{\baselineskip}

The exit of the gas nozzle was positioned $14\,\mathrm{mm}$ below the beam axis. A steel razor blade is introduced $1.2\,\mathrm{mm}$ into the gas flow and $4\,\mathrm{mm}$ above the nozzle edge at $- 32.4\pm 0.3^\circ$ angle in vertical direction relative to the laser axis to produce a shock front. The electron density without the shock feature was $1.5 \times 10^{18}\,\mathrm{cm}^{-3}$.

The gas target was characterised on-shot by a transverse optical probe, split from the main driver beam, used as shadowgraphy and coupled to a Mach-Zehnder interferometer to determine the density. 
The pathways are matched and a delay slide gives relative control over the timing.

The gas target and recombination light of the plasma channel are imaged by a Canon DSLR camera from the side and a CCD camera from the roof of the chamber.

\subsubsection{Scattering beam}

A second laser pulse is focused down tightly by an f/2 off-axis parabola (OAP) at the opposite edge of the gas jet in a geometry for head-on collisions. The f/2 OAP has a 21-mm-diameter hole in the centre to allow the electrons and radiation from ICS to propagate through. A round piece of plastic with radius $28\,\mathrm{mm}$ is also fitted around the hole to protect the optic from scattered light from the first laser. This reduces the on-target intensity by 16 degrees assuming a perfect top-hat laser profile. To protect the optic from potential debris a thin plastic layer with anti-reflective coating with a suitable hole (`pellicle') is attached to the OAP mount. The on-shot intensity is $8.17\pm0.13\,\mathrm{J}$. The focal spot measures XX NUMBER, which translates into a peak normalised vector potential of $a_0$ ~ NUMBER (400 micron at interaction). In the dataset presented, the beam size is defocused at about 400 micron size diameter, which translates into XX a0 at the interaction. The laser is linearly polarised in the vertical plane.

\subsubsection{Two-beam timing}

Both beams are timed to each other using spatial interferometry. A 90-degree dielectric knife-edge prism (Thorlabs) was inserted at the interaction point, deflecting both beams collinearly onto the CCD chip of a camera. Due to the different radii of curvature of both beams near the focus of the f/2 beam, a circular interference pattern emerges when the beams are overlapped in space and time. The timing procedure is explained in more details in the \nameref{Chap:Methods}.

\subsubsection{Particle and Radiation diagnostics}

The radiation and the electrons propagate through the hole in the f/2 OAP into a large aperture ($\sim 10\,\mathrm{cm} \times 60\,\mathrm{cm}$) permanent dipole magnet\footnote{designed by Dominik Hollatz (Jena)} of integrated magnet field $\int B(x) \mathrm{d}x = 0.35\,\mathrm{Tm}$. A linear translation stage carrying high-Z metal converter targets (bismuth, tungsten) can be driven into the beam path to intercept the electron beam and to produce gamma radiation from bremsstrahlung to calibrate the gamma-ray diagnostics in this experiment. As alternative a low-Z PTFE target of thickness $1.6$ mm is mounted on the same stage to allow synchronous electron and gamma-ray measurements.

The electrons are dispersed spectrally in the horizontal plane and leave the vacuum chamber through a two-layer wide-aperture vacuum window\footnote{designed and tested by Mechanical Engineering CLF, in particular Daniel Treverrow.} of dimensions $580 \,\mathrm{mm} \times 70\,\mathrm{mm}$. The first layer facing the vacuum is $25\,\mathrm{\mu m}$ of Kapton followed by $375\,\mathrm{\mu m}$ of Kevlar to provide additional mechanical stability and fibre support that holds as fail-safe.
A scintillating Lanex screen placed just after the window measures the spectrum of the dispersed electron beam. Due to the light material of the window and the short distance from the window to the screen, the electron beam is only little affected by small-angle scattering. The screen is at $1.61\,\mathrm{m}$ distance downstream of the interaction point. A second Lanex screen measures the spectrum $700\,\mathrm{mm}$ further downstream ($2.31\,\mathrm{m}$ from the interaction point) and can in conjunction with the first screen be used to account for a pointing of the beam (REF).
The electron spectrum is cut off at $220$ MeV due to the limiting aperture of the magnet as lower energetic electrons collide with the yoke from the inside and do not leave the magnet.
Both screens are imaged by cooled 16-bit CCD cameras (Andor Neo) equipped with objectives and bandpass filters. A third camera images a small region of interest on the first screen.
\vspace{\baselineskip}

Radiation traverses the magnet on the laser axis, then passes through a $120\,\mathrm{\mu m}$ aluminium laser beam block and the Kevlar-Kapton window. Bright radiation in the right bandwidth is also captured by the first Lanex screen as it extends beyond the axis. A $45 \times 45$ array of $1\,\mathrm{mm} \times 1\,\mathrm{mm} \times 10 \,\mathrm{mm}$ scintillating caesium-iodide (CsI) crystals coated in $TiO_2$  measures the profile of the radiation at the plane of the second Lanex screen, which is placed $700\,\mathrm{mm}$ downstream from the first screen. The stack covers a field of view corresponding to cone with half angle $11.7\,\mathrm{mrad}$. A scintillation signal produced by radiation on the first Lanex screen and the CsI profile screen can be used as on-shot reference for the laser axis for the electron spectrometer screens. Another $704\,\mathrm{mm}$ further downstream another elongated array of CsI crystals is positioned to measure the spectrum of the gamma radiation. Both stacks, the profile screen and the spectrometer, are imaged using cooled 14-bit EMCCD cameras (Andor iXons). Both diagnostics are described in more detail in the Methods section along with how their response is simulated in the Monte-Carlo code GEANT4.

\iffalse
The aperture of the magnet permits a field-of-view of $79\,\mathrm{mrad}$ (10cm/1.26m) and is hence less limiting than the OAP hole. On the laser axis a beam block of $10-12$ layers of standard kitchen aluminium foil of thickness $10.1 \pm 0.2 \,\mathrm{\mu m}$ each is attached to the Kevlar-Kapton window. Hard X-rays continue to propagate through the aluminium beam block and the vacuum window (permitting a FOV of $70/1.61 = 43.4 mrad$ , the LANEX screen that also covers the axis and are then after $700\,\mathrm{mm}$ incident on a $45 \times 45$ array of $1\,\mathrm{mm} \times 1\,\mathrm{mm} \times 10 \,\mathrm{mm}$ scintillating caesium-iodide crystals coated in $TiO_2$ that record the beam profile of the radiation burst. The top part of the stack is centred on the beam axis. The total stack covers $54.2/2.31 = 23.4\,\mathrm{mrad}$. A second $30 \times 30$ stack of  $1.5\,\mathrm{mm} \times 1.5\,\mathrm{mm} \times 10 \,\mathrm{mm}$ CsI crystals is placed on top of the other one, covering another $51.2\,\mathrm{mm}$ or $22.2\,\mathrm{mrad}$. Harder radiation propagates another $704\,\mathrm{mm}$ before hitting another array of caesium-idodide crystals (see Methods for dual axis spectrometer) with an aluminium front-plate that are arranged in longitudinal direction to measure the spectrum of the radiation. The front part of the spectrometer then catches $XX/3m$ FOV. The emission of the caesium-iodide crystals is imaged by Andor iXon cameras.
\fi


\section{Characterisation of Electron Spectra}

The electron spectra measured in this experiment were produced by producing a shock in the supersonic gas flow. For now we only use the first Lanex screen to determine the energy of the electron beam and ignore pointing effects. The image processing of the data is not further elaborated here but can be found in XX THESIS JASON KRIS XX METHODS.

In the dataset considered in this Chapter the properties of the electron beam vary strongly from shot-to-shot and produce beams of a wide range of shapes, maximum energy and energy spread. A few examples of electron spectra from the relevant dataset are provided in Figure \ref{linICS:Figs:Elec_example}.

The properties and the irregularity of the results at one fixed blade (and shock) position is not consistent with results from other laser systems where shock injection was used to produce reproducible electron beams with narrow energy spread. This will not be resolved as part of this Chapter and is beyond the scope of this thesis, but will be addressed in a separate publication (PENDING MATT?).
\vspace{\baselineskip}

\begin{figure}
\centering
\includegraphics[trim={4.8cm 0 5cm 0}, clip, width=1.0\columnwidth]{Example_Espec_CollisionMontage.png}
\caption{Example Especs, not in shooting order.}
\label{linICS:Figs:Elec_example}
\end{figure}

In some cases the electron beam shows signatures of strong vertical oscillations which indicate potential to act as a bright betatron source. In other cases beams with narrow energy spread at 1 GeV are produced. 

Over the course of a 326 shot dataset the maximum energy of the electron beam is about $915\pm177\,\mathrm{MeV}$ with some beams reaching 1.3 GeV. 
\EliasComm{Make a histogram maximum energy.}
\EliasComm{Make histogram vertical pointing.}
\EliasComm{make histogram energy spread.}
\EliasComm{Make histogram divergence?}
\EliasComm{Make histogram charge?}

The charge of the beam varies in particular due to the varying length of the bunch (spectrally) at $187.6 \pm 95.6\,\mathrm{pC}$.

DIVERGENCE CHECK DISTANCE.

\EliasComm{Make plot for examples of self- and ionisation-injection?}

\section{Gamma yield from scattering}

Yield on Gamma profile depends on energy deposition:

Felicie and Alec's paper

\begin{equation}
N_X = \frac{\sigma}{\pi w^2_0} N_L N_e,
\end{equation}
where $\sigma_T = 6.65 \times 10^{-25}\,\mathrm{m}^{-2}$ the Thomson scattering cross-section. Estimate size of the electron beam at the interaction? Is it before or after the interaction?

Estimate $N_L$. For average $a_0 \sim 0.3$ and photon energy $1.55\,\mathrm{eV}$, intensity XX, we get N photons per micron squared.
So X number of photons for a 100 pC beam?

Ignore temporal structure as focusing effects are negligible and we just assume a constant field.

\begin{equation}
N_X = \frac{\sigma}{\pi w^2_0} \left[\frac{E}{E_{ph}} \frac{\pi w^2_0}{\pi r^2}\right] \left[\frac{Q}{e}\right]
\end{equation}

\begin{equation}
N_X = \frac{\sigma}{\pi r^2} \frac{E}{E_{ph}}
\end{equation}

So in useful units

\begin{equation}
N_X = 5.32 \times 10^{10} E [J] Q [100pC] (r[\mu m])^{-2}
\end{equation}

For 10 Joules, 100 pC and 200 um radius we then get $1.33 \times 10^{7}$ photons which is of similar order as Chen and Kneip (betatron).
Source size? Due to homogeneous assumption the variation does not change intensity. If we make the beam smaller to say 50 um

For the parameters above we get $2.4 \pm 1.24 \times 10^{7}$ photons.

The average maximum energy is $915$ which translates at head-on collision into a peak gamma radiation of $19.89 \pm 3.85\,\mathrm{MeV}$.


\begin{equation}
a_0 \approx 0.85 \lambda [\mu m] \sqrt{I [10^{18} cm^{-2}]}
\end{equation}

The intensity is reduced to the hole in the beam but are we at that distance in the near field, so energy-loss is not important but energy flux (flat top).

We have a relationship for energy of the radiation times energy deposition in the gamma profile from GEANT

How much does this change in terms of energy deposition per electron or charge over 100 MeV or so?
What is the `contrast' of this method?

So we expect the energy deposition to be 

\begin{equation}
E_{dep} (\gamma) N_X = \int E_{dep} (\gamma) \frac{\sigma}{\pi \omega^2_0} N_L \frac{\mathrm{d}N_e}{\mathrm{d}\gamma}\mathrm{d}\gamma,
\end{equation}

Check if that aligns with results in terms of trends.

Check if the absolute calibration is of any use to estimate the numbers.

\section{Gamma spectra from linear inverse Compton scattering}

The electron beams shown above were collided with a 400 um laser spot at XX energy, i.e. at an intensity well below 1.

Given intensity of the interaction, we propose linear ICS as mechanism with an unperturbed electron beam.

\begin{equation}
E_{ph}' = E_{ph}\frac{2(1+\beta\cos \theta)\gamma^2}{1+a^2_0/2 + \gamma^2 \theta^2_0} = \frac{4\gamma^2}{1+a^2_0/2 + \gamma^2 \theta^2_0}
\end{equation}

Based that this is at a0 << 1, we will for now ignore non-linear effects.
Linear ICS in a head-on collision scenario predicts that photons will be emitted in a narrow 1/gamma cone ($\gamma = 1750$ for 900 MeV, so 0.6 mrad) at 

\begin{equation}
E_{ph}' =4\gamma^2 E_{ph}
\end{equation}

For a 5mrad beam the difference through the electron angles for a collimated light source is less than 0.001 percent. So ignore this.
For f/2 beam the angle is up to 14.04 degrees but since we are interacting with a small subset of the beam (few microns of 400, the angle should be taken as constant but it could vary by up to 3 percent, which again is not a large change.


For a Ti:Sa system at 1.55 eV central photon energy this converts XX into XX, see Figure XX.
\EliasComm{Make plot of Ephoton against electron energy to indicate range of energies achievable.}

\EliasComm{based on the divergences measured, estimate the bandwidth of the radiation produced.}
\EliasComm{based on the spot size and the electron charge, estimate the number of photons produced.}

Treat this as a 1-to-1 mapping of the spectra.
Using some of the examples shown in the previous section and the linear ICS spectrum (assuming over the energy range radiation is produced equally).

Check whether the cross section is very energy-dependent. I think it is a constant Thomson cross section and just the angle changes.




\begin{figure}
\centering
\includegraphics[trim={4.8cm 0 5cm 0}, clip, width=1.0\columnwidth]{Example_Espec_CollisionICSMontage.png}
\caption{Example Espec to ICS radiation. Maybe replace this with one main plot and different X-axis (one for electron energy, second for photon energy, plot second harmonic in red/other colour matching second x-axis.}
\end{figure}

For the first harmonic this gives in some cases narrow-energy radiation of 35 MeV with 4 MeV.
In other case this gives broadband radiation of few to 20 MeV radiation which is of interest for industrial applications.

In reality range of angles due to focusing effect and divergent electron beam.

In this regime nonlinear effects are not important and can be neglected.

For future experiments, this setup is in principle able to produce high-intensity interactions.

For this purpose, now for a range of $a_0$ show what the impact of nonlinearity would be in terms of broadening, redshift, harmonic.
Show an example of linear with weakly and moderately non-linear (based on redshift and harmonics)
\begin{figure}
\centering
\includegraphics[width=1.0\columnwidth]{PlaceHolder_perturbedResponse.png}
\caption{This is to indicate change in response.}
\end{figure}


... show the profile screen and gamma spec raw data


To measure the spectrum a CsI stack was used. The simulation of the detector response and the experimental calibration was done as described in Methods XX. In this case the spectral fitting is performed by assuming a 1-to-1 mapping of the spectra. As a comparison a mono-energetic photon is used to see a similar response.
... how to decide on successful collisions? Is that something too early to mention in this Chapter?

... use the linear spectra for the examples to fit to the response

... show that mono-energetic is also a good fit, so the response is dominated by the highest counts of photons, so without knowing the input is difficult



... try to perturb with nonlinearity and see whether it finds better, give estimate of $a_0$ (not good)
.. try to see when it would start to become visible and estimate an intensity (this could be used as intensity estimate, which it will be used for in the other section).


\begin{figure}
\centering
\includegraphics[width=1.0\columnwidth]{PlaceHolder_perturbedResponse.png}
\caption{This is for real experimental data.}
\end{figure}

... energy deposition in one line? For linear ICS this should be fairly straightforward.

... peak energy for the lot?

... in this energy range a converter gamma spectrometer would work very well (depending on the converter Li or so?)


\section{Using ICS as electron diagnostics}

\EliasComm{Pointing of beams related?}

\EliasComm{Intensity profile?}

\EliasComm{Examples of profiles and extra information?}

\begin{equation}
\theta_f = \theta_e + 1/(\gamma N^{1/2}_0)
\end{equation}
with laser oscillations $N_0 = 10$.

By comparing the divergence from the screen and the gamma profile (assuming laser parameter, can we say something?)

ICS is a useful source of radiation. In the linear regime and at these energies the beam is not perturbed and the energy loss is small compared to the energy of the electrons. For instance, a 1200 MeV electron beam emitting one photon loses at most 34 MeV, which is less than 3 percent, well below the measurement accuracy.

Since it does not perturb the electron beam and only interacts with some electrons due to the low photon density, ICS has been used previously as beam diagnostic as laser wire in recognition of a physical wire.
Due to the Lorentz boost radiation in ICS is emitted in a narrow forward cone and is hence dominated by the bunch itself. This means a head-on collision is acting similarly to an electron profile screen and represents a longitudinal integration of the transverse shape. In contrast to a profile lanex screen this does not scatter the beam (or only a small fraction). This could be used as pointing reference whenever the beam is expanded and an overlap is guaranteed.

This is something Dominik Hollatz (Jena) is working on, analysis pending for FUTURE REF PUB

On one hand this can be used as on-shot pointing reference if the lanex is placed in the same plane or it can be correlated, to know where the beam axis is.

In addition, this is more reliably telling us the pointing than betatron radiation. We can use this to characterise the beam.

We compare divergence on espec screen and in the same axis on profile screen (dispersion direction is horizontal). This should match up (check distances).

Let us consider a few interesting shots (20190208r015s212 with double bits, sligthly tilted shots, 20190208r015s037 for oscillations indicative only in one axis).

For now we assume naively that the integrated spectrum will then correspond to the horizontally integrated signal on the CsI profile screen. There is a slight weighting factor as the higher energy radiation will deposit more energy (convert Espec to ICS, then find linear scale for energy deposition to restore the shape).

We see that a small beam with some off-axis charge is recognised by the screen.

We see that a beam with large oscillations shows up as elongated profile which implies that it does not oscillate much in the other direction.

We see that a slightly bent beam shows up as tilted ellipse which indicates that the kink in the beam is in two dimensions.

These examples show that it can reveal, given enough energy and homogeneous illumination and spatial resolution, some information about the properties of the electron beam beyond the dispersion axis without perturbation.

\begin{figure}
\centering
\includegraphics[trim={4.8cm 0 5cm 0}, clip, width=1.0\columnwidth]{Example_LaserWire.png}
\caption{Laser Wire example Espec and Gamma Profile for off-axis charge.}
\end{figure}


This is also of interest as we know from PUBLICATION A0 (WITH OR WITHOUT TOM BLACKBURN) that we can infer a0 from the ellipticity of the gamma signal. If we can properly characterise the fluctuations in the pointing and ellipticity, we can also be reassured about which portion is due to source size and which is due to intensity.


... estimate what kind of intensity we would require if used the beam more economically.

... spatial resolution is 1 mm over 2.31 m, so 0.43 mrad. In theory can magnify geometrically.

... scattering? Does MeV radiation scatter significantly through these thin and low Z materials?

... are the intensities matching up?

\EliasComm{Pointing correlated?}

\section{Conclusion}