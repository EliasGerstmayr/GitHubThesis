\chapter[Linear ICS at Astra Gemini]{Linear Inverse Compton Scattering at Astra Gemini}
\label{Chap:linICS}
                                                                                                                                                                                                                                                                                                                                                                                                                                                                                    \section{Motivation}

Linear Inverse Compton scattering has been widely used as high-energy radiation source to study nuclear processes, diagnose beam properties (energy, emittance, polarisation) and recently in laser wakefield accelerators as source for imaging applications.

As lasers are more intense we now also access the regime where we can use linear ICS as a single-shot diagnostic.

In the context of radiation reaction studies, linear ICS, especially in the transient regime between linear and non-linear is useful to characterise electron beams and estimate the fluctuations.

If the gamma detector is sensitive enough this could also be used to estimate the intensity by considering the nonlinearity.

\section{Chapter Outline}

Linear ICS can be used as non-invasive beam diagnostic. This has been shown and is being considered. Usually using a focused beam to produce enough yield, but this assumes a stable source. Beams from LWFA have intrinsic fluctuations and a single-shot measurement is more desirable.
The fluctuations are in terms of pointing, divergence, energy, energy spread, shape and so on. In standard setups typically we measure the electron beam energy using a magnetic spectrometer. This gives us spatial information in the non-dispersion direction but destroys the component in the dispersion direction. We can introduce a spatial screen in the undispersed beam path to measure the 2D divergence and the pointing but the material will lead to scattering of the beam which blurs out the beam for the energy measurement. This also requires shielding the detector from laser light, which will scatter the beam more or if we use a plasma mirror this adds complexity. If we are interested in radiation production from linear ICS the screen and the tape will produce bremsstrahlung, which skews the data and makes a spectral retrieval difficult.

We present linear ICS as non-invasive beam diagnostics that provides us with 2D divergence and pointing data. The spectral resolution of the profile tells us more about the shape of the electron beam, its ellipticity and so on.

This is useful for radiation reaction experiments that require us to have a good knowledge and characterisation of the beam parameters in 3D (temporal as well). Intensity measurements rely on an ellipticity of a non-linear ICS signal, but without knowing or having characterised the fluctuations of the beam ellipticity we do not know how likely this was an intense interaction. Linear ICS also provides an avenue to overlap the beam easily at low intensities and then use the scanning method as proposed as beam diagnostics to close in onto the position of the beam.

A transverse probe and second view (top view) with high magnification can help to maintain overlap.

We start by introducing the electron beam parameters and properties in this experiment. We are using shock injection in a wild and very varying regime producing a wide range of electron beams which in turn are expected to produce a wide range of radiation and also produces interesting spatial features that we can image.

We then estimate the experiment conditions and what kind of radiation we expect in an interaction. Based on this and the electron characterisation we then calculate the total yield and compare this to the yield of betatron measurements. 

We then retrieve the spectrum and compare how well this fits.

Then we use the profile measurement to compare the electron spectrometer with the gamma profile measurements. 
We see that the profile changes with the electron beam.
We first look at the pointing. This does not correlate so well.
We then investigate the divergence. This correlates well within an error range which is related to spatial resolution and the energy range as well.
We now look at the spatial features. We find examples where we have off-axis contributions with features, visible in both.
For instance elliptical beams, several beamlets, tilt of the beam and several beams and how they are oriented in space. 
However, the beam seems inverted and the pointing does not match. Estimate how much intensity difference this would require. Seems unrealistic. We interact in plasma, maybe oscillation and momentum. Momentum unrealistic as consistently inverted. Oscillation matches as we interact half a betatron wavelength inside, intensity low. Data with beam outside interacting..
We can not see these things just using the electron spectrometer.
We can use this to diagnose the beam better on-shot, to estimate beam sizes, if we have a good spectral measurement to estimate intensity, better for low energy spread beams, to characterise shape and to then go for radiation reaction.

The overlap can be done to a certain stability but we can use optical response by two transverse views. Use the probe and main beam timing.
If counter-propagating we also relatively safely interact, especially using a defocused beam. Then raster scan and intensify.


\section{Experimental Setup}

This experiment was performed at the Astra Gemini facility in early 2019 using both laser arms of the dual laser beam facility.
A sketch of the relevant components of the setup are shown in Figure \ref{LinICS:Fig:SetupBlend}.

\begin{figure}
\centering
\includegraphics[width=0.9\columnwidth]{GeminiShock2019_ExpBlend_V1_Aug20_annotated.png}
\caption[Sketch of experimental setup to measure radiation from linear inverse Compton scattering.]{Sketch of the experimental setup aimed to measure radiation from linear inverse Compton scattering. The first laser pulse (red, left) is focused down by an $f/40$ OAP into a gas jet and accelerates electrons (blue) from LWFA, where the density perturbation induced by the blade affects the injection mechanism. A second laser is focused down by an $f/2$ OAP and counter-propagates with the electron beam performing linear inverse Compton scattering. Gamma radiation produced in the interaction (green) is emitted in the propagation direction of the electron beam and is measured downstream by a scintillating profile screen and a stack of scintillating crystals used as spectrometer (right). A converter target behind the $f/2$ OAP can be used to convert the electron beam via bremsstrahlung into an energetic calibration source for the gamma diagnostics. The electrons are dispersed horizontally by a permanent dipole magnet onto a set of lanex screens. The accelerator, the interaction point and the magnet are in vacuum, whereas the measurement screens and gamma diagnostics are located at air separated by a thin vacuum window (orange).} 
\label{LinICS:Fig:SetupBlend}
\end{figure}


\subsubsection{Laser wakefield accelerator}

The driver beam for the wakefield accelerator is focused by an $f/40$ off-axis parabola (OAP) onto the leading edge of a 15 mm conical supersonic helium gas jet. The measured pulse duration was $61\,\mathrm{fs}$ \textsc{fwhm} with an average energy on-target of $12.5 \pm 0.2\,\mathrm{J}$, reaching a peak power of $195\,\mathrm{TW}$. The size of the focal spot measured $48.6 \times 39.2\,\mathrm{\mu m}$ \textsc{fwhm}, amounting to a peak normalised vector potential of $a_0 = 1.88$ in vacuum\footnote{The focal spot and FROG analysis was conducted by Matthew Streeter (Imperial College)}. The laser is polarised in the horizontal plane.
\vspace{\baselineskip}

The exit of the gas nozzle was positioned $14\,\mathrm{mm}$ below the beam axis to avoid damage from the second, more divergent laser cone. A steel razor blade is introduced $4\,\mathrm{mm}$ above the nozzle edge at $- 32.4\pm 0.3^\circ$ angle in vertical direction relative to the laser axis to produce a shock front. The tip of the blade is inserted $1.2\,\mathrm{mm}$ into the gas flow from the same side that the driver beam enters from. 
The gas target was characterised on-shot by a transverse optical probe synchronised with the driver beam ending in a shadowgraphy and a Mach-Zehnder interferometry setup.
The interferometrically determined electron density\footnote{The interferometry data was evaluated by Cary Colgan (Imperial College)} without inserting the blade was $(1.4 \pm 0.2) \times 10^{18}\,\mathrm{cm}^{-3}$. When inserting the blade the density profile exhibits a sharp peak in density reaching about $(2.4 \pm 0.2) \times 10^{18}\,\mathrm{cm}^{-3}$, twice the ambient density\footnote{The plasma densities were interferometrically measured by a transvere optical probe.}.

The gas target and the recombination light emitted by the plasma channel are imaged by a Canon DSLR camera from the side and a CCD camera from the roof of the chamber.

\subsubsection{Scattering beam}

A second laser pulse is focused by an $f/2$ off-axis parabola (OAP) at the opposite edge of the gas jet in a geometry for head-on collisions. The $f/2$ OAP has a 21-mm-diameter hole in the centre to allow the electrons, laser light and radiation to propagate through. A plastic ring of outer radius $28\,\mathrm{mm}$ and inner radius $11\,\mathrm{mm}$ is fitted around the hole to protect the optic and the laser chain upstream from potentially in the plasma scattered or strongly defocused driver laser light. This reduces the on-target intensity by 16 percent assuming a perfect top-hat laser profile. To protect the optic from potential debris a thin plastic layer with anti-reflective coating (`pellicle') and a suitable hole is attached to the mount of the OAP. The on-shot energy of the laser is after the compressor was $9.73\pm 0.15\,\mathrm{J}$ and $8.17\pm0.13\,\mathrm{J}$ on-target due to the hole in the OAP. In the dataset presented, the beam size is defocused to a spot of about $400\pm50\,\mathrm{\mu m}$ diameter, which translates into $a_0 \sim 0.28 \pm 0.03$ at the interaction. The laser is linearly polarised in the vertical plane, cross-polarised to the wakefield driver beam.

\subsubsection{Two-beam timing}

Both laser beams are synchronised to each other using spatial interferometry \cite{Cole2018_RR}. A 90-degree dielectric knife-edge prism (Thorlabs MRAK25-E03) was inserted at the interaction point, deflecting both beams collinearly onto the CCD chip of a camera (AVT Manta G-033B) equipped with a X10 long working distance infinity-corrected microscope objective. Due to the different radii of curvature of both beams, especially near the focus of the $f/2$ beam, a circular interference pattern emerges when the beams are overlapped in space and time. The timing procedure is explained in more detail in the \nameref{Chap:Methods}.

\subsubsection{Particle and Radiation diagnostics}

The radiation, the remaining driver laser and the electrons propagate through the hole in the $f/2$ OAP into a large aperture ($10\,\mathrm{cm} \times 30\,\mathrm{cm}$) permanent dipole magnet\footnote{designed, and assembly and positioning supervised by Dominik Hollatz (Jena)} of integrated magnetic field strength $\int B(x) \mathrm{d}x = 0.35\,\mathrm{Tm}$. 

The electrons are dispersed in the horizontal plane and electrons of energy lower than 220 MeV collide with the yoke of the magnet resulting in an effective low-energy cut-off. The dispersed beam leaves the vacuum chamber through a two-layer wide-aperture vacuum window\footnote{designed and tested by the Mechanical Engineering Division at the CLF, in particular Daniel Treverrow.} of dimensions $580 \,\mathrm{mm} \,\times\, 70\,\mathrm{mm}$ (horizontal $\times$ vertical). The layer facing the vacuum consists of $25\,\mathrm{\mu m}$ of Kapton, the outside layer is $375\,\mathrm{\mu m}$ of Kevlar providing additional mechanical stability and fibre support that acts as fail-safe. A scintillating sensitive Lanex (Biomax) screen placed just after the window, at $1.61\,\mathrm{m}$ distance downstream of the interaction point, measures the spectrum of the dispersed electron beam. The screen also extends beyond the laser axis. The light and thin material of the window and the short distance to the Lanex screen minimises the impact of small-angle scattering on the measurement (REF).
A second standard Lanex screen measures the spectrum $700\,\mathrm{mm}$ further downstream ($2.31\,\mathrm{m}$ from the interaction point) and can in conjunction with the first screen be used to account for the pointing of the beam \cite{Soloviev2011_TWOSCREEN}.
The screens are each imaged by a cooled 16-bit CCD camera (Andor Neo) equipped with suitable objective and bandpass filter. %A third camera images a small region of interest on the first screen.
\vspace{\baselineskip}

High-Z metal converter targets (bismuth, tungsten) fixed to a $1.6\,\mathrm{mm}$ plastic (PTFE) base are mounted on a motorised linear stage between the $f/2$ OAP mount and the dipole magnet, and can be driven into the beam path to intercept the electron beam and to produce gamma radiation from bremsstrahlung \cite{Glinec2005_Brems}. This is used to calibrate the gamma-ray diagnostics in this experiment \cite{Behm2018_Gamma}. The high-Z targets are expected to enable efficient conversion of the electron beam energy into radiation, but does not allow measuring the electron spectrum at the same time. The PTFE base itself can also be used as converter that is less efficient but allows a synchronous measurement of the electron and gamma-ray beam. 

Radiation traverses the magnet on the laser axis, then passes through a $120\,\mathrm{\mu m}$ aluminium laser beam block and the Kevlar-Kapton window mentioned before. Bright radiation in the right bandwidth is also captured by the Lanex screen as it extends beyond the axis. A $45 \times 45$ array of $1\,\mathrm{mm} \times 1\,\mathrm{mm} \times 10 \,\mathrm{mm}$ scintillating with thallium doped caesium-iodide (CsI:Tl) crystals coated in $TiO_2$  measures the profile of the radiation $700 \pm 1\,\mathrm{mm}$ downstream from the plane of the first Lanex screen. The stack covers a field of view corresponding to a cone of half opening angle $11.7\,\mathrm{mrad}$. The spatial resolution including the coating separating the individual crystals is then $1.2\,\mathrm{mm}/2.31\,\mathrm{m} = 0.52\,\mathrm{mrad}$. %A scintillation signal produced by radiation on the first Lanex screen and the CsI profile screen can be used as on-shot reference for the laser axis for the electron spectrometer screens. 
$704 \pm 1\,\mathrm{mm}$ further downstream from the profile screen another elongated array of caesium-iodide crystals is positioned to measure the spectrum of the gamma radiation. Both scintillator arrays, the profile screen and the spectrometer, are imaged using cooled 14-bit EMCCD cameras (Andor iXons). The diagnostics are described in more detail in the \nameref{Chap:Methods} section.

\iffalse
The aperture of the magnet permits a field-of-view of $79\,\mathrm{mrad}$ (10cm/1.26m) and is hence less limiting than the OAP hole. On the laser axis a beam block of $10-12$ layers of standard kitchen aluminium foil of thickness $10.1 \pm 0.2 \,\mathrm{\mu m}$ each is attached to the Kevlar-Kapton window. Hard X-rays continue to propagate through the aluminium beam block and the vacuum window (permitting a FOV of $70/1.61 = 43.4 mrad$ , the LANEX screen that also covers the axis and are then after $700\,\mathrm{mm}$ incident on a $45 \times 45$ array of $1\,\mathrm{mm} \times 1\,\mathrm{mm} \times 10 \,\mathrm{mm}$ scintillating caesium-iodide crystals coated in $TiO_2$ that record the beam profile of the radiation burst. The top part of the stack is centred on the beam axis. The total stack covers $54.2/2.31 = 23.4\,\mathrm{mrad}$. A second $30 \times 30$ stack of  $1.5\,\mathrm{mm} \times 1.5\,\mathrm{mm} \times 10 \,\mathrm{mm}$ CsI crystals is placed on top of the other one, covering another $51.2\,\mathrm{mm}$ or $22.2\,\mathrm{mrad}$. Harder radiation propagates another $704\,\mathrm{mm}$ before hitting another array of caesium-idodide crystals (see Methods for dual axis spectrometer) with an aluminium front-plate that are arranged in longitudinal direction to measure the spectrum of the radiation. The front part of the spectrometer then catches $XX/3m$ FOV. The emission of the caesium-iodide crystals is imaged by Andor iXon cameras.
\fi

\section{Characterisation of Electron Spectra}



The electron beams from laser wakefield acceleration were in this experiment injected by density perturbations in the supersonic gas flow by introducing a steel blade. The spectrum is measured by scintillating Lanex screens as part of a magnetic spectrometer setup. The image processing of the data is not further elaborated here but can be found for instance in \cite{ColeThesis,PoderThesis} or the \nameref{Chap:Methods} of this thesis. For now we will only consider the first Lanex screen and ignore variations in energy due to global pointing of the beam as measurable with the second Lanex screen.

We consider two datasets: one is a series of shots at fixed conditions to investigate fluctuations in the accelerator performance. These electron beams were collided with the laser pulse and will in the following analysed in more detail for ICS.
The second dataset is a scan of the longitudinal position of the blade (z-scan) to analyse the effect of the shock position on the energy spectrum. This dataset is more of interest to understand the dynamics and behaviour of the accelerator in this setup.

\subsection{Fixed conditions}

The first dataset consists of 386 shots with electron beams, taken at constant backing pressure of the gas jet, fixed position of the blade (1.2 mm into the gas flow from the entry point of the driver beam). Despite the constant conditions, the properties of the electron beams vary strongly from shot-to-shot and produce beams of a wide range of shapes, maximum energy and energy spread. A few examples of electron spectra from the relevant dataset are provided in Figure \ref{linICS:Figs:Elec_example}, a waterfall plot of the integrated lineouts are shown in Figure \ref{linICS:Figs:fixed_waterfall}. 
The spectral length of the beams varies strongly with some starting below the measurement threshold of 220 MeV with significant amount of charge all the way up to 1 GeV, stretching 800 MeV of spectral range (see for instance the two panels on the left and right in Figure \ref{linICS:Figs:Elec_example}). These spectrally very wide electron beams show signatures of strong transverse oscillations which indicate potential to act as a bright betatron source. In other instances beams with narrow energy spread around 1 GeV are measured (central panels in Figure \ref{linICS:Figs:Elec_example}). 
For now we ignore potential fluctuations in the retrieved energy due to variations in the beam pointing and assume that the divergence of the beam in the dispersion direction is negligible.

\begin{figure}
\centering
\includegraphics[trim={4.0cm 0 5cm 0}, clip, width=1.0\columnwidth]{Example_Montage_twoSets.png}
\caption[Waterfall plot for electron spectra at fixed conditions.]{Waterfall plot of electron spectra in this dataset.}
\label{linICS:Figs:fixed_waterfall}
\end{figure}


\begin{figure}
\centering
\includegraphics[trim={4.8cm 0 5cm 0}, clip, width=1.0\columnwidth]{Example_Espec_CollisionMontage.png}
\caption[Examples of electron spectra measured in the experiment.]{Examples of electron spectra measured in the experiment. The y-axis is the dispersion axis and shows the energy, the x-axis indicates the divergence. The colour scale indicates the amount of charge in the beams and is fixed for all plots.}
\label{linICS:Figs:Elec_example}
\end{figure}

The striking variability of the electrons produced in this dataset at seemingly fixed experiment conditions are not consistent with experimental results reported from other LWFA experiments at other laser systems using shock injection (REF). These setups typically provide reproducible electron beams with narrow energy spread. 
\vspace{\baselineskip}

\begin{figure}
\centering
\includegraphics[trim={6cm 0 6cm 0}, clip, width=1.0\columnwidth]{linICS_ElectronProperties_Histograms.png}
\caption[Histogram of fluctuations in the electron beam properties in this dataset.]{Histograms showing the fluctuations of electron properties in course of this dataset (386 shots). Top left: Total charge of the electron beams measured from 220 MeV upwards. Top right: Maximum electron energy, defined as energy at which the spectral intensity falls to 10 percent of its peak value. Bottom left: FWHM Vertical divergence (non-dispersion axis) in mrad. Bottom right: Beam pointing fluctuation in mrad from the mean position. }
\label{linICS:Figs:Elec_histogram}
\end{figure}




The key properties of the electron beams in this dataset are summarised in Figure \ref{linICS:Figs:Elec_histogram} in form of histograms showing charge, maximum electron energy, vertical divergence and beam pointing.
The maximum energy of the electron beam, here defined as the energy when the spectral intensity reaches 10 percent of its peak value, was measured to be $944\pm139\,\mathrm{MeV}$, with some shots reaching up to 1.3 GeV. The charge of the beam varied in particular due to the varying spectral range of the bunch with a mean of $239 \pm 92\,\mathrm{pC}$.
The divergence of the beams was measured to be $2.7\pm 2.5\,\mathrm{mrad}$ and the vertical position of the beam centroid was fluctuating by $2.2\,\mathrm{mrad}$.
\vspace{\baselineskip}

The wide range of electron beams produced in this setup is interesting in the context of linear ICS. Since the properties of the beams this accelerator is able to produce vary strongly, the radiation they produce in a well-defined linear ICS interaction will also vary significantly. This means that this accelerator is in principle also able to produce a wide range of radiation spectra from broadband to strongly peaked radiation. 

\subsection{Blade z-scan}

An explanation of this behaviour is beyond the scope of this thesis and is not being developed within this work, but will be addressed in the future (SEE TALK EAAC).

Change in shock position changes the maximum energy.

From this we can get the accelerating gradient in the field.

Shock angle can be estimated. Since the shock position does not move the same distance as the blade the shock tilt has to change.
\begin{figure}
\centering
\includegraphics[width=0.8\columnwidth]{ShockMontage_bladez.jpg}
\caption[z-scan positions.]{z-scan for positions.}
\end{figure}

\begin{figure}
\centering
\includegraphics[width=0.5\columnwidth]{Bladezscan_Shock_Corr.png}\includegraphics[width=0.5\columnwidth]{Bladezscan_Shock_Angle.png}
\caption[z-scan positions.]{z-scan for positions.}
\end{figure}

\begin{figure}
\centering
\includegraphics[trim={4.0cm 0 5cm 0}, clip, width=1.0\columnwidth]{Example_EspecMontage_bladezscan.png}
\caption[z-scan positions.]{z-scan for positions.}
\end{figure}


\begin{figure}
\centering
\includegraphics[trim={0.0cm 0 0cm 0}, clip, width=1.0\columnwidth]{linICS_bladeZ_EnergyChargeCorrelation.png}

\includegraphics[trim={0.0cm 0 0cm 0}, clip, width=1.0\columnwidth]{linICS_bladeZ_EnergyCorrelation.png}
\caption[Waterfall plot for electron spectra at fixed conditions.]{Waterfall plot of electron spectra in this dataset.}
\end{figure}


\section{Gamma spectra from linear inverse Compton scattering}


A photon of energy $E_{ph}$ that is scattered from a relativistic electron is Doppler up-shifted due to the relativistic Lorentz boost. After an interaction shifts to an energy $E_{X}$ given by (REF):
\begin{equation}
E_{X} = E_{ph}\frac{2(1-\beta\cos \theta)\gamma^2}{1+a^2_0/2 + \gamma^2 \theta^2_0} = \frac{4\gamma^2}{1+a^2_0/2 + \gamma^2 \theta^2_0},
\label{linICS:eqns:full_linICS}
\end{equation}
where $\beta$ is XXX, $\theta$ the angle between the electron and the incoming photon, $a_0$ is the normalised vector potential and $\theta_0$ is XX.

The electron beams shown earlier (first dataset) were collided with a laser pulse at an intensity of $a_0 < 0.3$. This is sufficiently below $a_0 = 1$ such that we can ignore the production of higher harmonic radiation and only consider the fundamental harmonic from linear ICS (REF). Since also $a^2_0 \ll 1$ we can also ignore the term $a^2_0/2$ in the denominator, which accounts for red-shifting of the radiation when the longitudinal component of the electron motion becomes significant (see \nameref{Chap:Theory} for `figure-of-eight motion'). The electron beams are in most cases confined to a divergence cone of few milliradians. For a $4\,\mathrm{mrad}$ beam the difference introduced by the electron angles in the $\cos\theta$ term assuming a collimated light source is less than $0.001\%$, which means that we can ignore a broadening of the spectrum from electron angles as well.

For a $f/2$ beam the angle between the outer rays is up to 14.04 degrees which corresponds to a 3 percent spread in energy. Since we are only interacting with a small solid angle of the beam (few microns of 400) the rays are approximately collinear and we can ignore the this factor as well.

\begin{figure}
\centering
\includegraphics[trim={4.8cm 0 5cm 0}, clip, width=.5\columnwidth]{Egamma_ICS.png}
\caption[Gamma energy produced in scattering a 1.55 eV photon with a relativistic electron beam in a head-on collision.]{Gamma energy produced in scattering a 1.55 eV photon with a relativistic electron beam in a head-on collision.}
\end{figure}

Linear ICS theory predicts that in a head-on collision scattered photons will be emitted in a narrow cone of divergence $\sim 1/\gamma$, such that we will also ignore off-axis contributions.

Combining these assumptions and considering a head-on collision ($\theta = 180^\circ$) Equation \eqref{linICS:eqns:full_linICS} above simplifies to

\begin{equation}
E_{ph}' =4\gamma^2 E_{ph}.
\end{equation}

For $\gamma \sim 1750$ and $a_0 = 0.3$

\begin{equation}
\psi = \gamma^2 a_0 \frac{2 r_e \omega}{3c} \ll 1,
\end{equation}
which indicates that radiation reaction effects are negligible and the interaction with the laser pulse does not perturb the electron beam significantly (REF ALEC THOMAS AND FELICIE).

For a Ti:Sa system at central wavelength 800 nm, as used at Gemini, the corresponding energy carried by an individual photon amounts to 1.55 eV. Based on the simplified Equation above and also ignoring the bandwidth of the laser, we then obtain a quadratic one-to-one mapping of the electron energy to a corresponding gamma-ray energy (see Figure \ref{linICS:fig:ElecEToEgamma}). 
\vspace{\baselineskip}

Since the properties of the electron beam measured in this experiment vary strongly, we also expect the spectrum of the radiation produced from linear ICS to follow this behaviour. In this scenario the cross section is independent of the electron energy (REF) and the shape of the electron beam is hence preserved in the gamma spectrum. Two examples of electron spectra and the with Equation XX calculated corresponding gamma-ray spectra are shown in Figure XX. One electron beam is spectrally very broad with significant spectral intensity spread from 200 to 1000 MeV, resulting in a gamma-ray spectrum from few to XX MeV. The other beam is strongly peaked at 1.2 GeV with narrow energy spread, which has the potential to be converted into a narrow energy spread gamma-ray source around 35 MeV.


\begin{figure}
\centering
\includegraphics[trim={4.8cm 0 5cm 0}, clip, width=1.0\columnwidth]{Example_Espec_ICS.png}
\caption[Example of an electron spectrum with corresponding calculated ICS spectrum.]{Example of an electron spectrum. The to the electron energies corresponding gamma-ray energies are indicated in a second x-axis on the top.}
\label{linICS:fig:ElecEToEgamma}
\end{figure}

\section{Number of photons scattered}

Based on the conditions in the experiment, we can assume that the interaction is correctly described by linear ICS theory (REF).

In this case the number of photons scattered in an interaction or the number of produced gamma-rays, $N_X$, is simply given by the Thomson scattering cross-section (REF Felicie and Alec's paper)

\begin{equation}
\boxed{N_X = \frac{\sigma_T}{\pi w^2_0} N_L N_e,}
\end{equation}
where $\sigma_T = 6.65 \times 10^{-25}\,\mathrm{m}^{-2}$ the Thomson scattering cross-section, $N_e$ the number of electrons involved, $N_L$ the number of laser photons and $w_0$ the waist size. The cross-section and the total number of photons produced is independent of the electron and photon energy.

The Rayleigh length, $z_R$, of an $f/2$ beam is approximately $2.5 \lambda f^2_\# \approx 8\,\mathrm{\mu m}$.
At a diameter of $400\,\mu m$ the beam is at a distance $\sim800\mu m$ from the the focal plane, which corresponds to $\sim 100 z_R$. At this plane we can assume that we are in the near-field of the laser. As discussed before we ignore relative angles between the laser and the electron beam. We will also ignore focusing effect assuming that the duration of the interaction is short enough to avoid significant changes in intensity such that the electrons encounter a static photon field. Since the laser beam is much larger than the electron beam, we assume that the entire electron beam interacts and that all electrons encounter the same photon density, hence also contributes equally to the total radiation produced as the cross-section is constant across the energies.
Assuming a homogeneous perfect flat-top laser profile we estimate the photon density using the on-shot energy measurement (without the hole) and the laser beam size:

\begin{align}
N_X &= \frac{\sigma_T}{\pi w^2_0} &\left[\frac{E_J}{E_{ph}} \frac{\pi w^2_0}{\pi r^2}\right] &\left[\frac{Q}{e}\right],\\ \nonumber
&= \frac{\sigma_T}{\pi r^2}&\left[ \frac{E}{E_{ph}}\right]& \left[ \frac{Q}{e}\right],
\end{align}
where $E_J$ is the total laser energy, $E_{ph}$ the energy of a single photon and $r$ the radius of the beam at the plane of the interaction. The region of the interaction defined by the electron beam size $w_0$ cancels out as the photon density is constant across any region in this approximation.

In useful units we can write
\begin{equation}
\boxed{N_X = 5.32 \times 10^{10}\times  E [J] \times Q [100pC] \times \left(r[\mu m]\right)^{-2}.}
\end{equation}

Using this Equation we estimate for the conditions in the experiment ($E_J = 9.73\pm0.15$,$Q=239\pm92pC$ and $r=200um$) that $(1.3 \pm 0.5) \times 10^{7}$ photons are scattered. This is of similar order of magnitude as reported for bright betatron sources (REF REF Chen and Kneip), with a comparable pulse duration and source size. From our previous analysis we, however, expect to emit 1000-times higher photon energies due to the Lorentz boost. We expect that we are able to measure this order of magnitude of photons in this energy range.
\vspace{\baselineskip}

However, considering the experimental parameters it becomes clear that a linear Compton source can be achieved in a much more economical way by scaling down the energy and the beam size at the interaction whilst still preserving a permanent overlap within the shot-to-shot fluctuations of the laser and the accelerator. We can achieve the same number of scattered photons by keeping the photon density fixed. Moving from a beam radius of $200$ to $25um$, a factor of 8, the corresponding energy in the scattering beam could be reduced by a factor of 64, so around $150mJ$. Using an $f/2$ optic as before, this plane would be more than 10 Rayleigh lengths away from the focal plane and we might still be in the near-field of the laser, where the profile is flat and smooth, such that any change in response reflects the electron beam property. At the same time this keeps the beam large enough to interact at all times with the micron sized electron beam also considering relative pointing fluctuations. 
The beam size will also be large enough to minimise the impact of angle variations on the source parameters.

\section{Energy-dependent response}

So far we calculated the relation of electron energy to gamma energy. We also estimated the number of photons produced in an interaction. The cross-section for the scattering process is independent of electron and gamma energy.

However, whilst the cross-section for an interaction is independent of energy, the detected yield will not be constant across the energy bands and there will be a difference in the yield measured by a physical detector and the energy emitted by the source. We will use the gamma profile screen to measure the yield of the radiation. It is not a calorimeter and only absorbs a fraction of the total radiation, with the energy deposited per photon being energy-dependent. Since there is a range of mechanisms coming into play at the tens of MeV photon energy range, the energy deposition was simulated using GEANT4 for a range of photon energies. The average energy deposition as simulated for one individual photon in the profile stack is shown in Figure XX as a function of photon energy. Between $E_\gamma=1$ and 15 MeV the energy deposition increases steadily before slowing down but still increasing by 70 percent over the remaining energy range. This is important when we want to estimate the total number of photons from a measurement in the experiment, as we have to assume a spectral shape and scale it accordingly.

This is similar behaviour as usually cited for Lanex, where few MeV particles deposit most of their energy whereas at relativistic energies the deposition is almost flat. A variable energy deposition means that a profile screen has different properties.

Using the simulated result as a lookup-table for energy deposition and relating the photon energy to the electron energy, we can calculate the total energy deposition for a given electron spectrum as follows:

\begin{equation}
\boxed{\langle E_{dep} (\gamma) \rangle N_X = \int E_{dep} (\gamma) \frac{\sigma}{\pi \omega^2_0} N_L \frac{\mathrm{d}N_e}{\mathrm{d}\gamma}\mathrm{d}\gamma,}
\end{equation}
which becomes in particular important when the electron spectrum covers a wide energy range as it does in this case.

\begin{figure}
\centering
\includegraphics[trim={4.8cm 0 5cm 0}, clip, width=0.5\columnwidth]{Egamma_ICS.png}\includegraphics[trim={4.8cm 0 5cm 0}, clip, width=0.5\columnwidth]{Edep_Jena_1_100_MeV.png}

\includegraphics[trim={4.8cm 0 5cm 0}, clip, width=1.0\columnwidth]{Example_Espec_ICS_Edep.png}
\caption[Energy-dependent yield on detector.]{Left: Radiation produced in head-on ICS collision for different electron energies. Right: Photon energy deposition per photon in stack. Bottom: Example electron spectrum along with gamma-ray energy on a relative scale. Comparison with relative contribution to the measured yield.}
\end{figure}

We calibrate the yield of the detector with a radioactive source using caesium 137 (Cs137). Based on the Equation above and this calibration we estimate $2.6 \pm 1.4 \times 10^{7}$ photons above 1 MeV for electrons above 220 MeV (what energy range, which spectrum).

This also means that different parts of the beam contribute less or more to the measured profile. We see in Figure XX an example electron spectrum, its related gamma spectrum and the relative contributions in terms of total energy deposition. This means that the response of a detector will in this case be dominated by the high energy peak and the lower energy electrons and their radiation contribution is suppressed. If one is mainly interested in the properties of the high-energy electrons this is of advantage, as one would be in radiation reaction studies.

\section{Spectral Measurement in Experiment}


... show the profile screen and gamma spec raw data


To measure the spectrum a CsI stack was used. The simulation of the detector response and the experimental calibration was done as described in Methods XX. In this case the spectral fitting is performed by assuming a 1-to-1 mapping of the spectra. As a comparison a mono-energetic photon is used to see a similar response.

... use the linear spectra for the examples to fit to the response

... show that mono-energetic is also a good fit, so the response is dominated by the highest counts of photons, so without knowing the input is difficult

... show for a spectrum the relative contribution to the yield.

\begin{figure}
\centering
\includegraphics[trim={4.8cm 0 5cm 0}, clip, width=1.0\columnwidth]{GammaProfile_Espec_20190211r006s106.png}

\includegraphics[trim={4.8cm 0 5cm 0}, clip, width=0.5\columnwidth]{GammaSpec_Side_Fit_20190211r006s106.png}
\caption{Fitted mono-energetic response: 18.5 MeV or 15.5 MeV. Average energy 15.9 MeV.}
\end{figure}

\begin{figure}
\centering
\includegraphics[trim={4.8cm 0 5cm 0}, clip, width=1.0\columnwidth]{GammaProfile_Espec_20190208r015s037.png}

\includegraphics[trim={4.8cm 0 5cm 0}, clip, width=0.5\columnwidth]{GammaSpec_Side_Fit_20190208r015s037.png}
\caption{Fitted mono-energetic response: 12.5 MeV. Average energy 12.1 MeV}
\end{figure}


\begin{figure}
\centering
\includegraphics[trim={4.8cm 0 5cm 0}, clip, width=1.0\columnwidth]{GammaProfile_Espec_20190211r006s107.png}

\includegraphics[trim={4.8cm 0 5cm 0}, clip, width=0.5\columnwidth]{GammaSpec_Side_Fit_20190211r006s107.png}
\caption{Fitted mono-energetic response: 11 MeV. Average energy 11.4 MeV}
\end{figure}


\subsubsection{Nonlinearity}


For this purpose, now for a range of $a_0$ show what the impact of nonlinearity would be in terms of broadening, redshift, harmonic.
Show an example of linear with weakly and moderately non-linear (based on redshift and harmonics)
... try to perturb with nonlinearity and see whether it finds better, give estimate of $a_0$ (not good)
.. try to see when it would start to become visible and estimate an intensity (this could be used as intensity estimate, which it will be used for in the other section).

... use this in radiation reaction studies to move from linear to non-linear (before ellipticity becomes important)



\begin{figure}
\centering
\includegraphics[width=1.0\columnwidth]{PlaceHolder_perturbedResponse.png}
\caption{This is to indicate change in response.}
\end{figure}




\section{ICS as electron beam diagnostic}

Since the mechanism of linear ICS is understood very well in this regime, we can attempt using it to diagnose properties of the electron beam.

The laser pulse is interacting with the electron beam close to normal. Since the cross-section is energy-independent, we can assume that each part of the beam contributes a comparable number of photons to the total radiation emitted, under the caveat that the measured yield might vary due to the change in energy. This means this could act as non-invasive beam profile screen. In LWFA experiments a Lanex screen or different scintillator is used to measure the divergence and beam pointing of the undispersed electron beam. Placing a screen in the beam path, however, leads to small angle scattering of the electron beam and blurs out the beam, in particular for longer propagation distances. A different method is using a wire as position monitor in the beam path. This mainly gives a general pointing reference and when moved transversely can be used as measurement of the beam size at that plane, requiring shot-to-shot stability. This has also been done using linear ICS with a tightly focused laser pulse, also referred to as laser wire, as a sufficiently high photon density is required for a decent signal.

We present a single-shot on-shot beam profile measurement.

We will investigate how well spatial features in the vertical non-dispersed axis on the electron spectrometer screen relates to the measurement.
This also includes divergence.
We then use this method to investigate the dispersed axis.


\subsection{2D divergence measurements}






Using the gamma profile screen we can measure the divergence of the gamma beam. Radiation from linear ICS is emitted in a narrow cone of divergence $1/\gamma \sqrt{N_0}$. Since the electron beam also has its own divergence the measured divergence is an addition of both. 

\begin{equation}
\theta_f = \theta_e + 1/(\gamma N^{1/2}_0)
\end{equation}
with laser oscillations $N_0 = 10$.
Different parts of the beam contribute differently to the total gamma profile signal, but we will ignore this for now. To start with we will also assume that we have a constant $\gamma$ as changing the energy from 500 MeV to 1000 MeV would change the divergence from 0.32 to 0.16 mrad, so this is a small factor especially given the spatial resolution is of order 0.52 mrad. 
In Figure XX we compare the divergence of electron beam in non-dispersion direction (vertical) with the corresponding FWHM divergence measured with the gamma profile screen. We see that with some scattering the data follows a linear relationship. The slope of the fitted line is 1, which indicates that this is the electron component and our spatial calibrations work out well. The deviations from the line are then related to energy fluctuations, spatial resolution and measurement errors, for instance.The values are scattered around the fit parameter which could be a combination of change in energy (ESTIMATE ERRORS 0.2 mrad), FWHM fit error XX, and limited spatial resolution. The standard deviation from the fit is 0.81 mrad which is just around what the spatial resolution and the change of energy would provide. 

\begin{figure}
\centering
\includegraphics[width=1.0\columnwidth]{linICS_FWHM_Y_Espec_GP_2.png}
\caption{Clear correlation of FWHM at correlation coefficient 0.63.}
\end{figure}

The y-axis contribution is 1.4 mrad require 250 MeV and times 2, which is not the case in all situations, so not sure where this is coming from.
\vspace{\baselineskip}

Independently of the origin, we can use this to estimate the electron divergence in the dispersion-axis from this fit, and solving the previous equation:

\begin{equation}
\theta_e = \theta_\gamma - C
\end{equation}

For the same dataset this gives us an electron divergence of $2.2 \pm 1.4\,\mathrm{mrad}$ in the dispersion direction. In comparison it is $2.2 \pm 0.6\,\mathrm{mrad}$ in the other axis, which is very similar. 

This induces an error bar on the energy measurement, but on average the assumption that the divergence is symmetric would be close.

There are also examples where the electron beam has a high ellipticity and one could either over- or underestimate the properties.

By dividing the electron beam divergences, the calculated and the measured quantity, we obtain a value for the ellipticity.
It is $1.03\pm 0.5$. Add a histogram for this. Which means that on average the value is 1 but there are oscillations.


\begin{figure}
\centering
\includegraphics[trim={4.8cm 0 5cm 0}, clip, width=0.9\columnwidth]{linICS_ElectronFWHM_Ellipticity.png}
\caption[]{Ellipticity (X over Y).}
\end{figure}


Ellipticity is not perfectly related due to measurement errors but also due to using a constant offset for all despite varying energy. There is also indication that not everything is interacting at all times, so the real divergence might be underestimated.

This is also of interest as we know from PUBLICATION A0 (WITH OR WITHOUT TOM BLACKBURN) that we can infer a0 from the ellipticity of the gamma signal. If we can properly characterise the fluctuations in the pointing and ellipticity, we can also be reassured about which portion is due to source size and which is due to intensity. It is important to measure the divergence in two axes for this application.


\subsection{Beam Pointing}

The FWHM spatial features are well resolved and the spatial calibration is likely to be right. Somehow in general the pointing seems to be non-correlated. This is odd as the FWHM matches and indicates that we interact with the entire beam. There could be a slight different weighting of the energy deposition. Or the pointing changes throughout (if the interaction is inside within the bubble), but then FWHM would change also.

\begin{figure}
\centering
\includegraphics[trim={4.8cm 0 5cm 0}, clip, width=0.9\columnwidth]{linICS_Espec_GP_Pointing.png}
\caption[]{Pointing.}
\end{figure}

Beam pointing only correlated by 0.15, which is only very weakly. Maybe this is related to the inverted spectra as well. Looking at shadowgraphy, we are likely to interact within the plasma and hence the beam is developing and oscillations lead to inversion?



\subsection{Resolving spatial features}

WOULD HAVE TO SCAN TO GET THE SPATIAL COMPONENT TO THEN GET EMITTANCE.

Since it does not perturb the electron beam and only interacts with some electrons due to the low photon density, ICS has been used previously as beam diagnostic as laser wire in recognition of a physical wire.
Due to the Lorentz boost radiation in ICS is emitted in a narrow forward cone and is hence dominated by the bunch itself. This means a head-on collision is acting similarly to an electron profile screen and represents a longitudinal integration of the transverse shape. In contrast to a profile lanex screen this does not scatter the beam (or only a small fraction). This could be used as pointing reference whenever the beam is expanded and an overlap is guaranteed.


\begin{figure}
\centering
\includegraphics[trim={4.8cm 0 5cm 0}, clip, width=1.0\columnwidth]{linICS_LaserWire_Examples_2.png}
\caption{Laser Wire example Espec and Gamma Profile for off-axis charge.}
\end{figure}

\begin{figure}
\centering
\includegraphics[trim={4.8cm 0 5cm 0}, clip, width=0.9\columnwidth]{linICS_LaserWire_Examples_Ring.png}
\caption[Examples of gamma profile measurements and corresponding electron spectra.]{Examples of gamma profile measurements and corresponding electron spectra.}
\end{figure}

For lower energy spread we can ignore change in response of the detector to higher energies.

\begin{figure}
\centering
\includegraphics[trim={4.8cm 0 5cm 0}, clip, width=1.0\columnwidth]{linICS_LaserWire_Examples_Mono.png}

\includegraphics[trim={4.8cm 0 5cm 0}, clip, width=0.9\columnwidth]{linICS_Example_ICSEdepElecProfile_Intensity.png}
\caption[]{Intensity check.}
\end{figure}


\begin{figure}
\centering
\includegraphics[trim={4.8cm 0 5cm 0}, clip, width=1.0\columnwidth]{linICS_LaserWire_Examples_Tilt.png}

\includegraphics[trim={4.8cm 0 5cm 0}, clip, width=0.9\columnwidth]{linICS_Example_ICSEdepElecProfile_Intensity_2.png}
\caption[]{Intensity check.}
\end{figure}


\begin{figure}
\centering
\includegraphics[trim={4.8cm 0 5cm 0}, clip, width=1.0\columnwidth]{linICS_LaserWire_Examples_Wing.png}

\includegraphics[trim={4.8cm 0 5cm 0}, clip, width=0.9\columnwidth]{linICS_Example_ICSEdepElecProfile_Intensity_3.png}
\caption[]{Intensity check.}
\end{figure}


We see that a beam with large oscillations shows up as elongated profile which implies that it does not oscillate much in the other direction.

We see that a slightly bent beam shows up as tilted ellipse which indicates that the kink in the beam is in two dimensions.

These examples show that it can reveal, given enough energy and homogeneous illumination and spatial resolution, some information about the properties of the electron beam beyond the dispersion axis without perturbation.


... estimate what kind of intensity we would require if used the beam more economically.


How much does this change in terms of energy deposition per electron or charge over 100 MeV or so?
What is the `contrast' of this method?

PROPOSE HOW MUCH SMALLER WE COULD GO IN TERMS OF ENERGY AND SPOT SIZE AND WHAT KIND OF LASER COULD BE USE NOT TO SCAN BUT TO DO SINGLE-SHOT MEASUREMENTS.

MAYBE USE LONG FOCUSING OR THAT REDUCES ACCURACY IN TERMS OF FOCAL POINT.



\begin{figure}
\centering
\includegraphics[trim={4.8cm 0 5cm 0}, clip, width=1.0\columnwidth]{GammaProfile_Espec_20190208r015s092.png}
\caption[]{Another spatial example.}
\end{figure}




\subsubsection{Matching up features}

Trying to match up intensities. Assume that the divergences have to match up (scale x-axis accordingly and y-axis).

Using something with two features. Seeing that this does not seem to line up, there is a brighter side-lobe for some reason, also more distinct. This must be due to the crystal structure, even Gaussian filter makes this more distinct. Can average this out. Maybe variation in intensity. Using this example as the energies are all comparable. Increase in intensity by 50 percent, which would still mean $a_0 < 0.4$. This would realistically only be possible if the electron beam is expanded such that it can see variations in intensity.

\EliasComm{Intensity profile?}

\subsubsection{Explaining inversion}


Potential reasons:

... Fluctuations in intensity. Use some examples and show how much more intensity is required to make up for this change in intensity

... Since interacting in plasma the beams have some momentum, which then indicates which direction the radiation goes. Draw out the options: would be spread from outside to inside, but is it concentrated only in one angular direction. Also estimate transverse momentum.

... the beam is not perturbing the plasma much and we have half a betatron wavelength behind us to flip sides in any case. Estimate ranges and provide inversion and non-inversion samples.


Betatron wavelength:

\begin{equation}
\lambda_\beta = \lambda_p \sqrt{2\gamma}
\end{equation}

with plasma wavelength

\begin{equation}
\lambda_p = n_e ...
\end{equation}

The angle of the electrons is determined by the position in the oscillation. The largest component is going in z direction. There is a smaller transverse component. At the return point it is parallel (so the emission of the radiation should appear to be sent forward). In any other case there should be a component in positive or negative transverse direction. Naively if we assume a half betatron length of 800 um and a betatron radius of 1 micron, the ratio of $p_T$ and $p_z$ is 1:800, so we are talking about an average pointing of 1.25 mrad. This is a variation spatially of 2 mm. However, we are usually encountering at least four times that for the inversion. Do we have examples of other electron beams that are not doing that?



\subsubsection{Raster scan}

Use this to establish relative positioning.

Use this as preparation for non-linear ICS and RR.

Easier overlap, especially at shallow angles.

Use transverse diagnostics for position and then raster scan.

\section{Conclusion}