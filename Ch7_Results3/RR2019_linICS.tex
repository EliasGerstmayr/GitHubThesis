\chapter{Linear Inverse Compton Scattering at Astra Gemini}

\section{Motivation}

As outlined in the theory section a relativistic electron beam interacting with a laser pulse is an effective source of high energy radiation.
In the case of a very intense laser beam $a_0 \rightarrow 1$ the dominant process is non-linear inverse Compton scattering forming a synchrotron-like spectrum with hundreds of higher harmonics, redshifting orders, all convolved with the energy spread of the electron beam. Under these conditions the electrons lose a significant amount of energy and radiation reaction becomes significant.

In this context we will look at the low-intensity interaction being linear inverse Compton scattering producing a nearly monoenergetic response from each electron. In this scenario we are operating in the weakly non-linear regime where first harmonics become visible.

The gamma diagnostics will help us to characterise the spectrum, this time in two axes and hopefully also a shape-independent algorithm. The ability to diagnose a full 3D spectrum of the gamma ray burst and do this without assuming a shape is important. In a very linear regime the theory is well understood and an ideal test bed to trial these methods and the new diagnostics. The varying electron beams should result in a one to one mapping into the spectrum.
The varying shape and spreads should be picked up by the detector and could then in the future be used in high intensity interactions as another indicator (change of shape) for a model-dependent response.

\section{Experimental Setup}

This experiment was performed at the Astra Gemini facility in early 2019 using both laser arms of the dual laser beam facility.
\vspace{\baselineskip}


\EliasComm{Add sketch of experiment setup.}

\subsubsection{South beam and target}

The first laser pulse is focused with an f/40 off-axis parabola onto the edge of a 15 mm conical supersonic helium gas jet. The measured pulse duration was $61\,\mathrm{fs}$ \textsc{fwhm} with an average energy on-target of $12.5 \pm 0.2\,\mathrm{J}$, reaching a peak power of $195\,\mathrm{TW}$. The size of the focal spot was around $48.6 \times 39.2\,\mathrm{\mu m}$ \textsc{fwhm}, giving a peak normalised vector potential of $a_0 = 1.88$ in vacuum. The laser is polarised in the horizontal plane.
\vspace{\baselineskip}

The exit of the gas jet was positioned $\sim 14\,\mathrm{mm}$ below the beam axis. A steel razor blade is introduced into the gas flow $\sim 4\,\mathrm{mm}$ above the jet at $- 32.4\pm 0.3^\circ$ angle in vertical direction to produce a shock front. The electron density without the shock feature was $1.5 \times 10^{18}\,\mathrm{cm}^{-3}$.

The gas target was characterised on-shot by a transverse probe used as shadowgraphy and coupled to a Mach-Zehnder interferometry to determine the density. The beam was sourced from the transmission of the main beam through the first mirror, then re-collimated using an f/7 OAP and a lens to size XXX.
The pathways are matched and a delay slide gives independent control over the timing.

The gas target and recombination light of the plasma channel are also imaged by a Canon DSLR camera and a CCD chip from the top (Top View).

\subsubsection{North beam}

The second laser pulse is focused down tightly by an f/2 off-axis parabola (OAP) at the opposite of the gas jet in a geometry for head-on collisions. The f/2 OAP has a 21-mm-diameter hole in the centre to allow the electrons and radiation from ICS to propagate through. A round piece of plastic with radius $28\,\mathrm{mm}$ is also fitted around the hole to protect the optic from scattered light from the first laser. This reduces the on-target intensity by 16 degrees assuming a perfect top-hat laser profile. To protect the optic from potential debris a thin plastic layer with anti-reflective coating with a hole is attached to the OAP mount. The minimum spot size reached was about XX NUMBER, which translates into a peak normalised vector potential of $a_0$ ~ NUMBER. The laser is linearly polarised in the vertical plane. The beam can be defocused to larger spot sizes which makes an interaction less intense.

\subsubsection{Two-beam timing}

Both beams were timed to each other using spatial interferometry. A 90-degree knife-edge prism was inserted at the interaction point, deflecting both beams collinearly onto the CCD chip of a camera.

Two sets of timing measurements relying on spectral interferometry are also performed on-shot using transmitted beams, one in the laser area upstairs (before the beams enter the target area), and a second set within the chamber.

The in-chamber transmitted beams are also used as pointing references.

\subsubsection{Particle and Radiation diagnostics}

The radiation and the electrons propagate through the hole in the f/2 OAP into a large aperture ($\sim 10\,\mathrm{cm} \times$ XX ) permanent dipole magnet\footnote{designed by Dominik Hollatz (Jena)} of integrated magnet field $\int B(x) \mathrm{d}x = 0.35\,\mathrm{Tm}$. A linear translation stage carrying high-Z metal converter targets can be driven into the beam path to intercept the electron beam and to produce gamma radiation from bremsstrahlung. Targets were tungsten, bismuth on PTFE XX CHECK THESE BITS.
The electrons are dispersed spectrally in the horizontal plane and leave the vacuum chamber through a two-layer vacuum window\footnote{designed and tested by Mechanical Engineering CLF, in particular Daniel Treverrow.} of dimensions $580 \,\mathrm{mm} \times 70\,\mathrm{mm}$. The inside layer is $25\,\mathrm{\mu m}$ of kapton with $375\,\mathrm{\mu m}$ of kevlar to provide additional mechanical stability and a fibre layer that holds as failsafe.
A scintillating LANEX screen placed just after the window measures the spectrum with very little scattering from the window due to the short drift distance. A second LANEX screen measures the spectrum about 1 m further downstream and can be used to account for pointing of the beam.
The electron spectrum is cut off at around 200 MeV as lower energetic electrons hit the yoke of the magnet and do not make out of the magnet.
Both screens are imaged by Andor Neo cameras with a third camera looking at a smaller part of the first screen.
\vspace{\baselineskip}

The radiation propagates straight through the magnet and is damped by a 10-12 layer aluminium foil beam block (THICKNESS XX-XX) and the kevlar-kapton window. Hard and bright radiation is also captured by the first LANEX screen as it extends beyond the axis (aluminium on the back?). A scintillating stack of caesium-iodide crystals is measuring the profile of the radiation (two on top of each other? EXACT DIMENSIONS DESY AND JENA), another long stack of crystals can be used to retrieve the spectrum of the radiation (DUAL AXIS DIMENSIONS), both imaged by Andor iXon cameras. Both diagnostics are described in more detail in the Methods section and their GEANT modelling in the simulations chapter.


\section{Characterising Electron Spectra}

\subsubsection{General Shape}
The electron spectra were recorded by two LANEX screens using the large aperture dipole magnet as electron spectrometer. The first screen is positioned just outside the vacuum window.

The second screen is positioned about MM away from it. 

The tracking using these distances gives for zero divergence an energy range from XX MEV to XX MEV. The zero point was selected by
 on-axis radiation.
 
The electron beams considered are relying on shock injection, with the blade in various positions into the gas jet. The typical maximum energy was around 1.2 GeV with in some cases around 5 percent energy spread and in some cases a long tail down to around 200 MeV, filling the entire range. The total charge varied from XX pC to XX pC.

\subsubsection{Different Regimes and relation to shock position}

Decide which to use in set later.

For one regime characterise properties (energy, charge and stability).





\section{Gamma Background Yield}

\subsubsection{Gamma Profile Background Yield}


\subsubsection{Stitching together both parts of the spectrometer}

As described in previous Chapter.

\subsubsection{Correlate counts on gamma profile and gamma spec}






\section{Bremsstrahlung Measurements}

\subsubsection{Electron Spectra for Brems}

Here talk about using different bremsstrahlung materials in a known electron regime and see what the signal is and whether it matches some kind of fluctuations of the electron beam.

\subsubsection{Brems Targets}

\subsubsection{Gamma Profile and Spec Yield}

Correlations

\begin{figure}
\centering
\includegraphics[width=0.6\columnwidth]{20190211r009s060_GammaSpec_analysis1.png}

\includegraphics[width=0.6\columnwidth]{20190211r009s060_GammaSpec_analysis2.png}
\caption{Gamma Spec waterfall converter.}
\end{figure}

\begin{figure}
\centering
\includegraphics[width=0.6\columnwidth]{20190211r009s001-062_GammaSpec_Waterfall.png}
\caption{Gamma Spec waterfall converter.}
\end{figure}

\begin{figure}
\centering
\includegraphics[width=0.5\columnwidth]{compFePTFE.png}\includegraphics[width=0.5\columnwidth]{compvert.png}
\caption{Correction factor simulation and data.}
\end{figure}




\subsubsection{Spectral Response}

\subsubsection{GEANT sims}

\subsubsection{Corr factor}

Response and GEANT simulation.
Range of electron spectra.






\section{Estimating Interaction Conditions in Raster Scan}

\subsubsection{Give criterion threshold based on background}
Look at raster scan and show how we estimated the beam size.

\subsubsection{Raster scan to estimate beam size}

See how big the beam was and based on that estimate intensity.

\subsubsection{Comments on Timing using Probe?}

Show how the probe can be used in conjunction with Top View to maintain spatial alignment and temporal alignment over long time.

\begin{figure}

\centering
\includegraphics[width=0.8\columnwidth]{20190208r015s047_Probe_Slim.jpg}

\includegraphics[width=0.4\columnwidth]{20190208r015s047_PrettyPicIC.png}\includegraphics[width=0.4\columnwidth]{20190208r015s047_TopViewZoom.jpg}
\caption{Diagnostics on shot to characterise overlap.}
\end{figure}

\begin{figure}
\centering
\includegraphics[width=0.6\columnwidth]{20190211r005s077-160_EspecScan.png}
\caption{Raster scan at 400 microns. Maybe one example shot as well?}
\end{figure}

\begin{figure}
\centering
\includegraphics[width=0.6\columnwidth]{20190211r005s077-160_GammaProfile.jpg}
\caption{Raster scan at 400 microns. Maybe one example shot as well?}
\end{figure}

\begin{figure}
\centering
\includegraphics[width=0.6\columnwidth]{20190211r006s077-160_GammaSpec_Waterfall.png}
\caption{Raster scan at 400 microns. Maybe one example shot as well?}
\end{figure}

\section{Correlating Electron Properties and Gamma Yield}

Correlations of charge, energy and the gamma yield.

If the conditions were fairly stable there should be a linear trend - difference to the non-linear tightly focused case.

\begin{figure}
\centering
\includegraphics[width=0.6\columnwidth]{20190208r015_GPnorm.png}
\caption{Long run 20190208r015. Normalised Gamma Profile counts.}
\end{figure}

\section{Spectral Retrieval for linear ICS}

Based on the correction factor and the estimated interaction conditions fit close to mono-energetic spectra (times the electron spectrum) and see if that adds up.

Advantage is that we measure the identical electron spectrum and the ICS signal at the same time without perturbation.

\subsubsection{Second harmonic radiation}

At this intermediate level there might be some indication of second harmonic radiation.

Show how the response would change adding a realistic number of photons to the curve.

If it is a considerable change, apply this to the data and have some kind of measure of likelihood that some amount of energy was added.

If it is possible to pick up differences, it could be used to measure quite finely the intensity at the interaction.

\section{Using ICS as electron diagnostics}

\subsubsection{Tracking}

Indicate that Dominik will be able to use the ICS signal as on-shot reference for two-screen tracking.

\subsubsection{Electron Profile}

Can use this to get 2D electron profile.

\section{Conclusion}