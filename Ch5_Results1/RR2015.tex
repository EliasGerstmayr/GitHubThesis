\chapter{First Measurements of Radiation Reaction at Astra Gemini}

\section{Experimental Motivation}

\subsubsection{Inverse Compton Scattering}
In relativistic inverse Compton scattering (ICS) one (linear) or multiple (non-linear) photons scatter from a relativistic electron ($\epsilon/m_e c^2 = \gamma \gg 1$). The scattered photons experience a relativistic Doppler-shift and are re-emitted in a narrow cone of divergence $\sim 1/\gamma$ in the direction of the electron propagation, carrying a higher energy based on the electron energy and the intensity of the laser field. The emitted photon energy, $E_{ph}'$, is maximised in a head-on collision: $E_{ph}' = 2 \gamma^2 (1-\cos\theta)E_{ph}$ (see \nameref{Chap:Theory} for further details). Higher electron energies result in a stronger shift of the radiation, leading to a hardening of the spectrum and an increase of the corresponding energy loss the electrons experience (REF). A more intense laser field ($a_0 > 1$), on the other hand, enables non-linear interactions and increases the number of photons interacting at once with an electron (REF), resulting in the generation of $\sim a^3_0$ higher harmonics (REF) but also redshifting and broadening of the spectrum (REF). As a result the rapidly increasing number of higher harmonics blend together to form a broadband synchrotron-like spectrum.

ICS provides a promising route to generate bright burst of high energy radiation reaching 100's of keV or even MeV photon energy, suitable to image high-Z objects (REF) and stimulate nuclear transitions (REF), even at small-scale facilities (REF).
In addition, the high photon energies offer the opportunity to directly observe the energy loss, if the emitted photon carries an energy comparable to the energy of the electron, and to measure the effect of radiation reaction on the beam dynamics itself when the electric field in the rest frame of the electron approaches the critical field of QED, $E_{crit}$. Classical descriptions of radiation reaction exhibit the unphysical prediction of an unbounded emission spectrum, i.e. energy loss higher than the initial energy of the electron is allowed.
\vspace{\baselineskip}

\subsubsection{Inverse Compton Scattering using Laser Wakefield Accelerators}

In LWFA the intense laser driver is intrinsically synchronised to the relativistic electron beam that trails behind it, which makes LWFA well suited to study inverse Compton scattering at high intensities.
\vspace{\baselineskip}


\begin{figure}[h]
\centering
\includegraphics[width=0.8\columnwidth]{RR2015_Gmax_thiswork.pdf}
\caption{Some plot to place our work into context. Change colours as they look identical when printed and replace `this work' with a label. Check energies. I think our energies are higher but we quote critical energy instead.}
\end{figure}


Colliding-pulse or ICS experiments using LWFA have been successfully performed by different groups over the past years, gradually improving the control over the process, increasing the laser intensities and the energies of the electrons involved, and subsequently of the radiation generated \cite{TaPhuoc2012a,Chen2013a,Khrennikov2015,Powers2014,Sarri2014}.
The energy and the tunability of the produced radiation is a good indicator for the continuing progress of these experiments. More specifically of interest for measuring radiation reaction is the quantum non-linearity parameter, $\eta = 2\gamma E_L/E_{crit}$ (REF), that indicates the strength of the electric field of the laser in the rest frame of the electron relative to the critical electric field of QED, $E_{crit}$.
\vspace{\baselineskip}

\subsubsection{Inverse Compton Scattering using a plasma mirror}

A single intense laser pulse can be used to accelerate electrons to relativistic energies via LWFA and to scatter the electrons after backreflecting the laser from a close-to-normal plasma mirror (e.g. tape \cite{TaPhuoc2012a}). In this geometry the laser beam and the electrons are timed intrinsically. In \cite{TaPhuoc2012a} electrons of an energy $\sim 100\,\mathrm{MeV}$ were collided at a laser intensity of $a_0 \approx 1.2$ (weakly non-linear). The radiation measured was a broadband X-ray spectrum and reached up to $100'\mathrm{s}\,\mathrm{keV}$ photon energies. A lower electron energy spread and tunability was demonstrated at slightly lower electron energies and laser intensities in \cite{Powers2014} and\cite{Khrennikov2015}. By now the highest recorded gamma-ray energies from ICS in LWFA using this scheme have been achieved by Shaw et al. (REF) reaching up to $85 \,\mathrm{MeV}$ photon energies in a collision with electrons at energies up to $2\,\mathrm{GeV}$.
\vspace{\baselineskip}

Whilst this technique avoids issues with timing and overlap, the intensity of the laser pulse is limited as it is partly depleted when interacting with the electron bunch after driving a wake and is typically not tightly focused. At the same time the acceleration can not be pushed to its depletion limit as the laser pulse has to remain intense enough for a suitable interaction. Further problems might arise as controlling or measuring the wavefront of the depleted laser pulse is challenging. The electrons will also produce radiation from bremsstrahlung when passing through the plasma mirror, which overlays the ICS signal.
These details are important for a detailed measurement of radiation reaction, but might be secondary concerns if a well-defined spectrum is not important for the application.
At facilities housing the next generation of real PW-class lasers, the reflective scheme might still be able to yield very promising results to probe radiation reaction.
\vspace{\baselineskip}

\subsubsection{Inverse Compton Scattering with two lasers}

If two separate lasers are available or maybe one very powerful laser can be split into two parts, one can be used to accelerate electrons via LWFA and one to scatter off them. Electrons produced from LWFA are typically very short in duration (10's of femtoseconds, REF) and only few microns small (REF source size) when leaving the accelerating cavity. This allows focusing the second laser very tightly to reach high intensities and to interact with a large fraction of the electron bunch at comparable intensities without a large linear ICS background. Using two laser pulses opens the gateway to combine higher intensities and electron energies at the interaction point, and gives more control over the interaction itself. However, it is challenging to overlap the micron-sized electron bunch with the ultra-short tightly-focused laser pulse, compressed in time and space, and to maintain this alignment over an extended period of time.

A very powerful laser can be used to interact at slightly larger spot sizes, which improves the probability of interactions as it mitigates relative pointing fluctuations. If the laser spot is larger than the electron beam the theoretical modelling of the interaction can be approximated in 1D (REF).

In \cite{Chen2013a} relativistic electrons were scattered at an angle of $10$ degrees and gamma rays at an energy of around $1\,\mathrm{MeV}$ were produced.
In \cite{Sarri2014} electrons at an energy of $400 \, \mathrm{MeV}$ were successfully collided at an laser intensity of $a_0 \approx 2$, resulting in broadband radiation extending up to $10\,\mathrm{MeV}$. In YAN REF NAT PHOTONICS 2017 electrons of energy $\sim 200\,\mathrm{MeV}$ were collided with an intense laser at $a_0 \sim 12$ reaching photon energies just above $20\,\mathrm{MeV}$, resulting in the generation of over $500$ orders of higher harmonics.
\vspace{\baselineskip}


\begin{figure}[h]
\centering
\includegraphics[width=0.8\columnwidth]{RR2015_Eta_thiswork_lowera0.pdf}
\caption{Some plot to place our work into context. Move labels to left. Maybe make This work a different symbol? Change colours.}
\end{figure}



\subsubsection{Future LWFA Studies on Inverse Compton Scattering and Radiation Reaction}

Future laser facilities will be able to perform ICS studies at even higher electron energies and laser intensities.
Several new PW laser systems (ELI, EPACS, HERCULES) are now or soon starting their operation, with a current record for $\mathrm{fs}$ laser power of 10 PW (ELI REF) with future facilities already aiming for 100 PW (SULF PW REF). 
Maximum electron energies from LWFA on the other hand have reached 4 GeV in a single unguided stage and (4 GeV CORELS REF) and up to almost 10 GeV in a guided setup (REF GUIDED), staged REF?.

Combining both opens up a wide variety of research topics in the strongly non-linear quantum regime probing radiation reaction (REF), photon-photon scattering (REF) and pair production from the Breit-Wheeler mechanism (REF) or by approaching the Schwinger limit (REF). The high laser intensities will also enable studies of vacuum birefringence (REF).

Considering the cost and variety of facilities (REF), it is important to investigate the limitations of current all-optical ICS setups, the feasibility of future measurements and explore potential ways to improve them GUILLERMO REF.

\subsubsection{Inverse Compton Scattering at conventional accelerator facilities}

Inverse Compton Scattering has been employed at conventional accelerator facilities for some time as well, but at much lower laser intensities. Its main application is the measurement of the beam polarisation (polarimetry) in storage rings (REF), mostly related to the Sokolov-Ternov effect (REF).

As the field of applications for large-scale accelerators widens and the access to 10 TW laser systems and even commercially available PW-class lasers becomes easier, conventional accelerators combine their high-quality relativistic particle beams with intense laser pulses.
At XFELs, for instance, this is interesting for warm dense matter and HEDP studies, involving pump-probe measurements, where X-rays from an insertion device probe matter structures and changes in them in interaction with a laser.
In the wake of increasing interest in measuring QED phenomena in laser interactions ICS has also moved into the focus of conventional accelerator facilities. New dedicated projects continue where the seminal E144 experiment at SLAC left off (46 GeV electron beam with laser $a_0 \sim 0.3$) with LUXE at Hamburg, Munich, SFQED at FACET-II (30 TW, 13 GeV XX NUMBERS?). Whilst the high-quality beams at these facilities are superior to LWFA studies in terms of charge, energy and energy stability, these projects are facing other challenges due to significant amount of background noise in the accelerator tunnels, and typically longer and larger electron bunch sizes resulting in a significant overlay from linear scattering events in an ICS setup.

Another measurement of radiation reaction was performed at LHC (REF AARHUS), however using planar channeling in crystals instead of ICS.

\section{Chapter Outline}

The work presented in this chapter relates to an ICS experiment aimed at measuring radiation reaction performed at the Gemini laser facility in late 2015. The experimental team succeeded in colliding electrons of energy $\epsilon \approx 550\,\mathrm{MeV}$ at a laser intensity of $a_0 \approx 10$, reaching critical gamma-ray energies $\epsilon_{crit}$ in excess of $30\,\mathrm{MeV}$. This was the highest gamma-ray energy from an all optical ICS source published at that point and consists the first published measurement of radiation reaction in an LWFA setup \cite{Cole2018}. 
\vspace{\baselineskip}

This chapter includes significant contributions to this publication, in particular finding a method to identify successful collisions and characterising the electron spectra. As a result, the reader will find several parallels and in some instances a similar line of arguments between \cite{Cole2018} and this chapter.
\vspace{\baselineskip}

In order to make a measurement of radiation reaction we need to observe that when a laser-electron beam collision occurs the electron energy is lower. 
However, in a laser wakefield accelerator experiment there are two key problems: first, the electron spectrum varies from shot to shot, and second, not all attempted collisions will be successful.

To overcome these difficulties the statistical fluctuations of the electron beam need to be characterised when there is not a collision, and the sub-set of successful collisions needs to be identified.
\vspace{\baselineskip}

After an outline of the experimental setup, such a method to identify successful collisions is presented. For this purpose, the yield on the gamma detectors is correlated on a shot-to-shot basis with the energy in the electron beam. Without the second scattering beam, the radiation measured is produced from bremsstrahlung as dispersed electrons interact with the walls of the vacuum chamber. The second laser adds ICS as an additional source of radiation and particularly intense interactions produce a significant excess signal on the gamma detector. Over the course of 10 shots this method identifies 4 successful collisions with an excess signal of more than 5 standard deviations above background, i.e. at a probability of 1 in 3500000 to occur by chance one one shot, assuming a normal background distribution. 

The chapter continues with an analysis of the measured electron spectra of this dataset. The electron spectra consistently exhibit a distinct spectral feature in form of a sharp edge-like fall-off in charge at around $\sim 500\,\mathrm{MeV}$. 
For the identified 4 collisions the energy of the spectral feature is lower than the average of the shots without the scattering laser. 
Considering a larger dataset the statistical fluctuations of the spectral feature without the scattering laser are characterised and correlations to other experiment parameters explored. Based on this statistical analysis the combined probability of measuring 4 spectral features at low energies is about $0.3\%$.

Then key results from the analysis of the gamma spectra, first performed by Jason Cole (Imperial College London) and Keegan Behm (University of Michigan), are introduced. The spectra are deduced by simulating the response of an extended CsI scintillator stack to gamma radiation and comparing this to the measured signal. Details on the analysis can be found in \cite{Behm2018},\cite{Cole2018} or also in the \nameref{Chap:Methods} Chapter of this thesis.

The critical energy of the gamma spectrum is significantly higher on shots without the scattering laser as the detector response is dominated by bremsstrahlung. The critical energies of the 4 identified collisions are anti-correlated with the electron energy, all below $500\,\mathrm{MeV}$ which is not consistent with radiation produced after measuring the energy of the electron beam.
The chance for this to occur is 1 in 3000. This supports further that we have measured radiation reaction.
\vspace{\baselineskip}

Combining all of the previous results, we can confidently state that we successfully collided the electron beam with the laser pulse and that radiation reaction was measured. The inferred energy loss and the corresponding gamma spectra are placed in the context of different radiation reaction models. The measurements are in general agreement with models including energy loss from radiation reaction. At the $1 \sigma$ confidence level the data points agree better with the models including quantum corrections, but at these experiment parameters and uncertainties the models with and without quantum corrections overlap strongly.
\vspace{\baselineskip}

Finally, the electron spectra from this experiment are compared to a second measurement of radiation reaction. At the same laser system using a gas cell target electrons were accelerated to up to around $2\,\mathrm{GeV}$ energy and collided at $a_0\approx 10$ \cite{Poder2018}. The analysis of the stability of the electron energy and spectral shape of both measurements are used to indicate the scale of data points required to relate experimental results to specific models of radiation reaction at a high level of statistical confidence. This discussion is expanded to a wider parameter space and in more detail in C ARRAN REFS.

\section{Experimental Setup}

The experiment described in the following was performed at the dual $300\,\mathrm{TW}$ Ti:Sa Gemini laser system at the Central Laser Facility, Rutherford Appleton Laboratory, UK, in late 2015. Both arms provide two linearly polarised laser beams of central wavelength $800\,\mathrm{nm}$ at a pulse duration of $45\,\mathrm{fs}$ \textsc{FWHM} and collimated beam diameter of $\sim 150\,\mathrm{mm}$.

A sketch of the setup is shown in Figure \ref{RR15:figs:setup_sketch}.
\vspace{\baselineskip}

\begin{figure}[h]
\centering
\includegraphics[width=0.8\columnwidth]{Exp_setup_render_RR2.png}
\caption{Conceptual sketch of the experiment setup rendered with Blender.
Based on a sketch made by J. Cole (Imperial College) for \cite{Cole2018} and adapted for this work and for\cite{Behm2018}. 
%From left to right: a high intensity laser beam (red) focused with a f40 spherical mirror generates up to $\mathrm{GeV}$-scale electrons in a gas jet (LWFA). A second laser beam (red) is focused down tightly at the edge of the gas jet by an f2 off-axis parabola to scatter the electron beam shortly after it leaves the jet. The electrons (blue) are being dispersed by a dipole magnet and detected on a scintillating lanex screen (grey). Finally, gamma rays (green) propagate through a vacuum kapton-window and a lead aperture onto a stack of scintillating CsI crystals that are imaged by a camera.
}
\label{RR15:figs:setup_sketch}
\end{figure}


The first part of the experiment is set up to produce a relativistic electron bunch from laser wakefield acceleration (LWFA): a laser beam is focused down by an f/40 spherical mirror with $6\,\mathrm{m}$ focal length onto the leading edge of a $15\,\mathrm{mm}$ conical supersonic helium gas jet target. The top edge of the gas jet is positioned $6\,\mathrm{mm}$ below the laser axis in order to avoid damage to the nozzle from the second more divergent laser beam. Electrons are accelerated via LWFA and propagate further downstream where they are dispersed by a permanent dipole magnet of integrated field strength $\int B(x) \mathrm{d}x = 0.4\,\mathrm{Tm}$ onto a scintillating LANEX screen imaged by an Andor Neo camera to measure their spectrum. The typical \textsc{fwhm} focal spot of the driving laser pulse measures $37 \times 49 \,\mathrm{\mu m}$ with an energy on target of $(8.6 \pm 0.6)\,\mathrm{J}$, which corresponds to a normalised vector potential $a_0 = 1.9 \pm 0.1$. The electron density of the target was $(3.7 \pm 0.4) \times 10^{18} \,\mathrm{cm}^{-3}$. 
\vspace{\baselineskip}

The second laser beam is focused down tightly onto the opposite edge of the gas jet, at 180 degrees from the driver laser, using an f/2 off-axis parabola (OAP). This laser is used to scatter from the electron bunch accelerated through LWFA to generate a bright burst of gamma rays from inverse Compton scattering (ICS). The OAP is fitted with a central hole, $21\,\mathrm{mm}$ in diameter, to enable propagation of the electrons, gamma rays and the remaining laser light of the $f/40$ wakefield driver beam. In addition, a plastic ring of $28\,\mathrm{mm}$ radius around the hole protects the optics and the laser chain upstream from potential driver laser light scattered in an interaction with the plasma. The combined loss of reflective surface leads to a decrease in intensity of the flat-top beam of around $16\%$. The energy on target was typically $(10 \pm 0.6)\,\mathrm{J}$ focused into a spot of $2.4 \times 2.8\,\mathrm{\mu m}$ \textsc{fwhm}, corresponding to peak normalised vector potential of $a_0 = 24.7 \pm 0.7$. The peak value of the quantum non-linearity parameter, $\eta$, in a head-on collision is $\eta = 2\gamma a_0 \hbar \omega_0/m_e c^2$.
Based on the laser parameters the maximum values of $\eta$ achievable in this configuration are $\eta = 0.15$ for electrons at $0.5\,\mathrm{GeV}$ and $\eta = 0.3$ at $1\,\mathrm{GeV}$ electron energy.

The narrow cone of gamma rays from ICS propagates through the hole of the f/2 OAP, the XX cm wide aperture of the dipole magnet, then through a $50\,\mathrm{\mu m}$ aluminium laser beam block and finally leaves the vacuum chamber through a $250\,\mathrm{\mu m}$ thick kapton vacuum window. At air, the gamma rays are incident onto a stack of caesium-iodide (CsI) crystals that measures the spectrum of the high energy radiation and is imaged by an Andor iXon camera. More details on the composition of the detector can be found in \nameref{Chap:Methods} (labelled `RAL stack with steel front plate').
\iffalse
The stack is 33 crystals high and 47 crystals deep, each crystal $5\,\mathrm{mm} \times 5\,\mathrm{mm} \times 50\,\mathrm{mm}$, with the $5\,\mathrm{mm}\,\times\,5\,\mathrm{mm}$ sides facing to the side with respect to the gamma-ray axis and being imaged by an Andor iXon camera. The crystals are spaced by $1\,\mathrm{mm}$ aluminium dividers and the front side of the stack is fortified by a $9$-$\mathrm{mm}$-thick steel plate.
\fi
The entire stack is housed and shielded in a lead enclosure with a circular aperture of $15\,\mathrm{mm}$ diameter, corresponding to an acceptance angle of $\sim 6.8\,\mathrm{mrad}$ at $\sim 2.2\,\mathrm{m}$ from the interaction point.
\vspace{\baselineskip}

The two laser beams are overlapped in space and time using spatial interferometry: a reflective protected-aluminium 90-degree knife-edge prism is placed at the interaction point and reflects both counter-propagating laser beams collinearly onto a CCD camera chip equipped with a X10 long-working-distance microscope objective. Since the laser beams are cross-polarised, a polariser is added on a motorised rotation stage to enable interference. Varying the rotation of the polariser gives control over the relative brightness of the beams. The different radii of curvature of the f/40 and f/2 beams result in the formation of a circular interference pattern when the laser pulses overlap in both space and time. The overlap is then further improved by optimising the visibility of the fringes to a precision of around $\pm 30\,\mathrm{fs}$. More details on this technique can be found in \nameref{Chap:Methods}. 

\iffalse
Unfortunately, the overlapping the beams both in time and space at one point in time does not guarantee that they remain overlapped. Some studies performed at Gemini (by Oxford REF RAL REPORT SHALLOO) indicate that drifts exist in the system and the necessary precision of the alignment is only maintained for about half an hour. This has been greatly improved in recent years on other experiments. In addition, the precision of the prism position determines how accurately the interaction point is defined.
\fi
\vspace{\baselineskip}


%\begin{figure}[h]
%\centering
%\includegraphics[width=0.8\columnwidth]{scintillator.JPG}
%\caption{Scintillator array used in experiment. The CsI crystals are encased in an aluminium casing approximately $220\,\mathrm{mm}$ deep and $150\,\mathrm{mm}$ high with crystals  and an array of in total. The scintillator was placed into the beam axis, the short side facing the beam enabling the radiation to travel as long as possible through the array. The holed side (diameter of holes around $5\,\mathrm{mm}$ each) was imaged using a scientific camera.}
%\end{figure}


\section{Identifying Successful Collisions}
%\section{Correlating the Gamma-ray signal with the Electron Spectrum}

To identify successful collisions we compare the radiation yield from the electron beams with and without scattering laser.

The electrons will emit broadband bremsstrahlung with photon energies potentially reaching photon energies up to the maximum electron energy when interacting with matter in their trajectory. Especially high-Z materials such as the vacuum chamber walls are efficient converters. This radiation is in a spectral range comparable with the expected ICS signal and is also measured by the gamma-ray detectors. The electrons are dispersed upwards and collide with the roof of the aluminium vacuum chamber producing the main source of bremsstrahlung. It is located off-axis and a large fraction is shielded efficiently by blocking the direct line of sight with sufficient amounts of lead. This reduces the total background signal on the detector and improves the signal-to-noise ratio, but also allows a spectral retrieval of the bremsstrahlung component as it now only enters the stack from one defined side.
\vspace{\baselineskip}

\begin{figure}[h]
\centering
\includegraphics[height=0.25\columnwidth]{20151217r002_NullGroup.jpg}
\includegraphics[height=0.25\columnwidth]{20151217r002_CollGroup.jpg}
\includegraphics[height=0.25\columnwidth]{up_arrows.jpg}
\caption{Images of the gamma ray spectrometer (upper row) and electron spectrometer (bottom row) at shots with the colliding laser beam off (left) and on (right), using a constant colour scale. The shots with the colliding beam show bright signals on the scintillator array whilst the electron energies remain fairly similar throughout.}
\label{Results:Figs:NullColl:Montage}
\end{figure}


The energy emitted through bremsstrahlung by a relativistic electron is proportional to its energy squared (REF and see \nameref{Chap:Theory}). The total energy deposited in the caesium-iodide crystals of the gamma spectrometer converts linearly into scintillation light at an efficiency of $\approx 5 \times 10^4 \,\mathrm{MeV}^{-1}$ (REF). The total yield of the detected bremsstrahlung signal on the camera chip, $S_{BG}$, should then follow this relation:
\begin{equation}
S_{BG} = c_{BG} \int_{\gamma_{min}}^{\gamma_{max}} \left[\mathrm{d}N_e /\mathrm{d}\gamma\right]\,\gamma^2 \mathrm{d}\gamma = c_{BG} Q \left\langle \gamma^2 \right\rangle,
\end{equation}
where $c_{BG}$ is a constant that encapsulates the complicated details of the interaction and the experimental setup that remain the same over the course of the shots, such as the conversion efficiencies of electron energy to bremsstrahlung photons, photons depositing their energy in the detector crystals, number of photons emitted from the scintillator per energy deposited, viewing angle of the camera, collection efficiency of the imaging system and quantum efficiency of the camera. $\mathrm{d}N_e/\mathrm{d}\gamma$ is the charge distribution of the measured electron spectrum, $Q = \int (\mathrm{d}N_e/\mathrm{d}\gamma) \mathrm{d}\gamma$ the total charge and $\gamma$ is the relativistic Lorentz factor of the electrons. It was also assumed that other sources of background (dark field, stray light on the camera etc.) are removed efficiently and do not need to be included in this equation. By measuring the yield (total counts) on the gamma detector and the electron spectrum one can then determine $c_{BG}$ experimentally.
\vspace{\baselineskip}

%\begin{figure}
%\centering
%\includegraphics[width=0.8\columnwidth]{ElecQ2_CsI_null_Correlation.pdf}
%\caption{Electron energy squared times against signal strength (counts) measured from the CsI stack.}
%\end{figure}

\begin{figure}
\centering
\includegraphics[width=0.9\columnwidth]{ElecQ2_CsI_null_Correlation.pdf}
\caption{Squared energy of the electron beam on the x-axis vs. the gamma yield on the gamma detector in pixel counts for shots without the scattering beam. The shaded area indicates the $95\%$ confidence interval for the linear fit. The gradient is $c_{BG}$.}
\label{Results:Figs:NullColl:DeltaCsIVsSigmaE2Null}
\end{figure}



Figure \ref{Results:Figs:NullColl:DeltaCsIVsSigmaE2Null} shows the relation of experimentally measured $Q \left\langle\gamma^2\right\rangle$ to the total number of pixel counts (yield) on the gamma-ray detector for 10 shots without the scattering beam. An example of the raw data for the electron spectrum and the gamma detector can be seen on Figure \ref{fig:Cole_espec_example} and \ref{fig:Cole_gamma_example}. The data points follow a linear trend corresponding to $c_{BG}$ at a correlation coefficient of $0.71$. 
\vspace{\baselineskip}

On shots with a successful collision with the scattering beam, a burst of gamma rays from inverse Compton scattering will be produced. At the detector the signal is then a combination of bremsstrahlung as without the scattering beam, following the same relation and $c_{BG}$, and the ICS signature. The emitted energy of the ICS radiation and hence the produced detector signal, $S_{ICS}$, is also proportional to $\gamma^2$ similarly as for the bremsstrahlung process but also scales with the normalised vector potential $a_0$ in interaction with an electron with energy $\gamma$ (REF PRX THOMAS 2012 for $\gamma a_0^2 < 4.4 \times 10^5$, CORDE REV MOD 85, 2013):

\begin{equation}
S_{ICS} = c_{ICS} \int  a_0^2 (\gamma) \left[\mathrm{d}N_e /\mathrm{d}\gamma\right]\, \gamma^2 \mathrm{d} \gamma,
\end{equation}

where now $c_{ICS}$ similarly encapsulates all the complex physics of coupling constants, cross sections and conversion efficiencies. In this case the value of $a_0$ at the interaction is not constant from shot to shot as a varying overlap of the laser pulse and the electron beam will result in changing interaction conditions. $a_0$ can also vary throughout the interaction if the duration of the interaction is comparable to the time it takes for the laser pulse to pass through its focus, either due to a short Rayleigh length or long electron bunches. The better the overlap and the higher the intensity at the interaction, the stronger the ICS signal and the easier to distinguish will it be from the background bremsstrahlung. In addition, whilst we expect all of the electrons to produce bremsstrahlung, only a variable fraction of electrons will interact with the laser pulse. If the laser pulse is larger than the electron beam the variation in intensity the electrons experience will mainly be spectrally varying.

The total signal, $S_{total}$, combines to

\begin{equation}
S_{total} = S_{BG} + S_{ICS} = c_{BG} \int  \left[\mathrm{d}N_e /\mathrm{d}\gamma\right]\, \gamma^2 \mathrm{d} \gamma + c_{ICS}   \int a_0^2 (\gamma) \left[ \mathrm{d}N_e /\mathrm{d}\gamma\right]\, \gamma^2 \mathrm{d} \gamma.
\end{equation}

In Figure \ref{Results:Figs:NullColl:DeltaCsIVsSigmaE2+CsIVsE2} we now compare experimental data for shots with and without the scattering beam similarly as before in Figure \ref{Results:Figs:NullColl:DeltaCsIVsSigmaE2Null}. The shots with the scattering beam (orange) are more spread than the shots without the colliding beam (blue). Some of the data points with the colliding beam follow the linear relationship determined for the reference data relatively well and indicate poor overlap. A few shots exhibit a much higher detector signal than predicted by the fit for an electron spectrum of comparable charge and energy. This indicates good overlap.

To quantify how much higher the detected signal is than predicted by the background fit, it is more useful to look at the difference of measured to expected signal, i.e. by subtracting the expected bremsstrahlung background from the total signal to extract the ICS contribution:
\begin{equation}
S_{ICS} = S_{total} - c_{BG}Q\left\langle\gamma^2\right\rangle,
\end{equation}

or in relative terms:
\begin{equation}
S_{ICS,rel}(Q\left\langle\gamma^2\right\rangle)= \frac{S_{ICS}}{c_{BG}Q\left\langle\gamma^2\right\rangle} = \frac{S_{total} - c_{BG}Q\left\langle\gamma^2\right\rangle}{c_{BG}Q\left\langle\gamma^2\right\rangle}.
\end{equation}

For shots without the scattering beam the ICS signal $S_{ICS}$ determined by this method and $S_{ICS,rel}$ fluctuate around a mean of zero. Assuming a normal distribution we can calculate the standard deviation $\sigma_{BG}$ of the fluctuating background and estimate how likely a bright gamma signal has been produced by the characterised background signal:

\begin{equation}
S_{ICS, norm} = \frac{S_{ICS,rel}}{\sigma_{BG}}
\end{equation}

\begin{figure}
\centering
\includegraphics[width=0.5\columnwidth]{ElecQ2_CsI_Correlation.pdf}\includegraphics[width=0.5\columnwidth]{DCsI_sigma.pdf}
\caption{Left: Squared energy of the electron beam vs. the signal strength on the gamma detector in pixel counts. The reference shots without scattering beam are indicated in blue with a regression line. In orange shots with both laser beams on. Right: Shot number against the deviation from the expected gamma signal using the difference of the signal from the blue regression line on the left plot.}
\label{Results:Figs:NullColl:DeltaCsIVsSigmaE2+CsIVsE2}
\end{figure}

In Figure \ref{Results:Figs:NullColl:DeltaCsIVsSigmaE2+CsIVsE2} the shots are lined up in the order their data was taken in the experiment. The normalised signal above expected background, $S_{ICS, norm}$, is indicated on the y-axis. The y-axis now encapsulates the information of both the gamma signal and the electron spectrum.
As we can see most of the shots follow the expected trend within 2 standard deviations, with some exceeding the signal. 4 shots show a signal more than 5 standard deviations higher signal than expected from background. Based on a normal distribution the probability for one shot to be as bright as 5 standard deviations above the mean background is 1 in 3500000, which is very unlikely and hence a suitable confirmation that we have identified 4 successful collisions. 

This method not only allows the identification of successful collisions, but on the other hand enables us to also identify collisions where ICS was negligible if the signal aligns well enough with the background estimate. 



\section{Characterisation of the Electron Spectrum}

In the previous section the comparison of the measured gamma-ray signal with the expected bremsstrahlung noise produced by the electron beam enabled the identification of successful collisions. The next step is quantifying whether there was a measurable lower energy (potential energy loss) visible in the electron spectrum on these collisions. 

To estimate the energy loss accurately we have to take the intrinsic variations of the electron source and correlations to other fluctuating variables such as laser energy into account. For this purpose we will characterise the typical electron spectrum and its statistical fluctuations for shots without the scattering beam or where ICS is negligible based on the previous argument.


\subsubsection{Characterising a typical spectrum}


\begin{figure}
\centering
\includegraphics[trim={4.8cm 0 5cm 0}, clip, width=0.9\columnwidth]{Example_RR15Cole.png}
\caption[]{Example of a typical electron spectrum. The in dispersion and divergence axis integrated spectrum is shown on the respective axes. }
\label{fig:Cole_espec_example}
\end{figure}

A typical electron spectrum as measured on the experiment is shown in Figure \ref{fig:Cole_espec_example}. 

A detailed description of the treatment of the raw data (background subtractions, image transformations, tracking etc.) can be found in the Methods section and is not expanded here.

The characteristic electron spectrum consists of two components (see Figure \ref{fig:Cole_espec_example}): one is a low charge but high energy tail reaching $800-1000\,\mathrm{MeV}$, the other part contains a high charge at lower energy that falls off rapidly at an edge-like spectral feature at around $450-550\,\mathrm{MeV}$. The electron spectrum is cut off at $250\,\mathrm{MeV}$ due to limitations in the spectrometer setup. 
The high energy component resembles from its shape spectra from self-injection (REF) measured at Gemini before, whereas the high charge component with the distinct spectral feature is consistent with shock injection caused by either the nozzle design or potential damage to it. The shock fronts are visible on an optical image seen in Figure \ref{RR15:figs:prettypic} along with the damage on the nozzle. 
Shock injection has been observed by other groups introducing a blade (REF \cite{Tsai2018}), silicon wafer (REF) or a wire (REF BURZA 2013 PRSTAB) into the gas flow to generate a shock front.


\begin{figure}
\centering
\includegraphics[trim={4.6cm 0 5cm 0}, clip, width=.9\columnwidth]{ElecEdge_Example.png}
\caption{Lineout of an electron spectrum (red) normalised to its peak charge. The x-axis indicates the electron energy in $\mathrm{MeV}$ and is cut off at $800\,\mathrm{MeV}$ as high energy features will not be further investigated. The blue line is the first derivative of the spectrum with a sharp peak at the edge-like spectral feature around $600\,\mathrm{MeV}$. A circle indicates the determined position of the spectral feature. CHANGE STYLE AS THIS LOOKS A BIT OFF.}
\label{Results:Figs:Edge:ShotsNull}
\end{figure}



\begin{figure}
\centering
\includegraphics[width=0.8\columnwidth]{20151217r002s003_PrettyPic.jpg}
\caption[]{Pretty Pic place holder. Add second image without North beam and where shock is visible. Indicate the nozzle damage. CHANGE}
\label{RR15:figs:prettypic}
\end{figure}


The position of the spectral feature is identified by finding a maximum in the derivative of the integrated spectrum resulting from the sharp cut-off in the integrated spectrum as shown in Figure \ref{Results:Figs:Edge:ShotsNull}.



\subsubsection{Extending the dataset for the statistical analysis of the electron source}

The data set that includes the potential high-intensity interactions and its immediate null shots is relatively limited with 23 shots in total, 10 of which are reference data.
A small sample size like this (10) is not sufficient to draw reliable conclusions on the general character of the fluctuations. A larger data set from the same day at comparable conditions is taken into consideration. 

Here the laser beam was defocused to about $30\,\mathrm{\mu m}$ and translated relative to the electron beam in an attempt to establish their relative position. The scattering beam is active on all of these shots, but the interactions occur at low intensities $a_0 < 1$ and the overlap changes throughout the dataset. Using the gamma-ray signal as indication for significant overlap as developed in the previous section, we now identify shots that very closely align with the characterised bremsstrahlung background. Shots that are within 1 standard deviation of the background are identified as non-collision shots and are included as reference data. This new data set consists of additional 89 shots. This data set is used to investigate correlations with other experiment parameters and to characterise statistical fluctuations. 

Ultimately, we want to characterise the `intrinsic' fluctuations of the accelerator, i.e. the fluctuations that remain after we remove correlations.
\vspace{\baselineskip}

\subsubsection{Temporal drift in the electron energy}

\begin{figure}[h]
\centering
\includegraphics[width=0.5\columnwidth]{ElecEdge_Raster_Drift.pdf}\includegraphics[width=0.5\columnwidth]{ElecEdge_Raster_Slices_Drift.pdf}
\caption[]{Left: Distribution of the energy of the spectral feature for null shots or negligible ICS signal plotted against the relative time the data was taken. The energies are scattered around a line that slowly increases with time. A potential fitting regression line with a confidence interval (shaded area) is drawn. Right: By assuming a random distribution around a slowly drifting mean the data set was split into individual time slices. CHANGE TIME AXIS TO RELATIVE TIME}
\label{fig:Cole_drift_raw}
\end{figure}




When tracing the time the data was taken and the electron energy of the spectral feature a slow drift in the mean energy becomes apparent (see Figure \ref{fig:Cole_drift_raw}). 

The correlation coefficient for the driver laser energy over time is $-0.11$, and the confidence interval for the slope of the linear regression includes negative and positive values. This indicates that the mean laser energy does not drift significantly over this time period, and only fluctuates around a constant or slowly decreasing mean (see Figure REF NUMBER). 

Self-injection is coupled to the evolution and performance of the laser. It is unexpected that the mean electron energy increases over time without changes to the laser properties and energy (REF).

The appearance of the high charge component of the spectrum is consistent with other electron beams injected through shock injection at Gemini (NO REF AVAILABLE YET). The shock could be generated due to a damage in the gas nozzle or intrinsically due to its design. The drifting mean is consistent with a shock shifting position closer to the leading edge of the gas target, increasing the acceleration length and energy over time. This behaviour has been observed at lower intensities at other laser systems (REF). Unfortunately, there is no optical probe data, e.g. from a shadowgraphy, to provide conclusive evidence for this theory. Progressing damage to the nozzle could provide the seed for a shifting shock position.

\begin{figure}[h]
\centering
\includegraphics[width=0.5\columnwidth]{Cole_LaserTime_Corr.pdf}\includegraphics[width=0.5\columnwidth]{ElecEdge_Laser_Corr.pdf}
\caption[]{Left: On-shot driver laser energy over a 2 hour window. The actual data set extends further but the laser energy meter does not cover the entire time. Right: Laser energy on the shots with the respective drift-corrected relative energy shift. REPLACE WITH SMALL DATASET? LASER ENERGY VS UNCOMPRESSED LASER ENERGY, CHANGE IT TO ONE CONSISTENT LABEL. CHANGE TO NON 1E NOTATION.}
\end{figure}


We split the varying components into two parts: one is random noise fluctuating from shot to shot, the second is the mean around which the fast fluctuations take place, which is slowly increasing with time. The increase of the mean correlates well with time at a correlation coefficient of $0.86$ at a rate of $21.6\,\mathrm{MeV}$/hour. 
\vspace{\baselineskip}

We can check if the remaining fluctuations correlate with laser energy. There appears to be a positive correlation but only at lower confidence. The correlation coefficient is around 0.35 and the confidence interval for the slope encloses only very small gradients of order few percent over 4 J energy range (whereas the standard deviation is already scattered at 8 percent). When looking at dataset including the high-intensity interactions the gradient confidence interval for the slope even encloses zero which indicates no significant correlation. This does not seem to be a significant correlation in this context and we will hence ignore this factor in our analysis. 
\vspace{\baselineskip}

\subsubsection{Intrinsic statistical fluctuations}

Any deviation in energy $\Delta E$ from this slow drifting mean now represents a characteristic intrinsic fluctuation of the accelerator. In Figure XX NUMBER a histogram of relative differences in energies, $\Delta E/E$, is shown, along with an estimate of the underlying density distribution using a Kernel Density Estimate (KDE) and a Gaussian fit. The Gaussian fit agrees with the 99 percent confidence interval of the KDE, but there seems to be a slight skew of the distribution.

\begin{figure}[h]
\centering
\includegraphics[width=0.8\columnwidth]{ElecEdge_DE_drift_histo_errors.pdf}
\caption[]{Histogram of relative deviation of the spectral edge energy from the expected slowly varying mean ($\Delta E/E$). The total distribution is overlaid with a KDE, its $95\%$ confidence intervals and a Gaussian distribution (orange). The shaded area indicates a 2 sigma (95\%) confidence interval assuming a normal distribution.}
\end{figure}

Continuing with the assumption of a normal distribution we can assign probabilities to observing certain energies. The standard deviation of this distribution is $\sigma_{Cole} = 0.077$ or $7.7\%$. In other words this we expect 68 percent of all shots to have a value of $\Delta E/E$ between $\pm 7.7\,\%$ ($1\,\sigma_{Cole}$), 95 percent between $\pm 15.4\,\%$ ($2\,\sigma_{Cole}$) and so on. 


\subsubsection{Electron spectra in the collision data set}

\EliasComm{Here a comment on the low shots. Give mean and standard deviation for high collision shots.}

\begin{figure}[h]
\centering
\includegraphics[width=0.9\columnwidth]{ElectronSpectra2.pdf}
\caption{Waterfall plot from paper. Figure 4. Sort out licence or make myself.}
\end{figure}


Identified collisions have lower electron energy. Probability.


\section{Retrieval of gamma spectra}

Retrieval of gamma spectra by comparing shape of energy deposition in the scintillator stack with simulated response in GEANT.
More details can be found in REF, REF and Methods.

This has been done first by Jason Cole (Imperial College) and Keegan Behm (Michigan University).

\begin{figure}
\centering
\includegraphics[trim={4.8cm 0 5cm 0}, clip, width=0.9\columnwidth]{Example_RR15Cole_CsI.png}
\caption[]{Example of gamma-detector signal. The gamma rays propagate from the left into the stack and deposit their energy. Higher energy radiation penetrates the stack deeper. The y-axis indicates divergence.}
\label{fig:Cole_gamma_example}
\end{figure}


On shots without the scattering beam the critical energy of the distribution characterises the bremsstrahlung background.
On shots with the scattering beam with sufficiently bright signal, the overlaying ICS signal will be dominant (also as it is much brighter).

When there is only little signal or no scattering beam the energy is much higher due to the high energy bremsstrahlung.
When turning on the scattering laser the critical energy is much lower by XX and indicates the spectral emission is different for ICS.
In addition to the brighter signal this now confirms again a different process. The spectra can also be used to confirm the conditions at the interaction.

\begin{figure}
\centering
\includegraphics[width=0.8\columnwidth]{OnOff_Temp.pdf}
\caption{Right: Critical energy of gamma spectra for beam on and off. ENERGIES DONT SEEM TO MATCH UP WITH PAPER. WHAT ARE THE MOST RECENT NUMBERS?}
\label{Results:Figs:Gamma:OnOff}
\end{figure}






\section{Evidence of radiation reaction in the collision}

\subsubsection{Summary of results so far}
After having now characterised the electron spectra and the expected gamma signal, we can now look more closely at the identified successful collisions and potential indicators of radiation reaction.




To summarise, the dataset we investigate more closely consists of 13 shots with both beams on and 10 reference shots without the second scatterer beam.
Figure \ref{Results:Figs:NullColl:Montage} shows a montage of the raw data of the two relevant diagnostics, the electron spectrometer (bottom row) and the gamma-ray detector (top row) for reference shots (left) and dual-beam collisions (right). The electron spectrometer and the gamma-detector images are on the same colour scales respectively. The line-outs of the integrated and normalised electron spectra are also shown for convenience in Figure XX, along with a dotted line indicating the position of the distinct spectral feature.
\vspace{\baselineskip}

In section XX we decided on a selection criterion based on the signal on the gamma detector coupled to the electron beam energy to determine which shots were successful. 4 shots with a gamma signal 5 standard deviations or more above the expected background were identified. The bright burst of gamma rays lights up the CsI gamma detector as seen on the montages in Figure \ref{Results:Figs:NullColl:Montage}. The 4 bright shots are also indicated in Figure XX in red. The intrinsic variations of the electron source were characterised.
\vspace{\baselineskip}


\begin{figure}
\centering
\includegraphics[width=0.5\columnwidth]{DCsI_sigma_v_ElecEdge.pdf}
\caption{Position of the low-energy edges (x-axis) versus the gamma ray signal above the expected radiation background (y-axis) in units of standard deviation from the reference mean. Reference shots are shown in red and collision shots in blue. The edge positions are very scattered under both conditions with the majority of the collision shots exhibiting higher edges. The brightest gamma ray signals on collision seem to coincide with a low position of the edge. REPLACE THIS BY SNS JOINTPLOT}
\label{Results:Figs:Edge:PosVsCsI}
\end{figure}

\subsubsection{Statistical Significance of 4 successful shots and spectrum}

From Figure NUMBER HERE. We see that the electron spectral feature varies over the course of the shots, but all shots selected using our criterion have low electron energies. The probability for this, assuming a normal distribution as derived in the previous section, is around NUMBERS HERE for one shot and NUMBER TO THE FOUR for all four of them to occur together. If we are sufficiently convinced that these low electron energies are equivalent to energy loss from our mean, we can estimate the conditions at the interaction. For an electron energy of CHECK NUMBER $550\,\mathrm{MeV}$ we require a laser intensity of $a_0 \approx 10$ using PRX THOMAS 2012. This is much lower than the peak intensity of our laser pulse.
\vspace{\baselineskip}

A second angle at confirming the experiment conditions is the gamma spectrum. Jason Cole (Imperial College) and Keegan Behm (Michigan University) performed this analysis first and provided the results. More details on the analysis can be found in the Methods chapter or in \cite{Behm2018} outlining the modelling and analysis procedure applied in this specific case.
In Figure XX we see that the critical energy of the gamma spectrum is relatively stable around 45 MeV for all collision shots. A significant fraction of radiation seems to be emitted in all cases (relative to the background) and skew the energy estimate to this average number. If we relate these numbers to the electron beam measured, we can from both sides narrow down the exact conditions at the interaction. In addition, the critical energy of the gamma spectrum seems to be anti-proportional to the energy of the electron beam. The opposite correlation is expected if the electron energy was converted into radiation after the spectrometer screen as in the interaction with the vacuum chamber resulting in bremsstrahlung. This is another indicator that the energy loss and the measured high energy radiation are related. The probability for such a negative correlation by chance is XXX OUT OF XXXX.
\vspace{\baselineskip}

Both measurements (gamma-rays and electron energy loss) seem to be indicating that the laser intensity was around $a_0 \approx 10$ instead of the peak intensities of close to $a_0 \sim 25$ reached by the laser.

\subsubsection{Explaining lower laser intensity at the interaction}

The intensity of the laser pulse is significantly lower than the peak $a_0$ achievable in this geometry. This indicates XX.

The two laser pulses were synchronised at vacuum using spatial interferometry to an estimated accuracy of $30\,\mathrm{fs}$. This timing is only valid if both laser pulses travel through vacuum. In a shooting scenario, however, the driving laser pulse propagates through plasma to accelerate electrons via LWFA. The propagation in a medium reduces the group velocity of the laser pulse and delays its arrival relative to the vacuum timing.
In addition, we are trying to overlap the scattering beam with the electrons accelerated by the driver pulse, not the driver itself. The LWFA electron bunch trails behind the driving laser pulse and arrives even later at the interaction point. Since the scattering beam arrives before both at its focal plane and the designated interaction point, the real collision will occur beyond this point and at a slightly defocused spot size. 
To estimate the intensity of the laser pulse at the interaction we have to determine the real collision point and the size of the scattering beam at this plane. 
\vspace{\baselineskip}

The following analysis is based on Jason Cole (Imperial College) for \cite{Cole2018}.
We assume that the laser pulse travels through the plasma at the standard non-linear group velocity reduced by the etching velocity, $v_f \approx 1-\frac{3}{2}\frac{n_e}{n_c}$ (called front velocity in REF DECKER 1996 Phys Plasmas). Given a medium of thickness $d$ the laser pulse of the scattering beam travels in the same time a distance  $d'=d c/v_f > d$.

Both laser pulses now meet a distance $\delta z_1$ past the focal plane of the scattering beam:

\begin{equation}
\delta z_1 = \frac{1}{2} \left(d \frac{c}{v_f} - d\right) = \frac{d}{2} \left(1/\left(1-\frac{3}{2} \frac{n_e}{n_c}\right) -1\right) \approx \frac{d}{2}\left(1+\frac{3}{2} \frac{n_e}{n_c} -1\right) = \frac{3d}{4}\frac{n_e}{n_c}.
\end{equation} 

The electrons in turn trail behind the driving laser pulse around N plasma wavelengths and meet the scattering pulse midway:
\begin{equation}
\delta z_2 = \frac{1}{2} N \lambda_p = \frac{1}{2} N \frac{2 \pi c}{\omega_p} = \frac{1}{2} N \frac{2 \pi c}{\omega} \frac{\omega}{\omega_p} = N \frac{\lambda_0}{2} \sqrt{\frac{n_c}{n_e}}
\end{equation}

If we assume that the acceleration is close to its dephasing limit, $N = 1/2$.
Under these assumptions the electron bunch and the scattering beam meet at $\delta z$ distance from the intended collision point.

\begin{equation}
\delta z = \frac{3d}{4} \frac{n_e}{n_c} + N \frac{\lambda_0}{2}\sqrt{\frac{n_c}{n_e}},
\end{equation}
where in this context $d$ is the distance of the injection point from the front of the gas jet.
\vspace{\baselineskip}

For a density of XX ...
This amount of defocus then reaches an average intensity of NUMBER 12 instead of the peak intensity of close to 25 possible in this setup. We expect that the injection point varies depending on the evolution of the laser pulse on top of timing jitter. The fluctuation of relative timing and spatial overlap then results in a spread of interaction conditions.

\subsubsection{Estimating Collision Probability based on Jitter}

With the knowledge of the focal spot size at the interaction and realistic timing fluctuations, we can estimate the number of expected collisions in these conditions.
This has been done by Jason Cole (Imperial College) and Chris Baird (York) by inserting the jitter in time and space of the electron beam and the laser, in one case running a simple Monte-Carlo simulation, in the second case building this on the amount of overlap achieved in PIC simulations.

The number of successful collisions was XXX which is close to the 4 identified successful collisions.
\vspace{\baselineskip}


\subsubsection{Agreement of results with models of radiation reaction}

After having now narrowed down the conditions at the interaction, confirmed that the frequency of the interactions and the selected shots appear to match up, we can now look at the detailed relations of these shots and how they line up with actual models.


If we plot the electron spectrum (assuming post-interaction) against the critical energy of the gamma-ray spectrum, we see that they correlate negatively. This again supports the hypothesis that we are looking at ICS with radiation reaction. The background signal from bremsstrahlung, for instance, emits the radiation after the electron spectrometer and hence we would expect a positive correlation (a harder spectrum for higher energies). The fact that we see a negative correlation confirms our suspicion that the energy loss is converted into a gamma-ray signal before the measurement.

Now here some part about the models and how they relate, how the plot is made but this is all Jason's and Tom Blackburn's work and should be referenced as such.
The on-shot synchrotron-like spectrum and the shift of the cut-off edge in the electron spectrum were then combined to constrain the parameter space (normalised laser vector potential $a_0$ and electron energy) at the interaction, relying on a range of models.



\begin{figure}[h]
\centering 
\includegraphics[width=0.7\columnwidth]{EcritvsEf_withmean.pdf}
\caption[]{Adjusted from Figure 9 in \cite{Cole2018}.
}
\end{figure}


\EliasComm{Explanation of models}
\vspace{\baselineskip}

Some comments on the theory contours.
The data shows that the results match a model that requires radiation reaction. No radiation reaction overestimates the energy loss and the energy of the radiated gamma radiation. Our results are in agreement with all the models in question, more overlap with the models including quantum corrections which could be a semi-classical model with classical trajectories and merely reduced emitted power or a fully stochastic quantum model. This is all on a one sigma confidence level so a strong distinction is not possible, we require more data.
An analysis how much more data we would require to make a conclusion beyond the 2 sigma level will follow in the next section.
\vspace{\baselineskip}

In addition it becomes clear that the difference between a fully stochastic quantum model and a semi-classical model is not very visible in this case. There is in indication that the models depart at higher energy losses, i.e. higher laser intensities but it would also require a higher electron energy. This slow departure from each other in contrast to the relatively distinct classical model is founded in the electron observable which is the energy loss from a characterised mean. When looking at the mean energy loss for the quantum and the semi-classical model it becomes evident that by definition the values are identical. Even though the energy loss of the spectral feature is not the same thing it closely relates to the mean energy loss. A more sensitive observable that behaves very differently for both models is the variance of the spectrum or the shape of the spectrum. This will be closer illuminated in the next section. 
\vspace{\baselineskip}

\section{Electron beam stability and model distinction}

In the framework of the data set presented first indications of differences in the models become evident, but a definite discrimination is not within its reach. More data, stability, higher electron energies, laser intensities and additional observables could enable this as part of a future precision measurement.

The following section will highlight the importance of a stable electron source by comparing the data set presented (Cole et al. \cite{Cole2018}) with a second measurement of radiation reaction, performed at the same facility using a gas cell target (Poder et al. \cite{Poder2018}). For this purpose, the electron spectra of the second experiment undergo a statistical analysis similarly as described before.

\subsubsection{Experimental setup}

The experimental setup in \cite{Poder2018} is very similar to the experiment described at the beginning as it is based on the same facility, relying on similar optics and beam line designs. A sketch of the setup is show in Figure \ref{RR15:figs:exp_sketch_Poder}.


\begin{figure}[h]
\centering 
\includegraphics[width=0.9\columnwidth]{ExpSetup_Poder.pdf}%\includegraphics[width=0.9\columnwidth]{ExpSetup_GasJet_MagnetOnly.pdf}
\caption[]{Conceptual sketch of the experimental setup as used on the radiation reaction campaign \cite{Poder2018}. Adapted from Figure XX in \cite{Cole2018}.
%(from left to right): a laser pulse (in red) is focused by an f/40 spherical mirror onto the entrance of a gas target (gas jet or gas cell). The intense laser pulse drives a wakefield and accelerates electrons (blue) to relativistic energies. A second laser is focused with an f/2 off-axis parabola onto the exit of the gas target scattering the electrons and emitting a bright flash of gamma rays (green) from inverse Compton scattering. A permanent dipole magnet is used to disperse and characterise the energy of the electron beam on a scintillating LANEX screen. The gamma rays propagate through a kapton vacuum window (orange) onto a stack of caesium-iodide (CsI) crystals. The sketch is based on work by J. M. Cole, Imperial College, for \cite{Cole2018}.}
}
\label{RR15:figs:exp_sketch_Poder}
\end{figure}

The first laser is focused by an f/40 spherical mirror to a spot of \textsc{fwhm} dimensions $(59 \pm 2) \,\mathrm{\mu m} \times (67\pm 2)\,\mathrm{\mu m}$ into the $20\,\mathrm{mm}$ long gas cell filled with helium at an electron density of $2 \times 10^{18}\,\mathrm{cm}^{-3}$. The energy delivered on target is on average about $9\,\mathrm{J}$ reaching a peak normalised vector potential of $a_0 = 1.7$.
\vspace{\baselineskip}

The focal plane of the colliding laser was positioned approximately 1 cm downstream from the exit of the gas cell. The focusing optic was the identical f/2 OAP with a central hole as in the previously described setup. The energy on-target was measured to be $(8.8 \pm 0.7)\,\mathrm{J}$, already taking into account the loss in intensity due to the hole, at a \textsc{fwhm} pulse duration of $42\pm 3 \,\mathrm{fs}$. The intensity at the interaction point was $a_0\approx 10$ at a spot size \textsc{fwhm} $7\,\mathrm{\mu m}$.
\vspace{\baselineskip}

Both lasers were synchronised to about $40$ fs accuracy using spectral interferometry \cite{Corvan2016}.
\vspace{\baselineskip}

The electrons accelerated via LWFA are dispersed by a dipole magnet of integrated field strength $\int B(x) \mathrm{d}x \approx 0.15 \,\mathrm{Tm}$ onto a scintillating LANEX screen.
The gamma rays from ICS are measured by the same scintillator stack as described previously, but this time it is rotated such that the long side of the crystals is oriented in the longitudinal direction. The stack acts as a profile screen instead of a spectrometer and the brightness of the measured scintillation light is proportional to the total energy deposited in the crystal stack REF HERE.

\subsubsection{Characterisation of Electron Spectra}

A typical electron spectrum produced from the gas cell target can be seen in Figure \ref{fig:Poder_espec_example}. The average shape is an exponential spectrum reaching energies in excess of $1.5\,\mathrm{GeV}$. At the lower end the spectrum is cut off at around $400\,\mathrm{MeV}$ due to limitations in the magnetic electron spectrometer setup. The spectra from the gas cell originate from self-injection and lack a consistent distinctive feature comparable to the spectral edge described in the previous section related to shock injection. The data set considered consists of 19 shots.

\begin{figure}
\centering
\includegraphics[trim={4.8cm 0 5cm 0}, clip, width=0.9\columnwidth]{Example_RR15Poder.png}
\caption[]{Example of a typical electron spectrum.}
\label{fig:Poder_espec_example}
\end{figure}

The characteristic number assignted to each spectrum is the cut-off energy, which is as in \cite{Poder2018} defined as the energy at which the spectral intensity reaches 10 percent of its peak value.
\vspace{\baselineskip}

In this case, the only injection mechanism considered is self-injection. This process typically scales with the laser energy (REF), which is also confirmed in Figure \ref{fig:Poder_laser_corr}. The laser energy of the wakefield driver scales well linearly with the cut-off energy of the spectrum. The correlation is very strong at a correlation coefficient of $0.9$. The linear fit follows the equation $\epsilon_{cutoff} = 0.07\,\mathrm{GeV/J} \times E_{laser} + 0.57\,\mathrm{GeV}$.

After scaling the spectra according to the linear relation found in Figure \ref{fig:Poder_laser_corr}, we can analyse again the `intrinsic' fluctuations of the source. The distribution of cut-off energies in terms of $\Delta E/E$ follow a normal distribution after removing the correlation with the laser energy. The standard deviation of the cut-off energy distribution is $\sigma_{Poder} = 0.035$, so 95 percent of the expected energies will fall within $\pm 7\,\%$ of the mean energy.

\begin{figure}[h]
\centering
\includegraphics[width=0.8\columnwidth]{Poder_Laser_Corr.pdf}
\caption[]{Laser energy of the wakefield driver beam before pulse compression plotted against the cutoff energy of the electron spectrum. The data points clearly follow a linear trend drawn in the straight green line with a gradient of $0.07\,\mathrm{GeV/J}$. The shaded area indicates the 95 percent confidence interval of the fitting function. The correlation coefficient for a linear fit is $0.9$.}\label{fig:Poder_laser_corr}
\end{figure}


\subsubsection{Comparison of intrinsic fluctuations in both datasets}

By factoring out known correlations of the spectra, scaling them accordingly (slow drift for Experiment A (Cole et al.) and laser energy for Experiment B (Poder et al.)), and by normalising the spectra to their total charge, it is possible to compare the spectra. With this processed data set expected intrinsic or non-attributed fluctuations can be characterised and taken into consideration.


\begin{figure}
\centering
\includegraphics[width=0.8\columnwidth]{Comparison_ESpectra.pdf}
\caption[]{Averaged and scaled electron spectra for data from Cole (blue) and Poder (green). The spectra are normalised to a total charge of 1. The error bars indicate the energy dependent variance of the spectra.}\label{fig:comparison_especs}
\end{figure}

The averaged spectra from the two experiments are shown in Figure \ref{fig:comparison_especs}. The spectra were processed as outlined before and then averaged over all available data shots. The spectra are normalised by their total charge such that the integral of each spectrum is set to 1. As seen the typical energy reached in the gas cell from Poder et al. is significantly higher by at least a factor two for the majority of the charge distribution. The shaded error bar region around the scaled spectra indicates the standard deviation of the averaged spectrum at that particular energy. The higher electron energy also enables a potentially higher $\eta$ parameter in an interaction as it scales linearly with the electron energy.
\vspace{\baselineskip}

\begin{figure}
\centering
\includegraphics[width=0.8\columnwidth]{Comparison_Histo_errors.pdf}
\caption[]{Distribution of the relative energy deviations from the mean ($\Delta E/E$ for Cole et al. (blue) and Poder et al. (green) after scaling. The overlaid lines are kernel density estimates (KDEs). The shaded areas indicate the $\pm2$ sigma intervals assuming a normal distribution.}\label{fig:comparison_histo}
\end{figure}


The spread of cut-off energies for the gas cell data is narrower than the spread of energies of the edge feature produced with the gas jet target. The shaded areas in Figure \ref{fig:comparison_histo} indicate the $\pm 2$ sigma or 95 percent confidence intervals (assuming a normal distribution). The Poder data has a typical fluctuation of the cut-off energy of around 7 percent. The spectral feature from the gas jet varies with 15 percent by around the double amount. 


\begin{figure}
\centering
\includegraphics[width=0.5\columnwidth]{Comparison_Variance_Log.pdf}
\caption[]{Energy-dependent variance of the average spectra in introduced in figure 8. In blue the data taken on the experiment related to Cole et al., in green to Poder et al.. The total variance for Cole et al. was $1.5 \times 10^{-7}$ and $1.98 \times 10^{-8}$ in the case of Poder et al..}
\label{fig:comparison_varianceLog}
\end{figure}

Also discuss the variance of the spectra in this section.

\subsubsection{Relevance for measurements of radiation reaction}

Energy loss through LL given by THOMAS PRX 2012

\begin{equation}
\frac{\Delta \gamma}{\gamma_0} = \frac{\sqrt{\pi/2}\tau_0 t_L \omega_0^2 \gamma_0 a_0^2}{1+\sqrt{\pi/2 \tau_0 t_L \omega_0^2 \gamma_0 a_0^2}}
\end{equation}

$\tau_0 = 2 e^2/3m_e c^3 = 6.4 \times 10^{-24}\,\mathrm{s}$, pulse duration $t_L = 45\,\mathrm{fs}$, wavelength be $\lambda = 8--\,\mathrm{nm}$.

\begin{figure}
\centering
\includegraphics[width=0.5\columnwidth]{Histo_Cole_RR.pdf}\includegraphics[width=0.5\columnwidth]{Histo_Poder_RR.pdf}
\caption[]{.}
\end{figure}

Z-test:

\begin{equation}
Z = \frac{\bar{X_1} - \bar{X_2}}{\sqrt{\sigma_{X_1}^2 - \sigma_{X_2}^2}}
\end{equation}

$\sigma_X = \sigma/\sqrt{n}$ is standard deviation on the mean.

We assume an equal number of collisions and reference data $n$. If the collision likelihood is 1 in 3, the total number of shots is then $4n$.

\begin{equation}
n = Z^2 \frac{\sigma_1^2 + \sigma_2^2}{(\bar{X_1}-\bar{X_2})^2}
\end{equation}

\begin{figure}
\centering
\includegraphics[width=0.8\columnwidth]{Nshots_5sigma_RR.pdf}
\caption[]{.}
\end{figure}

For $5\sigma$ confidence $Z = 5$.

Use determined standard deviations and normal distribution of electron energies from the two datasets.
Based on those values, sample energies and calculate the energy after an interaction for the $a_0$ measured in experiment.
From these samples estimate the mean energy and standard deviation of the distribution after the interaction.
Use the Z-test to compare both distributions and estimate how likely it is to assign this distribution to the wrong model (in this case no RR).
\vspace{\baselineskip}

See that for the same mean energy the change in standard deviation affects how much overlap both distributions have and the likelihood of misattributing them. At the conditions for Cole we need of XX 15 NUMBER shots to reach a 5 sigma confidence that this is a distribution from radiation reaction.
For Poder at the same relative standard deviation would need of factor 10 fewer shots. At the current Poder conditions (energy) one shot would be sufficient under the assumption that the electron beam itself is characterised well enough. At $a_0 = 20$ both conditions are sufficient to quickly distinguish models within a shot, even for relatively large energy fluctuations of 10 percent.

\begin{figure}
\centering
\includegraphics[width=0.5\columnwidth]{Nshots_5sigma_RR_a0_20.pdf}\includegraphics[width=0.5\columnwidth]{Nshots_5sigma_RR_a0_range.pdf}
\caption[]{.}
\end{figure}

Some comment on that quantum model has smaller shift in energy, so need more shots actually.
Higher intensities shift significantly, but more challenging. So a good source helps.
Between the models need other observables, but explained in XX REFs.
Some comments on why number of shots is important: 4n shots. At repetition rate scans really take some time, drift and overlap.
Single shot means around 10 shots with characterisation. At the 10 levels we are at 50 shots in around 17 minutes, for 500 shots (100 collisions) 170 minutes, so almost 3 hours. Conditions might change and laser parameters etc., target exhaustion.

Motivate that looking at Jason's plot the quantum and semi-classical model is very close to each other. This is due to the matched emission power. Stochasticity and other factors will eventually make it possible to discriminate them, but since the parameters diverge very slowly the number of shots will remain high. Other parameters are more sensitive, see Theory and RIDGERS REF and other REFS.
One example is the shape. The variance is superior for Poder (XX against XX), but this will not be elaborated more in detail here.
\vspace{\baselineskip}

Finally, it should be noted that the interaction is more complicated. 
Knowledge of exact on-shot laser profile in time and space.
Knowledge of the electron beam properties in time and space, chirp.
Variation of spatial and temporal overlap.
Eta is not constant over the course of the interaction due to defocusing, energy loss, chirp in the beam, energy spread.
Also modelling constraints in the theory (LCFA).
\vspace{\baselineskip}


This line of thought has been explored more explicitly and in more detail in REF C ARRAN PPCF, SPIE 2019. The aim of these publications is to indicate which experimental observables are suited to discriminate different models of radiation reaction and to give an estimate on the number of shots one would require in an experiment taking fluctuations into account.

C ARRAN uses a Bayesian inference approach to determine how closely related the predictions of the different models are, the overlap of the models.


\iffalse

This means that if our observable is coupled to the energy of the electrons we will require a certain number of shots based on the typical variability of the source to say with a degree of confidence that a signal beyond this has been recorded. If we couple this with the probability to even collide two beams at all, the number of required additional shots increases by a significant amount.
\vspace{\baselineskip}

To put this into context: in \cite{Cole2018} 4 successful collisions were identified by consulting the gamma-ray detector signal which showed a bright signal, $5-10$ standard deviations above the expected background. The energies of these 4 successful collisions were about $15\,\%$ below the mean of the reference distribution. The cumulative probability for a normal distribution with standard deviation $\sigma_{Cole} = 0.077$ to see one shot with a relative deviation of the energy from the mean of $10\,\%$ or more is roughly $10\,\%$. 


For an individual shot this probability is non-negligible, but it holds in combination with the gamma-ray signal and over the $4$ collisions to confirm that this is to some extent as low due to radiation reaction. The cumulative probability for a normal distribution with standard deviation $\sigma_{Poder}=0.035$ on the other hand for the same scenario is only $0.2\,\%$ and holds the test by itself at a level of confidence equivalent to 2.7 as many shots at $\sigma_{Cole}$. At a repetition rate of $0.05\,\mathrm{Hz}$ (1 shot/20 s), typical for the Astra Gemini laser, a scan of 100 shots at $\sigma_{Poder}$ would take under 30 minutes, but would extend to $1.5$ hours to reach the same level of confidence. At this extended time-scale, however, even small drifts in alignment and timing can affect the reproducibility of the interaction conditions which requires few microns and femtoseconds precision and as a result call the validity of measurements into question. In addition, we have to factor in the potential deterioration of the gas target and laser performance over the course of the experiment. Time can become a crucial factor in experiments that require such a high level of complexity and sensitivity.

However, the error on the mean of the electron energy also propagates further into the estimate of the interaction conditions. If we manage to sustain the experiment conditions over the course of the experiment (spatio-temporal overlap, intensity of interaction, laser parameters and so on), the energy of the unperturbed electron beam is fluctuating according to a normal distribution and the energy of the post-interaction electron beam will follow the same normal distribution but shifted down by the average energy loss. For now we will ignore the potential cooling or heating effects of radiation reaction on the distribution function. In this scenario at a constant energy loss of $15\,\%$, we will measure a range of deviations from the unperturbed mean energy which we will call `energy loss', which is in reality the energy loss superimposed onto the normal distribution. The standard deviation of the Cole data $\sigma_{Cole}$ is half the size of the total energy loss, whereas for $\sigma_{Poder}$ it halves to $25\,\%$, which is still a significant range of values most of the measurements will show. That means that there is a significant error on the estimated energy loss for each measurement and a second diagnostic narrowing down the conditions at the interaction, for instance a spectral measurement of the gamma radiation, becomes imperative.

If we want to compare different models of radiation reaction based on this energy loss, an increased stability of the electron energy will help in two ways: firstly, the expected phase space of post-interaction energies for each model of radiation reaction at a fixed confidence level decreases, which might reduce the overlap of different models. Secondly, if the initial electron energy is sampled from a narrower distribution the error on that measurement decreases as well, reducing its footprint in phase space again.
\vspace{\baselineskip}


In some cases, however, even an increased level of confidence and more data is not resulting in an improved distinction of different models of radiation reaction. As seen in recent publications REFS HERE RIDGERS, measuring the difference between classical and models including some form of quantum corrections can be done based on the energy loss due to the reduced emission power in quantum systems (REF).

When trying to distinguish fully stochastic quantum models and a semi-classical model which includes a reduced emission power matched to the quantum model (Gaunt-factor), but uses a deterministic equation of motion, the emitted power is matched by definition and hence not a suitable figure of distinction. 
Instead, the shape of the spectral distribution, e.g. variance of the entire spectrum (not only of the energy of the spectral feature), becomes more important and another factor to discriminate the two, but again would require a certain level of stability and suitable measure of it. The stability of the spectral shape can, for instance, be quantified in terms of the energy dependent variance of the spectra in question. 

The total variance normalised to the energy range considered for the average gas jet data from Cole et al. is $1.5\times 10^{-7}$ whereas the result for Poder et al. is $1.98 \times 10^{-8}$. The energy dependent variance can be seen in Figure \ref{fig:comparison_especs} represented by the shaded error bars and in Figure \ref{fig:comparison_varianceLog} separately on a logarithmic scale. The base level of the variance is relatively low for the Poder data and decreases for high energies. For the Cole data especially the position of the characteristic feature, crucial for the two-beam measurements, appears prone to fluctuations. This makes the data set less suitable to track changes in the variance induced by radiation reaction. If the fluctuation is of similar amplitude or larger than the differences expected from the models, a definite distinction of a semi-classical and a fully stochastic quantum model at these conditions would then be very challenging, even with access to a larger data set. In Poder et al. a comparison of spectral shapes and attribution to models has qualitatively been indicated. However, due to the lack of a spectral measurement of the radiation and the large parameter space of unknowns, for instance spatio-temporal structure of the electron beam, more observables are required to definitively explain the differences between the observed electron spectra and predictions of various models.

\fi



\subsubsection{Conclusion}

From this analysis it appears that a gas cell is a better target and the energy dependencies can be easier correlated and factored out. On the other hand, the distinct feature and injection point of the electrons from shock-injection could be useful to maintain similar conditions over longer time. If the injection mechanism could be improved in terms of stability by a factor 2, it could outweigh the benefits of the gas cell.
In addition, REF C ARRAN indicates that low energy spread beams of suitable stability are in particular favourable for measurements of radiation reaction as they reduce the requirements for higher electron and laser energies. Literature shows that these are achievable through shock injection. Ideally a more stable low energy spread beam would be suited to discriminate quantum and non-quantum models.

Finally, a detailed knowledge of the exact interaction conditions is crucial and removes further uncertainty.
A gamma-profile arrangement as set up in Poder et al. allows a relatively model-independent measurement of the interaction intensity, whilst a spectral measurement of the ICS radiation might help to discriminate a classical radiation reaction model from quantum models that lead to a reduced emission power. Using both diagnostics on future experiments could greatly improve the understanding of the interaction and as a result improve confidence in modelling.

\section{Future Work}

Shock injection?

The expertise gained in these previous experiments paired with the technological capacities of the Astra-Gemini laser make it a realistic ambition to collide electrons at even higher relativistic energies ($\sim 2 \mathrm{GeV}$) and more intense laser fields ($a_0 \approx 25$) even at current laser facilities, expecting a gamma-ray spectrum with a critical energy close to $100\,\mathrm{MeV}$. These would enable probing radiation reaction at even higher energy losses -- possibly even reaching a regime where QED effects become relevant as the `quantumness' parameter $\eta$, indicating how much the interaction is dominated by QED effects \cite{Blackburn2014} (maybe another REF), approaches values close to $1$. In this regime we even expect gamma rays to combine with laser photons to produce electron-positron pairs (REF).
At the same time, this work in conjunction with C ARRAN REF indicates that by improving the electron beam quality differences in models can be measured even at comparable laser and electron energies, giving future experiments at Gemini or comparable laser facilities the opportunity to bridge the gap between these first measurements of radiation reaction and the future projects being worked on at the next generation experiments.

Future experiments at conventional facilities SFQED (SLAC), LUX (Hamburg), Cala (Munich) but also at high intensity laser facilities at even more non-linear conditions at CORELS (Gwangju), ELI pillars, Munich (FOR), maybe Gemini again, EPAC will be able to take precision measurements.

\section{Conclusion}

\subsubsection{Radiation Reaction}

Radiation reaction was for the first time measured in an all-optical setup by colliding a tightly focused laser beam of $a_0 \approx 10$ with a relativistic electron beam from LWFA of energy $\sim 500\,\mathrm{MeV}$ \cite{Cole2018} and up to $2\,\mathrm{GeV}$ \cite{Poder2018}. 

In \cite{Cole2018} the critical energy of the radiation reached an excess of $30\,\mathrm{MeV}$, the highest recorded gamma-ray energies from an all-optical setup at the time of the publication.

The measurements were in agreement with models including radiation reaction and indicate a better agreement at the $1\sigma$ level with models including quantum corrections. The intrinsic fluctuations of the experiment and the low number of successful collisions do not allow more confident association to one specific model.
\vspace{\baselineskip}

Successful high-intensity collisions are signalled by bright bursts of gamma radiation recorded on the gamma detectors.
The expected radiation background from bremsstrahlung was found by correlating the electron spectra and the signal on the gamma-detector on shots without the scattering laser. This characterisation helped identifying 4 particularly intense interactions that were considered in more detail.
\vspace{\baselineskip}

The energy of the electron spectra on those 4 identified shots was found to be on average lower than on the other shots.
By performing a statistical analysis of a larger set of electron spectra, investigating dependencies on other parameters such as slow drifts and laser energy, the likelihood of seeing these lower energies by chance was estimated and deemed significant.
\vspace{\baselineskip}

This information was combined with a spectral analysis of the gamma signal to confirm the conditions at the interaction from two avenues. The critical energy of the gamma spectra on the 4 successful collisions are negatively correlated with the measured electron energy. This is another indicator for radiation reaction and the probability to see this alignment by chance is XXX. 
\vspace{\baselineskip}

\subsubsection{Electron beam stability}

The statistical analysis of the stability of the electron beam has shown to be useful to estimate the significance of lower electron energies, but also provides a powerful tool to quantify the scale of data required to distinguish different models of radiation reaction.

A comparative analysis of the stability of electrons from a gas jet and a gas cell target, both used for radiation reaction measurements, was used to demonstrate it affects the scale of data required to reach a certain level of confidence in a measurement of radiation reaction .
In this case, the stability of the electron beam energy and of the spectral shape was superior for the data using a gas cell target \cite{Poder2018}, with half of the standard deviation on a normal distribution for the energies and a 10 times smaller total variance compared to the gas jet target used in \cite{Cole2018}.

The improved stability has to be weighed against the advantages of using a gas jet target, which are the ability to perform very close interactions without producing debris or damaging the target, as well as easy diagnostic access. The possibility to use shock injection for improved control over the injection point and potentially a lower energy spread is also speaking for gas jet targets.

This method can be expanded to other source parameters, including the gamma spectra, to identify observables that discriminate models more efficiently.
This is beyond the scope of this thesis, but can be found in C ARRAN REF.
