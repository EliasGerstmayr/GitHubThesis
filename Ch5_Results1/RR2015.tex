\chapter{First Measurements of Radiation Reaction at Astra Gemini}

\section{Motivation}

\EliasComm{Maybe start off by saying why ICS is promising and useful, then that it can be used to investigate radiation reaction.}
Inverse Compton scattering provides a promising route to generate bright burst of high energy radiation reaching 100's of keV or even MeV photon energy.
In ICS, the scattered photons are Doppler-shifted and re-emitted at a higher energy depending on the electron energy and the intensity of the laser field (see Theory REF for further details). Higher electron energies result in a stronger shift of the radiation, leading to a hardening of the spectrum and a corresponding energy loss of the electrons (REF). A more intense laser field enables non-linear interactions and increases the number of photons interacting with at once with an electron (REF), resulting in the generation of higher harmonics. The energy of the radiation is maximised in a head-on collision due to the Lorenz boost.

The high photon energy makes this not only a promising radiation source but offers the opportunity to measure radiation reaction if the emitted photon carries an energy comparable to the energy of the electron.
\vspace{\baselineskip}

Colliding-pulse experiments using LWFA have been successfully performed by different groups over the past years, gradually improving the control over the process and increasing the energies of the electrons and the radiation generated \cite{TaPhuoc2012a,Chen2013a,Khrennikov2015,Powers2014,Sarri2014}.
An indicator for the progress of these experiments, in terms of laser intensity and electron energy at the point of interaction, is the X-ray or gamma-ray spectrum produced through relativistic inverse Compton scattering (ICS) (REF). 
\vspace{\baselineskip}

In earlier experiments, a single laser pulse was used to accelerate electrons to relativistic energies and to scatter the electrons after reflecting the beam from a plasma mirror (e.g. tape \cite{TaPhuoc2012a}). In such a setup the laser beam and the electrons are timed intrinsically. A close to normal overdense plasma surface will then be able to overlap the laser beam and the electron bunch spatially. In \cite{TaPhuoc2012a} electrons of an energy around $100\,\mathrm{MeV}$ were collided at a laser intensity of $a_0 \approx 1.2$ (mildly non-linear). The radiation measured was broadband and reached up to $100\mathrm{s}\,\mathrm{keV}$. A lower energy spread and tunability was demonstrated at slightly lower electron energies and laser intensities in \cite{Powers2014} and\cite{Khrennikov2015}.
\vspace{\baselineskip}

Whilst this technique avoids issues with timing and overlap, the intensity of the laser pulse is limited as it is partly depleted when interacting with the electron bunch after driving a wake. At the same time the acceleration can not be pushed to its depletion limit as the laser pulse has to be remain intense enough for a suitable interaction. Further problems might arise as controlling or measuring the wavefront of the depleted laser pulse might be challenging.

Two separate lasers, one to accelerate and one to scatter off the electrons, offer the opportunity to achieve higher intensities and electron energies at the interaction and more control over the interaction. However, it is challenging to overlap the micron-sized electron bunch with the ultrashort laser pulse, probably focused down to a micron-scale focal spot to achieve high intensities, in time and space, and to maintain that alignment over an extended period of time.

In \cite{Chen2013a} the relativistic electrons were scattered at an angle of $10$ degrees and gamma rays at an energy of around $1\,\mathrm{MeV}$ were produced.
In \cite{Sarri2014} electrons at an energy of $400 \, \mathrm{MeV}$ were successfully collided at an laser intensity of $a_0 \approx 2$, resulting in broadband radiation extending up to $10\,\mathrm{MeV}$.
The energy loss of the electrons was in this case not significant enough to be able to probe radiation reaction, but is clearly non-linear.
\EliasComm{Ref Yan, Nat Photonics on multiphoton Compton.}
\vspace{\baselineskip}

There has been an effort to measure radiation reaction at conventional accelerator facilities where electron energies are high and stable. Experiment E-144. Linear regime though. A revived effort will look into high $\eta$ and high $a_0$. Up to now laser facilities have been pushing the very non-linear regime slowly reaching the quantum regime.
\vspace{\baselineskip}

The current work related to (REF Cole) succeeded in colliding electrons of around $550\,\mathrm{MeV}$ at a laser intensity of $a_0 \approx 10$, reaching critical gamma ray energies in excess of $30\,\mathrm{MeV}$. This was the highest gamma-ray energy from an all optical ICS source published at that point. This consists the first published measurement of radiation reaction in an LWFA setup (REF PRX Cole). The work of (Poder REF) collided electrons of around $2\,\mathrm{GeV}$ at similar intensities but lack a spectral measurement of the gamma rays to complete the picture.
By now the highest recorded gamma-rays from ICS in LWFA have been achieved by REF by using back-reflection of the laser pulse again and colliding them with electrons at energies up to $2\,\mathrm{GeV}$.

\EliasComm{Make a plot of electron energy and laser intensity, and place our work on this plot.}
\EliasComm{Reference the papers. See plot and papers from notebook.}
\EliasComm{Reference J Shaw paper on 85 MeV Compton?}
\EliasComm{Consider adding a contour for gamma-ray energies. Approximately (factors of a few missing due to varying laser photon energy).}
\EliasComm{Add a shaded area indicating where Gemini could operate, maybe indicate where ELI wants to operate at in the future.}

\begin{figure}[h]
\centering
\includegraphics[width=0.8\columnwidth]{RR2015_Eta_thiswork.pdf}
\caption{Some plot to place our work into context. Shade area for Gemini.}
\end{figure}

After a brief introduction to place this work into context, the author will describe the experiment setup, then characterise the electron spectrum and the signal on the gamma-ray detector for the shots without the scattering beam (`beam off'). These act as references for the colliding beam shots. Then these references will be compared to the shots with colliding beam shots and a clear deviation from those and an agreement with radiation reaction models is shown.

\vspace{\baselineskip}
The work presented within this chapter represent excerpts of analysis work done by the author, mainly related to process the data into a format which facilitates further analysis including the subtraction of background, correcting images for relative tilts in the optical systems used in experiment, scanning through data sets, identifying promising cases and categorising them by assigning characteristic quantities to them.

Other parts of the analysis, in particular the retrieval of the gamma spectrum of the inverse Compton signal, were performed by Jason Cole (Imperial College London) and Keegan Behm (University of Michigan). Their contributions will be acknowledged and indicated as such at the relevant parts of this chapter.

Finally, the author compares these results with another measurement of radiation reaction at higher electron energies performed a the same laser system a few weeks earlier. The results of this campaign have been published in (REF Poder). In particular, the author will compare the stability of the electron energy and the stability of the electron spectrum to draw conclusions for future experiments and indicate which parameters and scale of data points would be required to distinguish different models of radiation reaction with statistical significance.


\section{Experimental Setup}

The experiment described in the following was performed at the dual $300\,\mathrm{TW}$ Astra Gemini laser system at the Central Laser Facility, Rutherford Appleton Laboratory, UK, in late 2015.

A sketch of the experiment is shown in Figure \ref{RR15:figs:setup_sketch}.
\vspace{\baselineskip}

\begin{figure}[h]
\centering
\includegraphics[width=0.8\columnwidth]{Exp_setup_render_RR2.png}
\caption{Sketch of the experiment setup at the Gemini laser facility.
Based on a sketch made by J. Cole (Imperial College) and adapted for this work and similarly for K. Behm et al. REF SPECTROMETER 
%From left to right: a high intensity laser beam (red) focused with a f40 spherical mirror generates up to $\mathrm{GeV}$-scale electrons in a gas jet (LWFA). A second laser beam (red) is focused down tightly at the edge of the gas jet by an f2 off-axis parabola to scatter the electron beam shortly after it leaves the jet. The electrons (blue) are being dispersed by a dipole magnet and detected on a scintillating lanex screen (grey). Finally, gamma rays (green) propagate through a vacuum kapton-window and a lead aperture onto a stack of scintillating CsI crystals that are imaged by a camera.
}
\label{RR15:figs:setup_sketch}
\end{figure}


The first part of the experiment is a setup typical for laser wakefield acceleration (LWFA): a laser beam is focused by an f/40 spherical mirror with $6\,\mathrm{m}$ focal length onto the leading edge of a $15\,\mathrm{mm}$ conical supersonic helium gas jet target. Electrons are accelerated via LWFA and propagate further downstream where they are dispersed by a permanent dipole magnet of integrated field strength $\int B \mathrm{d}x = 0.4\,\mathrm{Tm}$ onto a scintillating LANEX screen imaged by an Andor Neo camera to measure their spectrum. The typical \textsc{fwhm} focal spot of the driving laser pulse measures $37 \times 49 \,\mathrm{\mu m}$ with an energy on target of $(8.6 \pm 0.6)\,\mathrm{J}$, which corresponds to a normalised vector potential $a_0 = 1.9 \pm 0.1$. The electron density of the target was $(3.7 \pm 0.4) \times 10^{18} \,\mathrm{cm}^{-3}$. PULSE DURATION, PEAK POWER XX NUMBER.
\vspace{\baselineskip}

The second laser beam is focused down tightly onto the opposite edge of the gas jet using an f/2 off-axis parabola (OAP) to scatter from the electron bunch accelerated by the first laser. This generates a bright burst of gamma rays from inverse Compton scattering (ICS). As this beam counter-propagates with the electron bunch, designed for a head-on collision, the focusing optic lies in the beam path of the first laser beam, the gamma rays and the electron trajectory. In order to enable their propagation, a hole, $21\,\mathrm{mm}$ in diameter, was added in the centre of the parabola used for the North beam. In addition, a plastic ring of $28\,\mathrm{mm}$ radius protecting the optics and the laser chain upstream from laser light scattered in an interaction with the plasma. The combined loss of reflective surface leads to a decrease in intensity of the flat-top beam of around $16\%$. The energy on target was typically $(10 \pm 0.6)\,\mathrm{J}$ focused into a spot of FWHM size of $2.4 \times 2.8\,\mathrm{\mu m}$, corresponding to peak $a_0 = 24.7 \pm 0.7$. PULSE DURATION PEAK POWER XX NUMBER. The peak $\eta$ value within reach in this scenario is $\eta = 2\gamma a_0 \hbar \omega_0/m_e c^2$.
In this case $\eta \approx 1.19 \times 10^{-5} \times E[MeV] a_0 \approx 0.3 \times E[GeV]$, i.e. $\eta \approx 0.15$ for $0.5\,\mathrm{GeV}$ and $0.3$ for $1\,\mathrm{GeV}$.

The collimated cone of gamma rays propagates without perturbation through the hole of the f/2 OAP, the aperture of the dipole magnet and through a kapton vacuum window into air. There the gamma rays are incident onto a stack of $33 \times 47$ caesium-iodide (CsI) crystals, each $5\,\mathrm{mm} \times 5\,\mathrm{mm} \times 50\,\mathrm{mm}$, that is imaged by an Andor iXon camera and used to measure the spectrum of the high energy radiation. 
\vspace{\baselineskip}


The two laser beams are overlapped in space and in time using spatial interferometry: a reflective knife-edge prism is used to reflect both counter-propagating laser beams by 90 degrees collinearly onto a CCD camera chip equipped with a X10 microscope objective. The different radii of curvature of the f/40 and f/2 beams result in the formation of interference fringes when the laser pulses overlap in both space and time. Once the fringes appear, their visibility can be optimised further to increase precision. The precision of the pulse overlap is around $\pm 30\,\mathrm{fs}$. Unfortunately, the overlapping the beams both in time and space at one point in time does not guarantee that they remain overlapped. Some studies by Oxford REF indicate that drifts exist in the system and the necessary precision of the alignment is only held for about half an hour.
\vspace{\baselineskip}


%\begin{figure}[h]
%\centering
%\includegraphics[width=0.8\columnwidth]{scintillator.JPG}
%\caption{Scintillator array used in experiment. The CsI crystals are encased in an aluminium casing approximately $220\,\mathrm{mm}$ deep and $150\,\mathrm{mm}$ high with crystals  and an array of in total. The scintillator was placed into the beam axis, the short side facing the beam enabling the radiation to travel as long as possible through the array. The holed side (diameter of holes around $5\,\mathrm{mm}$ each) was imaged using a scientific camera.}
%\end{figure}


\section{Characterisation of the Electron Spectrum}

In our experiment we measure two components of radiation reaction: one is energy loss of the electrons in the interaction with the laser pulse, the second is the radiation that this energy is being transferred into. In this section the author will characterise the electron spectrum, and its variation from shot-to-shot. A more detailed statistical analysis will follow later.


\subsubsection{General description of electron spectrum and finding the edge}
\begin{figure}
\centering
\includegraphics[width=0.8\columnwidth]{Elec_EdgeSpectrum_Null_V2.png}
\caption{Electron spectra of the reference shots (driver beam only) in red, normalised to their peak charge. The x-axis shows the electron energy in $\mathrm{MeV}$ and is cut off at $800\,\mathrm{MeV}$ as high energy features will not be further investigated. The blue line is the first derivative of the spectrum with significant maxima marked by the circles. The yellow circle indicates the peak selected to define the position of the edge.}
\label{Results:Figs:Edge:ShotsNull}
\end{figure}


A typical electron spectrum measured on the experiment using a gas jet target (Cole et al.) and the conditions described above are shown in Figure \ref{fig:Cole_espec_example}. The spectrum is integrated along the non-dispersed direction. 

The characteristic electron spectrum consists of two parts (see Figure \ref{fig:Cole_espec_example}): one is a low charge but high energy tail reaching $800-1000\,\mathrm{MeV}$, the other part contains a high charge at lower energy that falls off rapidly at an edge-like spectral feature at around $450-550\,\mathrm{MeV}$. The electron spectrum is cut off at $250\,\mathrm{MeV}$ due to limitations in the spectrometer setup.
The high energy electrons are consistent with self-injection (REF), whereas the high charge component with the distinct spectral feature is consistent with shock injection caused by damage to the gas nozzle (REF). 

Shock injection has been observed by other groups introducing a blade (REF \cite{Tsai2018}), silicon wafer (REF) or a wire (REF BURZA 2013 PRSTAB) into the gas flow to generate a shock front.



\begin{figure}
\centering
\includegraphics[angle=90, width=0.8\columnwidth]{example_espec_Cole_prelim.pdf}
\caption[]{Example of a typical electron spectrum. THIS HAS TO BE REPLACED WITH MY OWN PLOT.}
\label{fig:Cole_espec_example}
\end{figure}


\EliasComm{add a photo from PrettyPic indicating shock fronts.}

The position of edge feature in the spectrum was determined by using the derivative of the electron spectrum. The sharp cut-off in the spectrum results in a peak in the first derivative, which will be used as number for the energy of the edge.
An example is shown in Figure XX.

\subsubsection{Variation in shape?}

Show that this is so variable that this will not be good enough to distinguish models.


\section{Correlating the Gamma-ray signal with the Electron Spectrum}

The second key diagnostic to indicate radiation reaction is the gamma-ray detector.
Just like the electron spectrum might have misleading random noise attached to it that might be mistaken for a signal of radiation reaction or might conceal a signal, the gamma-ray detector also records a signal on the reference shots that has to be characterised.

In addition to radiation from the interaction of electrons and laser photons we expect background noise from the dispersed electrons interacting with matter in their trajectory, particularly high-Z materials, e.g. the vacuum chamber walls, resulting in Bremsstrahlung in a spectral range comparable with the expected inverse Compton signal. The Bremsstrahlung is expected to be produced mainly off-axis and can be efficiently shielded by blocking the direct line of sight with a sufficiently thick wall of lead. This has been confirmed in GEANT simulations (IMAGES? MAKE OWN OR BASED ON J. COLE) and in experiment (MAYBE ADD SOME PICTURE HERE OF BEFORE AND AFTER SHIELDING?). The remaining signal is expected to follow a scaling proportional to the energy and charge of the electron bunch. Interactions of electrons that produce bremsstrahlung follow a $\gamma^2$ relation for the radiation emitted (FLUX? ENERGY?). Assuming that the energy of the radiation is proportional to the energy deposited and the scintillating signal (REF OR XSEC) we expect something like this:
\begin{equation}
Background\,Signal = c_{BG} Q \left\langle\gamma^2\right\rangle,
\end{equation}
where $c_{BG}$ will encapsulate all the complicated details like conversion efficiencies of electron energy to photons, photons depositing their energy in the crystals, energy to scintillator photons, viewing angle, collection efficiency, quantum efficiencies and so on. $Q$ is the charge of the measured electron spectrum and $\gamma$ is the Lorentz factor of the electrons.
\EliasComm{add explicitly some equations for Bremsstrahlung and ICS or so to confirm the $\gamma^2$ relation.}
\EliasComm{back up the linear energy deposition somewhere, check Jason's paper.}

\begin{figure}
\centering
\includegraphics[width=0.8\columnwidth]{ElecQ2_CsI_null_Correlation.pdf}
\caption{Electron energy squared times against signal strength (counts) measured from the CsI stack.}
\end{figure}


Looking at figure NUMBER HERE which plots the $Q \left\langle\gamma^2\right\rangle$ against the total CsI signal (COUNTS XX) detected, we see that they appear to have a linear relationship as expected. Fitting a regression line and the confidence interval this seems to hold well at a value of NUMBER R.
We expect a signal following this trend on every shot without the scattering beam on. On shots with the scattering beam we expect a combined signal. The second beam will introduce, if the collision is successful, a burst of gamma-rays from inverse Compton scattering. The radiation mechanism will similarly as for Bremsstrahlung follow a $\gamma^2$:

\begin{equation}
CsI\,signal = c_{BG} Q \left\langle\gamma^2\right\rangle + c_{ICS} a_0^2 Q  \left\langle\gamma^2\right\rangle,
\end{equation}
where now $c_{ICS}$ encapsulates all the difficult physics of coupling constants, cross sections and so on. In this case $c_{ICS}$ is not a constant as the conditions of the interaction vary strongly due to the changing overlap of the laser pulse and the electron beam. The better the overlap and the higher the intensity at that interaction, the stronger the ICS signal will be and the easier to distinguish from the background Bremsstrahlung it gets.

\EliasComm{Either provide a reference for the $a^2_0$ relation or explain.}

If we now add the shots to the same plot with the beam on, we see that those shots are more scattered across the plane than the shots without the colliding beam. Some of the data points follow the trend relatively well, some have a much higher CsI signal than expected from an electron spectrum at that charge and energy. These could be potential good overlaps.

To qualify how much above the background these are, we will switch now to looking at the ICS signal mainly by subtracting the expected Bremsstrahlung background from the signal:
\begin{equation}
ICS\,signal = CsI\,signal - c_{BG} Q \left\langle\gamma^2\right\rangle.
\end{equation}

We then calculate for each of the reference data points the relative difference to the calculated regression line and characterise their behaviour statistically. Assuming a random normal distribution we calculate the standard deviation of the system.
In figure NUMBER PLOT FOR SIGNAL ABOVE... we now see the shots lined up with their actual shot number versus the CsI signal above the expected signal from the Bremsstrahlung background. The y-axis now encapsulates the information of the CsI signal and the electron spectrum.
As we can see most of the shots follow the expected trend very well, some are a few sigmas above but 4 shots are clearly more than 5 sigma (ADD WHAT THE CONFIDENCE WOULD BE) above the signal and we will consider this. The jitter of the electron beam and the laser pulse will be characterised later to check if a success quote of 4 out of 10 is matching our system.

In addition, this allows us to find more reference shots in a later dataset. We continued the campaign by defocusing the scattering beam and translate the beam relative to the electron beam. The signal on most of these shots is much lower but they were all taken with the scattering beam on. As we know that not all collisions are successful we can now determine which ones were by using the ICS signal above signal as selection criterion for additional references to analyse the electron spectrum statistically.


\section{Statistical Analysis of Electron Spectra}


We have to take the intrinsic variations of the electron source into account. In this section the author will present a statistical analysis of the shots without the scattering beam on, characterising the typical electron spectrum and the extent of its fluctuations. These shots will be from now on referred to as reference data or null shots. This is crucial as the energy loss of the electrons has to be distinguishable from the variations of the source.

The first data set of electron spectra was taken as immediate reference data for the two-beam interactions. This data set is relatively limited to 23 shots, 10 of which are reference data.

\EliasComm{Explain how the larger set of electron data was selected.}

However, the small sample size (10) is not sufficient to reliably make any predictions about the general character of the fluctuations. A larger data set from the same day at similar conditions is taken into consideration. This data set consists of 89 shots. When tracing the time the data was taken against the electron energy of the spectral feature a slow drift in the mean energy becomes apparent (see Figure \ref{fig:Cole_drift_raw}). The laser energy does not drift over the time period indicated and only fluctuates (see Figure 3). This speaks against self-injection which scales with the laser energy (REF) and is consistent with the hypothesis that the high charge component of the spectrum is injected through shock injection and that shocks were generated due to (progressing) damage in the gas nozzle.  It seems to degrade and moves the shock closer to the leading edge of the gas target, increasing the acceleration length and energy over time.
We split the varying components into two parts: one is some random (probably normal distributed) noise fluctuating from shot to shot, the second is the mean around which the fast fluctuations take place, which is slowly increasing with time. We calculate the increase of the mean and see that it correlates well with time at a correlation coefficient of 0.86.
Now we look at the deviation $\Delta E$ from this slow drifting mean in relative terms ($\Delta E/E$) as depicted in figure 6. The KDE overlay also indicates a normal distribution. 
\EliasComm{is it possible to put an error bar on the KDE? Might be a lot of statistics involved.}
\vspace{\baselineskip}


\begin{figure}[h]
\centering
\includegraphics[width=0.5\columnwidth]{ElecEdge_Raster_Drift.pdf}\includegraphics[width=0.5\columnwidth]{ElecEdge_Raster_Slices_Drift.pdf}
\caption[]{Distribution of the electron edge energies of the data shots plotted against time of day the data was taken. The energies are scattered around a line that slowly increases with time. A potential fitting regression line with a confidence interval (shaded area) is drawn. By assuming a random distribution around a slowly drifting mean the data set was split into individual time slices. Under this assumption the linear relation of energy and time holds very well at a correlation coefficient of $0.86$. The rate of increase is $21.6\,\mathrm{MeV}/\mathrm{hour}$.}
\label{fig:Cole_drift_raw}
\end{figure}


\begin{figure}[h]
\centering
\includegraphics[width=0.5\columnwidth]{Cole_LaserTime_Corr.pdf}\includegraphics[width=0.5\columnwidth]{ElecEdge_raster_drift_Laser_Corr.pdf}
\caption[]{Fluctuation of the laser energy over 2 hours of taking data. The time period does not cover the entire range of the data set as the diagnostic measuring the laser energy failed after 7 pm. The correlation coefficient is -0.11 and the confidence interval for the slope of the linear regression includes negative and positive values. This indicates that there is no real correlation of the two parameters, and the laser energy fluctuated around a constant mean. Right: Laser energy on the shots with the respective drift-corrected relative energy shift. A weak positive correlation was found but at a very low level (correlation coefficient $\sim 0.35$). This was hence not used in the following analysis.{\color{red}is the line and error bar correct. the error bars suggests a reasonably strong correlation?}}
\end{figure}



\begin{figure}[h]
\centering
\includegraphics[width=0.8\columnwidth]{ElecEdge_DE_drift_histo.pdf}
\caption[]{Histogram of relative deviation of the spectral edge energy from the expected slowly varying mean ($\Delta E/E$). The total distribution is overlaid with a KDE and resembles closely a Gaussian distribution (orange). The shaded area indicates a 2 sigma (95\%) confidence interval assuming a normal distribution.{\color{red} Can we put an error bar on the KDE?}}
\end{figure}


We can check if this correlates with laser energy. There appears to be a positive correlation but only with little confidence. The correlation coefficient is around 0.35 and the confidence interval for the slope .... We will hence ignore this factor in our analysis. 
\EliasComm{Expand this bit to justify points more}

Looking at this spectrum. Do we assume it is normal? We can test the hypothesis and get a high confidence that it is HERE SOME TESTS SCIPY.

Assuming this is a normal distribution we can assign probabilities to certain energies. The standard deviation of this distribution is $\sigma_{Cole} = 0.077$. In other words this means we expect 95 percent of the energies to be within $\pm 15.4\,\%$ ($2\,\sigma_{Cole}$) of the mean. 


\section{Estimating Collision Probability}

To verify whether our success quota for the collisions matches up, we will have a brief look at characterising the jitter of both laser beams or the electron beam relative to the laser beam.

Micron sized targets.

Something about F/2?
Something about F/40?
Do we assume the electron beam to follow the laser pulse?

electron pointing jitter rms = 3 mrad (in x and y)
electron beam divergence = 3 mrad 
electron waist =$ 1\, \mu m$
distance from waist to interaction = 1.5 mm
hit defined as overlap of laser and electron $>= 50\%$ of maximum overlap possible.

I did 1000 runs of 15 shots and calculated the overlap between the laser and electron beam, defining a hit as when the overlap $>= 0.5$ perfect alignment.  

The expected number of hits is $4 - 5$ (we saw 4 in the experiment)




I calculate the beam size assuming the ebeam behaves like a gaussian beam
so 
$w_{ebeam} = ebeam_{waist} * \sqrt( 1 + (z_{ebeam}/(zR_{ebeam}) )^2)$
and 
$zR_{ebeam} = ebeam_{waist}/divergence_{ebeam}$


Estimate collision probability here.


\section{Spectral Analysis of the Gamma-ray signal}

Geant simulations and so forth. Different algorithms maybe?

The spectral analysis of the gamma-ray signal was first performed by Jason Cole (Imperial College) and Keegan Behm (Michigan University).

In this process the scintillator response was modelled using the Monte-Carlo codes Geant4 \cite{Agostinelli2003} and MCNP \cite{Goorley2012} to retrieve the energy spectrum of the gamma radiation. The on-shot synchrotron-like spectrum and the shift of the cut-off edge in the electron spectrum were then combined to constrain the parameter space (normalised laser vector potential $a_0$ and electron energy) at the interaction, relying on a range of models.

A paper on the design of the detector and the analysis of the spectrum has been submitted for publication (REF Keegan paper).

Also present that on beam on and off the spectrum changes significantly and is hence beyond the level of absolute numbers very sensitive to the interaction.


\section{Estimating Intensity at Interaction}


Refer here to Oxford results/RAL report.

\EliasComm{Potentially already mention the time shift here.}


\section{Indicators of radiation reaction in the electron spectrum}

\begin{figure}
\centering
\includegraphics[height=0.25\columnwidth]{20151217r002_NullGroup.jpg}
\includegraphics[height=0.25\columnwidth]{20151217r002_CollGroup.jpg}
\includegraphics[height=0.25\columnwidth]{up_arrows.jpg}
\caption{Images of the gamma ray spectrometer (upper row) and electron spectrometer (bottom row) at shots with the colliding laser beam off (left) and on (right), using a constant colour scale. The shots with the colliding beam show bright signals on the scintillator array whilst the electron energies remain fairly similar throughout.}
\label{Results:Figs:NullColl:Montage}
\end{figure}

\EliasComm{Make a waterfall plot of the integrated spectra.}
Figure \ref{Results:Figs:NullColl:Montage} shows the visual data taken of the gamma ray detector (upper row) and the electron spectrometer (bottom row) with scatterer and driver (right) or driver only (left). A montage like figure \ref{Results:Figs:NullColl:Montage} can be a helpful tool to spot stark differences and features in a first step.


In the previous section we found a selection criterion to determine which shots were successful. The brightest CsI signals are found in shots NUMBERS. These four shots are selected accordingly and match roughly the statistics from the section on the relative jitter of the electron beam and the colliding laser pulse.


This method allows us to identify successful collisions independently from the electron signal which is due to its strong fluctuations less convincing by itself.

\begin{figure}
\centering
\includegraphics[width=0.5\columnwidth]{ElecQ2_CsI_Correlation.pdf}\includegraphics[width=0.5\columnwidth]{DCsI_sigma.pdf}
\caption{Left: The total energy of the electron beam squared (x-axis) vs. the number of CsI counts (y-axis) which is proportional to the energy of the photons. The photon energy radiated by the electrons is proportional to $\gamma^2$ which in turn is proportional to $E^2$. Hence, a linear relationship should be visible. As to be seen in the diagram, the shots with colliding beam off (red) are fitted well with a linear function whilst the data points for the collisions do not seem to fit on the same curve in many cases. This motivates that there seems to be a different process to take place when the colliding laser pulse is turned on. It can also be seen that the CsI counts for collision shots are scattered a lot, which could be related to different overlaps of laser pulse and electron beam resulting in a different range of radiation produced. Right: Shots in the order performed in experiment versus the gamma detector signal above the expected background signal according to the $E^2$-scaling described previously. The red dots are representing the reference shots with the driver beam only, whilst the blue dots are the collision shots with driver and scatterer. Especially shots 4,5,6 and 8 stand out remarkably above the background. The last few shots on the other hand seemed to have missed completely which could be related to a drift in alignment over time.}
\label{Results:Figs:NullColl:DeltaCsIVsSigmaE2+CsIVsE2}
\end{figure}


\begin{figure}
\centering
\includegraphics[width=0.8\columnwidth]{DCsI_sigma_v_ElecEdge.pdf}
\caption{Position of the low-energy edges (x-axis) versus the gamma ray signal above the expected radiation background (y-axis) in units of standard deviation from the reference mean. Reference shots are shown in red and collision shots in blue. The edge positions are very scattered under both conditions with the majority of the collision shots exhibiting higher edges. The brightest gamma ray signals on collision seem to coincide with a low position of the edge.}
\label{Results:Figs:Edge:PosVsCsI}
\end{figure}


From figure NUMBER HERE. We see that the electron spectral feature varies quite a bit over the course of the shots but that all shots selected using our criterion have low electron energies. The probability for this, assuming a normal distribution as derived in the previous section, is around NUMBERS HERE for one shot but NUMBER TO THE FOUR for all four of them to occur. If we are sufficiently convinced that these low electron energies are equivalent to energy loss from our mean, we can estimate the conditions at the interaction. For an electron energy of CHECK NUMBER $550\,\mathrm{MeV}$ we require a laser intensity of $a_0 \approx 10$. This is much lower than the peak intensity of our laser pulse.
\EliasComm{Add a plot for the spectral information.}
\EliasComm{At least indicate which model was used to first draw this conclusion. The ballpark will be similar. Semi-classical or LL?}
A look at the spectral information of the gamma-ray detector confirms these parameters. The spectrum recorded roughly matches this. We can from both sides narrow down the exact conditions at the interaction.
\vspace{\baselineskip}

The lower intensity is actually easily explained. The timing of the beams was done using spatial interferometry. The visibility of fringes was optimised which means that both laser pulses are timed to arrive within $30\,\mathrm{fs}$ of each other. This timing is true for both laser pulses travelling through vacuum. What we are actually aiming to time is the arrival of the scattering beam and the electron beam which is trailing behind the driving laser pulse. To estimate the intensity of the laser pulse at the interaction we have now to consider the delay of the laser pulse as it is travelling through a medium (plasma) to the point of injection and then, assuming that the electrons relatively quickly reaching a velocity close to the speed of light, the relative delay of the laser pulse and the electron beam. This delay is around half a plasma wavelength SEE REFS.

For the delay induced through the South beam travelling through a medium we use the group velocity $v_f \approx 1-\frac{3}{2}\frac{n_e}{n_c}$, which is the standard nonlinear group velocity in media minus the etching velocity (called front velocity in REF DEcker 1996 Phys Plasmas). The distance that the North beam travels in vacuum during the time the South beam travels through a medium of thickness $d$ is $d'=d c/v_f$. The new collision point is now

\begin{equation}
\delta z_1 = 0.5 (d c/v_f - d) = d/2 (1/(1-3/2 n_e/n_c) -1) \approx (1+3/2 n_e/n_c -1) = \frac{3d}{4}\frac{n_e}{n_c}
\end{equation} 

The electrons trail behind the South laser pulse around N plasma wavelengths (the factor half are as the collision happens midway):
\begin{equation}
\delta z_2 = 0.5 N \lambda_p = 0.5 N \frac{2 \pi c}{\omega_p} = 0.5 N \frac{2 \pi c}{\omega} \frac{\omega}{\omega_p} = 0.5 \lambda_0 \sqrt{\frac{n_c}{n_e}}
\end{equation}

As total delay we can estimate
ADD EQUATION HERE FOR DELAY AND A DRAWING TO SHOW THIS MORE CAREFULLY.

\begin{equation}
\delta z = \frac{3d}{4} \frac{n_e}{n_c} + N \frac{\lambda_0}{2}\sqrt{\frac{n_c}{n_e}},
\end{equation}
where d is the distance of injection point from front of gas jet
N is number of plasma periods between f/40 and electron bunch.

\EliasComm{Somewhere write about the delay of electrons to laser pulse (timing) and how to estimate it.}
\EliasComm{Add more details to explain this.}
\EliasComm{Maybe a comment why shock injection could be a useful tool as the injection point is less dependent on pulse evolution.}
\EliasComm{Maybe add a little sketch to indicate where all these delays come from.}

This amount of defocus then reaches an average intensity of NUMBER 10 instead of the peak intensity of close to 25 possible in this setup. A variation of overlap and the electron energy will then result in an accidental scan of our conditions.

Both measurements (gamma-rays and electron energy loss) seem to be indicating the conditions.
If we plot the electron spectrum (assuming post-interaction) against the critical energy of the gamma-ray spectrum, we see that they correlate negatively. This again supports the hypothesis that we are looking at ICS with radiation reaction. In a different mechanism after the measurement, like Bremsstrahlung, we would have expected a positive correlation (a harder spectrum for higher energies). The fact that we see a negative correlation confirms our suspicion that the energy loss is converted into a gamma-ray signal.

Now here some part about the models and how they relate, how the plot is made but this is all Jason's and Tom Blackburn's work and should be referenced as such.

FIGURE FROM THE PAPER, THEORY BLOBS.
\EliasComm{Add theory blobs.}
Some comments on the actual models?
\vspace{\baselineskip}

Some comments on the theory contours.
The data shows that the results match a model that requires radiation reaction. No radiation reaction overestimates the energy loss and the energy of the radiated gamma radiation. Our results are in agreement with all the models in question, more overlap with the models including quantum corrections which could be a semi-classical model with classical trajectories and merely reduced emitted power or a fully stochastic quantum model. This is all on a one sigma confidence level so a strong distinction is not possible, we require more data.
An analysis how much more data we would require to make a conclusion beyond the 2 sigma level will follow in the next section.
\vspace{\baselineskip}

In addition it becomes clear that the difference between a fully stochastic quantum model and a semi-classical model is not very visible in this case. There is in indication that the models depart at higher energy losses, i.e. higher laser intensities but it would also require a higher electron energy. This slow departure from each other in contrast to the relatively distinct classical model is founded in the electron observable which is the energy loss from a characterised mean. When looking at the mean energy loss for the quantum and the semi-classical model it becomes evident that by definition the values are identical. Even though the energy loss of the spectral feature is not the same thing it closely relates to the mean energy loss. A more sensitive observable that behaves very differently for both models is the variance of the spectrum or the shape of the spectrum. This will be closer illuminated in the next section. 
\EliasComm{add some references here to papers regarding radiation reaction. How much to add in theory and how much to add in experiment section?}

\EliasComm{Also remember that eta is not a constant in the interaction}

\section{Comparing Results Poder and Cole}

\EliasComm{Explain why comparing these results in the first place and the context of this experiment.}

\subsubsection{Experimental setup}

The experimental setup in \cite{Poder2018} was very similar as based on the same facility, using mostly the same optics and beam line.
As gas target a helium filled gas cell of length $20\,\mathrm{mm}$ was used. The electron density was $2 \times 10^{18}\,\mathrm{cm}^{-3}$. The focal spot from the f/40 spherical mirror was slightly larger at \textsc{fwhm} dimensions of $59 \times 67\,\mathrm{\mu m}$. The energy delivered on target was around $9\,\mathrm{J}$ reaching a normalised intensity of $a_0 = 1.7$. XX NUMBER FOR F/2. A sketch of the setup is show in Figure \ref{RR15:figs:exp_sketch_Poder}.
\vspace{\baselineskip}

\begin{figure}[h]
\centering 
\includegraphics[width=0.9\columnwidth]{ExpSetup_Poder.pdf}%\includegraphics[width=0.9\columnwidth]{ExpSetup_GasJet_MagnetOnly.pdf}
\caption[]{Sketch of the experimental setup as used on the radiation reaction campaign \cite{Poder2018} 
%(from left to right): a laser pulse (in red) is focused by an f/40 spherical mirror onto the entrance of a gas target (gas jet or gas cell). The intense laser pulse drives a wakefield and accelerates electrons (blue) to relativistic energies. A second laser is focused with an f/2 off-axis parabola onto the exit of the gas target scattering the electrons and emitting a bright flash of gamma rays (green) from inverse Compton scattering. A permanent dipole magnet is used to disperse and characterise the energy of the electron beam on a scintillating LANEX screen. The gamma rays propagate through a kapton vacuum window (orange) onto a stack of caesium-iodide (CsI) crystals. The sketch is based on work by J. M. Cole, Imperial College, for \cite{Cole2018}.}
}
\label{RR15:figs:exp_sketch_Poder}
\end{figure}

Gas cell targets have sometimes produced superior results from gas jets using the same laser system (Poder v Kneip, Osterhoff, \cite{Kuschel2018}) in terms of charge, shot-to-shot stability and maximum energy gain.
However, some of the most stable LWFA sources have been achieved using gas jet targets (Faure Nature 2006) and the open geometry allows for diagnostic access, colliding pulse experiments as well as a route to high repetition rate operation.
The variety of gas targets, their performance and the performance of the specific laser system make it difficult to compare gas cells and jets in general with each other. Comparisons can in most cases only be done between individual specimens and conclusions drawn from such a comparison might only be valid in this limited context. 

\subsubsection{Trends in Poder data}

The typical electron spectrum produced can be seen in Figure \ref{fig:Poder_espec_example}. The average shape is an exponential spectrum reaching energies in excess of $1.5\,\mathrm{GeV}$. At the lower end the spectrum is cut off at around $400\,\mathrm{MeV}$ due to limitations in the magnetic electron spectrometer setup. The spectra from the gas cell originate from self-injection and lack a sharp distinctive feature comparable to the spectral edge described in the previous section related to shock injection. This data set consists of 19 shots.

\begin{figure}
\centering
\includegraphics[angle=90, width=0.8\columnwidth]{example_espec_Cole_prelim.pdf}
\caption[]{Example of a typical electron spectrum. THIS HAS TO BE REPLACED WITH MY OWN PLOT AND WITH PODER DATA.}\label{fig:Poder_espec_example}
\end{figure}

Due to the lack of such a distinct feature, the characteristic number for each spectra of this experiment will now be the cutoff energy, which is as in \cite{Poder2018} defined as the energy at which the spectral intensity reaches 10 percent of its peak value.
\vspace{\baselineskip}

In this case the only injection mechanism we can suspect is self-injection. This process typically scales with the laser energy (REF), which is also confirmed in Figure \ref{fig:Poder_laser_corr}. As one can see is that the laser energy of the wakefield driver scales well linearly with the cutoff energy of the spectrum. The correlation coefficient is $0.9$. The linear fit follows the equation \\$\epsilon_{cutoff} = 0.07\,\mathrm{GeV/J} \times E_{laser} + 0.57\,\mathrm{GeV}$.

After scaling the spectra according to the linear relation found in Figure \ref{fig:Poder_laser_corr}, we can analyse the spectra similarly as before and see that this spectrum also follows a normal distribution. The standard deviation of this distribution is $\sigma_{Poder} = 0.035$, so 95 percent of the expected energies will fall within $\pm 7\,\%$ of the mean.

\begin{figure}[h]
\centering
\includegraphics[width=0.8\columnwidth]{Poder_Laser_Corr.pdf}
\caption[]{Laser energy of the wakefield driver beam before pulse compression plotted against the cutoff energy of the electron spectrum. The data points clearly follow a linear trend drawn in the straight green line with a gradient of $0.07\,\mathrm{GeV/J}$. The shaded area indicates the 95 percent confidence interval of the fitting function. The correlation coefficient for a linear fit is $0.9$.}\label{fig:Poder_laser_corr}
\end{figure}


\subsubsection{Comparing both datasets}

By factoring out known correlations of the spectra and scaling them accordingly (slow drift for Experiment A (Cole et al.) and laser energy for Experiment B (Poder et al.)), and by normalising the spectra to their total charge the spectra become comparable. With this processed data set expected intrinsic or unknown fluctuations can be characterised and taken into consideration.


\begin{figure}
\centering
\includegraphics[width=0.8\columnwidth]{Comparison_ESpectra.pdf}
\caption[]{Averaged and scaled electron spectra for data from Cole (blue) and Poder (green). The spectra are normalised to a total charge of 1. The error bars indicate the energy dependent variance of the spectra.}\label{fig:comparison_especs}
\end{figure}

The average spectrum of the two runs are shown in Figure \ref{fig:comparison_especs}. The spectra were processed as mentioned before and then averaged over the available data shots. As seen the typical energy reached in the gas cell from Poder et al. is significantly higher by at least a factor two for the majority of the charge distribution. The y-axis indicates a normalised charge, where the sum of the recorded spectra is set to 1. The Poder spectra appear to contain less charge since the total charge is distributed over a wider energy range. The shaded error bar region around the average spectrum indicates the standard deviation of the average spectrum at that particular energy.

\begin{figure}
\centering
\includegraphics[width=0.8\columnwidth]{Comparison_Histo.pdf}
\caption[]{Distribution of the relative energy deviations from the mean ($\Delta E/E$ for Cole et al. (blue) and Poder et al. (green) after scaling. The overlaid lines are kernel density estimates (KDEs). The shaded areas indicate the 2 sigma intervals assuming a normal distribution.}\label{fig:comparison_histo}
\end{figure}


The distribution of energies for the gas cell data is narrower than the data taken with a gas jet target. The shaded areas in Figure \ref{fig:comparison_histo} show the 2 sigma or 95 percent confidence intervals (assuming a normal distribution). The Poder data has a typical fluctuation of the cutoff energy of around 7 percent. The spectral feature from the gas jet vary with 15 percent by around the double amount. This means that if our observable is coupled to the energy of the electrons we will require a certain number of shots based on the typical variability of the source to say with a few sigma confidence that a signal beyond this has been recorded. If we couple this with the probability to even collide two beams at all, the number of required additional shots increases by a significant amount.
\vspace{\baselineskip}

To put this into context: in \cite{Cole2018} 4 successful collisions were identified by consulting the gamma-ray detector signal which showed a bright signal $5-10$ standard deviations above background. The energies of these 4 successful collisions were about $15\,\%$ below the mean of the reference distribution. The cumulative probability for a normal distribution with standard deviation $\sigma_{Cole} = 0.077$ to see one shot with a relative deviation of the energy from the mean of $10\,\%$ or more is roughly $10\,\%$. For an individual shot this probability is non-negligible, but it holds in combination with the gamma-ray signal and over the $4$ collisions to confirm that this is to some extent as low due to radiation reaction. The cumulative probability for a normal distribution with standard deviation $\sigma_{Poder}=0.035$ on the other hand for the same scenario is only $0.2\,\%$ and holds the test by itself at a level of confidence equivalent to 2.7 as many shots at $\sigma_{Cole}$. At a repetition rate of $0.05\,\mathrm{Hz}$ typical for the Astra Gemini laser a scan of 100 shots at $\sigma_{Poder}$ would take under 30 minutes, but would extend to $1.5$ hours to reach the same level of confidence. At this extended timescale, however, even small drifts in alignment and timing can affect the reproducibility of the interaction conditions which requires few microns and femtoseconds precision. In addition, we have to factor in the potential deterioration of the gas target and other components over the course of the experiment. Time is a crucial factor in experiments at such a high level of complexity and sensitivity.
\vspace{\baselineskip}

However, the error on the mean of the electron energy also propagates further into the estimate of the interaction conditions. If we sustain the experiment conditions over the course of the experiment (same spatio-temporal overlap, intensity of interaction, laser parameters and so on), the energy of the unperturbed electron beam is fluctuating according to a normal distribution and the energy of the post-interaction electron beam will follow the same normal distribution but shifted down by the average energy loss. For now we will ignore the potential cooling or heating effects of radiation reaction on the distribution function. In this scenario for a constant energy loss of $15\,\%$, we will measure a range of deviations from the unperturbed mean energy which we will call `energy loss', which is in reality the energy loss superimposed onto the normal distribution. The standard deviation of the Cole data $\sigma_{Cole}$ is half the size of the total energy loss, whereas for $\sigma_{Poder}$ it halves to $25\,\%$, which is still a significant range of values most of the measurements will show. That means that there is a significant error on the estimated energy loss for each measurement.

If we want to compare different models of radiation reaction based on this energy loss, an increased stability of our electron energy will help in two ways: firstly, the expected phase space of post-interaction energies for each model of radiation reaction at a fixed confidence level decreases, which might reduce the overlap of different models. Secondly, if the initial electron energy is sampled from a narrower distribution the error on that measurement decreases as well, reducing its footprint in phase space again.

\begin{figure}
\centering
\includegraphics[width=0.8\columnwidth]{Comparison_Variance_Log.pdf}
\caption[]{Energy-dependent variance of the average spectra in introduced in figure 8. In blue the data taken on the experiment related to Cole et al., in green to Poder et al.. The total variance for Cole et al. was $1.5 \times 10^{-7}$ and $1.98 \times 10^{-8}$ in the case of Poder et al..}
\end{figure}\label{fig:comparison_varianceLog}

In some cases, however, even an increased level of confidence and more data is not resulting in an improved distinction of different models of radiation reaction. As seen in recent publications REFS HERE, measuring the difference between classical and models including some form of quantum corrections can be done based on the energy loss. When trying to measure the difference between fully stochastic quantum models and a semi-classical model which includes a reduced emission power matched to the quantum model (Gaunt-factor), but using a deterministic equation of motion, the shape of the distribution, e.g. variance of the entire spectrum (not only of the energy of the spectral feature), becomes more important and another factor to discriminate. The stability of the shapes can, for instance, be quantified in terms of the energy dependent variance of the spectra in question. The total variance normalised to the energy range considered for the average gas jet data from Cole et al. is $1.5\times 10^{-7}$ whereas the result for Poder et al. is $1.98 \times 10^{-8}$. The energy dependent variance can be seen in Figure \ref{fig:comparison_especs} in the shaded error bars and in Figure \ref{fig:comparison_varianceLog} separately on a logarithmic scale. The base level of the variance is relatively low for Poder and decreases for high energies. For Cole especially the position of the characteristic feature, crucial for the two-beam measurements, is prone to fluctuations. This makes the data set unsuitable to track changes in the variance induced by radiation reaction and makes the distinction of a semi-classical and a fully stochastic quantum model at these conditions impossible even with access to a larger data set.

\section{Statistics for future experiments}

Maybe a section based on Chris Arran's work regarding how many shots we would require with a certain spectrum.

Add a table on how many shots are required at what width of the distribution and what electron and laser parameters. Or make a nice 2D plot.

As it became evident over the course of these past sections, the quality of our electron source and the lasers in use requires us to think about the statistics to make sure our statements to distinguish models are sufficiently backed up. 

In general we want higher electron energies and higher laser intensities to reach regimes where quantum and classical models clearly depart from eachother. The energy at these conditions should be fairly stable to allow a clear distinction. In addition we want a narrow energy spread and a stable spectrum also in terms of shape to be able to distinguish a fully stochastic quantum model and a semi-classical model. As we are not using a machine but an intrinsically delicate and highly non-linear acting laser-plasma accelerator, sometimes not all of these parameters can be at their optimum. Driving a wakefield very hard at maximum energy might give us higher energies but a pronounced feature as in COLE might disappear and the reproducibility might decrease. At lower energies narrow spread beams might be easier to reporduce and to control but the models will be closer to each other. In addition, there are potentially other effects such as chirp in the electron beam or a rapid decrease of $\eta$ or a variation of $\eta$ in a long scatterer pulse, that one has to consider. In a realistic setting one has to compromise and the extra effort to optimise one parameter might not necessarily increase the confidence much more. A simple analysis of the effect of the different variables should act as a handy reference for the experimentalist.

The variables will be: electron energy, electron energy spread, electron energy stability, electron shape stability, scattering laser intensity.
The hopeful result will be: number of shots required to distinguish a full quantum model from a semi-classical model or a classical model within 1, 2, 3, 4 and 5 sigma confidence.

\EliasComm{Comment on low energy spread beams and distinguishing models. Indicate that these kind of beams are achievable at Gemini. Reference the shock injection paper here}

\section{Conclusion}

The expertise gained in these previous experiments paired with the technological capacities of the Astra Gemini Laser make it a realistic ambition to collide electrons at even higher relativistic energies ($> 2 \mathrm{GeV}$) and more intense laser fields ($a_0 \approx 25$), expecting a gamma-ray spectrum with a critical energy close to $100\,\mathrm{MeV}$. These would enable probing radiation reaction as the energy loss due to the emission of radiation becomes significant in relation to the initial electron energy -- possibly even reaching a regime where QED effects become relevant as the `quantumness' parameter $\eta$, indicating how much the interaction is dominated by QED effects \cite{Blackburn2014} (maybe another REF), approaches values close to $1$. 
In this regime we expect QED cascades.


The author presented a statistical analysis of electron spectra from two similar experiments aimed at measuring radiation reaction, but using different gas targets. Two correlating factors, a slow drift in mean energy in case of Cole and the laser energy in case of Poder, were identified and considered in this analysis. The electron spectra from the Poder et al. campaign were more stable in terms of stability in energy and variance. The spectra did not only reach higher energies but the standard deviation of the energies were by a factor of $2$ more stable. The 2 sigma width of their distribution was $\pm 15\,\%$ for Cole and $\pm 7\,\%$ for Poder, which means that 95 percent of shots were expected to fall within this range. The variance of the spectrum was also more stable in the case of Poder et al. being 10 times lower than for the electron spectra observed in Cole et al.. Stable energy and shape of spectra is important.

Reproducibility is key to measure subtle changes in spectra.
A solid statistical analysis of the intrinsic fluctuations of the system, if it can not be neutralised by other measurements, is key to the analysis but also as decision helper how many data points are required in the end.
Taking references seriously is important for future measurements and the confidence in them.
A proper statistical analysis can help to estimate the number of data shots and references required to make firm statements, for instance in this case to distinguish different models.