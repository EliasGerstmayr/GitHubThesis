\chapter{First Measurements of Radiation Reaction at Astra Gemini}

\section{Motivation}

\subsubsection{General merits of Inverse Compton Scattering}
In relativistic inverse Compton scattering (ICS) one (linear) or multiple (non-linear) photons scatter from a relativistic electron ($\epsilon/m_e c^2 = \gamma \gg 1$). The scattered photons experience a relativistic Doppler-shift and are re-emitted in a narrow cone of divergence $\sim 1/\gamma$ in the direction of the electron propagation, carrying a higher energy based on the electron energy and the intensity of the laser field. The emitted photon energy, $E_{ph}'$, is maximised in a head-on collision: $E_{ph}' \sim 4 \gamma^2 E_{ph}$ (see Theory REF for further details). Higher electron energies result in a stronger shift of the radiation, leading to a hardening of the spectrum and an increase of the corresponding energy loss the electrons experience (REF). A more intense laser field ($a_0 > 1$), on the other hand, enables non-linear interactions and increases the number of photons interacting at once with an electron (REF), resulting in the generation of $\sim a^3_0$ higher harmonics (REF) but also redshifting and broadening of the spectrum (REF). As a result the rapidly increasing number of higher harmonics blend together to form a broadband synchrotron-like spectrum.

ICS provides a promising route to generate bright burst of high energy radiation reaching 100's of keV or even MeV photon energy, suitable to image high-Z objects and stimulate nuclear transitions, even at small-scale facilities.
In addition, the high photon energies offer the opportunity to directly observe the energy loss, if the emitted photon carries an energy comparable to the energy of the electron, and to measure the effect of radiation reaction itself when the electric field in the rest frame of the electron approaches the critical field of QED, $E_{crit}$.
\vspace{\baselineskip}

\iffalse
\subsubsection{Motivation Radiation Reaction}

Electrons emitting radiation is ubiquitous but still in most applications today the self-force does not reach significant levels and its impact on the equation of motion is negligible.

Radiation reaction is important in extreme conditions as they can be found in astrophysical environments, quasars, jets of relativistic particles scattering from the microwave background and limiting the energy of particles, just as the gamma rays from ICS interact with the CMB to produce electron-positron pairs from BW.

Synchrotrons today suffer from radiation losses but slowly over long periods. At future facilities, especially electrons, will suffer from so high energy losses that radiation reaction might also become a significant problem.
\fi

\subsubsection{Inverse Compton Scattering using Laser Wakefield Accelerators}

In LWFA the intense laser driver is intrinsically synchronised to the relativistic electron beam that trails behind it, which makes LWFA well suited to study inverse Compton scattering at high intensities.
\vspace{\baselineskip}

Colliding-pulse or ICS experiments using LWFA have been successfully performed by different groups over the past years, gradually improving the control over the process, increasing the laser intensities and the energies of the electrons involved, and subsequently of the radiation generated \cite{TaPhuoc2012a,Chen2013a,Khrennikov2015,Powers2014,Sarri2014}.
The energy of the produced radiation is a good indicator for the continuing progress of these experiments. More specifically of interest for measuring radiation reaction is the quantum non-linearity parameter, $\eta = 2\gamma E_L/E_{crit}$, that indicates the strength of the electric field of the laser in the rest frame of the electron relative to the critical electric field of QED, $E_{crit}$.
\vspace{\baselineskip}

\subsubsection{Inverse Compton Scattering using a plasma mirror}

A single intense laser pulse can be used to accelerate electrons to relativistic energies via LWFA and to scatter the electrons after backreflecting the laser from a close-to-normal plasma mirror (e.g. tape \cite{TaPhuoc2012a}). In this geometry the laser beam and the electrons are timed intrinsically. In \cite{TaPhuoc2012a} electrons of an energy $\sim 100\,\mathrm{MeV}$ were collided at a laser intensity of $a_0 \approx 1.2$ (mildly non-linear). The radiation measured was a broadband X-ray spectrum and reached up to $100'\mathrm{s}\,\mathrm{keV}$ photon energies. A lower electron energy spread and tunability was demonstrated at slightly lower electron energies and laser intensities in \cite{Powers2014} and\cite{Khrennikov2015}. By now the highest recorded gamma-ray energies from ICS in LWFA using this scheme have been achieved by Shaw et al. (REF) reaching up to $85 \,\mathrm{MeV}$ photon energies in a collision with electrons at energies up to $2\,\mathrm{GeV}$.
\vspace{\baselineskip}

Whilst this technique avoids issues with timing and overlap, the intensity of the laser pulse is limited as it is partly depleted when interacting with the electron bunch after driving a wake and is typically not tightly focused. At the same time the acceleration can not be pushed to its depletion limit as the laser pulse has to remain intense enough for a suitable interaction. Further problems might arise as controlling or measuring the wavefront of the depleted laser pulse might be challenging. The electrons will also produce radiation from bremsstrahlung when passing through the plasma mirror.
Whilst these details are important for a measurement of radiation reaction, they might be secondary concerns if simply radiation production is the aim.
In addition, at facilities housing the next generation of real PW-class lasers, the reflective scheme might still be able to yield very promising results to probe radiation reaction.
\vspace{\baselineskip}

\subsubsection{Inverse Compton Scattering with two lasers}

If two separate lasers are available or maybe one very powerful laser can be split into two parts, one can be used to accelerate electrons via LWFA and one to scatter off them. Electrons produced from LWFA are typically very short in duration (10's of femtoseconds, REF) and only few microns small when leaving the accelerating cavity. This allows focusing the second laser very tightly to reach high intensities and to interact with a large fraction of the electron bunch at comparable intensities. Using two laser pulses opens the gateway to combine higher intensities and electron energies at the interaction point, and gives more control over the interaction itself. However, it is challenging to overlap the micron-sized electron bunch with the ultra-short laser pulse, focused tightly to a micron-scale focal spot as well to achieve high intensities, in other words tightly compressed in time and space, and to maintain this alignment over an extended period of time.

In \cite{Chen2013a} relativistic electrons were scattered at an angle of $10$ degrees and gamma rays at an energy of around $1\,\mathrm{MeV}$ were produced.
In \cite{Sarri2014} electrons at an energy of $400 \, \mathrm{MeV}$ were successfully collided at an laser intensity of $a_0 \approx 2$, resulting in broadband radiation extending up to $10\,\mathrm{MeV}$. In YAN REF NAT PHOTONICS 2017 electrons of energy $\sim 200\,\mathrm{MeV}$ were collided with an intense laser at $a_0 \sim 12$ reaching photon energies just above $20\,\mathrm{MeV}$, resulting in the generation of over $500$ orders of higher harmonics.
\vspace{\baselineskip}


\begin{figure}[h]
\centering
\includegraphics[width=0.8\columnwidth]{RR2015_Eta_thiswork.pdf}
\caption{Some plot to place our work into context. Shade area for Gemini. Shade for ELI. Remove Gemini2019 plot...}
\end{figure}



\subsubsection{Future Studies on Inverse Compton Scattering and Radiation Reaction using LWFA}

Future laser facilities promise interactions at higher electron energies and laser intensities
 
.... several PW laser systems, 1 to 10 PW (recently reached)

... new facilities looking into this topic predominantly like ELI pillars

... several GeV of electron energies: Current LWFA energies for electrons of order 4 GeV (Korea) and guided up to almost 10 GeV.

.... SULF 100 PW.

... opens up new research topics: Strongly non-linear, quantum regime, photon-photon collider, BW, Schwinger limit.

... research into limitations of all-optical ICS. GUILLERMO REF.

\subsubsection{ICS at other types of facilities}

Used at conventional accelerators to determine polarisation of electron beam.

The increasing ease of access to ~10's TW laser systems, the advent of commercially available PW-class lasers and the ambitions of the LWFA have also spawned a renewed interest into the field of high-field phenomena at conventional accelerator facilities that can provide high-quality relativistic electron beams. What to do with an electron beam. XFELs. Lasers are interesting for warm and HEDP experiments, pump-probe, so are available already.
After the prominent E144 experiment at SLAC, performed during the 90's where a 46 GeV electron beam was collided with a laser $a_0 \sim 0.3$, which resulted in the production of positrons from a multi-photon interaction but in a linear regime.

Examples are for instance LUX at XFEL in Hamburg (Electron energies and intensities XXX NUMBER?), SFQED at FACET-II/SLAC (30 TW, 13 GeV XX NUMBERS?).
\vspace{\baselineskip}

\subsubsection{Chapter Outline}

The work presented in this chapter relates to an ICS experiment aimed at measuring radiation reaction performed at the Gemini laser facility in late 2015. The experimental team succeeded in colliding electrons of $\sim 550\,\mathrm{MeV}$ energy at a laser intensity of $a_0 \approx 10$, reaching critical gamma-ray energies in excess of $30\,\mathrm{MeV}$. This was the highest gamma-ray energy from an all optical ICS source published at that point and consists the first published measurement of radiation reaction in an LWFA setup \cite{Cole2018}. 
\vspace{\baselineskip}

This chapter includes significant contributions to this publication, in particular finding a method to identify successful collisions and characterising the electron spectra. As a result, the reader will find many parallels between \cite{Cole2018} and that this chapter follows a similar line of arguments in some instances. 
\vspace{\baselineskip}

After an outline of the experimental setup the chapter continues with presenting a method to identify successful collisions. For this purpose the yield on the gamma detectors is correlated on a shot-to-shot basis with the energy in the electron beam. On shots without a second scattering beam the radiation measured is produced from bremsstrahlung as dispersed electrons interact with the walls of the vacuum chamber. When the second laser is turned on, ICS is added as an additional radiation source and particularly intense interactions produce a significant excess signal on the gamma detector. 
There are collisions. To see whether there was also radiation reaction, we need to analyse the electron spectra.

The electron spectra are presented in the following section.
Edge feature. Small dataset, large fluctuations. Determining radiation reaction from low electron energies is a circular argument.
There seems to be a lower energy edge on collisions.

A statistical analysis of a larger dataset follows. The shots selected use the same criterion as before.
Removing drift, analysing laser energy correlation.
Finding normal distribution.
Likelihood for 4 low energy edges is XX and shots are collisions based on gamma signal.

The spectrum of the gamma radiation is shown. The retrieval of the gamma spectrum of the inverse Compton signal were performed by Jason Cole (Imperial College London) and Keegan Behm (University of Michigan). Details of the method can be found in \cite{Behm2018} or also in the Methods chapter of this thesis. Their contributions and work will be acknowledged and indicated as such again in the relevant parts of this chapter.
The spectrum changes from collision to non-collision and is anti-correlated with the electron energy on the 4 identified collisions.
The likelihood for this to occur by pure chance is XXX. This now backs up that we have seen radiation reaction.

Presenting models of radiation reaction and how the observed trend matches them.
\vspace{\baselineskip}

Finally, the electron spectra from this experiment are compared a second measurement of radiation reaction at higher electron energies performed at the same laser system. Electrons of around $2\,\mathrm{GeV}$ energy were collided at $a_0\sim 10$ but lack a spectral measurement of the gamma rays to complete the picture. The results of this campaign have been published in (REF Poder). Yhe stability of the electron energy and the stability of the electron spectrum of both measurements are used to draw conclusions for future experiments and to indicate which parameters and scale of data points would be required to distinguish different models of radiation reaction with statistical significance. This discussion is then continued more extensively and in more detail in C ARRAN REFS.


\section{Experimental Setup}

The experiment described in the following was performed at the dual $300\,\mathrm{TW}$ Ti:Sa Gemini laser system at the Central Laser Facility, Rutherford Appleton Laboratory, UK, in late 2015. Both arms provide two linearly polarised laser beams of central wavelength $800\,\mathrm{nm}$ at a pulse duration of $45\,\mathrm{fs}$ \textsc{FWHM} and collimated beam diameter of $\sim 150\,\mathrm{mm}$.

A sketch of the experiment is shown in Figure \ref{RR15:figs:setup_sketch}.
\vspace{\baselineskip}

\begin{figure}[h]
\centering
\includegraphics[width=0.8\columnwidth]{Exp_setup_render_RR2.png}
\caption{Conceptual sketch of the experiment setup rendered with Blender.
Based on a sketch made by J. Cole (Imperial College) for \cite{Cole2018} and adapted for this work and for\cite{Behm2018}. 
%From left to right: a high intensity laser beam (red) focused with a f40 spherical mirror generates up to $\mathrm{GeV}$-scale electrons in a gas jet (LWFA). A second laser beam (red) is focused down tightly at the edge of the gas jet by an f2 off-axis parabola to scatter the electron beam shortly after it leaves the jet. The electrons (blue) are being dispersed by a dipole magnet and detected on a scintillating lanex screen (grey). Finally, gamma rays (green) propagate through a vacuum kapton-window and a lead aperture onto a stack of scintillating CsI crystals that are imaged by a camera.
}
\label{RR15:figs:setup_sketch}
\end{figure}


The first part of the experiment is a setup to produce a relativistic electron bunch from laser wakefield acceleration (LWFA): a laser beam is focused down by an f/40 spherical mirror with $6\,\mathrm{m}$ focal length onto the leading edge of a $15\,\mathrm{mm}$ conical supersonic helium gas jet target. The gas jet is positioned $6\,\mathrm{mm}$ below the laser axis in order to avoid damage to the nozzle from the second more divergent laser beam. Electrons are accelerated via LWFA and propagate further downstream where they are dispersed by a permanent dipole magnet of integrated field strength $\int B(x) \mathrm{d}x = 0.4\,\mathrm{Tm}$ onto a scintillating LANEX screen imaged by an Andor Neo camera to measure their spectrum. The typical \textsc{fwhm} focal spot of the driving laser pulse measures $37 \times 49 \,\mathrm{\mu m}$ with an energy on target of $(8.6 \pm 0.6)\,\mathrm{J}$, which corresponds to a normalised vector potential $a_0 = 1.9 \pm 0.1$. The electron density of the target was $(3.7 \pm 0.4) \times 10^{18} \,\mathrm{cm}^{-3}$. 
\vspace{\baselineskip}

The second laser beam is focused down tightly onto the opposite edge of the gas jet at 180 degrees from the first laser using an f/2 off-axis parabola (OAP). This is aimed to scatter from the electron bunch accelerated by the first laser and to generate a bright burst of gamma rays from inverse Compton scattering (ICS). The OAP is fitted with a central hole, $21\,\mathrm{mm}$ in diameter, to enable propagation of the electrons, gamma rays and the remaining laser light of the $f/40$ wakefield driver beam. In addition, a plastic ring of $28\,\mathrm{mm}$ radius around the hole protects the optics and the laser chain upstream from potential laser light scattered in an interaction with the plasma. The combined loss of reflective surface leads to a decrease in intensity of the flat-top beam of around $16\%$. The energy on target was typically $(10 \pm 0.6)\,\mathrm{J}$ focused into a spot of $2.4 \times 2.8\,\mathrm{\mu m}$ \textsc{fwhm}, corresponding to peak normalised vector potential of $a_0 = 24.7 \pm 0.7$. The peak value of the quantum non-linearity parameter, $\eta$, in a head-on collision is $\eta = 2\gamma a_0 \hbar \omega_0/m_e c^2$.
In this case $\eta \approx 1.19 \times 10^{-5} \times E[MeV] a_0 \approx 0.3 \times E[GeV]$, i.e. $\eta \approx 0.15$ for $0.5\,\mathrm{GeV}$ and $0.3$ for $1\,\mathrm{GeV}$ electron energy.

The narrow cone of gamma rays from ICS propagates through the hole of the f/2 OAP, the aperture of the dipole magnet, then through a $50\,\mathrm{mm}$ aluminium laser beam block and finally leaves the vacuum chamber through a $250\,\mathrm{\mu m}$ thick kapton vacuum window. At air, the gamma rays are incident onto a stack of caesium-iodide (CsI) crystals that measures the spectrum of the high energy radiation. The stack is 33 crystals high and 47 crystals deep, each crystal $5\,\mathrm{mm} \times 5\,\mathrm{mm} \times 50\,\mathrm{mm}$, with the $5\,\mathrm{mm}\,\times\,5\,\mathrm{mm}$ sides facing to the side with respect to the gamma-ray axis and being imaged by an Andor iXon camera. The crystals are spaced by $1\,\mathrm{mm}$ aluminium dividers and the front side of the stack is fortified by a $9-\mathrm{mm}$-thick steel plate. The entire stack is housed and shielded in a lead enclosure with a circular aperture of $15\,\mathrm{mm}$ diameter, corresponding to an acceptance angle of $\sim 0.7\,\mathrm{mrad}$ at $\sim 2.2\,\mathrm{m}$ from the interaction point.
\vspace{\baselineskip}

The two laser beams are overlapped in space and time using spatial interferometry: a reflective 90-degree knife-edge prism is placed at the interaction point and reflects both counter-propagating laser beams collinearly onto a CCD camera chip equipped with a x10 long-working-distance microscope objective. Since the laser beams are cross-polarised, a polariser is needed to enable interference which also gives control over the relative brightness of the beams. The different radii of curvature of the f/40 and f/2 beams result in the formation of a circular interference pattern when the laser pulses overlap in both space and time. The overlap is then further improved by optimising the visibility of the fringes to a precision of around $\pm 30\,\mathrm{fs}$. 

Unfortunately, the overlapping the beams both in time and space at one point in time does not guarantee that they remain overlapped. Some studies performed at Gemini (by Oxford REF RAL REPORT SHALLOO) indicate that drifts exist in the system and the necessary precision of the alignment is only maintained for about half an hour. This has been greatly improved in recent years on other experiments. In addition, the precision of the prism position determines how accurately the interaction point is defined.
\vspace{\baselineskip}


%\begin{figure}[h]
%\centering
%\includegraphics[width=0.8\columnwidth]{scintillator.JPG}
%\caption{Scintillator array used in experiment. The CsI crystals are encased in an aluminium casing approximately $220\,\mathrm{mm}$ deep and $150\,\mathrm{mm}$ high with crystals  and an array of in total. The scintillator was placed into the beam axis, the short side facing the beam enabling the radiation to travel as long as possible through the array. The holed side (diameter of holes around $5\,\mathrm{mm}$ each) was imaged using a scientific camera.}
%\end{figure}


\section{Correlating the Gamma-ray signal with the Electron Spectrum}

We need a technique to identify successful collisions. Investigating the radiation yield from electron beams.

The dispersed electrons will emit broadband bremsstrahlung when interacting with matter in their trajectory, particularly high-Z materials such as the vacuum chamber walls. The radiation is in a spectral range comparable with the expected ICS signal and is also measured by the gamma-ray detectors. The main source of bremsstrahlung, the chamber roof, is located off-axis and a large fraction is shielded efficiently by blocking the direct line of sight with sufficient amounts of lead. This reduces the total background signal on the detector and later the signal-to-noise ratio, but also allows a spectral retrieval of the bremsstrahlung component as source of radiation is more constrained.
The energy emitted through bremsstrahlung by an electron is proportional to its energy squared and the total energy deposited in the caesium-iodide crystals of the gamma spectrometer converts linearly into scintillation light at an efficiency of $\approx 5 \times 10^4 \,\mathrm{MeV}^{-1}$ (REF). The total yield of the detected bremsstrahlung signal on the camera chip, $S_{BG}$, should then follow this relation:
\begin{equation}
S_{BG} = c_{BG} \int (\mathrm{d}N_e /\mathrm{d}\gamma)\,\gamma^2 \mathrm{d}\gamma = c_{BG} Q \left\langle \gamma^2 \right\rangle,
\end{equation}
where $c_{BG}$ is a constant that encapsulates the complicated details of the interaction and the experimental setup such as the conversion efficiencies of electron energy to bremsstrahlung photons, photons depositing their energy in the detector crystals, number of photons emitted from the scintillator per energy deposited, viewing angle of the camera, collection efficiency of the imaging system and quantum efficiency of the camera. $\mathrm{d}N_e/\mathrm{d}\gamma$ is the charge distribution of the measured electron spectrum, $Q = \int (\mathrm{d}N_e/\mathrm{d}\gamma) \mathrm{d}\gamma$ the total charge and $\gamma$ is the relativistic Lorentz factor of the electrons. 
\vspace{\baselineskip}

%\begin{figure}
%\centering
%\includegraphics[width=0.8\columnwidth]{ElecQ2_CsI_null_Correlation.pdf}
%\caption{Electron energy squared times against signal strength (counts) measured from the CsI stack.}
%\end{figure}


\begin{figure}
\centering
\includegraphics[width=0.5\columnwidth]{ElecQ2_CsI_Correlation.pdf}\includegraphics[width=0.5\columnwidth]{DCsI_sigma.pdf}
\caption{Left: Squared energy of the electron beam vs. the signal strength on the gamma detector in pixel counts. The reference shots without scattering beam are indicated in blue with a regression line. In orange shots with both laser beams on. Right: Shot number against the deviation from the expected gamma signal using the difference of the signal from the blue regression line on the left plot.}
%\caption{Left: The total energy of the electron beam squared (x-axis) vs. the number of CsI counts (y-axis) which is proportional to the energy of the photons. The photon energy radiated by the electrons is proportional to $\gamma^2$ which in turn is proportional to $E^2$. Hence, a linear relationship should be visible. As to be seen in the diagram, the shots with colliding beam off (red) are fitted well with a linear function whilst the data points for the collisions do not seem to fit on the same curve in many cases. This motivates that there seems to be a different process to take place when the colliding laser pulse is turned on. It can also be seen that the CsI counts for collision shots are scattered a lot, which could be related to different overlaps of laser pulse and electron beam resulting in a different range of radiation produced. Right: Shots in the order performed in experiment versus the gamma detector signal above the expected background signal according to the $E^2$-scaling described previously. The red dots are representing the reference shots with the driver beam only, whilst the blue dots are the collision shots with driver and scatterer. Especially shots 4,5,6 and 8 stand out remarkably above the background. The last few shots on the other hand seemed to have missed completely which could be related to a drift in alignment over time.}
\label{Results:Figs:NullColl:DeltaCsIVsSigmaE2+CsIVsE2}
\end{figure}


Figure \ref{Results:Figs:NullColl:DeltaCsIVsSigmaE2+CsIVsE2} shows the relation of $Q \left\langle\gamma^2\right\rangle$ to the total number of pixel counts (yield) on the gamma-ray detector. An example can be seen on Figure \ref{fig:Cole_espec_gamma_example}. The data points relating for shots without a scattering beam (in blue) seem to follow the expected linear trend, with $c_{BG}$ XX NUMBER. The correlation coefficient for the data has a value of XX NUMBER PM NUMBER. 
\vspace{\baselineskip}

On shots with the scattering beam on and if the collision is successful, we expect the second beam to produce a burst of gamma rays from inverse Compton scattering. At the detector we expect to measure a combined signal of bremsstrahlung background and the ICS signature. The emitted energy of the ICS radiation and hence the produced detector signal, $S_{ICS}$, is proportional to $\gamma^2$ similarly as for bremsstrahlung but also scales with the normalised vector potential (REF PRX THOMAS 2012 for GAMMA $<$ XXX, CORDE REV MOD 85, 2013):

\begin{equation}
S_{ICS} = c_{ICS} \int  a_0^2 (\gamma) (\mathrm{d}N_e /\mathrm{d}\gamma)\, \gamma^2 \mathrm{d} \gamma,
\end{equation}

and the total signal, $S_{total}$, then combines to

\begin{equation}
S_{total} = S_{BG} + S_{ICS} = c_{BG} \int  (\mathrm{d}N_e /\mathrm{d}\gamma)\, \gamma^2 \mathrm{d} \gamma + c_{ICS}   \int a_0^2 (\gamma) ( \mathrm{d}N_e /\mathrm{d}\gamma)\, \gamma^2 \mathrm{d} \gamma,
\end{equation}
where now $c_{ICS}$ similarly encapsulates all the complex physics of coupling constants, cross sections and conversion efficiencies. In this case the value of $a_0$ at the interaction is not constant from shot to shot as a varying overlap of the laser pulse and the electron beam will result in changing interaction conditions. In addition $a_0$ can vary throughout the interaction if the duration of the interaction XX, either due to short focal length and long electron bunches. The better the overlap and the higher the intensity at the interaction, the stronger the ICS signal and the easier to distinguish from the background bremsstrahlung it will be. In addition, whilst we expect all of the electrons to produce bremsstrahlung, only a variable fraction of electrons will interact with the laser pulse. If the laser pulse is larger than the electron beam the variation in intensity the electrons experience will mainly be spectrally varying.
\vspace{\baselineskip}

If we now add the shots to the same plot with the beam on, we see that those shots are more scattered across the plane than the shots without the colliding beam. Some of the data points follow the trend relatively well, some have a much higher detector signal than expected from an electron spectrum at that charge and energy. These could be potential good overlaps. We do not see a clear linear trend for the beam on shots as the overlap and energies vary from shot to shot.

To qualify how much above the background these are, it is more useful to look at the ICS signal mainly by subtracting the expected bremsstrahlung background from the signal:
\begin{equation}
S_{ICS} = S_{total} - S_{BG}.
\end{equation}

We then calculate for each of the reference data points the relative difference to the calculated regression line and characterise their behaviour statistically. Assuming a random normal distribution we calculate the standard deviation of the system.
In Figure \ref{Results:Figs:NullColl:DeltaCsIVsSigmaE2+CsIVsE2} we now see the shots lined up with their actual shot number versus the gamma signal above the expected signal from the bremsstrahlung background. The y-axis now encapsulates the information of both the gamma signal and the electron spectrum.
As we can see most of the shots follow the expected trend very well, some are a few standard deviations above but 4 shots are clearly more than 5 sigma (based on a normal distribution the probability for one shot to be above 5 sigma is $< 0.00001\%$) above the signal. The number of potentially successful collisions matches the jitter of the electron beam and the laser pulse as characterised by J. Cole and C. Baird for \cite{Cole2018}.

Hence, this method allows identifying successful collisions, but on the other hand enables us to also identify collisions where ICS was negligible or not occurring at all. We continued the campaign by defocusing the scattering beam and translate the beam relative to the electron beam. The signal on most of these shots is much lower but they were all taken with the scattering beam on. Not all shots result in successful collisions or interactions that modify the spectrum significantly. The gamma signal now enables us to identify shots without significant interactions and add them to the pool of reference shots to further characterise the electron variations.


\section{Characterisation of the Electron Spectrum}

In this experiment two components of radiation reaction are measured: one is the energy loss of the electrons in the interaction with the laser pulse, the second is the radiation that this energy is being transferred into. 

%In this section the author will characterise the electron spectrum, and its variation from shot-to-shot. A more detailed statistical analysis required due to the intrinsic variations of the electron beam will follow later. This will help distinguishing the intrinsic fluctuations of the electron spectrum from signatures of radiation reaction.



\subsubsection{General description of electron spectrum and finding the edge}

\begin{figure}
\centering
\includegraphics[width=0.8\columnwidth]{RawDataV.pdf}
\caption[]{Example of a typical electron spectrum and gamma-detector signal. Replace with own plots.}
\label{fig:Cole_espec_gamma_example}
\end{figure}


A typical electron spectrum measured on the experiment and a lineout are shown in Figure \ref{fig:Cole_espec_gamma_example}. 

A detailed description of the treatment of the raw data (background subtractions, image transformations, tracking etc.) can be found in the Methods section and is not expanded here.

The characteristic electron spectrum consists of two parts (see Figure \ref{fig:Cole_espec_gamma_example}): one is a low charge but high energy tail reaching $800-1000\,\mathrm{MeV}$, the other part contains a high charge at lower energy that falls off rapidly at an edge-like spectral feature at around $450-550\,\mathrm{MeV}$. The electron spectrum is cut off at $250\,\mathrm{MeV}$ due to limitations in the spectrometer setup. 
The high energy electrons are consistent with self-injection (REF), whereas the high charge component with the distinct spectral feature is consistent with shock injection caused by damage to the gas nozzle. The shock fronts are visible on an optical image seen in Figure \ref{RR15:figs:prettypic} along with the damage on the nozzle.

\begin{figure}
\centering
\includegraphics[width=0.8\columnwidth]{Elec_EdgeSpectrum_Null_V2.png}
\caption{Electron spectra of the reference shots (driver beam only) in red, normalised to their peak charge. The x-axis shows the electron energy in $\mathrm{MeV}$ and is cut off at $800\,\mathrm{MeV}$ as high energy features will not be further investigated. The blue line is the first derivative of the spectrum with significant maxima marked by the circles. The yellow circle indicates the peak selected to define the position of the edge. REPLACE THIS PLOT.}
\label{Results:Figs:Edge:ShotsNull}
\end{figure}
Shock injection has been observed by other groups introducing a blade (REF \cite{Tsai2018}), silicon wafer (REF) or a wire (REF BURZA 2013 PRSTAB) into the gas flow to generate a shock front.



\begin{figure}
\centering
\includegraphics[width=0.8\columnwidth]{20151217r002s003_PrettyPic.jpg}
\caption[]{Pretty Pic place holder. Add second image without North beam and where shock is visible. Indicate the nozzle damage.}
\label{RR15:figs:prettypic}
\end{figure}


The position of the spectral feature is in the following defined by using the derivative of the integrated spectrum. The sharp cut-off in the spectrum results in a peak in the first derivative, which will be used as number for the energy of the edge.
An example is shown in Figure \ref{Results:Figs:Edge:ShotsNull}.





\section{Statistical Analysis of Electron Spectra}

\subsubsection{Motivation}
In the previous sections we showed that by using both the electron spectrum and the gamma ray signal successful collisions can be identified. After identifying successful collisions the next step is quantifying whether there was a change in the gamma spectrum and whether there was a measurable amount of energy loss visible in the electron spectrum.


\begin{figure}[h]
\centering
\includegraphics[width=0.5\columnwidth]{ElecEdge_Raster_Drift.pdf}\includegraphics[width=0.5\columnwidth]{ElecEdge_Raster_Slices_Drift.pdf}
\caption[]{Distribution of the electron edge energies of the data shots plotted against time of day the data was taken. The energies are scattered around a line that slowly increases with time. A potential fitting regression line with a confidence interval (shaded area) is drawn. By assuming a random distribution around a slowly drifting mean the data set was split into individual time slices. Under this assumption the linear relation of energy and time holds very well at a correlation coefficient of $0.86$. The rate of increase is $21.6\,\mathrm{MeV}/\mathrm{hour}$.}
\label{fig:Cole_drift_raw}
\end{figure}

To put an accurate number on the energy loss we have to take into account the intrinsic variations of the electron source and correlations to other fluctuating variables such as laser energy. In this section the author will present a statistical analysis of the shots without the scattering beam on or where ICS was judged to be negligible, characterising the typical electron spectrum and the extent of its fluctuations. These shots will be from now on referred to as reference data or null shots. 

The set of electron spectra presented in Figure \ref{Results:Figs:NullColl:DeltaCsIVsSigmaE2+CsIVsE2} is the interesting data set including potential high-intensity interactions. This data set is relatively limited to 23 shots, 10 of which are reference data.


\begin{figure}[h]
\centering
\includegraphics[width=0.5\columnwidth]{Cole_LaserTime_Corr.pdf}\includegraphics[width=0.5\columnwidth]{ElecEdge_raster_drift_Laser_Corr.pdf}
\caption[]{Left: Driver laser energy over a 2 hour window. The actual data set extends further but the laser energy meter does not cover the entire time. The correlation coefficient is -0.11 and the confidence interval for the slope of the linear regression includes negative and positive values. This indicates that there is no real correlation of the two parameters, and the laser energy fluctuated around a constant mean. Right: Laser energy on the shots with the respective drift-corrected relative energy shift. A weak positive correlation was found but at a very low level (correlation coefficient $\sim 0.35$). This was hence not used in the following analysis.{\color{red}is the line and error bar correct. the error bars suggests a reasonably strong correlation?}}
\end{figure}


\subsubsection{Identify additional shots}

However, a small sample size (10) is not sufficient to draw reliable conclusions on the general character of the fluctuations. A larger data set from the same day at comparable conditions is taken into consideration. 

In contrast to the 10 reference shots where the scattering beam was turned off, the scattering beam was on for almost all subsequent shots. However, using the gamma-ray signal as indication for overlap, we selected shots that very closely (within one standard deviation) matched the expected trend of bremsstrahlung and made a new data set based on these reference shots. A significant interaction that would modify the spectrum would result in significant amount of additional radiation and is then discarded. This new data set consists of additional 89 shots. 
\vspace{\baselineskip}

\subsubsection{Check correlation with laser energy}

We can check if this correlates with laser energy. There appears to be a positive correlation but only at lower confidence. The correlation coefficient is around 0.35 and the confidence interval for the slope encloses only very small gradients of order few percent over 4 J energy range (whereas the standard deviation is already scattered at 8 percent). When looking at the smaller dataset the gradient confidence interval even encloses zero which indicates no correlation. This all indicates that this was not a very important factor in this context and we will hence ignore this factor in our analysis. 
\EliasComm{Expand this bit to justify points more}
\vspace{\baselineskip}

\subsubsection{Temporal Drifts}

When tracing the time the data was taken against the electron energy of the spectral feature a slow drift in the mean energy becomes apparent (see Figure \ref{fig:Cole_drift_raw}). The laser energy does not drift over the time period indicated and only fluctuates (see Figure REF NUMBER). This speaks against dominance of self-injection which scales with the laser energy (REF) and is consistent with the hypothesis that the high charge component of the spectrum is injected through shock injection and that shocks were generated due to (progressing) damage in the gas nozzle.  It seems to degrade and moves the shock closer to the leading edge of the gas target, increasing the acceleration length and energy over time. At the same time the average laser energy remains fairly stable.

\begin{figure}[h]
\centering
\includegraphics[width=0.8\columnwidth]{ElecEdge_DE_drift_histo.pdf}
\caption[]{Histogram of relative deviation of the spectral edge energy from the expected slowly varying mean ($\Delta E/E$). The total distribution is overlaid with a KDE and resembles closely a Gaussian distribution (orange). The shaded area indicates a 2 sigma (95\%) confidence interval assuming a normal distribution.}
\end{figure}


We split the varying components into two parts: one is some random (probably normal distributed) noise fluctuating from shot to shot, the second is the mean around which the fast fluctuations take place, which is slowly increasing with time. We calculate the increase of the mean and see that it correlates well with time at a correlation coefficient of 0.86.
Now we look at the deviation $\Delta E$ from this slow drifting mean in relative terms ($\Delta E/E$) as depicted in Figure XX NUMBER. The KDE overlay also indicates a normal distribution. 


Assuming this is a normal distribution we can assign probabilities to certain energies. The standard deviation of this distribution is $\sigma_{Cole} = 0.077$. In other words this means we expect 95 percent of the energies to be within $\pm 15.4\,\%$ ($2\,\sigma_{Cole}$) of the mean. 

\EliasComm{is it possible to put an error bar on the KDE? Might be a lot of statistics involved.}


\iffalse
\section{Estimating Collision Probability}

Something comparable to what Jason and Chris Baird did but without PIC simulations.

Check if the expected number of collisions matches the observed `bright' shots.
Base this on jitter data. Monte-Carlo simulations and define overlap threshold.


- F/40 spatial jitter\\
- F/2 spatial jitter\\
- F/2 temporal jitter\\
- F/40 temporal jitter\\
- electron divergence\\

Beam size of electrons similar to Gaussian beam?


$w_{e} (z) = w_{e,0} \sqrt{( 1 + (z_{e}/(z_{R,e}) )^2)}$\\
and \\
$z_{R,e} = w_{e,0}/\theta_{e}$

Estimate collision probability here.
\fi

\section{Gamma Spectra}



\section{Indicators of radiation reaction in the electron spectrum}

\subsubsection{Recap}
After having now characterised the electron spectra and the expected gamma signal, we can now more closely look at potential indicators of radiation reaction.

\begin{figure}[h]
\centering
\includegraphics[height=0.25\columnwidth]{20151217r002_NullGroup.jpg}
\includegraphics[height=0.25\columnwidth]{20151217r002_CollGroup.jpg}
\includegraphics[height=0.25\columnwidth]{up_arrows.jpg}
\caption{Images of the gamma ray spectrometer (upper row) and electron spectrometer (bottom row) at shots with the colliding laser beam off (left) and on (right), using a constant colour scale. The shots with the colliding beam show bright signals on the scintillator array whilst the electron energies remain fairly similar throughout.}
\label{Results:Figs:NullColl:Montage}
\end{figure}

\begin{figure}[h]
\centering
\includegraphics[width=0.9\columnwidth]{ElectronSpectra2.pdf}
\caption{Waterfall plot from paper. Figure 4. Sort out licence or make myself.}
\end{figure}

To summarise, the dataset we investigate more closely consists of 13 shots with both beams on and 10 reference shots without the second scatterer beam.
Figure \ref{Results:Figs:NullColl:Montage} shows a montage of the raw data of the two relevant diagnostics, the electron spectrometer (bottom row) and the gamma-ray detector (top row) for reference shots (left) and dual-beam collisions (right). The electron spectrometer and the gamma-detector images are on the same colour scales respectively. The line-outs of the integrated and normalised electron spectra are also shown for convenience in Figure XX, along with a dotted line indicating the position of the distinct spectral feature.
\vspace{\baselineskip}

The electron source has some intrinsic variations that make singling out a successful collision based on lower energy less convincing.
Hence, in section XX we decided on a selection criterion based on the signal on the gamma detector coupled to the electron beam energy to determine which shots were successful. 4 shots with a gamma signal 5 standard deviations or more above the expected background were identified. The bright burst of gamma rays lights up the CsI gamma detector as seen on the montages in Figure \ref{Results:Figs:NullColl:Montage}. The 4 bright shots are also indicated in Figure XX in red.
\vspace{\baselineskip}


\begin{figure}
\centering
\includegraphics[width=0.5\columnwidth]{DCsI_sigma_v_ElecEdge.pdf}\includegraphics[width=0.5\columnwidth]{RR15_GammaSpec_OnOff.png}
\caption{Left: Position of the low-energy edges (x-axis) versus the gamma ray signal above the expected radiation background (y-axis) in units of standard deviation from the reference mean. Reference shots are shown in red and collision shots in blue. The edge positions are very scattered under both conditions with the majority of the collision shots exhibiting higher edges. The brightest gamma ray signals on collision seem to coincide with a low position of the edge. Right: Critical energy of gamma spectra for beam on and off.}
\label{Results:Figs:Edge:PosVsCsI}
\end{figure}

\subsubsection{Statistical Significance of 4 successful shots and spectrum}

From Figure NUMBER HERE. We see that the electron spectral feature varies quite a bit over the course of the shots but that all shots selected using our criterion have low electron energies. The probability for this, assuming a normal distribution as derived in the previous section, is around NUMBERS HERE for one shot but NUMBER TO THE FOUR for all four of them to occur. If we are sufficiently convinced that these low electron energies are equivalent to energy loss from our mean, we can estimate the conditions at the interaction. For an electron energy of CHECK NUMBER $550\,\mathrm{MeV}$ we require a laser intensity of $a_0 \approx 10$ using PRX THOMAS 2012. This is much lower than the peak intensity of our laser pulse.
\vspace{\baselineskip}

A second angle at confirming the experiment conditions is the gamma spectrum. Jason Cole (Imperial College) and Keegan Behm (Michigan University) performed this analysis first and provided the results. More details on the analysis can be found in the Methods chapter or in \cite{Behm2018} outlining the modelling and analysis procedure applied in this specific case.
In Figure XX we see that the critical energy of the gamma spectrum is relatively stable around 45 MeV for all collision shots. A significant fraction of radiation seems to be emitted in all cases (relative to the background) and skew the energy estimate to this average number. If we relate these numbers to the electron beam measured, we can from both sides narrow down the exact conditions at the interaction.
\vspace{\baselineskip}

Both measurements (gamma-rays and electron energy loss) seem to be indicating that the laser intensity was more in a ballpark of $a_0 \approx 10$ instead of the peak intensities of close to $a_0 \sim 25$ we aimed for.

\subsubsection{Explaining lower intensity}
The lower intensity is actually easily explained. The timing of the beams was done using spatial interferometry. The visibility of fringes was optimised which means that both laser pulses are timed to arrive within $30\,\mathrm{fs}$ of each other. This timing is true for both laser pulses travelling through vacuum. What we are actually aiming to time is the arrival of the scattering beam and the electron beam which is trailing behind the driving laser pulse. To estimate the intensity of the laser pulse at the interaction we have now to consider the delay of the laser pulse as it is travelling through a medium (plasma) to the point of injection and then, assuming that the electrons relatively quickly reaching a velocity close to the speed of light, the relative delay of the laser pulse and the electron beam. This delay is around half a plasma wavelength SEE REFS.

For the delay induced through the South beam travelling through a medium we use the group velocity $v_f \approx 1-\frac{3}{2}\frac{n_e}{n_c}$, which is the standard non-linear group velocity in media minus the etching velocity (called front velocity in REF DECKER 1996 Phys Plasmas). The distance that the North beam travels in vacuum during the time the South beam travels through a medium of thickness $d$ is $d'=d c/v_f$. The new collision point is now

\begin{equation}
\delta z_1 = \frac{1}{2} (d \frac{c}{v_f} - d) = \frac{d}{2} (1/(1-\frac{3}{2} \frac{n_e}{n_c}) -1) \approx \frac{d}{2}(1+\frac{3}{2} \frac{n_e}{n_c} -1) = \frac{3d}{4}\frac{n_e}{n_c}
\end{equation} 

The electrons trail behind the South laser pulse around N plasma wavelengths (the factor half are as the collision happens midway):
\begin{equation}
\delta z_2 = 0.5 N \lambda_p = 0.5 N \frac{2 \pi c}{\omega_p} = 0.5 N \frac{2 \pi c}{\omega} \frac{\omega}{\omega_p} = 0.5 \lambda_0 \sqrt{\frac{n_c}{n_e}}
\end{equation}

As total delay we can estimate
ADD EQUATION HERE FOR DELAY AND A DRAWING TO SHOW THIS MORE CAREFULLY.

\begin{equation}
\delta z = \frac{3d}{4} \frac{n_e}{n_c} + N \frac{\lambda_0}{2}\sqrt{\frac{n_c}{n_e}},
\end{equation}
where d is the distance of injection point from front of gas jet
N is number of plasma periods between f/40 and electron bunch.

\EliasComm{Maybe add a little sketch to indicate where all these delays come from.}

This amount of defocus then reaches an average intensity of NUMBER 10 instead of the peak intensity of close to 25 possible in this setup. A variation of overlap and the electron energy will then result in an accidental scan of our conditions.

The injection point varies depending on the evolution of the laser pulse and has to be measured separately. Shock injection could provide a definite and known injection location. The position of the shock could be measured by an optical probe on-shot.
\vspace{\baselineskip}

With this knowledge of the focal spot size and the timing fluctuations, we can also estimate the number of collisions we expect.
This has been done by Jason Cole (Imperial College) and Chris Baird (York) by inserting the jitter in time and space of the electron beam and the laser, in one case running a simple Monte-Carlo simulation, in the second case building this on the amount of overlap achieved in PIC simulations.

The number of successful collisions was of order 3-4 shots which is in the ballpark of our 4 successful collisions.
\vspace{\baselineskip}


\subsubsection{Agreement with models}

After having now narrowed down the conditions at the interaction, confirmed that the frequency of the interactions and the selected shots appear to match up, we can now look at the detailed relations of these shots and how they line up with actual models.


If we plot the electron spectrum (assuming post-interaction) against the critical energy of the gamma-ray spectrum, we see that they correlate negatively. This again supports the hypothesis that we are looking at ICS with radiation reaction. The background signal from bremsstrahlung, for instance, emits the radiation after the electron spectrometer and hence we would expect a positive correlation (a harder spectrum for higher energies). The fact that we see a negative correlation confirms our suspicion that the energy loss is converted into a gamma-ray signal before the measurement.

Now here some part about the models and how they relate, how the plot is made but this is all Jason's and Tom Blackburn's work and should be referenced as such.
The on-shot synchrotron-like spectrum and the shift of the cut-off edge in the electron spectrum were then combined to constrain the parameter space (normalised laser vector potential $a_0$ and electron energy) at the interaction, relying on a range of models.



\begin{figure}[h]
\centering 
\includegraphics[width=0.7\columnwidth]{EcritvsEf.pdf}
\caption[]{Theory blobs. Sort out licence, Figure 9 in \cite{Cole2018}.
}
\end{figure}
Some comments on the actual models?
\vspace{\baselineskip}

Some comments on the theory contours.
The data shows that the results match a model that requires radiation reaction. No radiation reaction overestimates the energy loss and the energy of the radiated gamma radiation. Our results are in agreement with all the models in question, more overlap with the models including quantum corrections which could be a semi-classical model with classical trajectories and merely reduced emitted power or a fully stochastic quantum model. This is all on a one sigma confidence level so a strong distinction is not possible, we require more data.
An analysis how much more data we would require to make a conclusion beyond the 2 sigma level will follow in the next section.
\vspace{\baselineskip}

In addition it becomes clear that the difference between a fully stochastic quantum model and a semi-classical model is not very visible in this case. There is in indication that the models depart at higher energy losses, i.e. higher laser intensities but it would also require a higher electron energy. This slow departure from each other in contrast to the relatively distinct classical model is founded in the electron observable which is the energy loss from a characterised mean. When looking at the mean energy loss for the quantum and the semi-classical model it becomes evident that by definition the values are identical. Even though the energy loss of the spectral feature is not the same thing it closely relates to the mean energy loss. A more sensitive observable that behaves very differently for both models is the variance of the spectrum or the shape of the spectrum. This will be closer illuminated in the next section. 
\vspace{\baselineskip}

Finally, it should be noted that the interaction is more complicated. 
Knowledge of exact on-shot laser profile in time and space.
Knowledge of the electron beam properties in time and space, chirp.
Variation of spatial and temporal overlap.
Eta is not constant over the course of the interaction due to defocusing, energy loss, chirp in the beam, energy spread.
Also modelling constraints in the theory (LCFA).

\section{Comparing Results Poder and Cole}

In the framework of the data set presented first indications of differences in the models become evident, but a definite discrimination is not within its reach. More data, stability, higher electron energies, laser intensities and additional observables could enable this as part of a future precision measurement.

The author will attempt to indicate the importance of a stable electron source in the following section by comparing the data set presented with a second measurement of radiation reaction taken at the same facility and published in Poder et al \cite{Poder2018}. For this purpose, the electron spectra of the second experiment are included in a statistical analysis as well.
In this context, the choice of targetry (gas jet vs gas cell), injection mechanisms and diagnostics will also be discussed.

\subsubsection{Experimental setup}

The experimental setup in \cite{Poder2018} is very similar to the experiment described at the beginning as it is based on the same facility, relying on similar optics and beam line designs. A sketch of the setup is show in Figure \ref{RR15:figs:exp_sketch_Poder}.


\begin{figure}[h]
\centering 
\includegraphics[width=0.9\columnwidth]{ExpSetup_Poder.pdf}%\includegraphics[width=0.9\columnwidth]{ExpSetup_GasJet_MagnetOnly.pdf}
\caption[]{Conceptual sketch of the experimental setup as used on the radiation reaction campaign \cite{Poder2018} . Adapted from Figure XX in \cite{Cole2018}.
%(from left to right): a laser pulse (in red) is focused by an f/40 spherical mirror onto the entrance of a gas target (gas jet or gas cell). The intense laser pulse drives a wakefield and accelerates electrons (blue) to relativistic energies. A second laser is focused with an f/2 off-axis parabola onto the exit of the gas target scattering the electrons and emitting a bright flash of gamma rays (green) from inverse Compton scattering. A permanent dipole magnet is used to disperse and characterise the energy of the electron beam on a scintillating LANEX screen. The gamma rays propagate through a kapton vacuum window (orange) onto a stack of caesium-iodide (CsI) crystals. The sketch is based on work by J. M. Cole, Imperial College, for \cite{Cole2018}.}
}
\label{RR15:figs:exp_sketch_Poder}
\end{figure}

The first laser is focused by an f/40 spherical mirror to a spot of \textsc{fwhm} dimensions $(59 \pm 2) \,\mathrm{\mu m} \times (67\pm 2)\,\mathrm{\mu m}$ into the $20\,\mathrm{mm}$ long gas cell filled with helium at an electron density of $2 \times 10^{18}\,\mathrm{cm}^{-3}$. The energy delivered on target is on average about $9\,\mathrm{J}$ reaching a peak normalised vector potential of $a_0 = 1.7$.
\vspace{\baselineskip}

Gas cell targets have sometimes produced superior results from gas jets using the same laser system (Poder v Kneip, Osterhoff, \cite{Kuschel2018}) in terms of charge, shot-to-shot stability and maximum energy gain.
However, some of the most stable LWFA sources have been achieved using gas jet targets (Faure Nature 2006) and the open geometry allows for diagnostic access, colliding pulse experiments as well as a route to high repetition rate operation.
The variety of gas targets, their performance and the performance of the specific laser system make it difficult to compare gas cells and jets in general with each other. Comparisons can in most cases only be done between individual specimens and conclusions drawn from such a comparison might only be valid in this limited context. 
\vspace{\baselineskip}

The second laser is focused down at about 1 cm downstream of the exit of the gas cell with an f/2 OAP, again with a hole in the centre. The energy on-target was measured to be $(8.8 \pm 0.7)\,\mathrm{J}$, already taking into account the loss in intensity due to the hole, at a \textsc{fwhm} pulse duration of $42\pm 3 \,\mathrm{fs}$. The intensity at the interaction point was estimated to be of order $a_0\approx 10$ at a spot size \textsc{fwhm} $7\,\mathrm{\mu m}$.
\vspace{\baselineskip}


Both lasers were synchronised to about $40$ fs accuracy using spectral interferometry \cite{Corvan2016}.
\vspace{\baselineskip}

The electrons accelerated via LWFA are dispersed by a dipole magnet of integrated field strength $\int B(x) \mathrm{d}x \approx 0.15 \,\mathrm{Tm}$ onto a scintillating LANEX screen.
The gamma rays from ICS are recorded by the same CsI stack as described previously, but this time rotated such that the long side of the crystals is oriented in the longitudinal direction. The stack acts as a profile screen instead of a spectral measurement and the brightness of the recorded scintillation light is proportional to the total energy deposited in the crystal stack REF HERE.

\subsubsection{Trends in Poder data}

The typical electron spectrum produced can be seen in Figure \ref{fig:Poder_espec_example}. The average shape is an exponential spectrum reaching energies in excess of $1.5\,\mathrm{GeV}$. At the lower end the spectrum is cut off at around $400\,\mathrm{MeV}$ due to limitations in the magnetic electron spectrometer setup. The spectra from the gas cell originate from self-injection and lack a sharp distinctive feature comparable to the spectral edge described in the previous section related to shock injection. This data set consists of 19 shots.

\begin{figure}
\centering
\includegraphics[angle=90, width=0.8\columnwidth]{example_espec_Cole_prelim.pdf}
\caption[]{Example of a typical electron spectrum. THIS HAS TO BE REPLACED WITH MY OWN PLOT AND WITH PODER DATA.}\label{fig:Poder_espec_example}
\end{figure}

Due to the lack of such a distinct feature, the characteristic number for each spectra of this experiment will now be the cut-off energy, which is as in \cite{Poder2018} defined as the energy at which the spectral intensity reaches 10 percent of its peak value.
\vspace{\baselineskip}

In this case, the only injection mechanism considered is self-injection. This process typically scales with the laser energy (REF), which is also confirmed in Figure \ref{fig:Poder_laser_corr}. The laser energy of the wakefield driver scales well linearly with the cut-off energy of the spectrum. The correlation coefficient is $0.9$. The linear fit follows the equation $\epsilon_{cutoff} = 0.07\,\mathrm{GeV/J} \times E_{laser} + 0.57\,\mathrm{GeV}$.

After scaling the spectra according to the linear relation found in Figure \ref{fig:Poder_laser_corr}, we can analyse the spectra similarly as before and see that this spectrum also follows a normal distribution. The standard deviation of the cut-off energy distribution is $\sigma_{Poder} = 0.035$, so 95 percent of the expected energies will fall within $\pm 7\,\%$ of the mean energy.

\begin{figure}[h]
\centering
\includegraphics[width=0.8\columnwidth]{Poder_Laser_Corr.pdf}
\caption[]{Laser energy of the wakefield driver beam before pulse compression plotted against the cutoff energy of the electron spectrum. The data points clearly follow a linear trend drawn in the straight green line with a gradient of $0.07\,\mathrm{GeV/J}$. The shaded area indicates the 95 percent confidence interval of the fitting function. The correlation coefficient for a linear fit is $0.9$.}\label{fig:Poder_laser_corr}
\end{figure}


\subsubsection{Comparing both datasets}

By factoring out known correlations of the spectra, scaling them accordingly (slow drift for Experiment A (Cole et al.) and laser energy for Experiment B (Poder et al.)), and by normalising the spectra to their total charge, the spectra become comparable. With this processed data set expected intrinsic or non-attributed fluctuations can be characterised and taken into consideration.


\begin{figure}
\centering
\includegraphics[width=0.8\columnwidth]{Comparison_ESpectra.pdf}
\caption[]{Averaged and scaled electron spectra for data from Cole (blue) and Poder (green). The spectra are normalised to a total charge of 1. The error bars indicate the energy dependent variance of the spectra.}\label{fig:comparison_especs}
\end{figure}

The average spectra from the two experiments are shown in Figure \ref{fig:comparison_especs}. The spectra were processed as mentioned before and then averaged over the available data shots. As seen the typical energy reached in the gas cell from Poder et al. is significantly higher by at least a factor two for the majority of the charge distribution. The y-axis indicates a normalised charge, where the sum of the recorded spectra is set to 1, which is why the Poder spectra appear to contain less charge as the total charge is distributed over a wider energy range. The shaded error bar region around the average spectrum indicates the standard deviation of the average spectrum at that particular energy. The higher electron energy also enables a potentially higher $\eta$ parameter in an interaction as it scales linearly with the electron energy.
\vspace{\baselineskip}

\begin{figure}
\centering
\includegraphics[width=0.8\columnwidth]{Comparison_Histo.pdf}
\caption[]{Distribution of the relative energy deviations from the mean ($\Delta E/E$ for Cole et al. (blue) and Poder et al. (green) after scaling. The overlaid lines are kernel density estimates (KDEs). The shaded areas indicate the $\pm2$ sigma intervals assuming a normal distribution.}\label{fig:comparison_histo}
\end{figure}


The spread of energies for the gas cell data is narrower than the data taken with a gas jet target. The shaded areas in Figure \ref{fig:comparison_histo} indicate the $\pm 2$ sigma or 95 percent confidence intervals (assuming a normal distribution). The Poder data has a typical fluctuation of the cut-off energy of around 7 percent. The spectral feature from the gas jet vary with 15 percent by around the double amount. This means that if our observable is coupled to the energy of the electrons we will require a certain number of shots based on the typical variability of the source to say with a few sigma confidence that a signal beyond this has been recorded. If we couple this with the probability to even collide two beams at all, the number of required additional shots increases by a significant amount.
\vspace{\baselineskip}

To put this into context: in \cite{Cole2018} 4 successful collisions were identified by consulting the gamma-ray detector signal which showed a bright signal, $5-10$ standard deviations above the expected background. The energies of these 4 successful collisions were about $15\,\%$ below the mean of the reference distribution. The cumulative probability for a normal distribution with standard deviation $\sigma_{Cole} = 0.077$ to see one shot with a relative deviation of the energy from the mean of $10\,\%$ or more is roughly $10\,\%$. 

For an individual shot this probability is non-negligible, but it holds in combination with the gamma-ray signal and over the $4$ collisions to confirm that this is to some extent as low due to radiation reaction. The cumulative probability for a normal distribution with standard deviation $\sigma_{Poder}=0.035$ on the other hand for the same scenario is only $0.2\,\%$ and holds the test by itself at a level of confidence equivalent to 2.7 as many shots at $\sigma_{Cole}$. At a repetition rate of $0.05\,\mathrm{Hz}$ (1 shot/20 s), typical for the Astra Gemini laser, a scan of 100 shots at $\sigma_{Poder}$ would take under 30 minutes, but would extend to $1.5$ hours to reach the same level of confidence. At this extended time-scale, however, even small drifts in alignment and timing can affect the reproducibility of the interaction conditions which requires few microns and femtoseconds precision and as a result call the validity of measurements into question. In addition, we have to factor in the potential deterioration of the gas target and laser performance over the course of the experiment. Tim can become a crucial factor in experiments that require such a high level of complexity and sensitivity.
\vspace{\baselineskip}

However, the error on the mean of the electron energy also propagates further into the estimate of the interaction conditions. If we manage to sustain the experiment conditions over the course of the experiment (spatio-temporal overlap, intensity of interaction, laser parameters and so on), the energy of the unperturbed electron beam is fluctuating according to a normal distribution and the energy of the post-interaction electron beam will follow the same normal distribution but shifted down by the average energy loss. For now we will ignore the potential cooling or heating effects of radiation reaction on the distribution function. In this scenario at a constant energy loss of $15\,\%$, we will measure a range of deviations from the unperturbed mean energy which we will call `energy loss', which is in reality the energy loss superimposed onto the normal distribution. The standard deviation of the Cole data $\sigma_{Cole}$ is half the size of the total energy loss, whereas for $\sigma_{Poder}$ it halves to $25\,\%$, which is still a significant range of values most of the measurements will show. That means that there is a significant error on the estimated energy loss for each measurement and a second diagnostic narrowing down the conditions at the interaction, for instance a spectral measurement of the gamma radiation, becomes imperative.

If we want to compare different models of radiation reaction based on this energy loss, an increased stability of the electron energy will help in two ways: firstly, the expected phase space of post-interaction energies for each model of radiation reaction at a fixed confidence level decreases, which might reduce the overlap of different models. Secondly, if the initial electron energy is sampled from a narrower distribution the error on that measurement decreases as well, reducing its footprint in phase space again.
\vspace{\baselineskip}

\begin{figure}
\centering
\includegraphics[width=0.5\columnwidth]{Comparison_Variance_Log.pdf}
\caption[]{Energy-dependent variance of the average spectra in introduced in figure 8. In blue the data taken on the experiment related to Cole et al., in green to Poder et al.. The total variance for Cole et al. was $1.5 \times 10^{-7}$ and $1.98 \times 10^{-8}$ in the case of Poder et al..}
\end{figure}\label{fig:comparison_varianceLog}

In some cases, however, even an increased level of confidence and more data is not resulting in an improved distinction of different models of radiation reaction. As seen in recent publications REFS HERE RIDGERS, measuring the difference between classical and models including some form of quantum corrections can be done based on the energy loss due to the reduced emission power in quantum systems (REF).

When trying to distinguish fully stochastic quantum models and a semi-classical model which includes a reduced emission power matched to the quantum model (Gaunt-factor), but uses a deterministic equation of motion, the emitted power is matched by definition and hence not a suitable figure of distinction. 
Instead, the shape of the spectral distribution, e.g. variance of the entire spectrum (not only of the energy of the spectral feature), becomes more important and another factor to discriminate the two, but again would require a certain level of stability and suitable measure of it. The stability of the spectral shape can, for instance, be quantified in terms of the energy dependent variance of the spectra in question. 

The total variance normalised to the energy range considered for the average gas jet data from Cole et al. is $1.5\times 10^{-7}$ whereas the result for Poder et al. is $1.98 \times 10^{-8}$. The energy dependent variance can be seen in Figure \ref{fig:comparison_especs} represented by the shaded error bars and in Figure \ref{fig:comparison_varianceLog} separately on a logarithmic scale. The base level of the variance is relatively low for the Poder data and decreases for high energies. For the Cole data especially the position of the characteristic feature, crucial for the two-beam measurements, appears prone to fluctuations. This makes the data set less suitable to track changes in the variance induced by radiation reaction. If the fluctuation is of similar amplitude or larger than the differences expected from the models, a definite distinction of a semi-classical and a fully stochastic quantum model at these conditions would then be very challenging, even with access to a larger data set. In Poder et al. a comparison of spectral shapes and attribution to models has qualitatively been indicated. However, due to the lack of a spectral measurement of the radiation and the large parameter space of unknowns, for instance spatio-temporal structure of the electron beam, more observables are required to definitively explain the differences in the measured electron spectra.
\vspace{\baselineskip}

This line of thought has been explored more explicitly and in more detail in REF C ARRAN PPCF, SPIE 2019. The aim of these publications is to indicate which experimental observables are suited to discriminate different models of radiation reaction and to give an estimate on the number of shots one would require in an experiment taking fluctuations into account.


\subsubsection{Summary}

From this analysis it appears that a gas cell is a better target and the energy dependencies can be easier correlated and factored out. On the other hand, the distinct feature and injection point of the electrons from shock-injection could be useful to maintain similar conditions over longer time. If the injection mechanism could be improved in terms of stability by a factor 2, it could outweigh the benefits of the gas cell.
In addition, REF C ARRAN indicates that low energy spread beams of suitable stability are in particular favourable for measurements of radiation reaction as they reduce the requirements for higher electron and laser energies. Literature shows that these are achievable through shock injection. Ideally a more stable low energy spread beam would be suited to discriminate quantum and non-quantum models.

Finally, a detailed knowledge of the exact interaction conditions is crucial and removes further uncertainty.
A gamma-profile arrangement as set up in Poder et al. allows a relatively model-independent measurement of the interaction intensity, whilst a spectral measurement of the ICS radiation might help to discriminate a classical radiation reaction model from quantum models that lead to a reduced emission power. Using both diagnostics on future experiments could greatly improve the understanding of the interaction and as a result improve confidence in modelling.


\section{Conclusion}

\subsubsection{Radiation Reaction}

Radiation reaction was measured in an all-optical setup for the first time by colliding a tightly focused laser beam of $a_0 \approx 10$ with a relativistic electron beam from LWFA of energy $\sim 500\,\mathrm{MeV}$ \cite{Cole2018} and up to $2\,\mathrm{GeV}$ \cite{Poder2018}. 

In \cite{Cole2018} the critical energy of the radiation reached an excess of $30\,\mathrm{MeV}$, the highest recorded gamma-ray energies from an all-optical setup at the time of the publication.

The measurements indicated first differences in the models of radiation reaction, but were not significant to definitively distinguish them.
\vspace{\baselineskip}

Successful high-intensity collisions are signalled by bright bursts of gamma radiation recorded on the gamma detectors.
The expected radiation background from bremsstrahlung was found by correlating the electron spectra and the signal on the gamma-detector on shots without ICS. This characterisation helped singling out 4 particularly intense interactions that were considered in more detail.
\vspace{\baselineskip}

The energy of the electron spectra on those 4 selected shots was found to be lower on average than on the other shots.
By performing a statistical analysis of a larger set of electron spectra, investigating dependencies on other parameters such as slow drifts and laser energy, the likelihood of seeing these lower energies was estimated and deemed significant.
\vspace{\baselineskip}

This information was combined with a spectral analysis of the gamma signal\footnote{performed by Jason Cole (Imperial College) and Keegan Behm} to confirm the conditions at the interaction from two avenues. The results are compatible with models including radiation reaction, but do not deviate enough in these experimental conditions to express a definite preference for a certain model.
\vspace{\baselineskip}

In the course of this analysis the stability of the accelerator and laser parameters were found to be of crucial importance to make more significant measurements of radiation reaction in the future. The importance of complementary measurements and observables is also very important as each angle narrows down the experimental uncertainties to focus on uncertainties introduced by modelling and theory.

\subsubsection{Statistical analysis of electron spectra}

The author then presented a comparative statistical analysis of electron spectra from both experiments aimed at measuring radiation reaction to demonstrate the differences at this explicit example. Two correlating factors, a slow drift in mean energy in case of Cole and the laser energy in case of Poder, were identified and considered in this analysis. The electron spectra from the Poder et al. campaign were more stable in terms of stability in energy and variance. The spectra did not only reach higher energies but the standard deviation of the energies were by a factor of $2$ more stable. The 2 sigma width of their distribution was $\pm 15\,\%$ for Cole and $\pm 7\,\%$ for Poder, which means that 95 percent of shots were expected to fall within this range. The variance of the spectrum was also more stable in the case of Poder et al. being 10 times lower than for the electron spectra observed in Cole et al..

Whilst the stability of the electron source was superior in Poder et al., the use of gas jet targets in conjunction with shock injection offer benefits in terms of stability of the injection point and the prospect of low energy spread electron beams that could facilitate discriminating radiation reaction models (REF CARRAN).

\subsubsection{Future ambitions and projects}

The expertise gained in these previous experiments paired with the technological capacities of the Astra-Gemini laser make it a realistic ambition to collide electrons at even higher relativistic energies ($\sim 2 \mathrm{GeV}$) and more intense laser fields ($a_0 \approx 25$) even at current laser facilities, expecting a gamma-ray spectrum with a critical energy close to $100\,\mathrm{MeV}$. These would enable probing radiation reaction as the energy loss due to the emission of radiation becomes significant in relation to the initial electron energy -- possibly even reaching a regime where QED effects become relevant as the `quantumness' parameter $\eta$, indicating how much the interaction is dominated by QED effects \cite{Blackburn2014} (maybe another REF), approaches values close to $1$. In this regime we even expect gamma rays to combine with laser photons to produce electron-positron pairs.
At the same time, this work in conjunction with C ARRAN REF indicates that by improving the electron beam quality differences in models can be measured even at comparable laser and electron energies, giving future experiments at Gemini or comparable laser facilities the opportunity to bridge the gap between these first measurements of radiation reaction and the future projects being worked on at the next generation experiments.

Future experiments at conventional facilities SFQED, LUX but also at high intensity laser facilities at even more non-linear conditions at CORELS (Gwangju), ELI pillars, Munich (FOR), maybe Gemini again, will be able to take precision measurements.