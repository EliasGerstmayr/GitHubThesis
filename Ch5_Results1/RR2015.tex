\chapter{First Measurements of Radiation Reaction at Astra Gemini}

In the following section the author describes the experiment setup and some of the key findings of an experimental campaign performed at the Astra Gemini laser system at the Central Laser Facility (CLF) in late 2015. In this experiment a highly relativistic electron bunch, accelerated through laser wakefield acceleration (LWFA), was collided with a secondary scatter beam at high intensities. This consists the first published measurement of radiation reaction in an LWFA setup (REF Cole).
\vspace{\baselineskip}

After a brief introduction for some context, the author will describe the experiment setup, then characterise the electron spectrum and the signal on the gamma-ray detector for the shots without the scattering beam (`beam off'). These act as references for the colliding beam shots. Then these references will be compared to the shots with colliding beam shots and a clear deviation from those and an agreement with radiation reaction models is shown.
\vspace{\baselineskip}

Finally, the author compares these results with another measurement of radiation reaction at higher electron energies performed a the same laser system a few weeks earlier. The results of this campaign have been published in (REF Poder). In particular the author will compare the stability of the electron energy and the stability of the electron spectrum to draw conclusions for future experiments on what parameters and how many data points would be required to make solid statements about different radiation reaction models.

\vspace{\baselineskip}
The work presented within this chapter represent excerpts of analysis work done by the author, mainly related to process the data into a format which facilitates further analysis including the subtraction of background, correcting images for relative tilts in the optical systems used in experiment, scanning through data sets, identifying promising cases and categorising them by assigning characteristic quantities to them.


\section{Motivation}

Colliding-pulse experiments using LWFA have been successfully performed by different groups over the past years, improving the control over the process and the energies involved gradually \cite{TaPhuoc2012a,Chen2013a,Khrennikov2015,Powers2014,Sarri2014}.
An indicator for the progress of these experiments, in terms of laser intensity and electron energy at the point of interaction, is the X-ray spectrum produced through relativistic inverse Compton scattering (ICS) (REF). In the case of ICS, the scattered photons are Doppler-shifted and re-emitted at a higher energy depending on the electron energy and the intensity of the laser field (see Theory REF for further details). Higher electron energies result in a stronger shift of the radiation, leading to a hardening of the spectrum and a corresponding energy loss of the electrons (REF). A more intense laser field enables non-linear interactions and increases the number of photons interacting with the electrons (REF). The energy of the radiation is maximised in a head-on collision due to the Lorenz boost.
\EliasComm{add references but also talk about harmonics...}
\EliasComm{somewhere mention the efforts of the community to collide an accelerator beam and the success of the SLAC experiment. What is the difference here?}
\vspace{\baselineskip}

In earlier experiments, a single laser pulse was used to accelerate electrons to relativistic energies and to scatter the electrons after reflecting the beam from a plasma mirror (e.g. tape \cite{TaPhuoc2012a}). In such a setup the laser beam and the electrons are timed intrinsically. A close to normal overdense plasma surface will then be able to overlap the laser beam and the electron bunch spatially. In \cite{TaPhuoc2012a} electrons of an energy around $100\,\mathrm{MeV}$ were collided at a laser intensity of $a_0 \approx 1.2$. The radiation measured was broadband and reached up to $100\mathrm{s}\,\mathrm{keV}$. A lower energy spread and tunability was demonstrated at slightly lower electron energies and laser intensities in \cite{Powers2014} and\cite{Khrennikov2015}.
\vspace{\baselineskip}

Whilst this technique avoids issues with timing and overlap, the intensity of the laser pulse is limited as it is partly depleted when interacting with the electron bunch after driving a wake. At the same time the acceleration can not be pushed to its limit as the laser pulse has to be remain intense enough for a suitable interaction. Further problems might arise as controlling the wavefront of the depleted laser pulse might be challenging.

Two separate lasers, one to accelerate and one to scatter off the electrons, offer the opportunity to achieve higher intensities and electron energies at the interaction. However, it is challenging to overlap the micron-sized electron bunch with the ultrashort laser pulse, probably focused down to a micron-scale focal spot to achieve high intensities, in time and space, and to keep that alignment.

In \cite{Chen2013a} the relativistic electrons were scattered at an angle of $10$ degrees and gamma rays at an energy of around $1\,\mathrm{MeV}$ were produced.
In \cite{Sarri2014} electrons at an energy of $400 \, \mathrm{MeV}$ were successfully collided at an laser intensity of $a_0 \approx 2$, resulting in broadband radiation extending up to $10\,\mathrm{MeV}$.
The energy loss of the electrons was in this case not significant enough to be able to probe radiation reaction.
\EliasComm{Ref Yan, Nat Photonics on multiphoton Compton.}
\vspace{\baselineskip}

The current work related to (REF Cole) succeeded in colliding electrons of around $550\,\mathrm{MeV}$ at a laser intensity of $a_0 \approx 10$, reaching gamma ray energies in excess of $30\,\mathrm{MeV}$. This was the highest gamma-ray energy from an all optical ICS source published at that point. The work of (Poder REF) collided electrons of around $2\,\mathrm{GeV}$ at similar intensities but lack a spectral measurement of the gamma rays.
By now the highest recorded gamma-rays have been achieved by REF by using back-reflection of the laser pulse again and colliding them with electrons at energies up to $2\,\mathrm{GeV}$.

\EliasComm{Make a plot of electron energy and laser intensity, and place our work on this plot.}
\EliasComm{Reference the papers. See plot and papers from notebook.}
\EliasComm{Reference J Shaw paper on 85 MeV Compton?}
\EliasComm{Consider adding a contour for gamma-ray energies. Approximately (factors of a few missing due to varying laser photon energy).}

\begin{figure}[h]
\centering
\includegraphics[width=0.8\columnwidth]{RR2015_Eta_thiswork.pdf}
\caption{Some plot to place our work into context.}
\end{figure}

\section{Experiment Setup}

The experiment described in the following was performed at the dual $300\,\mathrm{TW}$ Astra Gemini laser system at the Central Laser Facility, Rutherford Appleton Laboratory, UK, in late 2015.
\vspace{\baselineskip}

\begin{figure}[h]
\centering
\includegraphics[width=0.8\columnwidth]{Exp_setup_render_RR1.png}
\caption{Sketch of experiment setup at the Astra Gemini laser facility (based on a drawing by Jason Cole). From left to right: a high intensity laser beam (red) focused with a f40 spherical mirror generates up to $\mathrm{GeV}$-scale electrons in a gas jet (LWFA). A second laser beam (red) is focused down tightly at the edge of the gas jet by an f2 off-axis parabola to scatter the electron beam shortly after it leaves the jet. The electrons (blue) are being dispersed by a dipole magnet and detected on a scintillating lanex screen (grey). Finally, gamma rays (green) propagate through a vacuum kapton-window and a lead aperture onto a stack of scintillating CsI crystals that are imaged by a camera.}
\end{figure}
\EliasComm{add labels to the sketch.}
The first part of the experiment is a LWFA setup. The first beam (referred to as `South' or `Driver' beam) of Gemini is focused down to a spotsize of NUMBER with a 6m focal length f/40 spherical mirror. The beam is focused onto the leading edge of a 15 mm conical supersonic helium gas jet target to generate a beam of relativistic electrons. The energy on target was typically around $(8.6 +- 0.6)\,\mathrm{J}$, which corresponds to $a_0 = 1.9 +- 0.1$. The electron density of the target was around $(3.7 +- 0.4) \times 10^{18} \,\mathrm{cm}^{-3}$. Gas cells provide a more stable environment for LWFA, resulting in more reproducibility and sometimes higher electron energies (see REF Kuschel 2018 for instance). However, in this case a gas jet was chosen over a gas cell as the focus of the second beam was aimed to be as close to the electron source as possible in order to interact with a small spot size, i.e. high number density of electrons. Using a gas cell in this case could lead to damage on components as the tightly focused f/2 beam diverges quickly as well.
\vspace{\baselineskip}

The second beam (referred to as `North beam' or `Scatterer') is focused down tightly onto the opposite edge of the gas jet using an f/2 off-axis parabola. The energy on target was typically $(10 +- 0.6)\,\mathrm{J}$ focused into a spot of FWHM size of $2.4 \times 2.8\,\mathrm{\mu m}$, corresponding to peak $a_0 = 24.7 +- 0.7$. As this beam counter-propagates with the electron bunch, the focusing optic lies in the beam path of the first beam and the electron trajectory. In order to enable their propagation a hole, several $\mathrm{mm}$ in diameter, was added in the centre of the parabola used for the North beam. This leads to a decrease in intensity of the beam (add NUMBER), which, however, is relatively small as the profile of the beam is not Gaussian but a flat top profile.
\vspace{\baselineskip}

\EliasComm{add pulse duration.}

\begin{figure}
\centering
\includegraphics[width=0.8\columnwidth]{quantumness_contour_block8.png}
\caption{`Quantumness parameter' $\eta$ \cite{Blackburn2014} for given laser intensity and electron energy in terms of the relativistic Lorentz factor $\gamma$. The contour lines show which parameters are required to reach a certain value of $\eta$. The green box indicates roughly which regimes are realistically within reach at the Astra Gemini laser.}
\end{figure}


The two laser beams were overlapped in space in time by using spatial interferometry: a knife-edge prism was used to reflect both counterpropagating beams by 90 degrees collinearly onto a CCD camera chip equipped with a x10 microscope objective. The different radii of curvature of the f/40 and f/2 beams resulted in the formation of interference fringes when the laser pulses overlapped in time. Once the fringes appeared, their visibility was optimised. The precision of the pulse overlap is around $+- 30\,\mathrm{fs}$. Unfortunately, the overlapping the beams both in time and space at one point in time does not guarantee that they remain overlapped. Some studies by Oxford REF indicate that drifts exist in the system and the necessary precision of the alignment is only held for about half an hour.
\vspace{\baselineskip}

The main diagnostics utilised in this experiment can be divided into two parts:
\vspace{\baselineskip}

Firstly, the particle diagnostics, in this case an electron energy spectrometer. In this experiment electrons were accelerated to relativistic energies. A permanent dipole magnet of field strength NUMBER dispersed the electrons according to their energy, where higher energetic electrons are deflected less than lower energetic ones (see Theory). A scintillating slanex creen in a certain distance from the magnet and tilted by $45$ degrees to the laser axis was used to detect the dispersed electrons, where again electrons closer to the laser axis are more energetic. The lanex screen was imaged from the top by a camera on the roof of the vacuum chamber.
\vspace{\baselineskip}

Secondly, the radiation or gamma-ray diagnostic: several meters NUMBER from the interaction point on laser axis an array of scintillator crystals (caesium iodide CsI, doped with thallium) was placed. Incoming photons would penetrate the stack whilst depositing energy in the crystals, in a characteristic distribution related to the spectrum. The photons emitted from the scintillator were then detected by another camera. A photograph of the crystal stack is provided in figure NUMBER REF.

\begin{figure}[h]
\centering
\includegraphics[width=0.8\columnwidth]{scintillator.JPG}
\caption{Scintillator array used in experiment. The CsI crystals are encased in an aluminium casing approximately $220\,\mathrm{mm}$ deep and $150\,\mathrm{mm}$ high with crystals each $4.8\,\mathrm{mm} \times 4.8\,\mathrm{mm} \times 50\,\mathrm{mm}$ and an array of $33 \times 47$ in total. The scintillator was placed into the beam axis, the short side facing the beam enabling the radiation to travel as long as possible through the array. The holed side (diameter of holes around $5\,\mathrm{mm}$ each) was imaged using a scientific camera.}
\end{figure}

\EliasComm{regarding the scintillator stack, add some dimensions to the figure.}
\EliasComm{add some comment on timing being off.}

\EliasComm{maybe talk here about some raw data.}

\section{Characterisation of the Electron Spectrum}

In our experiment we measure two components of radiation reaction: one is energy loss of the electrons in the interaction with the laser pulse, the second is the radiation that this energy is being transferred into. In this section the author will characterise the electron spectrum, and its variation from shot-to-shot. A more detailed statistical analysis will follow later.

Since this experiment was not performed at a conventional accelerator facility we have to take the intrinsic variations of the electron source into account. In this section the author will present a statistical analysis of the shots without the scattering beam on, characterising the typical electron spectrum and the extent of its fluctuations. These shots will be from now on referred to as reference data or null shots. This is crucial as the energy loss of the electrons has to be distinguishable from the variations of the source.

\begin{figure}
\centering
\includegraphics[width=0.8\columnwidth]{Elec_EdgeSpectrum_Null_V2.png}
\caption{Electron spectra of the reference shots (driver beam only) in red, normalised to their peak charge. The x-axis shows the electron energy in $\mathrm{MeV}$ and is cut off at $800\,\mathrm{MeV}$ as high energy features will not be further investigated. The blue line is the first derivative of the spectrum with significant maxima marked by the circles. The yellow circle indicates the peak selected to define the position of the edge.}
\label{Results:Figs:Edge:ShotsNull}
\end{figure}

\EliasComm{add some raw data.}

\EliasComm{Make some waterfall plots.}

Spectra in general. Edge feature.
The energy of the electrons is being measured by dispersion through a dipole magnet onto a scintillating lanex screen. The electron spectra are all composed of two very distinct parts: one is a low charge high energy tail at around $1\,\mathrm{GeV}$, the second is a high charge but low energy feature falling off steeply in an edge, sometimes it is more explicitly a quasi-monoenergetic peak. This spectral `edge' feature is very distinct and is found consistently on all shots analysed. Because of this consistence and also because it contains a large part of the charge of the beam this is a useful indicator. Examples of the spectrum can be found in FIGURE NEEDS DOING. 

The position of the cut-off, from here on referred to as the `edge', is determined by the maximum of the first derivative of the electron spectrum (see figure \ref{Results:Figs:Edge:ShotsNull}).


We believe that the high energy tail results from self-injection early in the acceleration, whereas the low energy edge might be due to accidental shock injection caused by damage on the gas nozzle. Some photographs at the interaction seem to show a shock front closer to the centre of the gas jet which would be consistent with the energies achieved. In addition, the sharp falloff of the edge speaks for a rapid change in plasma density resulting in turn in a rapid injection mechanism.

As can be seen from these examples (SHOW FIGURE) it becomes evident that the position of this edge is varying in energy from shot to shot. This might be linked to experimental factors we recorded or is beyond or measurement or intrinsically random. First we will try to find correlations with common factors and then characterise the remaining fluctuations as carefully as possible. 

ADD A PLOT OF THE LASER ENERGY CORRELATION HERE.

\vspace{\baselineskip}
Laser energy and failing correlation. Excluded as parameter to normalise to.
A common cause of varying energy in LWFA is the energy of the laser. Stronger fields can lead to earlier wavebreaking and also extend the efficient drive of the wakefield. As seen in figure NUMBER we do not see any strong correlation between the edge energy and the laser energy. While there appears to be a weak positive correlation between the two, the confidence in this is low (PROVIDE HERE SOME R value). If we consider the larger dataset the result remains fairly similar at a confidence of R=... NUMBER.
As we believe that the injection is probably caused by a shock in the gas it is not expected that the energy of the feature correlates with fluctuations in laser energy as it might do for pure self-injection (see later comparison with Poder and REF). A second factor affecting the injection and acceleration besides the laser would be the gas target itself. Unfortunately, we do not have on-shot measurements of the density and visible confirmation of the shock front to confirm the variation. The fluctuation of the general gas density, however, is relatively small (PUT A NUMBER HERE), so we assume that the main impact of the fluctuation is due to the variation of the shock front. This is a somewhat random effect and as we did not measure it we have to characterise the fluctuations.


MAKE A PLOT HERE OF THE GENERAL HISTOGRAM AT THIS POINT.

EXPLAIN THAT WE NEED ADDITIONAL REFERENCE SHOTS.

\section{Correlating the Gamma-ray signal with the Electron Spectrum}

The second key diagnostic to indicate radiation reaction is the gamma-ray detector.
Just like the electron spectrum might have misleading random noise attached to it that might be mistaken for a signal of radiation reaction or might conceal a signal, the gamma-ray detector also records a signal on the reference shots that has to be characterised.

In addition to radiation from the interaction of electrons and laser photons we expect background noise from the dispersed electrons interacting with high-Z materials in their trajectory, e.g. the vacuum chamber walls, resulting in Bremsstrahlung in a spectral range comparable with the expected inverse Compton signal. The Bremsstrahlung is expected to be produced mainly off-axis and can be efficiently shielded by blocking the direct line of sight with a sufficiently thick wall of lead. This has been confirmed in GEANT simulations (IMAGES?) and in experiment (MAYBE ADD SOME PICTURE HERE OF BEFORE AND AFTER SHIELDING?). The remaining signal is expected to follow a scaling proportional to the energy and charge of the electron bunch. Interactions of electrons that produce generation follow a $\gamma^2$ relation for the radiation emitted. Assuming that the energy of the radiation is proportional to the energy deposited and the scintillating signal (REF) we expect something like this:
\begin{equation}
Background\,Signal = c_{BG} Q \left\langle\gamma^2\right\rangle,
\end{equation}
where $c_{BG}$ will encapsulate all the complicated details like conversion efficiencies of electron energy to photons, photons depositing their energy in the crystals, energy to scintillator photons, viewing angle and so on. $Q$ is the charge of the measured electron spectrum and $\gamma$ is the Lorentz factor of the electrons.
\EliasComm{add explicitly some equations for Bremsstrahlung and ICS or so to confirm the $\gamma^2$ relation.}
\EliasComm{back up the linear energy deposition somewhere, check Jason's paper.}

\begin{figure}
\centering
\includegraphics[width=0.8\columnwidth]{ElecQ2_CsI_null_Correlation.pdf}
\caption{.}
\end{figure}


Looking at figure NUMBER HERE which plots the $Q \gamma^2$ against the total CsI signal detected, we see that they appear to have a linear relationship as expected. Fitting a regression line and the confidence interval this seems to hold well at a value of NUMBER R.
We expect a signal following this trend on every shot without the scattering beam on. On shots with the scattering beam we expect a combined signal. The second beam will introduce, if the collision is successful, a burst of gamma-rays from inverse Compton scattering. The radiation mechanism will similarly as for Bremsstrahlung follow a $\gamma^2$:

\begin{equation}
CsI\,signal = c_{BG} Q \left\langle\gamma^2\right\rangle + c_{ICS} a_0^2 Q  \left\langle\gamma^2\right\rangle,
\end{equation}
where now $c_{ICS}$ encapsulates all the difficult physics of coupling constants, cross sections and so on. In this case $c_{ICS}$ is not a constant as the conditions of the interaction vary strongly due to the changing overlap of the laser pulse and the electron beam. The better the overlap and the higher the intensity at that interaction, the stronger the ICS signal will be and the easier to distinguish from the background Bremsstrahlung it gets.

If we now add the shots to the same plot with the beam on, we see that those shots are more scattered across the plane than the shots without the colliding beam. Some of the data points follow the trend relatively well, some have a much higher CsI signal than expected from an electron spectrum at that charge and energy. These could be potential good overlaps.

To qualify how much above the background these are, we will switch now to looking at the ICS signal mainly by subtracting the expected Bremsstrahlung background from the signal:
\begin{equation}
ICS\,signal = CsI\,signal - c_{BG} Q \left\langle\gamma^2\right\rangle.
\end{equation}

We then calculate for each of the reference data points the relative difference to the calculated regression line and characterise their behaviour statistically. Assuming a random normal distribution we calculate the standard deviation of the system.
In figure NUMBER PLOT FOR SIGNAL ABOVE... we now see the shots lined up with their actual shot number versus the CsI signal above the expected signal from the Bremsstrahlung background. The y-axis now encapsulates the information of the CsI signal and the electron spectrum.
As we can see most of the shots follow the expected trend very well, some are a few sigmas above but 4 shots are clearly more than 5 sigma (ADD WHAT THE CONFIDENCE WOULD BE) above the signal and we will consider this. The jitter of the electron beam and the laser pulse will be characterised later to check if a success quote of 4 out of 10 is matching our system.

\begin{figure}
\centering
\includegraphics[width=0.8\columnwidth]{ElecQ2_CsI_Correlation.pdf}
\caption{The total energy of the electron beam squared (x-axis) vs. the number of CsI counts (y-axis) which is proportional to the energy of the photons. The photon energy radiated by the electrons is proportional to $\gamma^2$ which in turn is proportional to $E^2$. Hence, a linear relationship should be visible. As to be seen in the diagram, the shots with colliding beam off (red) are fitted well with a linear function whilst the data points for the collisions do not seem to fit on the same curve in many cases. This motivates that there seems to be a different process to take place when the colliding laser pulse is turned on. It can also be seen that the CsI counts for collision shots are scattered a lot, which could be related to different overlaps of laser pulse and electron beam resulting in a different range of radiation produced.}
\label{Results:Figs:NullColl:CsIVsE2}
\end{figure}

This method allows us to identify successful collisions independently from the electron signal which is due to its strong fluctuations less convincing by itself.

\begin{figure}
\centering
\includegraphics[width=0.8\columnwidth]{DCsI_sigma.pdf}
\caption{Shots in the order performed in experiment versus the gamma detector signal above the expected background signal according to the $E^2$-scaling described previously. The red dots are representing the reference shots with the driver beam only, whilst the blue dots are the collision shots with driver and scatterer. Especially shots 4,5,6 and 8 stand out remarkably above the background. The last few shots on the other hand seemed to have missed completely which could be related to a drift in alignment over time.}
\label{Results:Figs:NullColl:DeltaCsIVsSigmaE2}
\end{figure}

In addition, this allows us to find more reference shots in a later dataset. We continued the campaign by defocusing the scattering beam and translate the beam relative to the electron beam. The signal on most of these shots is much lower but they were all taken with the scattering beam on. As we know that not all collisions are successful we can now determine which ones were by using the ICS signal above signal as selection criterion for additional references to analyse the electron spectrum statistically.

\section{Spectral Analysis of the Gamma-ray signal}

Geant simulations and so forth. Different algorithms maybe?

The spectral analysis of the gamma-ray signal was first performed by Jason Cole (Imperial College) and Keegan Behm (Michigan University).

In this process the scintillator response was modelled using the Monte-Carlo codes Geant4 \cite{Agostinelli2003} and MCNP \cite{Goorley2012} to retrieve the energy spectrum of the gamma radiation. The on-shot synchrotron-like spectrum and the shift of the cut-off edge in the electron spectrum were then combined to constrain the parameter space (normalised laser vector potential $a_0$ and electron energy) at the interaction, relying on a range of models.

A paper on the design of the detector and the analysis of the spectrum has been submitted for publication (REF Keegan paper).

Also present that on beam on and off the spectrum changes significantly and is hence beyond the level of absolute numbers very sensitive to the interaction.

\section{Statistical Characterisation of the Electron Spectrum}

Having found the selection criterion for successful collisions we can use the same method to determine which shots with both beams were complete misses and can hence be treated as reference data as well to build up a more solid statistical characterisation of the shot-to-shot fluctuations.

The spectra that are being considered for this analysis are two sets of data that were taken at identical experiment conditions as the data shots that will be discussed later. The first part are around INSERT NUMBER shots that were taken within a few minutes of the data shots and were introduced earlier, the second part of INSERT NUMBER `shots' were taken within an hour of the collision data set of interest. In total INSERT NUMBER of reference shots were included.
CHANGE THIS SECTION LATER.
\vspace{\baselineskip}

The selection of the additional shots.
ADD A PLOT WITH THE DATA OF ALL?
The additional reference shots are based on their CsI signal output. If the CsI signal matches the expected Bremsstrahlung background level within 5 sigma, we deem this to be sufficient to say that the interaction did not take place or so weakly that it will not result in any visible effect onto the electron spectrum. 5 sigma corresponds to more than 99 percent confidence that this is the case and matches the references. CHECK THE NUMBERS. This criterion gives us another 76 CHECK NUMBERS of shots from the data set taken within an hour of the original data set we were considering.

\begin{figure}
\centering
\includegraphics[width=0.8\columnwidth]{ElecEdge_raster_null_predrift_histo.pdf}
\caption{.}
\end{figure}


\EliasComm{Make some waterfall plots.}


When plotting the energy of all shots in a histogram we become aware that these seem to have a higher energy on average than the ones we considered earlier. The distribution is relatively wide but peaks at higher energies. If we assume that our hypothesis is correct about the origin of the spectral feature this might be related to a progressing degradation of the nozzle.
\vspace{\baselineskip}

Drift in energy over time. Characterisation how fast the drift is. Additional 76 reference shots.
When looking at figure NUMBER TO TIME DRIFT it appears that the energy of the spectral feature fluctuates around a slowly over time increasing mean. The correlation of the electron mean energy with the recorded time of the shots is relatively strong with SOME NUMBERS HERE. This is another argument for the hypothesis that a damage to the nozzle caused this injection is a slow drift in the energy of the spectral feature over time. A quick look at the laser energies also reveals that they did not increase over that time.
 
\begin{figure}
\centering
\includegraphics[width=0.8\columnwidth]{ElecEdge_Raster_Drift.pdf}
\caption{.}
\end{figure}
According to this the mean energy increases at a rate of around MEV PER TIME. The first set of data taken around the collision shots was taken in the course of around 20 NUMBER minutes which means the effect of the slow drift is negligible in contrast to the shot-to-shot fluctuations.
The drift could be related to further degradation of the nozzle and a moving of the shock front towards the front of the nozzle. The spectral feature increases in energy while the self-injection high energy tail remains somewhat constant. Towards the end of the measurements we also observe how the high energy tail disappears as the spectral feature is very close. This could indicate that the shock injection suppresses the self-injection mechanism at that point.

\begin{figure}
\centering
\includegraphics[width=0.8\columnwidth]{ElecEdge_DE_drift_histo.pdf}
\caption{Drift corrected deviation from mean energy.}
\end{figure}
\EliasComm{add some numbers on sigma and width.}

\vspace{\baselineskip}
Variation of edge feature after considering drift.
Even though the slow drift is not of interest for the collision dataset itself we can use it to make the larger dataset in itself more comparable and take a closer look at the fluctuations around the slowly varying mean. We take the linear interpolation of the mean as reference point and calculate the deviation from this mean $\Delta E$ from the expected mean at that point in time. We then normalise the deviation by the expected mean energy which gives us a relative deviation rather than absolute deviation in $\mathrm{MeV}$. SOMETHIGN ABOUT FAVOURING HIGH ENERGY FLUCTUATIONS TO BE MORE PROMINENT. The results from the complete dataset can be seen in FIGURE NUMBER HISTO DE/E. From a first glance the distribution looks close to normal. A normal kernel density estimate also seems to fit well. SOME WAY TO QUANTIFY THIS?
Histograms for variation over time: normal distribution KDE. Consider different shape, what is the effect?
Lorentz function instead of Gaussian?

Some numbers: what is the variation, sigma, mean and so on.
\EliasComm{Include all null shots, not just raster in the analysis!}
\vspace{\baselineskip}

TRY TO CORRELATE AGAIN FOR LASER ENERGY. FAILING.
Now that we corrected the slow drift we could have another try at verifying that there is no strong correlation to the laser energy. 
Looking at figure NUMBER here we see that this extended set of data does again not correlate very strongly with the laser energy on-shot, before and after correcting for the drift in the mean energy of the spectral feature.

\begin{figure}
\centering
\includegraphics[width=0.8\columnwidth]{ElecEdge_Raster_Slices_Drift.pdf} 
\caption{.}
\end{figure}

\EliasComm{Explain that correlation was similarly strong but one was time-varying but the laser energy was relatively constant over the duration of the experiment. Maybe also show a plot laser energy versus time. Explain how we then assumed that we are dealing with a slow drift of a normal distribution. We calculated the mean of the time slices and found a good correlation around 0.8. Explain also plausibly why not then on top of this treatment use the slicing for the laser energy correlation.}


\begin{figure}
\centering
\includegraphics[width=0.4\columnwidth]{ElecEdge_raster_Laser_Corr.pdf} \includegraphics[width=0.4\columnwidth]{ElecEdge_raster_drift_Laser_Corr.pdf}
\caption{Left raw elec edge enery, right the drift corrected relative energy variation around the mean. No apparent strong correlation, strongly scattered.}
\end{figure}

\vspace{\baselineskip}
Decide not to use drift in itself for the analysis as the relevant shots were taken within the interval of around 20 minutes only.
For this part of the distribution the histogram looks as follows ADD HISTOGRAM. The mean energy is NUMBER MEV.
Since we have shown that the fluctuations in the electron energy are sufficiently described by a normal distribution we can easily put a number on how probable it is for spectra to be of a certain energy, and to make sure our statements are statistically significant.

\section{Analysis of the spatio-temporal jitter?}

To verify whether our success quota for the collisions matches up, we will have a brief look at characterising the jitter of both laser beams or the electron beam relative to the laser beam.

Micron sized targets.

Something about F/2?
Something about F/40?
Do we assume the electron beam to follow the laser pulse?

electron pointing jitter rms = 3 mrad (in x and y)
electron beam divergence = 3 mrad 
electron waist =$ 1\, \mu m$
distance from waist to interaction = 1.5 mm
hit defined as overlap of laser and electron $>= 50\%$ of maximum overlap possible.

I did 1000 runs of 15 shots and calculated the overlap between the laser and electron beam, defining a hit as when the overlap $>= 0.5$ perfect alignment.  

The expected number of hits is $4 - 5$ (we saw 4 in the experiment)




I calculate the beam size assuming the ebeam behaves like a gaussian beam
so 
$w_{ebeam} = ebeam_waist * \sqrt( 1 + (z_{ebeam}/(zR_{ebeam}) )^2)$
and 
$zR_ebeam = ebeam_waist/divergence_ebeam$

\section{Indicators of radiation reaction in the electron spectrum}

\begin{figure}
\centering
\includegraphics[height=0.25\columnwidth]{20151217r002_NullGroup.jpg}
\includegraphics[height=0.25\columnwidth]{20151217r002_CollGroup.jpg}
\includegraphics[height=0.25\columnwidth]{up_arrows.jpg}
\caption{Images of the gamma ray spectrometer (upper row) and electron spectrometer (bottom row) at shots with the colliding laser beam off (left) and on (right), using a constant colour scale. The shots with the colliding beam show bright signals on the scintillator array whilst the electron energies remain fairly similar throughout.}
\label{Results:Figs:NullColl:Montage}
\end{figure}


Figure \ref{Results:Figs:NullColl:Montage} shows the visual data taken of the gamma ray detector (upper row) and the electron spectrometer (bottom row) with scatterer and driver (right) or driver only (left). A montage like figure \ref{Results:Figs:NullColl:Montage} can be a helpful tool to spot stark differences and features in a first step.


In the previous section we found a selection criterion to determine which shots were successful. The brightest CsI signals are found in shots NUMBERS. These four shots are selected accordingly and match roughly the statistics from the section on the relative jitter of the electron beam and the colliding laser pulse.

\begin{figure}
\centering
\includegraphics[width=0.8\columnwidth]{DCsI_sigma_v_ElecEdge.pdf}
\caption{Position of the low-energy edges (x-axis) versus the gamma ray signal above the expected radiation background (y-axis) in units of standard deviation from the reference mean. Reference shots are shown in red and collision shots in blue. The edge positions are very scattered under both conditions with the majority of the collision shots exhibiting higher edges. The brightest gamma ray signals on collision seem to coincide with a low position of the edge.}
\label{Results:Figs:Edge:PosVsCsI}
\end{figure}

\EliasComm{Make some waterfall plots.}

From figure NUMBER HERE. We see that the electron spectral feature varies quite a bit over the course of the shots but that all shots selected using our criterion have low electron energies. The probability for this, assuming a normal distribution as derived in the previous section, is around NUMBERS HERE for one shot but NUMBER TO THE FOUR for all four of them to occur. If we are sufficiently convinced that these low electron energies are equivalent to energy loss from our mean, we can estimate the conditions at the interaction. For an electron energy of CHECK NUMBER $550\,\mathrm{MeV}$ we require a laser intensity of $a_0 \approx 10$. This is much lower than the peak intensity of our laser pulse.

A look at the spectral information of the gamma-ray detector confirms these parameters. The spectrum recorded roughly matches this. We can from both sides narrow down the exact conditions at the interaction.
\vspace{\baselineskip}

The lower intensity is actually easily explained. The timing of the beams was done using spatial interferometry. The visibility of fringes was optimised which means that both laser pulses are timed to arrive within $30\,\mathrm{fs}$ of each other. This timing is true for both laser pulses travelling through vacuum. What we are actually aiming to time is the arrival of the scattering beam and the electron beam which is trailing behind the driving laser pulse. To estimate the intensity of the laser pulse at the interaction we have now to consider the delay of the laser pulse as it is travelling through a medium (plasma) to the point of injection and then, assuming that the electrons relatively quickly reaching a velocity close to the speed of light, the relative delay of the laser pulse and the electron beam. This delay is around half a plasma wavelength SEE REFS.

For the delay induced through the South beam travelling through a medium we use the group velocity $v_f \approx 1-\frac{3}{2}\frac{n_e}{n_c}$, which is the standard nonlinear group velocity in media minus the etching velocity (called front velocity in REF DEcker 1996 Phys Plasmas). The distance that the North beam travels in vacuum during the time the South beam travels through a medium of thickness $d$ is $d'=d c/v_f$. The new collision point is now

\begin{equation}
\delta z_1 = 0.5 (d c/v_f - d) = d/2 (1/(1-3/2 n_e/n_c) -1) \approx (1+3/2 n_e/n_c -1) = \frac{3d}{4}\frac{n_e}{n_c}
\end{equation} 

The electrons trail behind the South laser pulse around N plasma wavelengths (the factor half are as the collision happens midway):
\begin{equation}
\delta z_2 = 0.5 N \lambda_p = 0.5 N \frac{2 \pi c}{\omega_p} = 0.5 N \frac{2 \pi c}{\omega} \frac{\omega}{\omega_p} = 0.5 \lambda_0 \sqrt{\frac{n_c}{n_e}}
\end{equation}

As total delay we can estimate
ADD EQUATION HERE FOR DELAY AND A DRAWING TO SHOW THIS MORE CAREFULLY.

\begin{equation}
\delta z = \frac{3d}{4} \frac{n_e}{n_c} + N \frac{\lambda_0}{2}\sqrt{\frac{n_c}{n_e}},
\end{equation}
where d is the distance of injection point from front of gas jet
N is number of plasma periods between f/40 and electron bunch.

\EliasComm{Somewhere write about the delay of electrons to laser pulse (timing) and how to estimate it.}
\EliasComm{Add more details to explain this.}

This amount of defocus then reaches an average intensity of NUMBER 10 instead of the peak intensity of close to 25 possible in this setup. A variation of overlap and the electron energy will then result in an accidental scan of our conditions.

Both measurements (gamma-rays and electron energy loss) seem to be indicating the conditions.
If we plot the electron spectrum (assuming post-interaction) against the critical energy of the gamma-ray spectrum, we see that they correlate negatively. This again supports the hypothesis that we are looking at ICS with radiation reaction. In a different mechanism after the measurement, like Bremsstrahlung, we would have expected a positive correlation (a harder spectrum for higher energies). The fact that we see a negative correlation confirms our suspicion that the energy loss is converted into a gamma-ray signal.

Now here some part about the models and how they relate, how the plot is made but this is all Jason's and Tom Blackburn's work and should be referenced as such.

FIGURE FROM THE PAPER, THEORY BLOBS.

Some comments on the actual models?
\vspace{\baselineskip}

Some comments on the theory contours.
The data shows that the results match a model that requires radiation reaction. No radiation reaction overestimates the energy loss and the energy of the radiated gamma radiation. Our results are in agreement with all the models in question, more overlap with the models including quantum corrections which could be a semi-classical model with classical trajectories and merely reduced emitted power or a fully stochastic quantum model. This is all on a one sigma confidence level so a strong distinction is not possible, we require more data.
An analysis how much more data we would require to make a conclusion beyond the 2 sigma level will follow in the next section.
\vspace{\baselineskip}

In addition it becomes clear that the difference between a fully stochastic quantum model and a semi-classical model is not very visible in this case. There is in indication that the models depart at higher energy losses, i.e. higher laser intensities but it would also require a higher electron energy. This slow departure from each other in contrast to the relatively distinct classical model is founded in the electron observable which is the energy loss from a characterised mean. When looking at the mean energy loss for the quantum and the semi-classical model it becomes evident that by definition the values are identical. Even though the energy loss of the spectral feature is not the same thing it closely relates to the mean energy loss. A more sensitive observable that behaves very differently for both models is the variance of the spectrum or the shape of the spectrum. This will be closer illuminated in the next section. 
\EliasComm{add some references here to papers regarding radiation reaction. How much to add in theory and how much to add in experiment section?}

\EliasComm{Also remember that eta is not a constant in the interaction}

\section{Comparing Results Poder and Cole}

The results from Cole show that energy stability and reproducibility of the spectrum is crucial to make solid and significant statements about radiation reaction. A strongly varying electron source requires stringent statistical analysis and sufficient data to increase confidence and to back up claims. A more stable source in terms of energy will allow us to further distinguish a fully classical model from a model using any kind of quantum corrections. As mentioned earlier, however, the energy loss will solely not be enough to give us a strong distinction between quantum models. The spectral information of the gamma-ray signal will be a crucial information but the shape of the electron spectrum will be important too, given the source is stable enough in its shape from shot to shot to allow for comparison.
Ideally, we would measure the electron spectrum before and after interaction. A technique to measure both on-shot has been proposed in BAIRD PAPER REF but would be challenging as only a small part of the beam would interact and disappear under the rest of the spectrum.
\vspace{\baselineskip}

The second published result on radiation reaction (see PODER REF) takes the shape of the spectrum into account.
Setup, experiment conditions.

The data presented previously and published in COLE REF will be processed again as described previously: the selection of 80 NUMBER CHECK reference shots is being treated for the slow drift in the mean energy of the spectral feature. The variable then calculated is then the relative deviation from that slow varying drift. A correlation with the laser energy was not found which also matches our hypothesis that we are looking at shock injection in a gas jet target (helium). The distinct edge feature is set around $550$ to $600\,\mathrm{MeV}$ CHECK NUMBER. An example average spectrum of all references (corrected for their energy and normalised to their total charge) is found in FIGURE REF.
\vspace{\baselineskip}

The electron spectra used as references in the PODER REF are generated purely by self-injection REF HERE in helium gas cell target. Gas cells are known to be more stable REF HERE in their shot-to-shot variation. The shape of the spectrum is more stretched and with less distinct features, rolling off slowly in exceed of $2\,\mathrm{GeV}$. The measured quantity here will be the cutoff energy of the spectrum which will be defined here as 95 percent of the total charge recorded CHECK NUMBERS as the spectrum has no distinct features as the edge in COLE DATA. Since we are looking at a self-injected electron beam the energy of the electron beam is expected to scale to the laser energy. As also shown in PODER REF this is true. A plot of the cutoff energy against the laser energy of the driver beam can be seen.
The reference spectra used will be scaled according to their laser energy and a representative mean spectrum (normalised by charge) can be seen in FIGURE REF. The correlation coefficient for laser energy versus cutoff energy is around $0.9$ which indicates a very strong indication for a linear relationship between the two variables. MAYBE SOME ADDITIONAL NUMBERS HERE WHAT THIS MEANS?
\vspace{\baselineskip}

\begin{figure}
\centering
\includegraphics[width=0.8\columnwidth]{Poder_Laser_Corr.pdf}
\caption{.}
\end{figure}

One claim is that gas cells are more stable, which in our case would also mean potentially more suitable for these experiments.
First we will compare the shot-to-shot fluctuation of both sources regarding their stability in energy.
In FIGURE REF NUMBER histograms of the relative deviation of the energy (edge and cutoff) are shown. The blue histogram and the respective KDE has been shown before in the section and is the distribution of the COLE REF electron spectra. The green histogram and the KDE relate to the data taken on the PODER campaign. Both distributions are normalised and not representative of the actual number of shots taken. One observation is that both distributions are well described by a normal distributions. The confidence for normal distribution KDE is NUMEBR HERE CHECK. The second part is that the distribution from the Poder campaign is narrower. The width of the Cole data is NUMBER INSERT HERE with a standard deviation of NUMBER INSERT HERE, whereas the Poder data has a widht of NUMEBR HERE.
This means the energy of the Poder data using the cell is more stable by a factor of NUMBER HERE. INSERT SOME EXAMPLES WHAT THAT MEANS IN STATISTICAL TERMS. WHAT IS THE CONFIDENCE INTERVAL FOR VARIATION IN WHAT PERCENTAGE?
THE standard deviation for COle is 0.077, whereas the Poder std is 0.035, so half of that. This means that assuming normal distribution in the case of Cole data $68$ percent of the shots will be within $+-$ 8 percent or for 95 percent $+-$ 16 percent. For the Poder data the interval will be less than half of that size, i.e. $+-$ 4 and $+-$ 7 percent, respectively. For the full width half max of the distribution this means using $FWHM = 2.355\sigma$, that $FWHM_{Cole}=0.18$ and $FWHM_{Poder}=0.08$, so 18 and 8 percent relative energy shift respectively.
\vspace{\baselineskip}

The second part is looking at the shape of the spectra. While the COLE PAPER has an electron spectrum with a very distinct feature which is desirable, the analysis in the previous parts has shown that the shape of this changes significantly from shot to shot.
One way to quantify this variation is by looking at the energy dependent variance of the spectra. These are shown in terms of the error bars in the exemplary average spectra SEE FIGURE and in FIGURES WITH LIN AND LOG.
The variance is significantly higher for the COLE DATA by a factor of NUMBER HERE. EXPLAIN WHAT THIS MEANS. CORRECT A BIT AS PODER WAS NOT MEV CORRECTED.

Electron spectra: energy, stability in energy, stability in shape, ideal shape?

\begin{figure}[h]
\centering
\includegraphics[width=0.4\columnwidth]{Comparison_ESpectra.pdf}\includegraphics[width=0.4\columnwidth]{Comparison_Histo.pdf}
\caption{A comparison of the spectra.}
\end{figure}

\begin{figure}[h]
\centering
\includegraphics[width=0.4\columnwidth]{Comparison_Variance_Lin.pdf}\includegraphics[width=0.4\columnwidth]{Comparison_Variance_Log.pdf}
\caption{A comparison of the variances.}
\end{figure}

\EliasComm{add some comments on the spectrum in general. Number of shots. Setup.}
\EliasComm{have to correct Kris spectra by MEV term, as the units were different.}
\EliasComm{add something about chirp?}

\section{Statistics for future experiments}

Maybe a section based on Chris Arran's work regarding how many shots we would require with a certain spectrum.

Add a table on how many shots are required at what width of the distribution and what electron and laser parameters. Or make a nice 2D plot.

As it became evident over the course of these past sections, the quality of our electron source and the lasers in use requires us to think about the statistics to make sure our statements to distinguish models are sufficiently backed up. 

In general we want higher electron energies and higher laser intensities to reach regimes where quantum and classical models clearly depart from eachother. The energy at these conditions should be fairly stable to allow a clear distinction. In addition we want a narrow energy spread and a stable spectrum also in terms of shape to be able to distinguish a fully stochastic quantum model and a semi-classical model. As we are not using a machine but an intrinsically delicate and highly non-linear acting laser-plasma accelerator, sometimes not all of these parameters can be at their optimum. Driving a wakefield very hard at maximum energy might give us higher energies but a pronounced feature as in COLE might disappear and the reproducibility might decrease. At lower energies narrow spread beams might be easier to reporduce and to control but the models will be closer to each other. In addition, there are potentially other effects such as chirp in the electron beam or a rapid decrease of $\eta$ or a variation of $\eta$ in a long scatterer pulse, that one has to consider. In a realistic setting one has to compromise and the extra effort to optimise one parameter might not necessarily increase the confidence much more. A simple analysis of the effect of the different variables should act as a handy reference for the experimentalist.

The variables will be: electron energy, electron energy spread, electron energy stability, electron shape stability, scattering laser intensity.
The hopeful result will be: number of shots required to distinguish a full quantum model from a semi-classical model or a classical model within 1, 2, 3, 4 and 5 sigma confidence.

\section{Conclusion}

The expertise gained in these previous experiments paired with the technological capacities of the Astra Gemini Laser make it a realistic ambition to collide electrons at even higher relativistic energies ($> 2 \mathrm{GeV}$) and more intense laser fields ($a_0 \approx 25$), expecting a gamma-ray spectrum with a critical energy close to $100\,\mathrm{MeV}$. These would enable probing radiation reaction as the energy loss due to the emission of radiation becomes significant in relation to the initial electron energy -- possibly even reaching a regime where QED effects become relevant as the `quantumness' parameter $\eta$, indicating how much the interaction is dominated by QED effects \cite{Blackburn2014} (maybe another REF), approaches values close to $1$.
