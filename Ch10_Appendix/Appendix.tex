\chapter{Appendix}

\section{Notation and Glossary}

\section{Nabla Operator and Vector Identities}

\section{Maxwell's equations}

\section{Relativistic Transformations}

Boosts

\section{Radiation Integrals}

\section{Feynman diagrams and convention}

Different particles and types

Time direction in this thesis from left to right

\subsection{Mandelstam variables}

\begin{align}
s & = (p + p')^2 &= (k + k')^2; \nonumber\\
t &= (k-p)^2 &= (k' - p')^2; \nonumber\\
u &= (k'-p)^2 &= (k-p')^2.
\end{align}

\subsection{Trace Identities}


\subsection{Calculating Matrix Elements from Feynman diagrams}

\section{Derivation Klein-Nishina equation}


\begin{figure}[h]
\centering
\includegraphics[width=0.4\columnwidth]{Compton_Feyn1.pdf}\includegraphics[width=0.4\columnwidth]{Compton_Feyn2.pdf}
\caption{Feynman diagrams for Compton scattering. Time axis from left to right. s- and u-channel.}
\end{figure}

See the Feynman diagrams for this.

This derivation is based on Peskin and Schroeder.

Based on the Feynman diagrams find the following expression for the matrix element:

\begin{align}
iM &= \bar{u}(p')(-ie\gamma^\mu)\epsilon_\mu^*(k') \frac{i(\cancel{p} + \cancel{k} + m)}{(p+k)^2 - m^2}(-ie\gamma^\nu)\epsilon_\nu(k)u(p)\nonumber\\
&+ \bar{u}(p')(-ie\gamma^\nu)\epsilon_\nu(k) \frac{i(\cancel{p} - \cancel{k}' + m)}{(p - k')^2 - m^2}(-ie\gamma^\mu) \epsilon_\mu^*(k')u(p)\\
&= -ie^2 \epsilon_\mu^*(k')\epsilon_\nu(k) \bar{u}(p') \left[\frac{\gamma^\mu (\cancel{p} + \cancel{k} +m)\gamma^\nu}{(p+k)^2 - m^2} + \frac{\gamma^\nu(\cancel{p}-\cancel{k'}+m)\gamma^\mu}{(p-k')^2-m^2}\right] u(p).
\end{align}

\begin{equation}
iM = -ie^2\epsilon_\mu^*(k')\epsilon_\nu(k)\bar{u}(p')\left[\frac{\gamma^\mu \cancel{k} \gamma^\nu + 2 \gamma^\mu p^\nu}{2p\cdot k} + \frac{-\gamma_\nu \cancel{k'} \gamma^\mu + 2 \gamma^\nu p^\mu}{-2p\cdot k'}\right]u(p).
\end{equation}


\begin{equation}
\Sigma \epsilon_\mu^* \epsilon_\nu \longrightarrow - g_{\mu \nu}.
\end{equation}


Average squared amplitude over initial electron and photon polarisations. This is the spin-averaged Klein-Nishina equation.

\begin{equation}
\frac{1}{4} \Sigma_{spins} |M|^2 = 2 e^4 \left[\frac{p \cdot k'}{p \cdot k} + \frac{p \cdot k}{p \cdot k'} + 2 m^2 \left(\frac{1}{p \cdot k}-\frac{1}{p \cdot k'}\right) + m^4 \left( \frac{1}{p \cdot k} - \frac{1}{p \cdot k'}\right)^2   \right].
\end{equation}

Replacing $p \cdot k = m \omega$ and $p \cdot k' = m\omega'$

\begin{equation}
\frac{d \sigma}{d \cos \theta} = \frac{\pi \alpha^2}{m^2} \left(\frac{\omega'}{\omega}\right)^2 \left[ \frac{\omega'}{\omega} + \frac{\omega}{\omega'} - \sin^2 \theta \right].
\end{equation}

\section{Scintillators}

Exponential decay. Fast (prompt) and slow response. Dual exponential.

scintillation wavelength
light yield
stopping power
resolution
decay time
material

Calibration using a radioactive source


\subsection{CsI(Tl)}
Caesium-iodide doped with thallium

Hallide crystal. Inorganic alkali halide.

CsI (NIST webbook and STAR):

Density 4.51 g/cc
Atomic fraction 0.488451 for 53 and 0.511549 for 55 (fraction by weight).

One of the brightest scintillators.

Slow scintillaotr, with 0.6 microsecs to 3.5 microsecon components.

This is based on St Gobain (deviations from manufacturers possible).

Wavelength of maximum emission around 550 nm (suitable for photodiode detection) and cutoff around 320 nm.
Primary decay time around 1000 ns.
Light yield around 54 photons/keV gamma.

Best light output at room temperature.

\subsection{LYSO}
LYSO ($Lu_{1.8}Y_{.2} SiO_5:Ce$)

Oxide crystal

St Gobain:
Density 7.1 g/cc
Wavelength emission maximum at 420 nm
Decay time around 45 ns (faster than CsI)
Light yield around 27.6 photons/keV.

\subsection{LANEX}
Commercially available scintillator, frequently used in dentistry.

