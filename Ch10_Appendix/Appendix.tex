\chapter{Appendix}

\section{Notation and Glossary}

Vectors will be noted in bold letters. 

In equations using variable names $\times$ will be the cross product and $\cdot$ will notate the dot product. 
In occasions using specific numbers as in $3 \times 10^{18}$ the $\times$ will just be a normal multiplication.

\section{Nabla Operator and Vector Identities}

In Cartesian coordinates
\begin{equation}
\nabla = (\partial x, \partial y, \partial z)^t
\end{equation}

In cylindrical coordinates

In spherical coordinates


\section{Maxwell's equations}

In differential form

\begin{align}
\nabla \cdot \mathbf{E} &= \frac{\rho}{\epsilon_0}\\
\nabla \cdot \mathbf{B} &= 0\\
\nabla \times \mathbf{E} &= - \frac{\partial \mathbf{B}}{\partial t}\\
\nabla \times \mathbf{B} &= \mu_0 \mathbf{J} + \mu_0 \epsilon_0 \frac{\partial \mathbf{E}}{\partial t}
\end{align}

In integral form

Combining these:

Wave equations.

\section{Relativistic Transformations}

Boosts

\section{Radiation Integrals}

Lienard Wiechart potentials and fields.

Jackson:
\begin{equation}
\mathbf{E} (\mathbf{r}, t) = \frac{e}{4 \pi \epsilon_0} \left[\frac{\mathbf{n}-\mathbf{\beta}}{\gamma^2 R^2 (1-\mathbf{\beta} \cdot \mathbf{n})^3}+\frac{\mathbf{n}\times\left[(\mathbf{n}-\mathbf{\beta})\times\dot{\mathbf{\beta}}\right]}{cR(1-\mathbf{\beta}\cdot\mathbf{n})^3}\right]_{retarded}
\end{equation}

\begin{equation}
\mathbf{B}(\mathbf{r},t) = \frac{\mathbf{n}\times\mathbf{E}}{c}
\end{equation}

Poynting vector
\begin{equation}
\mathbf{S} = \mu_0^{-1} \mathbf{E}\times\mathbf{B} = (c \mu_0)^{-1} \mathbf{E} \times \mathbf{n} \times \mathbf{E}
\end{equation}

\begin{equation}
[\mathbf{S} \cdot \mathbf{n}]_{retarded} = \frac{e^2}{16 \pi c \epsilon_0} \left[ \frac{1}{R^2}|\frac{\mathbf{n} \times [(\mathbf{n}-\mathbf{\beta})\times\dot{\mathbf{\beta}}]}{(1-\mathbf{\beta}\cdot \mathbf{n})^3}|^2\right]_{retarded}
\end{equation}

\begin{equation}
\frac{dP(t)}{d\Omega} = R^2 [\mathbf{S}\cdot\mathbf{n}]_{retarded}
\end{equation}

Total power radiated:
\begin{equation}
P = \frac{e^2 \gamma^6}{6 \pi c \epsilon_0} \left( |\mathbf{\dot{\beta}}|^2 - |\mathbf{\beta} \times \dot{\mathbf{\beta}}|^2\right)
\end{equation}

%\begin{equation}
%\frac{d^2 I}{d\omega d\Omega} = \frac{e^2 \omega^2}{16 \pi^3 c \epsilon_0} |\int_{-\inf}^\inf \mathbf{n} \times (\mathbf{n} \times \mathbf{\beta}) e^{i\omega (t'- \mathbf{n}\cdot\mathbf{r}(t')/c)} dt|^2
%\end{equation}

\section{Feynman diagrams and convention}

Different particles and types

Time direction in this thesis from left to right

Rules for QED

Vertex

Fermion propagator

Photon propagator

External field line photon

External field line fermion

Momentum conservation

\subsection{Mandelstam variables}

\begin{align}
s & = (p + p')^2 &= (k + k')^2; \nonumber\\
t &= (k-p)^2 &= (k' - p')^2; \nonumber\\
u &= (k'-p)^2 &= (k-p')^2.
\end{align}

\subsection{Trace Identities}


\subsection{Calculating Matrix Elements from Feynman diagrams}

\section{Derivation Klein-Nishina equation}


\begin{figure}[h]
\centering
\includegraphics[width=0.4\columnwidth]{Compton_Feyn1.pdf}\includegraphics[width=0.4\columnwidth]{Compton_Feyn2.pdf}
\caption{Feynman diagrams for Compton scattering. Time axis from left to right. s- and u-channel.}
\end{figure}

See the Feynman diagrams for this.

This derivation is based on Peskin and Schroeder.

Based on the Feynman diagrams find the following expression for the matrix element:

\begin{align}
iM &= \bar{u}(p')(-ie\gamma^\mu)\epsilon_\mu^*(k') \frac{i(\cancel{p} + \cancel{k} + m)}{(p+k)^2 - m^2}(-ie\gamma^\nu)\epsilon_\nu(k)u(p)\nonumber\\
&+ \bar{u}(p')(-ie\gamma^\nu)\epsilon_\nu(k) \frac{i(\cancel{p} - \cancel{k}' + m)}{(p - k')^2 - m^2}(-ie\gamma^\mu) \epsilon_\mu^*(k')u(p)\\
&= -ie^2 \epsilon_\mu^*(k')\epsilon_\nu(k) \bar{u}(p') \left[\frac{\gamma^\mu (\cancel{p} + \cancel{k} +m)\gamma^\nu}{(p+k)^2 - m^2} + \frac{\gamma^\nu(\cancel{p}-\cancel{k'}+m)\gamma^\mu}{(p-k')^2-m^2}\right] u(p).
\end{align}

\begin{equation}
iM = -ie^2\epsilon_\mu^*(k')\epsilon_\nu(k)\bar{u}(p')\left[\frac{\gamma^\mu \cancel{k} \gamma^\nu + 2 \gamma^\mu p^\nu}{2p\cdot k} + \frac{-\gamma_\nu \cancel{k'} \gamma^\mu + 2 \gamma^\nu p^\mu}{-2p\cdot k'}\right]u(p).
\end{equation}


\begin{equation}
\Sigma \epsilon_\mu^* \epsilon_\nu \longrightarrow - g_{\mu \nu}.
\end{equation}


Average squared amplitude over initial electron and photon polarisations. This is the spin-averaged Klein-Nishina equation.

\begin{equation}
\frac{1}{4} \Sigma_{spins} |M|^2 = 2 e^4 \left[\frac{p \cdot k'}{p \cdot k} + \frac{p \cdot k}{p \cdot k'} + 2 m^2 \left(\frac{1}{p \cdot k}-\frac{1}{p \cdot k'}\right) + m^4 \left( \frac{1}{p \cdot k} - \frac{1}{p \cdot k'}\right)^2   \right].
\end{equation}

Replacing $p \cdot k = m \omega$ and $p \cdot k' = m\omega'$

\begin{equation}
\frac{d \sigma}{d \cos \theta} = \frac{\pi \alpha^2}{m^2} \left(\frac{\omega'}{\omega}\right)^2 \left[ \frac{\omega'}{\omega} + \frac{\omega}{\omega'} - \sin^2 \theta \right].
\end{equation}

\section{Derivation linear Breit-Wheeler cross section}

\section{Bessel Functions}

Functions solving the differential equations ....

The Bessel function of first order transforms:
...

The generalized Bessel function
\begin{equation}
J_n(x,y) = \Sigma 
\end{equation}

Recursion

First order for radiation integrals (see Synchrotron/Betatron and in generalized form ICS)


\section{Scintillators}

A convenient indirect detection method is using scintillators of various kinds. For electron beams from LWFA typically LANEX scintillators are being used. This is due to their thin substrate layer which gives a decent spatial resolution (about 100 microns for standard LANEX). Thicker versions of this scintillator, e.g. KODAK Biomax, give a higher light yield and hence signal sensitivity without losing much spatial resolution. Lanex is at the same time light-tight and hence more favourable to transparent scintillators with comparable spatial resolution. At the same time Lanex can be put on a plastic substrate making it flexible, available in large sizes and harder to break.

The signal from a relativistic ~100 pC electron beam, however, tends to be much brighter than the X-rays and gamma rays as they do not deposit as much energy and the flux is lower. More direct measurements based on silicon detectors or thicker scintillators is employed. Harder radiation requires thicker and more dense scintillators to absorb enough energy.


Typically one assumes that the energy deposited translates linearly into photons. Caesium iodide doped with thallium (CsI(Tl)) is one of the brightest scintillators available. The main emission spectrum is centred around 546 nm (green) conveniently matching the peak of quantum efficiency of many silicon-based cameras.

Exponential decay. Fast (prompt) and slow response. Dual exponential.

scintillation wavelength
light yield
stopping power
resolution
decay time
material

Calibration using a radioactive source


\subsection{CsI(Tl)}
Caesium-iodide doped with thallium

Hallide crystal. Inorganic alkali halide.

CsI (NIST webbook and STAR):

Density 4.51 g/cc
Atomic fraction 0.488451 for 53 and 0.511549 for 55 (fraction by weight).

One of the brightest scintillators.

Slow scintillaotr, with 0.6 microsecs to 3.5 microsecon components.

This is based on St Gobain (deviations from manufacturers possible).

Wavelength of maximum emission around 550 nm (suitable for photodiode detection) and cutoff around 320 nm.
Primary decay time around 1000 ns.
Light yield around 54 photons/keV gamma.

Best light output at room temperature.

\subsection{LYSO}
LYSO ($Lu_{1.8}Y_{.2} SiO_5:Ce$)

Oxide crystal

St Gobain:
Density 7.1 g/cc
Wavelength emission maximum at 420 nm
Decay time around 45 ns (faster than CsI)
Light yield around 27.6 photons/keV.

\subsection{LANEX}
Commercially available scintillator, frequently used in dentistry.

