\chapter{Characterisation of a LWFA Bremsstrahlung Source for a Measurement of the Linear Breit-Wheeler Process}

\section{Motivation and Introduction}

The Breit-Wheeler (BW) process is the inverse process of the more commonly observed pair annihilation, where electrons and positrons decay into two photons. In BW the two (linear BW) or more (non-linear BW) photons combine to produce an electron-positron pair, in other words matter is created from light. The linear BW process is represented by a simple tree-level Feynman diagram. However, despite its simplicity it is the last tree-level diagram of QED that has not been definitively observed in its pure form, i.e. without the presence of other particles or external fields.

The non-linear process, on the other hand, has been measured and confirmed previously at an experiment at SLAC in the 90s where an ultrarelativistic electron beam of $46.6\,\mathrm{GeV}$ energy was scattered with an intense laser pulse $a_0 \sim 0.3$, resulting in highly energetic gamma photons from inverse Compton scattering. These gamma photons in turn interacted with $n \approx 5$ REF laser photons to produce electron-positron pairs. At intensities above $a_0 > 0.2$ the researchers detected $69 \pm 9$ positrons after XX SHOTS NUMBER that match the momentum profile for the nonlinear Breit-Wheeler process and the Trident process.
\vspace{\baselineskip}

Whilst the pair annihilation process is easily measured in radioactive decays as its cross section is very high at low energies and an abundance of electrons in ordinary matter provides an abundance of partners to annihilate with. Pair creation process from radiation, on the other hand, requires high energy photons or a very high photon density to overcome the production threshold of twice the electron rest mass energy $2m_e c^2 \approx 1\,\mathrm{MeV}$ in a suitable frame and more energy to provide some momentum to the produced particles. However, finding two bright photon sources at suitable energies at the same place is very challenging and hence mainly mediated pair production processes (Trident, Bethe-Heitler) have been measured.
\vspace{\baselineskip}

\begin{figure}
\includegraphics[width=.9\columnwidth]{2018QED_Chicane.JPG}
\caption{In pursuit of creating matter from light. Photo taken during the experiment campaign aimed at measuring the BW process at the Gemini facility in 2018. View upstream on the gamma-ray axis, indicated by a green alignment laser, through the magnetic chicane that is designed to transport electron-positron pairs from BW to single particle detectors.}
\end{figure}

High energy radiation in the keV range can be produced from highly relativistic electrons, for instance at XFELs and synchrotron sources using bending magnets and insertion devices. An alternative thermal X-ray source can be provided by sufficiently heating solid targets, for instance hohlraums or foils.

Directed energetic gamma radiation can be generated by passing electrons through solid material to produce bremsstrahlung or relying on inverse Compton scattering, which is for instance the method used in the previously E144 experiment.
\vspace{\baselineskip}

In the experiment presented in this work, two lasers are used to provide simultaneously an X-ray and a gamma ray source to overcome the mass threshold and to produce electron-positron pairs from the linear Breit-Wheeler process. The conceptual idea is based on REF PIKE WORK. The asymmetric distribution of the photon energies means that produced pairs will be strongly directed and easier to distinguish from noise.

The X-rays are generated by heating a solid germanium target, emitting an X-ray spectrum peaked at $1.5\,\mathrm{keV}$. The gamma ray source is produced by bremsstrahlung from LWFA electrons.
\vspace{\baselineskip}

The following sections will outline the experiment setup aimed at measuring the linear Breit-Wheeler process, and take the reader through some of the data taken during the experimental campaign and the considerations relevant for its analysis.

The description of the experiment setup will convince the reader immediately of the complexity of this project and will also make it evident that a large number of researchers is required to tackle this challenge, with responsibilities for different parts of the design, execution and analysis of various diagnostics being shared amongst its contributors. 

The author will focus on the characterisation of the electron beam and the gamma ray source produced with it, their impact on the BW cross section, and variability on and between different shot days. In particular, the choice of converter targets for the bremsstrahlung production and the sensitivity of the spectral measurement of the gamma rays are discussed.

In some parts the author will refer to X-ray spectra, diagnostics for single particle detection (SPD), beam transport for the produced pairs, and signal-to-noise ratios, but will not cover these parts in extensive detail. It is pursued and will be presented in some form by other colleagues, in other words it does not reflect the primary work of the author and exceeds the limits of this particular piece of work. Contributions of fellow colleagues will be highlighted as such if they are used in the course of this work. 

Nonetheless, in a brief section towards the end, the author will attempt to place this work in the context of the full analysis of this campaign, summarise its findings and relevance for the project, and direct the reader to relevant publications.

\section{Experiment Setup}

\subsubsection{General Aim and Location}

The experiment was performed at the dual 300 TW Ti:Sa Gemini laser at the Central Laser Facility, UK, in early 2018.
The aim of this setup is to produce and to measure electron-positron pairs from the linear BW process. For this purpose, high energy gamma rays are collided with a thermal X-ray source. The South arm of the Gemini laser provides the gamma source, the North beam the X-rays.
\vspace{\baselineskip}

\subsubsection{North beam and X-ray source}
In this experiment, the North beam is not fully temporally compressed to its usual 40-45 fs duration but is stretched to a pulse duration of $40\,\mathrm{ps}$ FWHM XX ERRORBAR? in order to effectively heat the solid target. The beam is focused down onto a solid target using an f/2 off-axis parabola (OAP) with a phase plate in the collimated beam that smooths and increases the size of the focal spot to about $75\,\mathrm{\mu m} \times 210\,\mathrm{\mu m}$. The target is a $25\,\mathrm{\mu m}$ kapton tape with etched holes reducing the thickness to $5\,\mathrm{\mu m}$ that are coated with $100\,\mathrm{nm}$ thick germanium (Ge) dots. When the germanium is heated, it produces a plasma plume that emits a thermal bath of X-rays at the interaction point. The motion of the tape drive is motorised and a fresh Ge target is provided on each shot. The tape drive was designed by Brendan Kettle (Imperial College), and the target and fabrication techniques were developed by the Target Fabrication Division at the Central Laser Facility, more specifically Sam Astbury and Chris Spindloe. IS THERE A PUBLICATION/PATENT ON THIS? THERE WERE SOME PLANS.

The X-ray source is diagnosed by an X-ray pinhole camera measuring the flux and the source size, and by a camera attached to a crystal measuring the spectrum of the X-rays in a $\sim 700\,\mathrm{eV}$ window from just under $1.3\,\mathrm{keV}$ to $2\,\mathrm{keV}$ including the L-shell emission of Ge at $1.5\,\mathrm{keV}$. IS THIS THE RIGHT EMISSION LINE? Both cameras were Andor DX420 in-vacuum X-ray cameras.
The X-ray source is the first component for the two-photon interaction. 
\vspace{\baselineskip}

\subsubsection{South beam, electrons and Gamma source}
The South laser is focused down by a $6\,\mathrm{m}$ focal length f/40 OAP onto the edge of a 17.5 mm length gas cell target\footnote{The gas cell has a variable length from few to 20 XX mm length. The target was designed by Nelson Lopes, formerly Imperial College, now IST.} filled with helium and 2 percent nitrogen at an average electron density of XX NUMBER. The typical average spot size was around $44\,\mathrm{\mu m} \times 53\,\mathrm{\mu m}$ at $5.51\pm 0.64\,\mathrm{J}$ energy on target at a pulse duration of XX NUMBER FS THIS WAS NOT MEASURED? IF SO $45 \pm 5\,\mathrm{fs}$, peak intensity $I_0 = 2.75 \times 10^{18} \,\mathrm{W/cm^3}$ corresponding to a normalised vector potential $a_0 = 1.13$ (this is based on focal spots from the 28th March and Matt's analysis MAYBE REDO ANALYSIS?). The energy on target was limited by the damage threshold of the mirrors in the focusing beam.
The laser drives a wakefield and accelerates electrons, injected predominantly via ionisation injection REF HERE, to relativistic energies. The remaining laser exiting the cell is disposed of by a replenishable kapton tape that acts as a plasma mirror (REF).
\vspace{\baselineskip}

The electrons are dispersed vertically by a permanent dipole magnet of integrated field strength $\int B dx = 0.4\,\mathrm{Tm}$ onto a scintillating LANEX screen measuring their energy and charge. The yoke of the magnet is blocking the path of dispersed lower energy electrons and as a result the outgoing spectrum has a low energy cutoff at about $300\,\mathrm{MeV}$. The screen is imaged by an Andor Neo camera equipped with a TV lens and a bandpass filter around $546 \pm 10\,\mathrm{nm}$ XX IS THIS CORRECT?. A motorised stage with high-Z foils of various thicknesses can be driven into the path of the electrons and act as bremsstrahlung converter foils. This interaction produces copious amounts of directed bremsstrahlung in propagation direction and stops most of the electrons in the process. Divergent gamma rays are likely to collide with components and apertures producing particles near axis with comparable properties as the BW pairs. A tungsten collimator of length $100\,\mathrm{mm}$, inner diameter $2\,\mathrm{mm}$ and outer diameter $20\,\mathrm{mm}$ is employed to reduce the emitted gamma-ray burst to its central part at low divergence in order to reduce noise. The exit of the collimator is placed at 280 mm distance from the converter target and hence apertures the beam down to a field of view of $2\,\mathrm{mm}/0.28\,\mathrm{m} = 7.14\,\mathrm{mrad}$. In addition, a thick block of tungsten obstructs the direct line of sight from the converter target to the Ge target drive. Only a collimated bright burst of gamma-rays from bremsstrahlung is incident on the interaction point, providing the second component for the two-photon interaction.
\vspace{\baselineskip}

\begin{figure}
\centering
\includegraphics[width=1.0\columnwidth]{BW2018_render_V3_annotated3.png}
%\includegraphics[width=.9\columnwidth]{BW2018_render_minV1.png}
\caption{Sketch of the experiment setup indicating the main components.}
\end{figure}

The gamma-ray burst propagates almost unperturbed by the interaction through the aperture of a lead wall, which shields noise generated around the interaction point, and then through a large aperture dipole magnet. It then passes through the central one out of three $125\,\mathrm{\mu m}$ thick kapton windows at the end of the chamber into air. The flange holds three apertures spanned with kapton, with a central $80\,\mathrm{mm}$ diameter circular aperture and two slits of dimensions $150\,\mathrm{mm}\,\times\, 35\,\mathrm{mm}$ on both sides. 

At air, a stack of $20 \times 20$ caesium-iodide (CsI) crystals of dimensions $2\,\mathrm{mm}\,\times\,2\mathrm{mm}\,\times20\,\mathrm{mm}$, each individually wrapped in aluminium foil ($\sim 15-20\,\mathrm{\mu m}$) and held together in a $1\,\mathrm{cm}$ thick aluminium casing, measures the transverse profile and yield of the gamma-ray signal. The total transverse area is $40\,\mathrm{mm}\times 40\,\mathrm{mm}$ which corresponds to an acceptance angle of $11.8\,\mathrm{mrad}$ based on a distance of $3.39\,\mathrm{m}$ from the converter targets to the profile screen. 
Another larger stack of $33 \times 47$ CsI crystals doped with thallium each of dimensions $5\,\mathrm{mm}\, \times\, 5\,\mathrm{mm}\,\times\,50\,\mathrm{mm}$ and separated by $1\,\mathrm{mm}$ aluminium spacers measures the decay of the signal in propagation direction to deduce the spectrum. 
Both CsI arrays are imaged by sensitive Andor iXon cameras equipped with suitable objectives and bandpass filters.
\vspace{\baselineskip}

\subsubsection{Chicane and Single Particle Detectors}

Potential electron-positron pairs from the photon-photon interaction are emitted preferably in the propagation direction of the gamma-ray burst. The pairs enter the field of a large aperture permanent dipole magnet of XX FIELD STRENGTH NUMBER that disperses the electrons and positrons horizontally in opposite directions, leaving the vacuum through the the other two kapton apertures at the end of the target chamber. The aperture of the magnet produces a low energy cutoff at about 220 MeV CHECK NUMBER ESPECIALLY BECAUSE OF WINDOW AND UPPER CUTOFF which is a slightly larger acceptance range than the vacuum window would provide. The dispersed electrons and positrons are then caught by an oppositely polarised permanent magnet of FIELD STRENGTH XX NUMBER on each side that bends the electrons or positrons respectively onto a narrow aperture of a lead-shielded enclosure and onto a set of single-particle detectors. The single-particle detectors are CsI arrays attached to sensitive CCD cameras. In addition, there are two TimePix silicon detectors in front of the CsI array on the positron side.

\EliasComm{Will there be a comment on the potential RSI here? This will depend on the status and progress of this work.}



\section{Bremsstrahlung Converter Targets}


The gamma signal depends strongly on the properties of the electron beam in terms of energy, brightness, divergence and pulse duration.
However, the choice of converter material is also crucial and needs to be optimised for this specific purpose.
In this case, we want to efficiently convert the electron beam into a narrow beam of gamma rays whilst also maintaining a low level of noise from secondary radiation and particles. To reduce the noise level several components are used to cut down the gamma signal, particularly of interest is the tungsten collimator that was briefly mentioned in the overview of the experiment setup.

The collimator permits a free field of view of $7.14\,\mathrm{mrad}$ full angle.

As indicator for gamma yield and divergence we will rely on the gamma profile monitor. An example of a gamma signal with and without collimator is shown in Figure \ref{BW:figs:Gprofile_collimator_INOUT}. The circular shadow seen on the profile monitor matches the predicted field of vow (GIVE NUMBER HERE).


\begin{figure}
\centering
\includegraphics[width=.5\columnwidth]{2018QED_GProfile_CollimatorIN_example.png}\includegraphics[width=.5\columnwidth]{2018QED_GProfile_CollimatorOUT_example.png}
\caption{Examples of gamma profile with converter in and out for divergence field of view. Replace this with real plots.}
\label{BW:figs:Gprofile_collimator_INOUT}
\end{figure}

Despite its low divergence the collimated gamma will be around $6.2\,\mathrm{mm}$ wide at the interaction point, about $865\,\mathrm{mm}$ downstream from the converter target. The X-ray source is much closer to the interaction point and will be smaller at around XX MM NUMBER.
The spatial overlap of both photon sources is hence guaranteed on each shot. The temporal overlap is also secured as the North beam is a $40\,\mathrm{ps}$ long heater and timing fluctuations at Gemini have been measured to be of order of 10's of $\mathrm{fs}$ REF HERE.

As a result the aim is to optimise the flux of gamma photons to interact with the ambient X-ray bath.

\subsubsection{Choice of converter material, overview on materials, yield vs. divergence consideration}

As choices of materials we decided on tungsten (W) and bismuth (Bi) as they both are high-Z materials, which means energy can be efficiently converted to radiation within thinner targets resulting in less scattering and divergence. W and Bi are also readily available and easy to handle. 
The aim is to find the ideal combination where most or all of the electron energy is transferred into radiation at a low divergence and high flux.

More specifically, 2 mm and 4 mm tungsten, 0.5 and 1 mm bismuth were mounted on 0.5 mm PTFE, all on a motorised linear stage that allows switching between the materials or no converter material at all.
\vspace{\baselineskip}

\EliasComm{Maybe write down bremsstrahlung equation here to explain the trends.}

The quantities we measure are the total counts on the gamma profile monitor (yield) and the width of the gamma signal in terms of the FWHM in the vertical and horizontal axis. This is being tested for all 4 converter targets in two electron regimes (one higher energy less charge from lower density plasma, the other higher charge but lower energy from a high density plasma). 10 shots were taken at each configuration (backing pressure A, converter 1...).

The results can be seen in Figure \ref{BW:fig:converter_yield_v_div}.
\vspace{\baselineskip}

From the results we can see a couple of trends. The divergence and yield for each material is larger when the solid target is thicker, i.e. the measured gamma yield is higher for 1 mm of bismuth than for 0.5 mm, and higher for 4 mm than for 2 mm of tungsten, and similarly so for the divergence.
This makes sense since more material leads to more scattering and hence an increased divergence. At the same time it indicates that more energy is converted into photons.

At the same time we see that the higher charge but lower energy electrons provide a lower yield of gamma rays at higher divergences. This also appears to be sensible as the initial divergence of the electrons is higher. The yield scales quadratically with the electron energy whilst the photon number relies only linearly on the electron charge. A higher energy electron beam, even at slightly lower charge, can hence provide a higher yield.
\vspace{\baselineskip}

Comparing the bismuth and the tungsten targets we see that the divergence for bismuth is in general lower than for tungsten. This is probably related to the thinner targetry. Bismuth has a higher Z number and can provide a similar conversion efficiency at lower scattering rates.
We also observe that the yield does not increase much more from bismuth 1 mm to any of the tungsten targets whilst preserving a lower divergence. This indicates that most of the electrons already transferred all or a large fraction of their energy into gamma radiation and more material mainly leads to scattering and showers of radiation. The yield of bismuth increases by around 50 percent from 0.5 to 1 mm thickness. This also coincides with a faint electron signal on the spectrometer screens and confirms that somewhere around those thicknesses most of the electron energy is converted to radiation, reaching saturation.

\begin{equation}
Bremstrahlung \sim Z^2 N(E) E^2
\end{equation}

Another result can be seen in Figure \ref{BW:figs:converter_pointing}. Here the pointing stability of the gamma signal is shown. There is some scatter but most of the targets operate at a comparable level and do not show a clear trend. This makes sense, as the general pointing mainly depends on the electron pointing and should not depend on the material.
%\begin{figure}[h]
%\centering
%\includegraphics[width=.5\columnwidth]{2018QED_ConverterGSpec_Yield.png}\includegraphics[width=.5\columnwidth]{2018QED_TimePixNoise.png}
%\caption{Left: Gamma Spec yield. Right: Noise on TimePix.}
%\end{figure}

\begin{figure}
\centering
\includegraphics[trim={4.9cm 0 5cm 0}, clip, width=.9\columnwidth]{2018QED_Converter_YieldVDiv.png}
\caption{Gamma Profile Yield and divergence. Use mrad instead of mm for FWHM and present a combined plot for x and y axis. The distance to gamma profile from the converter target is about 3.4 m. So 20/3.4 for a full angle, so less than 3 mrad half angle for 0.5 mm bismuth and so on.}
\label{BW:fig:converter_yield_v_div}
\end{figure}


\begin{figure}
\centering
\includegraphics[trim={4.9cm 0 5cm 0}, clip, width=.5\columnwidth]{2018QED_GConverter_PointingMM.png}\includegraphics[trim={5cm 0 5cm 0}, clip, width=.5\columnwidth]{2018QED_GConverter_PointingSigma.png}
\caption{Left: Centroid position on detector in mm, coordinate zero point is the bottom left corner of the detector. Right: Standard deviation of pointing fluctuation from mean value in mrad.}
\label{BW:figs:converter_pointing}
\end{figure}





\subsubsection{Throughput Collimator for Flux}

Considering the collimator and the measured divergences of the gamma rays, we can now estimate what fraction of the gamma signal that will make it through the collimator. A more divergent beam will lose more intensity but if it comes with higher yield it might still prevail.

We will use the measured FWHM divergence values and model the intensity profile after a Gaussian. We will also assume perfect alignment and ignore pointing for now as it is in the same ballpark for the converters and should not depend on the material.

Bismuth 0.5 mm has a lower divergence and most of its energy makes it through in terms of fraction based on the initial input, followed by 1 mm bismuth. If weighed by the total flux of photons related to that converter, the higher yield of the thicker bismuth target increases the total flux. The 2 mm thick W target provides a similar throughput (in absolute numbers) as the thin bismuth target. The thicker W target reaches close to similar levels as 1 mm of bismuth but not fully.

Based on these results 1 mm bismuth appears to be the best suited candidate for this endeavour.


\EliasComm{Add a bar chart on pure yield and throughput.}

\subsubsection{Noise Considerations}

When inserting the W collimator and thick block to reduce the noise levels, the number of gamma rays that make it to the detectors is decreased significantly. This makes it more difficult to characterise how the noise at the single particle detectors (SPDs) scales as the range is limited.

Using the same setup we are now looking at noise events reaching the SPDs, in this case only one of the TimePix detectors, and how it correlates with divergence and yield. Our hypothesis is that a more divergent gamma ray beam will interact with more components and apertures in the beam path, generating radiation and particles that are closely related to the BW signal and will because of this resemblance be detected by the SPDs.
\vspace{\baselineskip}

Figure XX shows a plot of the total counts on the gamma detector (yield) agains the number of registered particle events or `clusters' on the TimePix detector.

The TimePix results show the level of noise measured for the different converters. It shows that, despite having a similar yield as 1 mm bismuth, the noise levels are much higher. 
The yield is correlated with noise and so is the divergence. More photons produce more noise and more divergent beams interact with more matter and apertures, producing more noise.

On that scale 1 mm bismuth combines lower noise levels with high yields. Only 0.5 mm bismuth performs better in terms of signal to noise but at cost of lower yields. Assuming that the introduction of the collimator will get rid of most divergence related noise 1 mm bismuth is a better choice as it preserves high flux.

\EliasComm{Introduce TimePix and what the numbers mean.}
\EliasComm{Need appropriate legend for all data points or combine them.}
\EliasComm{Fit some lines to support the linear trends?}

\begin{figure}
\centering
\includegraphics[trim={4.9cm 0 5cm 0}, clip, width=.5\columnwidth]{2018QED_Converter_YieldvNoise.png}\includegraphics[trim={5cm 0 5cm 0}, clip, width=.5\columnwidth]{2018QED_Converter_DivvNoise.png}
\caption{Left: Gamma Spec yield. Right: Noise on TimePix.}
\end{figure}

\begin{figure}
\centering
\includegraphics[trim={4.9cm 0 5cm 0}, clip, width=.5\columnwidth]{2018QED_SignalNoise_Transmission.png}\includegraphics[trim={5cm 0 5cm 0}, clip, width=.5\columnwidth]{2018QED_SignalNoise_AbsTransmission.png}
\caption{Left: Gamma Spec yield. Right: Noise on TimePix.}
\end{figure}


\subsubsection{Summary and Decision on Material}

Best flux throughput for 1 mm bismuth. Noise is also lower although the levels will change for collimator and W block conditions.
1 mm bismuth also takes care of all electrons and potentially reduces noise from inside the collimator (dispersed remaining electrons).

Electron regime intermediate (low density but backing pressure had to be increased due to leak). Higher energy is more preferable than a bit more charge and higher initial divergence.

In theory, even thinner target and higher Z would be favourable (as long as all electrons are converted) but Bi is the last non-radioactive material.




\section{Characterisation of Electron Spectra for BW shot days}

\subsubsection{Application of electrons in this context and need for characterisation}

In this experiment the energy of relativistic electrons is converted into directed and highly energetic bremsstrahlung.
Whenever bremsstrahlung is produced from a converter target, the spectrum of the electrons can not be measured simultaneously as they are almost entirely stopped by the high-Z material in order to produce a high flux of gamma rays. 

Hence, the electron spectra are characterised without the converter target before and after full BW data runs instead. The gamma signal, on the other hand, is measured during the BW data runs.
\vspace{\baselineskip}

The relevant shot days were the 6th, 9th and 10th May, hence the analysis of the electron spectra will focus particularly on these days.

\begin{figure}
\centering
\includegraphics[width=.9\columnwidth]{2018QED_ElecSpecs_montage.jpg}
\caption{Examples of raw electron spectra acquired on the experiment. The spectra are on a linear spatial scale, where the top is closer to the beam axis, i.e. indicates higher electron energies. Replace this by one row of raw and processed images or maybe even just one row of example spectra on a linear energy scale. Or just one spectrum with linear energy scale plus lineout in both dimensions to show where quantities come from.}
\label{BW:figs:elec_raw}
\end{figure}

\subsubsection{Raw images and general note on processing data}

Raw images of typical electron signals on the lanex are seen in Figure \ref{BW:figs:elec_raw}.
The images are treated for background and are projective transformed to account for the image distortion caused by the relative angle of camera and screen. By measuring the field components of the magnet and the distances accurately the spatial scale on the lanex screen is related to the energy of the electrons. The captured scintillation light is cross-calibrated with an image plate to obtain a number for the absolute charge of the beam. A more detailed description of the intermediate steps is found in the Methods section.
\vspace{\baselineskip}

\subsubsection{Investigation of main quantities and variation on the days}

The performance of the wakefield accelerator can vary from day to day and is hence characterised separately for each of the relevant shot day, in some cases several datasets taken at different times were available.

The key quantities of interest to characterise the electron beam with and to then compare the shot days with each other are charge, energy, divergence, pointing and the stability of those properties from shot to shot.
\vspace{\baselineskip}

An overview of these quantities describing the accelerator performance for the relevant days are presented in Table \ref{BW:tables:elec_days} and again in a more graphic comparison in Figure \ref{BW:figs:elec_variations}. The lineouts for the average electron spectra for those days are shown in Figure \ref{BW:figs:elec_average_spec}.
\vspace{\baselineskip}

In general, 
... large energy spread as typical for ionisation injection at Gemini, no quasi-mono
... lower energy than typically (related to energy on-target)
... performance comparable 9th and 10th with plateau up to 550 MeV or so
... on the 6th the signal extends to 700 MeV and quite stable.

Looking at the key quantities:
... charge was higher on the 6th at XX 25 pC pm.
... energy as mentioned
... divergence similar levels but more stable over the dataset and at the lower end
... similar properties for the pointing



\begin{figure}
%\includegraphics[width=.5\columnwidth]{2018QED_ElecSpecs.png}
\centering
\includegraphics[trim={4.9cm 0 5cm 0}, clip, width=.9\columnwidth]{2018QED_ElecSpecs_average}
\caption{Average spectra for the relevant shot days, 6th (blue), 9th (purple) and 10th (green). The spectra are normalised to charge and are not representative (SHOULD MAKE THIS NOT NORMALISED ACTUALLY) of relative differences. The shaded areas indicate 1 standard deviation from the average spectrum.}
\label{BW:figs:elec_average_spec}
\end{figure}

\begin{figure}
\centering
\includegraphics[trim={7.5cm 0 6.5cm 0}, clip, width=1.0\columnwidth]{2018QED_Espec_Variations.png}
\caption{Average electron charge, energy, divergence and pointing on the relevant shot days. The error bars indicate 1 standard deviation from the average value.}
\label{BW:figs:elec_variations}
\end{figure}



\begin{table}
\centering
\begin{tabular}{r|r|r|r|r|r|r}
Date & Run & Charge [pC] & $E_{max}$ [MeV] & $\theta$ [mrad] & $\Delta_{centroid}$ [mrad] & $N_{shots}$\\ \hline \hline
%5 & 2 & $20 \pm 10.8$ & $599 \pm 57$ & $3.14 \pm 0.75$ & 1.25 & 32\\ \hline
6 & 1 & $26.26 \pm 3.8$ & $709 \pm 46$ & $2.26 \pm 0.29$ & 0.62 & 10\\ \hline
9 & 1 & $14.73 \pm 5.5$ & $565 \pm 43$ & $2.5 \pm 0.72$ & 0.93 & 18\\ 
9 & 4 & $7.7 \pm 4.4$ & $551 \pm 16$ & $1.7\pm 0.31$ & 0.81 & 12\\ \hline
10 & 1 & $11.55 \pm 2.7$ & $511 \pm 19$ & $2.84 \pm 0.93$ & 0.90 & 22\\ 
10 & 2 & $15.24 \pm 5.1$ & $512 \pm 26$ & $2.3 \pm 0.52$ & 0.50 & 6\\ 
10 & 3 & $9.56 \pm 3.5$ & $535 \pm 21$ & $2.82 \pm 0.91$ & 1.55 & 21
\end{tabular}
\caption{Details on runs. Electron properties.}
\label{BW:tables:elec_days}
\end{table}




\section{Spectral Analysis of the Gamma Signal}

\subsubsection{General bits on the diagnostic, signal extraction and characterisation}
\EliasComm{Add some raw data and examples from analysis.}

Electrons characterised in the previous section are converted to a burst of gamma rays by inserting a high-Z converter foil in the beam path, in this context mainly 1 mm of bismuth. This is to balance the need for a decent amount of gamma rays and a low divergence of the resulting beam. Thick targets result in the highest yield but the beam is more divergent and more signal will be cut out by the collimator etc. and more noise is to be expected.
\vspace{\baselineskip}


\EliasComm{Add a plot showing the raw data from GammaSpec for different converters.}

\EliasComm{Add a plot for correction factor.}

The tungsten collimator only lets radiation through an aperture of XX NUMBER. In addition the thick tungsten block aims to half the signal to block any gamma rays from hitting the Ge target drive. The alignment of these components will change from day to day based on human error.
\vspace{\baselineskip}

The detector used to measure the spectrum of this gamma flux is a stack of scintillating crystals as described in Methods imaged by an Andor iXon camera.
RAL stack with spectral measurement and plastic, gamma profile in the path. 
\vspace{\baselineskip}

Signal extraction. Show an example.
Selecting the crystals and get a pixelated response.
\vspace{\baselineskip}


\subsubsection{Detector response in experiment, shot to shot and day by day}

Seeing not really much change in experimental response.
Give a number on critical energy. Any idea of brightness (absolute number?).
The curves are fairly similar. Might see an overall change in the spectrum but everything within a crystal. Check if there is a difference between the days.
Absolute calibrations for gamma profile is possible but might be harder for gamma spectrometer due to the components in between.
\vspace{\baselineskip}

The response is not sensitive to such highly energetic radiation or the number of photons is much lower and hence the response is based on that.
This means the detector is not suitable to discriminate a spectrum to such detail and also is not suitable to deduce a gamma spectrum independent of assuming its shape. Luckily bremsstrahlung is well understood. It is essential to have many reference electron shots. Also a lesson for the future to take data in between.
\vspace{\baselineskip}

\subsubsection{GEANT simulation and correction factors using bremsstrahlung}

Simulating response of detectors using GEANT.
Monoenergetic photons in a different range and present the response curves.
\vspace{\baselineskip}

Correcting using brems.
Get a brems run using a thick converter and simulate the response based on that.
Correct out the crystal efficiencies and viewing angle. Reduce this section maybe based on Methods.
\vspace{\baselineskip}

Correlating Gamma profile and spec counts. No change in response for those detectors.
Seeing that this is very linear the results from gamma profile assessment are still or again valid.
This is not useful to distinguish energies.
\vspace{\baselineskip}



\subsubsection{General critical energy fitting}

\subsubsection{Comparison yield per day with and without blocks}

\subsubsection{Simulated spectra from Electrons and variation in resulting gamma spectra}

\subsubsection{Characterisation of fluctuations and lack of change in response}

The produced spectrum can be simulated using the electron spectra characterised earlier. GEANT. Typical gamma spectra produced are shown in FIGURE XX. 
Show that the responses for all electron spectra is fairly similar in simulations and the resulting spectra vary mostly at a low photon level and high energies.
The response of the detector only results in small variations.
\vspace{\baselineskip}

The experimental results also show only small variations which is consistent with simulations. The systematic offset is including the viewing angle, efficiency of crystals and so on.
Does the flux vary as much as the electron charge? (for the shots with W block, collimator and converter in the correlation is washed out and there is no clear indication that these things are linked). Add some plots for that as well.
Stability of gamma flux?


\begin{figure}
\centering
\includegraphics[width=.5\columnwidth]{2018QED_ElecSpecs_examples2.png}\includegraphics[width=.5\columnwidth]{2018QED_GammaSpec_simspec2.png}
\caption{Left: Representative extreme examples of electron spectra from one day. Right: Resulting gamma spectrum based on GEANT simulations. Maybe show an average with shaded error bars instead. Add a third electron spectrum and gamma result for the other extreme.}
\end{figure}

Add the differences from day to day somewhere as well as comparison.

\begin{figure}
\centering
\includegraphics[width=.5\columnwidth]{2018QED_ElecSpecs.png}\includegraphics[width=.5\columnwidth]{2018QED_GammaSpec_simresp.png}

\includegraphics[width=.5\columnwidth]{2018QED_GammaSpec_expresp.png}\includegraphics[width=.5\columnwidth]{2018QED_GammaSpec_Average_expsim.png}

\caption{Left: Lineouts for most electron spectra. Right: Simulated detector responses. BLeft: Experimental responses. BRight: Comparison average responses.}
\end{figure}


What is the statistical variation of the signal on each day. Use the electron spectra to put a number onto this.
Use plot of relative and absolute number of photons above a threshold. Sum them up and also sum up number times energy as indicator for high energy photons.

Provide a plot with numbers on fluctuation of number of photons to expect in total and in particular above a threshold and how that varies shot-to-shot and day by day.

Provide some plots from the GEANT simulation (visualisation).

\begin{figure}
\centering
\includegraphics[trim={7.5cm 0 6.5cm 0}, clip, width=1.0\columnwidth]{2018QED_GSpec_Variations.png}

\includegraphics[width = 0.9\columnwidth]{2018QED_GEANT_AvGSpec.png}
\caption{Variations of gamma signal (based on GEANT simulations). Average gamma spectrum (not scaled by yield). Make this one 4 plot panel.}
\end{figure}


\section{Performance and Stability of the Bremsstrahlung Source}

This part is about the spectra on the BW data days.
\vspace{\baselineskip}



\section{Impact of Spectral Fluctuations on the total BW cross section}

\subsubsection{Simple calculation on impact on the total cross section based on the GEANT spectra}

Cross section using number of photons. How many photons above a threshold and how is this affected?

Here some estimates from Robbie whether this change in flux is affecting the number of BW pairs or not.
If not, this is somewhat a description that the detector is not sensitive enough but it does not matter.
If it is somewhat important, it becomes crucial to develop a spectrometer that is sensitive to these photons.
This is the part about the shot-to-shot fluctuations.

Using the new spectra with photons only up to a certain energy. 

Put a plot average fraction of photons above an energy on a plot.
\vspace{\baselineskip}


\begin{figure}
\centering
\includegraphics[width=.5\columnwidth]{BWxsection2.png}\includegraphics[width=.5\columnwidth]{BWxsection_2D.png}
\caption{BW cross section at a fixed and variable X-ray photon energy.}
\end{figure}

\begin{figure}
\centering
\includegraphics[width=.9\columnwidth]{BW_Xsec_NxNg.png}
\caption{BW cross section for experimental gamma and X-ray spectra with photon numbers.}
\end{figure}


\begin{figure}
\centering
\includegraphics[width=.5\columnwidth]{2018QED_Xsec_VariationAvDay.png}\includegraphics[width=.5\columnwidth]{2018QED_Xsec_VariationMorton.png}
\caption{Day to day fluctuation of relative total cross section: Left using the measured X-ray spectra (average per day) with limited energy window. Right: using Morton simulations over a larger energy window. Add data points for different relative photon-photon angle.}
\end{figure}

The part about the additional photons on the 6th should be discussed here as well.
By how much, assuming an X-ray spectrum, do we expect the total cross section to rise with a change in gamma spectrum as estimated from day to day?
How much more likely is seeing electron-positron pairs based on this?

Make a simple equation basing this on one Xray photon energy (1.5 keV) and put a threshold on the photons we need from the gamma source.
Then see how much the cross section increases and check if it increases by more than the general exponential spectrum.
Be clear that a proper calculation needs to convolve the X-ray spectra, gamma spectra, efficiencies and so on.

Looking at the cross section plot the cross section decreases towards higher energies at a constant X-ray energy.
The spectrum and the number of photons also become less important at the higher energy end of photons.
The key question is how the X-ray spectrum evolves around the right energies and if there is a large abundance of lower energy photons that can interact at high cross section with the tail of photons.

Multiply the curve for 1.5 keV with the gamma spectra to get the total cross section. Then multiply this with the respective average charge to get an estimate how much more likely it would be to see a positron.

\begin{equation}
N_{BW, day}|_{theta} \sim \int N_{ph, X} (E_{X}) N_{ph, \gamma} (E_{\gamma}) \sigma_{BW} (E_{X}, E_{\gamma}, \theta) \mathrm{d}E_{\gamma}\mathrm{d}E_{X}
\end{equation}

This approach is just a simple calculation to demonstrate the difference we expect.
Of course one also has to fold in the X-ray spectrum over a wider energy range along with other factors like higher noise levels due to the increased charge and so on.

What angle are we assuming? There will be a range of photon angles but what is the design main angle?
According to an Evernote note (X-ray diagnostics survey by BK I assume) the angle was roughly 40 degrees, so main axis is 140 degrees but with a large spread of relative angles I would assume. If the beam has good emittance (or if there is a different term for gamma beams) then the divergence will determine the range of angles. The collimator lets through 3.57 mrad half angle which is 0.2 degrees, so the main range of angles will come from the X-ray source.

The closer to angle is to head-on collision to more the peak of the cross-section shifts towards lower gamma energies. 
If the X-ray spectrum is mainly centred around 1.5 keV the additional 800 MeV does not have an impact.
If the full X-ray spectrum (for instance Morton simulations) are included then a large pool of X-rays are available at lower energies to be paired up with the high energy photons of 600-800 MeV.


Maybe a comment about designing your experiment such that parts of the X-ray spectrum we can not measure are being blocked by material such that the gamma components we can not resolve are not impacting on the cross section. The angle is also a handle on equalising the cross section across the window of gamma energies such that few high energy photons still access the same pool of photons and the cross section is low.

Weakness of wakefield accelerator are the intrinsic fluctuations of the electron beam. Better stability by improving the e-beam but for instance by using bremsstrahlung we already cut out some of the uncertainty. See simulations that show that fluctuations in the spectral shape are not impacting the overall gamma spectrum much. 
If we now tailor the X-rays accordingly and choose the angle of the collision appropriately we can mitigate the impact of these fluctuations (maybe at a marginal cost of produced positrons) and have more stable conditions at the experiment.

\section{Future Outlook: A gamma spectrometer to discriminate high energy photons}

\subsubsection{Explanation why detector is not sensitive to the changes seen in the simulated spectra}
Here some ideas and simulations on how to improve the gamma spectrometer design to determine more accurately the spectrum.

Ideas: Insert different materials?
Check cross sections.
Pair spectrometer works at which energies?
Is there a compact way to measure this?
Would this also be of advantage for radiation reaction? 
\vspace{\baselineskip}

The elongation of the signal for the dual axis spectrometer was to measure the decay at higher resolution. However, most of the decay happens early in the crystal and the length of the detector only produces uncertainties and problems to consider when extracting the signal.
A cut off at reasonable crystal length using heavier materials could help.
Test in GEANT for range of gamma photons what the output is after a certain number of photons. A thicker lead piece could cancel out lower energies and ensure the higher energy tail is considered properly.

The problem is that in some scenarios the number of photons at higher energies is much lower and the signal is swamped. Having two parts, one to determine the overall critical energy, and a second one only looking at the high energy behind some shielding, would be an option.
\vspace{\baselineskip}

Show that the problem is the following:
above 10's of MeV the cross section for pair production increases and is relatively flat over long time. What happens is that gammas convert into electrons which convert into gammas in showers and so on. In most cases a high number of low few MeV gamma rays is present and the main indicator for the initial energy is how many times these showers continue to proceed and where they deposit energy.

\subsubsection{Based on problem present ideas to make the detector more sensitive along with GEANT simulations.}

The approach to check how we can distinguish things beyond the peak of the main number of photons is using GEANT and checking the input gamma, electrons and positrons and see how the distribution evolves going through the profile stack and the gamma spectrometer.
Based on this decide on a tactic to have a third part (in addition to profile, critical energy measurement) which allows separately to look at just very high energies and see for a selected energy band if the flux changes. This could be adjusted and optimised for certain energies.
If this does not work, point out that there are alternative designs based on pair spectrometers including magnets which might be a better option.


\section{Results of the Breit-Wheeler Analysis}


\subsubsection{Discussion of how this work fits in the big picture of the complete BW analysis}
\EliasComm{Here a brief discussion of the findings of the complete BW analysis and if there was any sign of BW, reference publications that came out of this and so on.}

How much to talk about different parts?
Single particle detectors, tungsten blob, TimePix?

There are some positron particles, noise etc..
Positrons can be created from Bethe-Heitler, wall apertures, windows and so on.

Decision on whether it is significant or not.



\section{Conclusion}

LWFA dual laser setup to produce BW pairs. Some indication that we saw pairs consistent with the experiment conditions at the optimum.
\vspace{\baselineskip}

Stability and robustness of gamma/X-ray source to a certain degree.
\vspace{\baselineskip}

Limits of gamma spectrometer design and potential improvements.
An improved detector design presented in this chapter.
