\chapter{Measurement of the Linear Breit-Wheeler Process using LWFA}

\section{Motivation}

Last tree-level diagram that was not measured.

Why this scenario? Need to high energy photons to overcome the threshold. Using an asymmetric distribution the resulting particle momentum is strongly directed. There are also other processes that produce positrons and they have to be separated out.

In this case we use a thermal X-ray source coupled to a gamma ray:

\begin{equation}
n_\gamma = \frac{2\chi(3)}{\pi^2 \lambda^3} \left(\frac{k_B T}{m_e c^2}\right)^3
\end{equation} 

Coupled to a Bremsstrahlung source (high gamma!).

\section{Experiment Setup}

Maybe also a Blender diagram of the setup.

The aim of this experimental setup is producing and measuring Breit-Wheeler pairs. For this purpose high energy gamma rays are collided with a thermal X-ray source. In this scenario the South arm of the Gemini laser provides the gamma source, the North beam the X-ray bath.
\vspace{\baselineskip}

The North beam in this experiment is not compressed to its usual 40-45 fs duration but is stretched to around 60 ps. The beam is focused down using an f/2 OAP but a phase plate smoothens and enlargens the focal spot to several millimetres size. This beam is then incident onto a tape drive with Germanium dots, producing a plasma emitting a thermal bath of X-rays at a second interaction point, called secondary target chamber centre (SCC). This is the first component for the two-photon interaction. The X-ray source is diagnosed by an X-ray pinhole camera measuring the flux and the source size, and by a camera attached to a crystal measuring the spectrum of the X-rays in a certain spectral range.
\vspace{\baselineskip}

The South laser is focused down by an f/40 OAP onto the edge of a 17.5 mm length gas cell target filled with helium and 2 percent nitrogen. The average spot size was around XX NUMBERS at XX NUMBERS energy on target (limited by the damage threshold of the mirrors), corresponding to a normalised vector potential $a_0$ of XX NUMBERS.
The laser drives a wakefield and accelerates an electron bunch injected via ionisation injection to relativistic energies. The remaining laser exiting the cell is disposed of by a tape drive that acts as a plasma mirror.

The electrons are dispersed vertically by a XX NUMBERS permanent dipole magnet onto a scintillating lanex screen to measure their energy and charge. A motorised stage with high-Z foils of different thicknesses can be driven into the path of the electrons and act as converter foils. This interaction produces copious amounts of directed bremsstrahlung in propagation direction and stops most of the electrons in the process. A tungsten collimator reduces the emitted gamma-ray burst to its central part which is close to collimated. This reduces noise generated further downstream by pre-emptively converting the excess of gamma rays into positrons (which are dispersed and dumped by the dipole magnet). A thick block of tungsten also blocks the direct line of sight to the Ge target drive. Only a collimated bright burst of gamma-rays from bremsstrahlung are incident on the second interaction point, providing the second component for the two-photon interaction.
\vspace{\baselineskip}

The gamma ray burst propagates almost unperturbed by the interaction through a kapton-kevlar window. At air a stack of CsI crystals measures the transverse profile of the gamma-ray signal and another larger stack of crystals measures the decay of the signal in propagation direction to deduce the spectrum.
\vspace{\baselineskip}

Potential electron-positron pairs from this interaction propagate preferably in the propagation direction of the gamma-ray burst. The pairs enter the field of a large aperture permanent dipole magnet that disperses the electrons and positrons horizontally in opposite directions, leaving the vacuum through the same wide kapton-kevlar window, where they are then caught by one magnet on each side that bend the electrons or positrons respectively onto single-particle detectors.
\vspace{\baselineskip}

This chapter will focus on the characterisation of the electron spectra and the gamma-ray signal. The single-particle detectors, the exact shielding setup and the characterisation of the X-ray source are hence not further elaborated in more detail.

\section{Electron Spectra Measurement}

In this experiment the energy of relativistic electrons is converted into directed and highly energetic bremsstrahlung.
Whenever bremsstrahlung is produced from a converter target, the spectrum of the electrons can not be measured as they are almost entirely stopped by the high-Z material. The electron spectra, their charge and stability are characterised before and after data runs attempting BW production.

The electron beam is dispersed by the a permanent dipole magnet onto a scintillating lanex screen. Electrons with energy below XX NUMBERS hit the yoke of the magnet and can hence not be measured on the lanex screen. The screen is imaged with an Andor Neo camera equipped with a TV lens and suitable filters to reduce ambient laser light.
\vspace{\baselineskip}

Raw images of typical electron signals are seen in FIGURE XX, lineouts of electron spectra on an energy axis can be seen in FIGURE XX.
The electrons produced with ionisation injection in this context have a large energy spread from continuous injection reaching maximum energies of about XX. The performance of the accelerator varies from day to day but typically either an almost flat spectrum up to XX MeV or a distinct peak at XX MEV was produced (see figure). Depending on the day the average charge was XX pC which is relatively low for the Gemini laser but might be linked to the limited energy on-target that could be delivered.

The energy of the beams and the charge was stable to within XX some numbers here.

\begin{figure}
\includegraphics[width=.9\columnwidth]{2018QED_ElecSpecs_montage.jpg}
\caption{Montage of electron spectra (screen raw).}
\end{figure}

\begin{figure}
\includegraphics[width=.5\columnwidth]{2018QED_ElecSpecs.png}\includegraphics[width=.5\columnwidth]{2018QED_ElecSpecs.png}
\caption{Left: 2D example of spectrum. Right:  Lineouts for electron spectra (maybe waterfall?).}
\end{figure}

\section{Bremsstrahlung Characterisation - Spectral Measurement Gamma Rays}

RAL stack with spectral measurement and plastic. GEANT.

Used in experiment was mainly bismuth 1 mm. This is to balance the need for a decent amount of gamma rays and the divergence of the resulting beam.
Show that the responses for all electron spectra is fairly similar in simulations and the resulting spectra vary mostly at a low photon level and high energies.
The experimental results also show only small variations which is consistent with simulations.

Does the flux vary as much as the electron charge?

This means the detector is not suitable to discriminate a spectrum to such detail and also is not suitable to deduce a gamma spectrum independent of assuming its shape.

Here some estimates from Robbie whether this change in flux is affecting the number of BW pairs or not.
If not, this is somewhat a description that the detector is not sensitive enough but it does not matter.
If it is somewhat important, it becomes crucial to develop a spectrometer that is sensitive to these photons.

\begin{figure}
\centering
\includegraphics[width=.5\columnwidth]{2018QED_ElecSpecs_examples.png}\includegraphics[width=.5\columnwidth]{2018QED_GammaSpec_simspec.png}
\caption{Left: Representative extreme examples of electron spectra. Right: Resulting gamma spectrum based on GEANT simulations.}
\end{figure}

\begin{figure}
\includegraphics[width=.5\columnwidth]{2018QED_ElecSpecs.png}\includegraphics[width=.5\columnwidth]{2018QED_GammaSpec_simresp.png}

\includegraphics[width=.5\columnwidth]{2018QED_GammaSpec_expresp.png}\includegraphics[width=.5\columnwidth]{2018QED_GammaSpec_Average_expsim.png}

\caption{Left: Lineouts for most electron spectra. Right: Simulated detector responses. BLeft: Experimental responses. BRight: Comparison average responses.}
\end{figure}


\section{Future Outlook: A gamma spectrometer to discriminate high energy photons}

Here some ideas and simulations on how to improve the gamma spectrometer design to determine more accurately the spectrum.


\section{Conclusion}

LWFA dual laser setup to produce BW pairs.

Stability and robustness of gamma/X-ray source.
Limits of gamma spectrometer design and potential improvements.
