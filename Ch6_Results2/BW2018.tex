\chapter{Characterisation of a LWFA Bremsstrahlung Source for a Measurement of the Linear Breit-Wheeler Process}
\label{Chap:BW}


\section{Motivation and Introduction}

The Breit-Wheeler (BW) process is the inverse process of the more commonly observed pair annihilation, where an electron and a positron decay into two photons. In BW, two (linear BW) or more (non-linear BW) photons combine to produce an electron-positron pair - in other words matter is created from light. The linear BW process is represented by a simple tree-level Feynman diagram. However, despite its simplicity it is the last tree-level diagram of QED that has not been definitively observed in isolation in vacuum, i.e. without the presence of other particles or external fields.
\vspace{\baselineskip}
\iffalse
its pure form
The non-linear process, on the other hand, has been measured and confirmed previously as part of the E144 experiment at SLAC in the `90s where an ultrarelativistic electron beam of $46.6\,\mathrm{GeV}$ energy was scattered with an intense laser pulse $a_0 \sim 0.3$, resulting in highly energetic gamma photons from inverse Compton scattering \cite{Bula1996_RR,Burke1997_RR}. These gamma photons in turn interacted with $n \approx 5$ laser photons to produce electron-positron pairs. At intensities above $a_0 > 0.2$ the researchers detected $69 \pm 9$ positrons after XX SHOTS NUMBER that match the momentum profile for the non-linear Breit-Wheeler process and the Trident process.
\vspace{\baselineskip}
\fi

Pair annihilation is easily measured in radioactive decays as its cross section is very high at low energies ($\sigma \propto E^{-1}$) and an abundance of electrons in ordinary matter provides an abundance of partners for positrons to annihilate with. Pair creation from radiation, on the other hand, requires high energy photons and/or a very high photon density, for instance a tightly focused laser, to overcome the production threshold of twice the electron rest mass energy $2m_e c^2 \approx 1\,\mathrm{MeV}$ and an excess of energy to provide the generated pair with sufficient momentum. However, delivering two bright photon sources at suitable energies at the same location is very challenging and hence mainly mediated pair production processes in presence of a background field have been measured instead \cite{Heitler1954_BH,Chen2009_BH,Sarri2015_pairs,Bula1996_RR}.
\vspace{\baselineskip}

\begin{figure}
\centering
\includegraphics[width=.7\columnwidth]{2018QED_Chicane.JPG}
\caption{In pursuit of creating matter from light. Photo taken during the experiment campaign aimed at measuring the linear Breit-Wheeler process at the Gemini facility in 2018. View upstream on the gamma-ray axis, indicated by a green alignment laser, through the magnetic chicane that is designed to transport generated electron-positron pairs from the Breit-Wheeler process to single particle detectors.}
\end{figure}

High energy radiation in the $\mathrm{keV}$ range can be produced from highly relativistic electrons, for instance at XFELs and synchrotron sources using bending magnets and insertion devices, and equivalently in plasma accelerators through betatron radiation. A thermal X-ray source can be provided by sufficiently heating solid targets, for instance hohlraums or foils (REF).

Directed energetic gamma radiation can be generated by passing electrons through solid material to produce bremsstrahlung (REF) or relying on inverse Compton scattering (REF).
\vspace{\baselineskip}

In the experiment presented in this work, two lasers are used to provide simultaneously an X-ray and a gamma-ray source to overcome the mass threshold and to produce electron-positron pairs from the linear Breit-Wheeler process. The conceptual idea is based on \cite{Pike2014_BW,PikeThesis}. The asymmetric distribution of the photon energies means that produced pairs will be strongly directed along the gamma-ray axis and easier to separate from noise. This is referred to as pair beaming \cite{Ribeyre2018_BW}.

The X-rays are generated by heating a solid germanium target, emitting an X-ray spectrum peaked at $1.5\,\mathrm{keV}$ (L-shell emission). The gamma ray source is produced by bremsstrahlung from LWFA electrons \cite{Glinec2005_Brems}.

\section{Chapter Outline}

... Exp setup

... optimise yield

... optimise noise 

... characterise electrons

.. check gammas related

.. check variations on days

... analyse the impact on the BW cross-section

.. Final work/future work: new detector and estimate what needed to improve resolution

The following sections will outline the experiment setup aimed at measuring the linear Breit-Wheeler process, and take the reader through some of the data taken during the experimental campaign and the considerations relevant for its analysis.

The description of the experiment setup will convince the reader immediately of the complexity of this project and will also make it evident that a large number of researchers is required to tackle this challenge, with responsibilities for different parts of the design, execution and analysis of various diagnostics being shared amongst its contributors. 

This chapter will focus on the characterisation of the electron beam and the gamma ray source produced with it, their impact on the BW cross section, and variability on and between different shot days. In particular, the choice of converter targets for the bremsstrahlung production and the sensitivity of the spectral measurement of the gamma rays are discussed.

In some parts there are references to X-ray spectra, diagnostics for single particle detection (SPD), beam transport for the produced pairs, and signal-to-noise ratios, but this will not be covered in extensive detail. It is pursued and will be presented in some form by other colleagues and exceeds the scope of this thesis. Explicit contributions by fellow colleagues will be highlighted as such if they are used in the course of this work. 

Nonetheless, a brief section at the end will attempt to place this work in the context of the full analysis of this campaign, summarise its findings and relevance for the project, and direct the reader to relevant publications for further information.

\section{Experimental Setup}

The experiment was performed at the dual 300 TW Ti:Sa Gemini laser at the Central Laser Facility, UK, in early 2018.
The aim of this setup is to produce and to measure electron-positron pairs from the linear BW process. For this purpose, high-energy gamma rays are collided with a X-ray source from a hot plasma. The South arm of the Gemini laser provides the gamma source, the North beam the X-rays.
A sketch of the setup is provided in Figure \ref{BW:fig:exp_sketch}.

\begin{figure}
\centering
\includegraphics[width=1.0\columnwidth]{BW2018_render_V3_annotated3.png}
%\includegraphics[width=.9\columnwidth]{BW2018_render_minV1.png}
\caption{Sketch of the experiment setup indicating the main components.}
\label{BW:fig:exp_sketch}
\end{figure}

\subsubsection{North beam and X-ray source}

In this experiment, the North beam is temporally not fully compressed to its usual $45\,\mathrm{fs}$ duration but is stretched to a pulse duration of $40\,\mathrm{ps}$ \textsc{fwhm} in order to effectively heat a solid target. The beam is focused by an f/2 off-axis parabola (OAP) with a phase plate (REF) in the collimated beam that smooths and also increases the size of the focal spot to an ellipse of about $75\,\mathrm{\mu m} \times 210\,\mathrm{\mu m}$. The target is a $25\,\mathrm{\mu m}$ thick kapton tape with etched holes reduced to $5\,\mathrm{\mu m}$ thickness. The thinned out parts of the tape are coated with $100\,\mathrm{nm}$ thick germanium (Ge) dots\footnote{Target and fabrication techniques were developed by the Target Fabrication Division at the Central Laser Facility, in particular Sam Astbury and Chris Spindloe.}. When the germanium is heated, it produces a plasma plume that emits X-rays with a distinct peak at $1.5\,\mathrm{keV}$ due to an L-shell emission (REF). The motion of the tape drive is motorised\footnote{The tape drive was designed and built by Brendan Kettle (Imperial College).} and a fresh Ge target is provided on each shot. 

The X-ray source is diagnosed by a pinhole camera measuring the flux and the source size. A camera attached to a crystal measures the spectrum of the X-rays in a $\sim 700\,\mathrm{eV}$ window from just under $1.3\,\mathrm{keV}$ to $2\,\mathrm{keV}$ including the L-shell emission of Ge at $1.5\,\mathrm{keV}$. Both cameras are back-illuminated deep-depletion X-ray in-vacuum CCD cameras (Andor DX420-BR-DD).
The X-ray source is the first component for the two-photon interaction. 

\subsubsection{South beam, electrons and gamma source}

The South laser is focused down by a $6\,\mathrm{m}$ focal length f/40 OAP onto the edge of a $17.5\,\mathrm{mm}$ long gas cell target\footnote{Variable gas cell from 1 to 20 mm, designed and built by Nelson Lopes (Imperial College/IST).} filled with helium and 2 percent nitrogen at an average electron density of XX NUMBER. The typical average spot size is $44\,\mathrm{\mu m} \times 53\,\mathrm{\mu m}$ at $5.51\pm 0.64\,\mathrm{J}$ energy on target at a pulse duration of $45 \pm 5\,\mathrm{fs}$, peak intensity $I_0 = 2.75 \times 10^{18} \,\mathrm{W/cm^3}$ corresponding to a normalised vector potential $a_0 = 1.13$. The energy on target was limited by the damage threshold of the mirrors in the focusing beam.
The laser drives a wakefield and accelerates electrons, injected predominantly via ionisation injection \cite{McGuffey2010_ION,Pak2010_ION}, to relativistic energies. The remaining laser exiting the cell is disposed of by another replenishable kapton tape that acts as a plasma mirror \cite{Kapteyn1991_PM}.
\vspace{\baselineskip}

The electrons are dispersed vertically by a permanent dipole magnet of integrated field strength $\int B dx = 0.4\,\mathrm{Tm}$ onto a scintillating LANEX screen measuring their energy and charge. The yoke of the magnet is blocking the path of dispersed lower energy electrons and as a result the outgoing spectrum has a low energy cut-off at about $300\,\mathrm{MeV}$. The screen is imaged by an Andor Neo camera equipped with an objective and a bandpass filter transmitting $546 \pm 10\,\mathrm{nm}$. A motorised stage with high-Z foils of various thicknesses can be driven into the path of the electrons and the foils act as bremsstrahlung converters (REF). This interaction produces copious amounts of directed bremsstrahlung in propagation direction and stops most of the electrons in the process (REF). Divergent gamma rays are likely to collide with components and apertures along the beam axis producing leptons with comparable properties as the BW pairs. A tungsten collimator of length $100\,\mathrm{mm}$, inner diameter $2\,\mathrm{mm}$ and outer diameter $20\,\mathrm{mm}$ is employed to reduce the emitted gamma-ray cone to its central part at low divergence in order to reduce noise. The exit of the collimator is placed at $280\,\mathrm{mm}$ distance from the converter target and apertures the beam down to a field of view of $2\,\mathrm{mm}/0.28\,\mathrm{m} = 7.14\,\mathrm{mrad}$. In addition, a thick block of tungsten obstructs the direct line of sight from the converter target to the Ge target drive, bisecting the signal to avoid gamma rays the kapton tape and producing pairs in the process. Only a collimated bright `half moon' of gamma rays from bremsstrahlung is incident on the interaction point, providing the second component for the two-photon interaction.
\vspace{\baselineskip}

The gamma-ray burst propagates almost unperturbed by the interaction through the aperture of a lead wall, which shields noise generated around the interaction point, and then through a large aperture dipole magnet. It then passes through the central one out of three $125\,\mathrm{\mu m}$ thick kapton windows at the end of the chamber into air. The flange holds three apertures spanned with kapton, with a central $80\,\mathrm{mm}$ diameter circular aperture and two slits of dimensions $150\,\mathrm{mm}\,\times\, 35\,\mathrm{mm}$ flanking it on either side. 

At air, a stack of $20 \times 20$ caesium-iodide (CsI) crystals of dimensions $2\,\mathrm{mm}\,\times\,2\mathrm{mm}\,\times\,20\,\mathrm{mm}$, each individually wrapped in aluminium foil ($\sim 15-20\,\mathrm{\mu m}$) and held together in a $1\,\mathrm{cm}$ thick aluminium casing, measures the transverse profile and yield of the gamma-ray signal. The total transverse area is $40\,\mathrm{mm}\times 40\,\mathrm{mm}$ which corresponds to an acceptance angle of $11.8\,\mathrm{mrad}$ based on a distance of $3.39\,\mathrm{m}$ measured from the converter targets to the profile screen. 
Another larger stack of $33 \times 47$ CsI crystals doped with thallium each of dimensions $5\,\mathrm{mm}\, \times\, 5\,\mathrm{mm}\,\times\,50\,\mathrm{mm}$ and spaced by $1\,\mathrm{mm}$ aluminium spacers measures the decay of the signal in propagation direction to deduce the spectrum. 
Both CsI arrays are imaged by sensitive Andor iXon cameras equipped with suitable objectives and bandpass filters.
The arrays are described in more detail in the \nameref{Chap:Methods}.
\vspace{\baselineskip}

\subsubsection{Magnetic Chicane and Single Particle Detectors}

Potential electron-positron pairs from the photon-photon interaction are emitted preferably in the propagation direction of the gamma-ray burst. The pairs enter the field of the permanent large aperture ($\sim 10\,\mathrm{cm}$ gap) dipole magnet with integrated field $\int B (x) \mathrm{d}x =0.35\,\mathrm{Tm}$ that disperses the electrons and positrons horizontally in opposite directions and sends them through the two kapton slits at the end of the target chamber into air. The yoke of the magnet produces a low energy cut-off at about 220 MeV CHECK NUMBER ESPECIALLY BECAUSE OF WINDOW AND UPPER CUT-OFF which is a slightly larger acceptance range than the vacuum window would provide. A lead wall further cuts down the spectrum to a 500-600 MeV window XX CHECK. The dispersed electrons and positrons XX WHICH ENERGY RANGE XX are then caught by an oppositely polarised permanent magnet\footnote{The design of the magnets and the chicane as well as the supervision of the magnet assembly and positioning was performed by D. Hollatz (Jena).} of field strength $B\sim 0.5\,\mathrm{T}$ on each side that bends the electrons or positrons, respectively, onto a narrow aperture of a lead-shielded enclosure and onto a set of single-particle detectors that are housed within. On each arm a CsI array attached to a sensitive CCD camera (PICO) acts as as a single-particle detector\footnote{Developed by Jena.} with additional two TimePix silicon detectors\footnote{Provided by the Medipix Collaboration, CERN} on the positron side.






\section{Characterisation of Bremsstrahlung Converter Targets}


The gamma signal depends strongly on the properties of the electron beam in terms of energy, brightness, divergence and pulse duration, but is also shaped by the choice of the converter material.
In this case, the aim is the efficient conversion of the electron beam into a narrow cone of gamma rays whilst also maintaining a low level of noise from secondary radiation and particles. To reduce the noise level several tools are used to cut down specific components of the gamma signal, particularly of interest is the collimator that was mentioned in the overview of the experiment setup.

The $10\,\mathrm{mm}$ long tungsten collimator efficiently blocks even highly energetic radiation. Its aperture permits a field of view of $7.14 \pm 0.1 \,\mathrm{mrad}$ full angle.

As indicator for gamma yield and divergence we will rely on the gamma profile monitor. An example of a gamma signal with and without collimator is shown in Figure \ref{BW:figs:Gprofile_collimator_INOUT}. The circular shadow seen on the profile monitor matches the predicted field of view (GIVE NUMBER HERE AS COMPARISON).


\begin{figure}
\centering
\includegraphics[width=.5\columnwidth]{2018QED_GProfile_CollimatorIN_example.png}\includegraphics[width=.5\columnwidth]{2018QED_GProfile_CollimatorOUT_example.png}
\caption{Examples of gamma profile with converter in and out for divergence field of view. Replace this with real plots.}
\label{BW:figs:Gprofile_collimator_INOUT}
\end{figure}

Despite its low divergence the collimated gamma will be around $6.2\pm 0.1\,\mathrm{mm}$ wide at the interaction point, about $865\pm 4.5\,\mathrm{mm}$ downstream from the converter target. The X-ray source is much closer to the interaction point and will be smaller at around XX MM NUMBER.
The spatial overlap of both photon sources is hence guaranteed on each shot. The temporal overlap is also secured as the North beam is a $40\,\mathrm{ps}$ long heater and timing fluctuations at Gemini have been measured to be of order of 10's of $\mathrm{fs}$ REF HERE.

As a result the aim is to now optimise the yield of gamma photons to interact with the ambient X-ray bath.

\subsubsection{Gamma yield for different materials and electron regimes}

As choices of materials we decided on tungsten (W) and bismuth (Bi) as they both are high-Z materials, which means energy can be efficiently converted to radiation within thinner targets resulting in less scattering and divergence. Both materials are also readily available and easy to handle. 
The aim is to find the ideal combination where most or all of the electron energy is transferred into radiation at a low divergence and high yield.

More specifically, $2\,\mathrm{mm}$ and $4\,\mathrm{mm}$ tungsten, $0.5$ and $1\,\mathrm{mm}$ bismuth were mounted on $0.5\,\mathrm{mm}$ plastic (PTFE), all on a motorised linear stage that allows switching between the materials or removing the converter targets from the beam path completely.
\vspace{\baselineskip}

The quantities we measure are the total counts on the gamma profile monitor (yield) and the width of the gamma signal in terms of the \textsc{fwhm} in the vertical and horizontal axis. This is being investigated for all 4 converter targets in two different electron regimes (one higher energy less charge from lower density plasma, the other higher charge but lower energy from a high density plasma). 10 shots were taken at each configuration (backing pressure A, converter 1...).

The results can be seen in Figure \ref{BW:fig:converter_yield_v_div}.
\vspace{\baselineskip}

From the results we can see a couple of trends. The divergence and yield for each material is larger when the solid target is thicker, i.e. the measured gamma yield is higher for $1\,\mathrm{mm}$ of bismuth than for $0.5\,\mathrm{mm}$, and higher for $4 \,\mathrm{mm}$ than for $2\,\mathrm{mm} $of tungsten, and similarly so with respect to the divergence.
More material leads to more scattering and an increase in divergence. At the same time it indicates that more energy is converted into photons when using thicker targets.

We see that the electrons produced in a higher density plasma (higher charge, lower energy) provide a lower yield of gamma rays at higher divergences. This also appears to be sensible as the initial divergence of the electrons is higher. The yield scales quadratically with the electron energy whilst the photon number relies only linearly on the electron charge. A higher energy electron beam, even at slightly lower charge, can hence provide a higher yield.

\begin{equation}
Bremstrahlung \sim Z^2 N(E) E^2
\end{equation}

Comparing the bismuth and the tungsten targets we see that the divergence for bismuth is in general lower than for tungsten. This is probably related to the thinner target. Bismuth has a higher Z-number and can provide a similar conversion efficiency at lower scattering rates.
We also observe that the yield does not increase much more from bismuth $1\,\mathrm{mm}$ to any of the tungsten targets, whilst preserving a $\sim\,10\%$ lower divergence. This indicates that most of the electrons already transferred all or a large fraction of their energy into gamma radiation and more material mainly leads to more scattering. The yield of bismuth, on the other hand, increases by around 50 percent from $0.5$ to $1\,\mathrm{mm}$ thickness. In some instances for $0.5\,\mathrm{mm}$, a faint electron signal is observed on the spectrometer screens. This confirms that somewhere around those thicknesses most of the electron energy is converted to radiation, reaching saturation.


Another result can be seen in Figure \ref{BW:figs:converter_pointing}. Here the pointing stability of the gamma signal is shown. There is some scatter but most of the targets operate at a comparable level and do not show a clear trend. This makes sense, as the general pointing mainly depends on the electron pointing and should not depend on the material.

\begin{figure}
\centering
\includegraphics[trim={4.9cm 0 5cm 0}, clip, width=.9\columnwidth]{2018QED_Converter_YieldVDiv.png}
\caption{Gamma Profile Yield and divergence. Maybe switch to a combined plot for x and y axis.}
\label{BW:fig:converter_yield_v_div}
\end{figure}


\begin{figure}
\centering
\includegraphics[trim={4.9cm 0 5cm 0}, clip, width=.5\columnwidth]{2018QED_GConverter_PointingMM.png}\includegraphics[trim={5cm 0 5cm 0}, clip, width=.5\columnwidth]{2018QED_GConverter_PointingSigma.png}
\caption{Left: Centroid position on detector in mm, coordinate zero point is the bottom left corner of the detector. Right: Standard deviation of pointing fluctuation from mean value in mrad. This looks a bit empty. Replace this by a bar chart?}
\label{BW:figs:converter_pointing}
\end{figure}


\subsubsection{Throughput Collimator for Yield}

Considering the collimator and the measured divergences of the gamma rays, we can now estimate what fraction of the gamma signal that will make it through the collimator. A more divergent beam will lose more intensity but if it comes with higher yield it might still prevail.

We will use the measured \textsc{fwhm} divergence values and model the intensity profile after a Gaussian. We will also assume perfect alignment and ignore pointing for now as it is in the same ballpark for the converters and should not depend on the material.

Bismuth at $0.5\,\mathrm{mm}$ has a lower divergence and most of its energy makes it through in terms of fraction based on the initial input, followed by 1 mm bismuth. If weighed by the total flux of photons related to that converter, the higher yield of the thicker bismuth target increases the total flux. The 2 mm thick W target provides a similar throughput (in absolute numbers) as the thin bismuth target. The thicker W target reaches close to similar levels as 1 mm of bismuth but not fully.

Based on these results 1 mm bismuth appears to be the best suited candidate for this endeavour.

\begin{figure}
\centering
\includegraphics[trim={4.9cm 0 5cm 0}, clip, width=.9\columnwidth]{2018QED_AbsTransmission.png}
\caption{Bar chart transmission through collimator. Maybe add plasma to labels. Change y label to something more concise.}
\label{BW:fig:abs_transmission}
\end{figure}


\subsubsection{Noise Considerations}

When inserting the W collimator and thick W block to reduce the noise levels, the number of gamma rays that make it to the detectors is decreased significantly. This makes it more difficult to characterise how the noise at the single particle detectors (SPDs) scales as the range is limited.

Using the same setup, we are now looking at noise events reaching the SPDs, in this case only one of the TimePix detectors\footnote{Method and numbers provided by B. Kettle looking at the TimePix and finding events that look like particles, will be called clusters.}, and how it correlates with the divergence and yield of the gamma-ray signal. Our hypothesis is that a more divergent gamma-ray beam will interact with more components and apertures in the beam path, generating radiation and particles that are closely related to the BW signal and will because of this resemblance be detected by the SPDs.
\vspace{\baselineskip}

Figure \ref{BW:figs:converter_noise_div_yield} shows a plot of the total counts on the gamma detector (yield) against the number of registered particle events or `clusters' on the TimePix detector, and how the divergence is related to those noise events.

Linear trend of yield and noise on TimePix.

W data points lie on one straight line. Bi events are parallel or lower gradient, in general less noise for comparable yields.

For each material a thicker target results in more divergence. More divergence results in more noise. This confirms the need of the collimator, W block and so on and confirms the ideas behind the setup.

Looking at yield over the number of clusters, the bismuth targets perform the best when looking at fraction of flux transmitted through the collimator.
A thinner target produces less noise and bismuth provides a decent yield.

When scaling the y-axis to the absolute flux transmitted we see again that 1 mm bismuth performs similarly well as 4 mm of tungsten in terms of transmitted flux. However, we see that the ratio of yield to noise is better for bismuth due to the lower divergence.

The collimator will cut out the more divergent parts of the beam, so the noise levels will be comparable for the materials.

Looking at Figure \ref{BW:fig:Yield_with_Coll} we see that introducing the collimator cuts down the gamma yield and subsequently also the number of noise events registered on the TimePix detector. When comparing the trend of these events with the linear relations derived from the converter analysis, we see that the yield per noise further improved as we are decreasing the divergence dramatically.


\EliasComm{Need appropriate legend for all data points or combine them.}


\begin{figure}
\centering
\includegraphics[trim={4.9cm 0 5cm 0}, clip, width=.5\columnwidth]{2018QED_Converter_YieldvNoise_fits.png}\includegraphics[trim={5cm 0 5cm 0}, clip, width=.5\columnwidth]{2018QED_Converter_DivvNoise.png}
\caption{Left: Gamma Spec yield against particle-like cluster events on the TimePix detector. Right: Divergence of the gamma ray beam against noise events on the TimePix detector. Maybe fit each dataset differently or fit materials together?}
\label{BW:figs:converter_noise_div_yield}
\end{figure}

\begin{figure}
\centering
\includegraphics[trim={4.9cm 0 5cm 0}, clip, width=.5\columnwidth]{2018QED_SignalNoise_Transmission.png}\includegraphics[trim={5cm 0 5cm 0}, clip, width=.5\columnwidth]{2018QED_SignalNoise_AbsTransmission.png}
\caption{Left: yield over noise against estimated fraction of transmitted flux through the collimator aperture. Right: Yield over noise against estimated throughout of flux through the collimator.}
\end{figure}


\begin{figure}
\centering
\includegraphics[trim={4.9cm 0 5cm 0}, clip, width=.9\columnwidth]{2018QED_YieldvNoise_CollIN.png}
\caption{Noise after inserting collimator.}
\label{BW:fig:Yield_with_Coll}
\end{figure}

\subsubsection{Summary and Decision on Material}

Best flux throughput for 1 mm bismuth. Noise is also lower although the levels will change for collimator and W block conditions.
1 mm bismuth also removes all electrons and potentially reduces noise from inside the collimator (dispersed remaining electrons).

Electron regime intermediate (low density but backing pressure had to be increased due to leak). Higher energy is more preferable than a bit more charge and higher initial divergence.

In theory, even thinner target and higher Z would be favourable (as long as all electrons are converted) but Bi is the readily available material on the periodic table.

The trends confirm that we need the collimator and W block. We also see that the yield over noise improves.


\section{Characterisation of Electron Spectra for BW shot days}

\subsubsection{Application of electrons in this context and need for characterisation}

In this experiment the energy of relativistic electrons from LWFA is converted into directed and highly energetic bremsstrahlung.
Whenever the bremsstrahlung is produced using a converter target, the spectrum of the electrons can not be measured simultaneously as they are almost entirely stopped by the high-Z material in order to produce a high flux of gamma rays. 

Hence, the electron spectra are characterised without the converter target before and after full BW data runs instead. The gamma signal, on the other hand, is measured during the BW data runs.
\vspace{\baselineskip}

The relevant shot days were the 6th, 9th and 10th May, hence the analysis of the electron spectra and their characterisation will focus particularly on these days.

\begin{figure}
\centering
\includegraphics[width=.9\columnwidth]{2018QED_ElecSpecs_montage.jpg}
\caption{Examples of raw electron spectra acquired on the experiment. The spectra are on a linear spatial scale, where the top is closer to the beam axis, i.e. indicates higher electron energies. Replace this by one row of raw and processed images or maybe even just one row of example spectra on a linear energy scale. Or just one spectrum with linear energy scale plus lineout in both dimensions to show where quantities come from.}
\label{BW:figs:elec_raw}
\end{figure}

\subsubsection{Raw images and general note on processing data}

Raw images of typical electron spectra measured on the LANEX screen are shown in Figure \ref{BW:figs:elec_raw}.
The images are treated for background and image distortions caused by the relative angle of camera and screen. By measuring the field components of the magnet and the distances accurately the spatial scale on the LANEX screen is related to the energy of the electrons. The captured scintillation light is cross-calibrated with image plate measurements to obtain an absolute number for charge of the beam. A more detailed description of the procedure is found in the Methods section.
\vspace{\baselineskip}

\subsubsection{Investigation of main quantities and variation on the days}



\begin{figure}
%\includegraphics[width=.5\columnwidth]{2018QED_ElecSpecs.png}
\centering
\includegraphics[trim={2cm 0 5cm 0}, clip, width=.9\columnwidth]{2018QED_ElecSpecs_average}
\caption{Average spectra for the relevant shot days, 6th (blue), 9th (purple) and 10th (green). The spectra are normalised to charge and are not representative (SHOULD MAKE THIS NOT NORMALISED ACTUALLY) of relative differences. The shaded areas indicate 1 standard deviation from the average spectrum.}
\label{BW:figs:elec_average_spec}
\end{figure}

\begin{figure}
\centering
\includegraphics[trim={7.5cm 0 6.5cm 0}, clip, width=1.0\columnwidth]{2018QED_Espec_Variations.png}
\caption{Average electron charge, energy, divergence and pointing on the relevant shot days. The error bars indicate 1 standard deviation from the average value.}
\label{BW:figs:elec_variations}
\end{figure}


The performance of the wakefield accelerator can vary from day to day and is hence characterised separately for each of the relevant shot day, in some cases several datasets were taken on one day.

The key quantities of interest to characterise the electron beam with and to then compare the shot days with each other are charge, energy, divergence, pointing and the stability of those properties from shot to shot.
\vspace{\baselineskip}

An overview of these quantities describing the accelerator performance for the relevant days are presented in Table \ref{BW:tables:elec_days} and again in a more graphic comparison in Figure \ref{BW:figs:elec_variations}. The lineouts for the average electron spectra for those days are shown in Figure \ref{BW:figs:elec_average_spec}.
\vspace{\baselineskip}




\begin{table}
\centering
\begin{tabular}{r|r|r|r|r|c|r}
Date & Run & Charge [pC] & $E_{max}$ [MeV] & $\theta$ [mrad] & $\sigma_X$ [mrad] & $N_{shots}$\\ \hline \hline
%5 & 2 & $20 \pm 10.8$ & $599 \pm 57$ & $3.14 \pm 0.75$ & 1.25 & 32\\ \hline
6 & 1 & $26.26 \pm 3.8$ & $709 \pm 46$ & $2.26 \pm 0.29$ & 0.62 & 10\\ \hline
9 & 1 & $14.73 \pm 5.5$ & $565 \pm 43$ & $2.5 \pm 0.72$ & 0.93 & 18\\ 
9 & 4 & $7.7 \pm 4.4$ & $551 \pm 16$ & $1.7\pm 0.31$ & 0.81 & 12\\ \hline
10 & 1 & $11.55 \pm 2.7$ & $511 \pm 19$ & $2.84 \pm 0.93$ & 0.90 & 22\\ 
10 & 2 & $15.24 \pm 5.1$ & $512 \pm 26$ & $2.3 \pm 0.52$ & 0.50 & 6\\ 
10 & 3 & $9.56 \pm 3.5$ & $535 \pm 21$ & $2.82 \pm 0.91$ & 1.55 & 21
\end{tabular}
\caption{Details on runs. Electron properties. Divergence $\theta$, Pointing $\sigma_X$.}
\label{BW:tables:elec_days}
\end{table}



In general, 

... large energy spread as typical for ionisation injection at Gemini, no quasi-mono

... lower energy than typically (related to energy on-target)

... performance comparable 9th and 10th with plateau up to 550 MeV or so

... on the 6th the signal extends to 700 MeV and quite stable.


Looking at the key quantities:

... charge was higher on the 6th at XX 25 pC pm.

... energy as mentioned

... divergence similar levels but more stable over the dataset and at the lower end

... similar properties for the pointing





\section{Characterisation of the Gamma Signal}

\subsubsection{General bits on the diagnostic, signal extraction and characterisation}
\EliasComm{Add some raw data and examples from analysis.}

\EliasComm{Add a plot showing the raw data from GammaSpec for different converters.}

\EliasComm{This section will be about the gamma yield from the profile and fluctuations from it, but also about the gamma spectrometer.}

The detector used to measure the spectrum of this gamma flux is a stack of scintillating crystals as described in Methods imaged by an Andor iXon camera.
RAL stack with spectral measurement and plastic, gamma profile in the path. 
\vspace{\baselineskip}

Signal extraction. Show an example.
Selecting the crystals and get a pixelated response.

\subsubsection{Yield of gamma spectrometer and profile on the different days}

Show yield and variations of this per day.

Show correlation of gamma profile and spectrometer yield, meaning we can use either as indicator for yield.

Look at yield with and without block (is the correlation to the electron charge more pronounced?).


\subsubsection{Detector response in experiment}

Look at the detector response in general (without corrections)

Get a better background subtraction done to avoid a fluctuation low signal tail.

Maybe make a plot of average spectra. Do they have a slight change in peak?

It seems that despite the change in spectra, the response is not changing as much.

This means the detector is not suitable to discriminate a spectrum to such detail and also is not suitable to deduce a gamma spectrum independent of assuming its shape. Luckily bremsstrahlung is well understood. It is essential to have many reference electron shots. Also a lesson for the future to take data in between.

\subsubsection{GEANT simulation and correction factors using bremsstrahlung}

Simulating response of detectors using GEANT.
Monoenergetic photons in a different range and present the response curves.
\vspace{\baselineskip}

Correcting using brems.
Get a brems run using a thick converter and simulate the response based on that.
Correct out the crystal efficiencies and viewing angle. Reduce this section maybe based on Methods.
\vspace{\baselineskip}

Correlating Gamma profile and spec counts. No change in response for those detectors.
Seeing that this is very linear the results from gamma profile assessment are still or again valid.
This is not useful to distinguish energies.

\subsubsection{Characterisation of fluctuations and lack of change in response}

The produced spectrum can be simulated using the electron spectra characterised earlier. GEANT. Typical gamma spectra produced are shown in FIGURE XX. 
Show that the responses for all electron spectra is fairly similar in simulations and the resulting spectra vary mostly at a low photon level and high energies.
The response of the detector only results in small variations.
\vspace{\baselineskip}

The experimental results also show only small variations which is consistent with simulations. The systematic offset is including the viewing angle, efficiency of crystals and so on.
Does the flux vary as much as the electron charge? (for the shots with W block, collimator and converter in the correlation is washed out and there is no clear indication that these things are linked). Add some plots for that as well.
Stability of gamma flux?


\begin{figure}
\centering
\includegraphics[width=.5\columnwidth]{2018QED_ElecSpecs_examples2.png}\includegraphics[width=.5\columnwidth]{2018QED_GammaSpec_simspec2.png}
\caption{Left: Representative extreme examples of electron spectra from one day. Right: Resulting gamma spectrum based on GEANT simulations. Maybe show an average with shaded error bars instead. Add a third electron spectrum and gamma result for the other extreme.}
\end{figure}

Add the differences from day to day somewhere as well as comparison.

\begin{figure}
\centering
\includegraphics[width=.5\columnwidth]{2018QED_ElecSpecs.png}\includegraphics[width=.5\columnwidth]{2018QED_GammaSpec_simresp.png}

\includegraphics[width=.5\columnwidth]{2018QED_GammaSpec_expresp.png}\includegraphics[width=.5\columnwidth]{2018QED_GammaSpec_Average_expsim.png}

\caption{Left: Lineouts for most electron spectra. Right: Simulated detector responses. BLeft: Experimental responses. BRight: Comparison average responses.}
\end{figure}


What is the statistical variation of the signal on each day. Use the electron spectra to put a number onto this.
Use plot of relative and absolute number of photons above a threshold. Sum them up and also sum up number times energy as indicator for high energy photons.

Provide a plot with numbers on fluctuation of number of photons to expect in total and in particular above a threshold and how that varies shot-to-shot and day by day.

Provide some plots from the GEANT simulation (visualisation).

\begin{figure}
\centering
\includegraphics[trim={7.5cm 0 6.5cm 0}, clip, width=1.0\columnwidth]{2018QED_GSpec_Variations.png}

\includegraphics[width = 0.9\columnwidth]{2018QED_GEANT_AvGSpec.png}
\caption{Variations of gamma signal (based on GEANT simulations). Average gamma spectrum (not scaled by yield). Make this one 4 plot panel.}
\end{figure}


\subsubsection{General critical energy fitting for the spectra}

Using the results from the correction curve, now try to use a fitting routine based on brems spectra to give some kind of critical energy for the spectra.

See whether there is a change in energy over the days. Based on the qualitative measurements the fitting parameter should only change marginally over the days.

\section{Impact of Spectral Fluctuations on the total BW cross section}

\subsubsection{Simple calculation on impact on the total cross section based on the GEANT spectra}

Relate measured quantities to the cross section for the BW process.

Based on the measured X-ray spectrum (measured and simulation), estimated fluctuation of the gamma spectrum and the known remaining experimental conditions, by how much does the cross section vary on a daily basis (including charge, flux etc.).

\begin{figure}
\centering
\includegraphics[width=.5\columnwidth]{BWxsection2.png}\includegraphics[width=.5\columnwidth]{BWxsection_2D.png}
\caption{BW cross section at a fixed and variable X-ray photon energy.}
\end{figure}

\begin{figure}
\centering
\includegraphics[trim={4.9cm 0 5cm 0}, clip, width=1.0\columnwidth]{BW_Xsec_NxNg.png}
\caption{BW cross section for experimental gamma and X-ray spectra with photon numbers.}
\end{figure}


\begin{figure}
\centering
\includegraphics[width=.5\columnwidth]{2018QED_Xsec_VariationAvDay.png}\includegraphics[width=.5\columnwidth]{2018QED_Xsec_VariationMorton.png}
\caption{Day to day fluctuation of relative total cross section: Left using the measured X-ray spectra (average per day) with limited energy window. Right: using Morton simulations over a larger energy window. Add data points for different relative photon-photon angle.}
\end{figure}

The number of pairs produced from BW scales something like this

\begin{equation}
N_{BW, day}|_{theta} \sim \int N_{ph, X} (E_{X}) N_{ph, \gamma} (E_{\gamma}) \sigma_{BW} (E_{X}, E_{\gamma}, \theta) \mathrm{d}E_{\gamma}\mathrm{d}E_{X}
\end{equation}

The main angle between the two sources is about 40 degrees (XX ERROR ON THIS?). However, in general we have to fold in the angular distribution of both photon sources. The collimator has a field of view of $7.14 \pm 0.1 \,\mathrm{mrad}$ or $\pm 0.2^\circ$ from the main axis. The main contribution of angular spreads comes from the X-ray source.

The closer the angle is to a head-on collision the more the peak of the cross-section shifts towards lower gamma energies. 
If the X-ray spectrum is mainly centred around 1.5 keV and lower energies are suppressed the additional 800 MeV does not have a significant impact.
If the full X-ray spectrum (for instance Morton simulations) are included then a large pool of X-rays are available at lower energies to be paired up with the high energy photons of 600-800 MeV. We see the effect of this in Figure XX showing the fluctuation of the number of expected pairs on the days based on different angles. The differences flatten out.

The X-ray yield was higher on the XXth whilst the gamma yield and energy was higher on the 6th. These fluctuations balance each other out and the collisional geometry further mitigates the fluctuation.


\section{Future Outlook: A gamma spectrometer to discriminate high energy photons}

Here some ideas and simulations on how to improve the gamma spectrometer design to determine the spectrum more accurately.

\subsubsection{Explanation why detector is not sensitive to the changes seen in the simulated spectra}


Show that the problem is the following:
above 10's of MeV the cross section for pair production increases and is relatively flat over long time. What happens is that gammas convert into electrons which convert into gammas in showers and so on. In most cases a high number of low few MeV gamma rays is present and the main indicator for the initial energy is how many times these showers continue to proceed and where they deposit energy. Most spectra convert to an almost exponential shower of particles and radiation, which means that most of the deposited energy from different energy gamma rays will be deposited through similar radiation. Resolving a low number of higher energy photons will be difficult as large fraction of the energy deposited will be identical to other energies.
\vspace{\baselineskip}

Show cross sections.

Show investigations in GEANT:

- looked at the experimental setup in GEANT and measured for different input energies at different parts of the spectrometer the radiation and particle spectra

Compton scattering also does not work anymore due to sensitivities (lower energies see \cite{Corvan2014_Gamma}).

\begin{figure}
\centering
\includegraphics[width=.9\columnwidth]{CsI_MassAttenuation.png}
\caption{Mass attenuation coefficient for CsI shows it becomes quite flat at hundreds of MeV just producing pairs. This is from a presentation and I would have to plot this myself probably again. All the data is on the NIST though.}
\end{figure}


\subsubsection{Present Ideas on how to improve the detector}

Ideas: 

- insert different materials at various positions (one part of the spectrometer measures the main part of the spectrum, a part further in only shows energies above a threshold?)

- use magnets either as pair spectrometers (see Gianluca's design) or ...

- ... to `clean up' the signal by removing generated pairs from the system (requires a lot of space)

- should stay compact
\vspace{\baselineskip}

First results did not look promising. The low number of high energy particles etc. is really swamped by other signals.
To really make a difference one would have to kick out low energy particles several times. From experience I can not just put one big magnet around the spectrometer as the magnetic field will affect the scintillators.
Using Gianluca's pair spectrometer is also not feasible as for a compact version of this design the higher energies will not be deflected sufficiently and the number of particles is very low, will be swamped by other signal.

Use gamma profile as converter or even the vacuum window (less divergence) and then disperse. Having the pair measurements and the array response will constrain the spectrum. 


\begin{figure}
\centering
\includegraphics[width=.9\columnwidth]{CombinedSpecSketch.png}
\caption{Combined design.}
\end{figure}

\subsubsection{Point out how to mitigate fluctuations if not measurable.}

A comment about designing your experiment such that parts of the X-ray spectrum we can not measure are being blocked by material such that the gamma components we can not resolve are not impacting on the cross section. The angle is also a handle on equalising the cross section across the window of gamma energies such that few high energy photons still access the same pool of photons and the cross section is low.

Flaw of wakefield accelerator are the intrinsic fluctuations of the electron beam. Better stability by improving the e-beam but for instance by using bremsstrahlung we already cut out some of the uncertainty. See simulations that show that fluctuations in the spectral shape are not impacting the overall gamma spectrum much. 
If we now tailor the X-rays accordingly and choose the angle of the collision appropriately we can mitigate the impact of these fluctuations (maybe at a marginal cost of produced positrons) and have more stable conditions at the experiment.


\section{Results of the Breit-Wheeler Analysis}


\subsubsection{Discussion of how this work fits in the big picture of the complete BW analysis}
\EliasComm{Here a brief discussion of the findings of the complete BW analysis and if there was any sign of BW, reference publications that came out of this and so on.}

How extensive?

Described most of the setup and the different diagnostics.

Maybe:

... reiterate the setup and the different components.

... emphasise that we have seen from the analysis that the gamma and X-ray sources were working...

... and that the diagnostics were working as well.

... explain that the single particle diagnostics were behaving in an expected way and we investigated their behaviour using

... the tungsten blob shots

... the magnet chicane (tracking)

... noise behaviour with different blocks in.

... mention that we see a number of events on each shot based on Bethe-Heitler...

... and there are some events above that expected background

... these events are particularly visible on the 6th where the gamma conditions and the alignment was working properly.



\section{Conclusion}

LWFA dual laser setup to produce BW pairs. Some indication that we saw pairs consistent with the experiment conditions at the optimum.
\vspace{\baselineskip}

Stability and robustness of gamma/X-ray source to a certain degree.
\vspace{\baselineskip}

Limits of gamma spectrometer design and potential improvements.

An improved detector design presented in this chapter or alternatively show:

- that electrons from LWFA are not as stable in every scenario

- but it can be mitigated by using bremsstrahlung which varies only a bit depending on the spectral shape

- that it is very hard to measure small fluctuations in high energy radiation at the scale of 100s of MeV

- but that choosing the right collision geometry this effect can also be mitigated or by allowing only a small window of energies for the X-ray spectrum
