\chapter{Measurement of the Linear Breit-Wheeler Process using LWFA}

\section{Motivation}

The Breit-Wheeler process is the inverse process of pair annihilation (electron-positrons decaying into two photons) and during it electron-positron pairs are created from two photons. The simple tree-level diagram is the 2-photon process also called linear Breit-Wheeler.
Despite its simplicity the linear Breit-Wheeler process is the last tree-level diagram of QED that has not been observed in experiment so far.
The non-linear process, where more than two photons combine their energies to create an electron-positron pair, has been observed at SLAC in the 90s where n=XX NUMBER REF E144 and NUMBER pairs were measured over NUMBER shots.
\vspace{\baselineskip}

Whilst the pair annihilation process is frequently observed in radioactive decays and its cross section is very high at low energies (also an abundance of electrons are available to annihilate with), the pair creation process from radiation requires high energy photons to overcome the production threshold of twice the electron rest mass energy $2m_e c^2$ in a suitable frame. Two abundant sources of high energy photons at high intensities are very hard to come by and hence mainly mediated pair production processes (Trident, Bethe-Heitler) have been measured.
\vspace{\baselineskip}

High energy radiation in the keV range can be produced from highly energetic electrons, for instance at XFELs, synchrotron sources or by heating suitable plasmas. High energetic gamma radiation can be produced from converting electrons to bremsstrahlung or relying on inverse Compton scattering. The E144 experiment scattered a highly relativistic electron beam with a laser resulting in highly energetic gamma radiation which in turn was able to interact with multiple laser photons to generate electron-positron pairs from the nonlinear Breit-Wheeler process.
\vspace{\baselineskip}

In this experiment two lasers are used to provide an X-ray and a gamma ray source to probe the linear Breit-Wheeler process. Using an asymmetric distribution the resulting particle momentum is strongly directed. There are also other processes that produce positrons and they have to be separated out.
Proposed by REF was a laser heated hohlraum but here a thermal X-ray source was produced by heating a plasma with one laser and producing a relativistic electron beam from LWFA to convert into bremsstrahlung.
\vspace{\baselineskip}

The following chapter will outline the experiment setup aimed at measuring the linear Breit-Wheeler process.
The author will focus on the characterisation of the electron beam and the gamma ray source, in particular an analysis of different shot days to estimate the likelihood of measuring pairs. Since the analysis of the single particle detectors, X-ray diagnostics and so on was done by other colleagues these results might be referred to but will not be explained in too long detail. Refer to future publications and theses.
Very important discussions on signal-to-noise, noise sources including the pair production from other processes are also not included in this chapter but will in near time be found in related publications.

One section will show that the design of the gamma spectrometer is suitable to get a grasp of the overall energy of the radiation but, in particular in case of an exponential spectrum, is not sensitive to smaller number of high energy photons in the tail.
A final simulation section addresses this issue and proposes changes to the detector to pick up smaller changes.

\section{Experiment Setup}

The aim of this experimental setup is to produce and measure Breit-Wheeler pairs. For this purpose high energy gamma rays are collided with a thermal X-ray source. In this scenario the South arm of the Gemini laser provides the gamma source, the North beam the X-ray bath.
\vspace{\baselineskip}

The North beam in this experiment is not compressed to its usual 40-45 fs duration but is stretched to around 40 ps FWHM to effectively heat the solid target. The beam is focused down using an f/2 off-axis parabola (OAP) but a phase plate smoothens and enlargens the focal spot to several millimetres size providing a smooth and homogeneous spot at the target. This beam is then incident onto a tape drive with germanium (Ge) dots, producing a plasma that emits a thermal bath of X-rays at the interaction point. By moving the tape drive a new Ge target is provided on each shot. This is the first component for the two-photon interaction. The X-ray source is diagnosed by an X-ray pinhole camera measuring the flux and the source size, and by a second camera attached to a crystal measuring the spectrum of the X-rays on the spectral range of the line emission of Ge (around 1.5 keV).
\vspace{\baselineskip}

The South laser is focused down by a $6\,\mathrm{mm}$ focal length f/40 OAP onto the edge of a 17.5 mm length gas cell target filled with helium and 2 percent nitrogen. The typical average spot size was around XX NUMBERS at XX NUMBERS energy on target at a pulse duration of XX NUMBER FS, corresponding to a normalised vector potential $a_0$ of XX NUMBERS (just about 1). The energy was limited by the damage threshold of the mirrors in the focusing beam.
The laser drives a wakefield and accelerates electrons injected via ionisation injection REF HERE to relativistic energies. The remaining laser exiting the cell is disposed of by a replenishable kapton tape that acts as a plasma mirror.
\vspace{\baselineskip}

The electrons are dispersed vertically by a XX NUMBERS permanent dipole magnet onto a scintillating lanex screen to measure their energy and charge. A motorised stage with high-Z foils of various thickness can be driven into the path of the electrons and act as converter foils. This interaction produces copious amounts of directed bremsstrahlung in propagation direction and stops most of the electrons in the process. A tungsten collimator reduces the emitted gamma-ray burst to its central part which is close to collimated. This reduces noise generated further downstream by pre-emptively converting the excess of gamma rays into positrons (which are dispersed and dumped by the dipole magnet). A thick block of tungsten also blocks the direct line of sight to the Ge target drive. Only a collimated bright burst of gamma-rays from bremsstrahlung are incident on the second interaction point, providing the second component for the two-photon interaction.
\vspace{\baselineskip}

\begin{figure}
\includegraphics[width=.9\columnwidth]{BW2018_render_V3.png}
\caption{Sketch of the experiment setup.}
\end{figure}

\EliasComm{Add some photos of the setup to give an idea of how crowded the setup was. Include the magnet laser photo.}

The gamma ray burst propagates almost unperturbed by the interaction through a thin wide aperture kapton-kevlar ($375\,\mathrm{mu m}$ kevlar, $25\,\mathrm{\mu m}$ kapton) window of size $580\,\mathrm{mm} \times 70\,\mathrm{mm}$. At air a stack of 400 cesium-iodide crystals of dimensions $2\,\mathrm{mm}\,\times\,2\mathrm{mm}\,\times20\,\mathrm{mm}$ each individually wrapped in aluminium foil ($\sim 15-20\,\mathrm{mu m}$ in an $1\,\mathrm{cm}$ thick aluminium casing measures the transverse profile of the gamma-ray signal. The total transverse area is about $20\,\mathrm{mm}\times 20\,\mathrm{mm}$ which corresponds to an acceptance angle of XX NUMBER MRAD (USE DISTANCE FROM FOIL). 
Another larger stack of $XX \times XX$ CsI crystals measures the decay of the signal in propagation direction to deduce the spectrum. 
Both scintillator arrays are imaged by sensitive Andor iXon cameras equipped with suitable objectives.
\vspace{\baselineskip}

Potential electron-positron pairs from this interaction propagate preferably in the propagation direction of the gamma-ray burst. The pairs enter the field of a large aperture permanent dipole magnet of XX FIELD STRENGTH NUMBER that disperses the electrons and positrons horizontally in opposite directions, leaving the vacuum through the same wide kapton-kevlar window. The aperture of the magnet produces a low energy cutoff at about 220 MeV CHECK NUMBER which is a slightly larger acceptance range than the vacuum window would provide. The dispersed electrons and positrons are then caught by an oppositely polarised permanent magnet of FIELD STRENGTH XX NUMBER on each side that bends the electrons or positrons respectively onto a set of shielded single-particle detectors. The single-particle detectors are CsI arrays attached to sensitive CCD cameras and two layers of TimePix silicon detectors.
\vspace{\baselineskip}

This chapter will focus on the characterisation of the electron spectra and the gamma-ray signal. The single-particle detectors, the exact shielding setup and the characterisation of the X-ray source are hence not further elaborated in more detail. This information can be found in references REF FUTURE?

\section{Electron Spectra Measurement}

In this experiment the energy of relativistic electrons is converted into directed and highly energetic bremsstrahlung.
Whenever bremsstrahlung is produced from a converter target, the spectrum of the electrons can not be measured simultaneously as they are almost entirely stopped by the high-Z material in order to produce an as high as possible flux of gamma rays. The electron spectra, their charge and stability are characterised before and after data runs attempting BW production. The gamma signal is measured during the BW data runs and both are linked together.

The electron beam is dispersed by the a permanent dipole magnet onto a scintillating lanex screen. Electrons with energy below XX NUMBERS hit the yoke of the magnet and can hence not be measured on the lanex screen. The screen is imaged with an Andor Neo camera equipped with a TV lens and suitable filters to reduce ambient laser light.
\vspace{\baselineskip}

Raw images of typical electron signals are seen in FIGURE XX, lineouts of electron spectra on an energy axis can be seen in FIGURE XX.
The electrons produced with ionisation injection in this context have a large energy spread from continuous injection reaching maximum energies of around 800 MeV in best conditions. The performance of the accelerator varies from day to day but typically either an almost flat spectrum up to 800 MeV or a distinct peak at 400 MEV was produced (see figure). Depending on the day the average charge was 10's of pC which is relatively low for the Gemini laser but might be linked to the limited energy on-target that could be delivered. The response and the imaging system was cross-calibrated with an image plate providing an absolute calibration of the images.
\vspace{\baselineskip}

The performance of the accelerator varied from day to day and is characterised accordingly.
The measured quantities are charge, energy, divergence and pointing of the beams along with the stability of these quantities from shot to shot.

Main shot days were the 6th, 9th and 10th. Compare the charge, mean and max energy as well as the pointing.

Differences in energy and charge over the days and runs.
Differences in pointing and divergences.

How many shots. These shots were taken before data runs.

\begin{figure}
\includegraphics[width=.9\columnwidth]{2018QED_ElecSpecs_montage.jpg}
\caption{Montage of electron spectra (screen raw).}
\end{figure}

\begin{figure}
\includegraphics[width=.5\columnwidth]{2018QED_ElecSpecs.png}\includegraphics[width=.5\columnwidth]{2018QED_ElecSpecs.png}
\caption{Left: 2D example of spectrum. Right:  Lineouts for electron spectra (maybe waterfall?). Maybe average spectra for day with shaded errorbars.}
\end{figure}

\begin{figure}
\includegraphics[width=.9\columnwidth]{2018QED_Espec_Variations.png}
\caption{Electron charge and energy over shot days.}
\end{figure}

Main shot days were the 6th, 9th and 10th April. Compare the charge, mean and max energy as well as the pointing.

It turns out that the charge and stability in terms of pointing was the best on the 6th of April.
The energy was also significantly higher.

The results of the analysis are seen in figure XXXX with the relevant numbers in table XXXX.


\begin{table}
\centering
\begin{tabular}{l|l|l|l|l|l|l}
Date & Run & Charge & Max Energy & Divergence & Pointing & Nshots\\ \hline \hline
5 & 2 & $20 \pm 10.8$ & $599 \pm 57$ & $3.14 \pm 0.75$ & 1.25 & X\\ \hline
6 & 1 & $26.26 \pm 3.8$ & $709 \pm 46$ & $2.26 \pm 0.29$ & 0.62 & X\\ \hline
9 & 1 & $14.73 \pm 5.5$ & $565 \pm 43$ & $2.5 \pm 0.72$ & 0.93 & X\\ 
9 & 4 & $7.7 \pm 4.4$ & $551 \pm 16$ & $1.7\pm 0.31$ & 0.81 & X\\ \hline
10 & 1 & $11.55 \pm 2.7$ & $511 \pm 19$ & $2.84 \pm 0.93$ & 0.9 & X\\ 
10 & 2 & $15.24 \pm 5.1$ & $512 \pm 26$ & $2.3 \pm 0.52$ & 0.5 & X\\ 
10 & 3 & $9.56 \pm 3.5$ & $535 \pm 21$ & $2.82 \pm 0.91$ & 1.55 & X
\end{tabular}
\caption{Details on runs. Electron characteristica.}
\end{table}

The charge, energy, divergence and pointing varies from day to day and in some cases the performance even degraded over the course of the day.
Seeing in the figures it appears that the electrons on the 6th were in particular better in various dimensions:
The charge is close to doubled with respect to other days and relatively stable. The maximum energy is increased by 100-200 MeV than on other days.
At the same time a decent divergence at a stable level is paired with a good pointing stability of smaller than 1 mrad, which is not the optimum but a very good value in the subset of data taken.

The shape of the spectra (see average spectra) shows a distinct peak at few hundred MeV and a tail in some cases extending to 500-550 MeV on most days and in case of the 6th even up to 800 MeV.

\section{Gamma Profile Assessment}

Look at gamma profile and see that from day to day the alignment changes.

Show that the 6th had the ideal alignment based on this.

Show the pointing varies and the total alignment.

Show the gamma yield is varying from day to day but not as much as the electron charge which indicates that the alignment is washing out any correlation.

\section{Bremsstrahlung Characterisation - Spectral Measurement Gamma Rays}

\EliasComm{Add some raw data and examples from analysis.}

Electrons characterised in the previous section are converted to a burst of gamma rays by inserting a high-Z converter foil in the beam path, in this context mainly 1 mm of bismuth. This is to balance the need for a decent amount of gamma rays and a low divergence of the resulting beam. Thick targets result in the highest yield but the beam is more divergent and more signal will be cut out by the collimator etc. and more noise is to be expected.

\EliasComm{Add a plot showing the raw data from GammaSpec for different converters.}

The tungsten collimator only lets radiation through an aperture of XX NUMBER. In addition the thick tungsten block aims to half the signal to block any gamma rays from hitting the Ge target drive. The alignment of these components will change from day to day based on human error.
\vspace{\baselineskip}

The detector used to measure the spectrum of this gamma flux is a stack of scintillating crystals as described in Methods imaged by an Andor iXon camera.
RAL stack with spectral measurement and plastic, gamma profile in the path. 
\vspace{\baselineskip}

Signal extraction. Show an example.
Selecting the crystals and get a pixelated response.
\vspace{\baselineskip}

Simulating response of detectors using GEANT.
Monoenergetic photons in a different range and present the response curves.
\vspace{\baselineskip}

Correcting using brems.
Get a brems run using a thick converter and simulate the response based on that.
Correct out the crystal efficiencies and viewing angle. Reduce this section maybe based on Methods.
\vspace{\baselineskip}

Correlating Gamma profile and spec counts. No change in response for those detectors.
Seeing that this is very linear the results from gamma profile assessment are still or again valid.
This is not useful to distinguish energies.
\vspace{\baselineskip}

Seeing not really much change in experimental response.
Give a number on critical energy. Any idea of brightness (absolute number?).
The curves are fairly similar. Might see an overall change in the spectrum but everything within a crystal. Check if there is a difference between the days.
Absolute calibrations for gamma profile is possible but might be harder for gamma spectrometer due to the components in between.
\vspace{\baselineskip}

The response is not sensitive to such highly energetic radiation or the number of photons is much lower and hence the response is based on that.
This means the detector is not suitable to discriminate a spectrum to such detail and also is not suitable to deduce a gamma spectrum independent of assuming its shape. Luckily bremsstrahlung is well understood. It is essential to have many reference electron shots. Also a lesson for the future to take data in between.
\vspace{\baselineskip}

The produced spectrum can be simulated using the electron spectra characterised earlier. GEANT. Typical gamma spectra produced are shown in FIGURE XX. 
Show that the responses for all electron spectra is fairly similar in simulations and the resulting spectra vary mostly at a low photon level and high energies.
The response of the detector only results in small variations.
\vspace{\baselineskip}

The experimental results also show only small variations which is consistent with simulations. The systematic offset is including the viewing angle, efficiency of crystals and so on.
Does the flux vary as much as the electron charge? (for the shots with W block, collimator and converter in the correlation is washed out and there is no clear indication that these things are linked). Add some plots for that as well.
Stability of gamma flux?


\begin{figure}
\centering
\includegraphics[width=.5\columnwidth]{2018QED_ElecSpecs_examples.png}\includegraphics[width=.5\columnwidth]{2018QED_GammaSpec_simspec.png}
\caption{Left: Representative extreme examples of electron spectra from one day. Right: Resulting gamma spectrum based on GEANT simulations.}
\end{figure}

\begin{figure}
\centering
\includegraphics[width=.5\columnwidth]{2018QED_ElecSpecs.png}\includegraphics[width=.5\columnwidth]{2018QED_GammaSpec_simresp.png}

\includegraphics[width=.5\columnwidth]{2018QED_GammaSpec_expresp.png}\includegraphics[width=.5\columnwidth]{2018QED_GammaSpec_Average_expsim.png}

\caption{Left: Lineouts for most electron spectra. Right: Simulated detector responses. BLeft: Experimental responses. BRight: Comparison average responses.}
\end{figure}

What is the statistical variation of the signal on each day. Use the electron spectra to put a number onto this.
Use plot of relative and absolute number of photons above a threshold. Sum them up and also sum up number times energy as indicator for high energy photons.

Provide a plot with numbers on fluctuation of number of photons to expect in total and in particular above a threshold and how that varies shot-to-shot and day by day.

Provide some plots from the GEANT simulation (visualisation).

\begin{figure}
\centering
\includegraphics[width=.5\columnwidth]{2018QED_GSpec_Variations.png}
\caption{Variations of gamma signal (based on GEANT simulations).}
\end{figure}



\section{Cross section and Gamma Flux}

Cross section using number of photons. How many photons above a threshold and how is this affected?

Here some estimates from Robbie whether this change in flux is affecting the number of BW pairs or not.
If not, this is somewhat a description that the detector is not sensitive enough but it does not matter.
If it is somewhat important, it becomes crucial to develop a spectrometer that is sensitive to these photons.
This is the part about the shot-to-shot fluctuations.

Using the new spectra with photons only up to a certain energy. 

Put a plot average fraction of photons above an energy on a plot.
\vspace{\baselineskip}

\begin{figure}
\centering
\includegraphics[width=.5\columnwidth]{2018QED_BWXsec.png}
\caption{BW cross section.}
\end{figure}


The part about the additional photons on the 6th should be discussed here as well.
By how much, assuming an X-ray spectrum, do we expect the total cross section to rise with a change in gamma spectrum as estimated from day to day?
How much more likely is seeing electron-positron pairs based on this?

Make a simple equation basing this on one Xray photon energy (1.5 keV) and put a threshold on the photons we need from the gamma source.
Then see how much the cross section increases and check if it increases by more than the general exponential spectrum.
Be clear that a proper calculation needs to convolve the X-ray spectra, gamma spectra, efficiencies and so on.

\section{Discussion of results of BW analysis}

\EliasComm{Here a brief discussion of the findings of the complete BW analysis and if there was any sign of BW, reference publications that came out of this and so on.}

How much to talk about different parts?
Single particle detectors, tungsten blob, TimePix?

There are some positron particles, noise etc..
Positrons can be created from Bethe-Heitler, wall apertures, windows and so on.

Decision on whether it is significant or not.


\section{Future Outlook: A gamma spectrometer to discriminate high energy photons}

Here some ideas and simulations on how to improve the gamma spectrometer design to determine more accurately the spectrum.

Ideas: Insert different materials?
Check cross sections.
Pair spectrometer works at which energies?
Is there a compact way to measure this?
Would this also be of advantage for radiation reaction? 
\vspace{\baselineskip}

The elongation of the signal for the dual axis spectrometer was to measure the decay at higher resolution. However, most of the decay happens early in the crystal and the length of the detector only produces uncertainties and problems to consider when extracting the signal.
A cut off at reasonable crystal length using heavier materials could help.
Test in GEANT for range of gamma photons what the output is after a certain number of photons. A thicker lead piece could cancel out lower energies and ensure the higher energy tail is considered properly.

The problem is that in some scenarios the number of photons at higher energies is much lower and the signal is swamped. Having two parts, one to determine the overall critical energy, and a second one only looking at the high energy behind some shielding, would be an option.
\vspace{\baselineskip}

Show that the problem is the following:
above 10's of MeV the cross section for pair production increases and is relatively flat over long time. What happens is that gammas convert into electrons which convert into gammas in showers and so on. In most cases a high number of low few MeV gamma rays is present and the main indicator for the initial energy is how many times these showers continue to proceed and where they deposit energy.

The approach to check how we can distinguish things beyond the peak of the main number of photons is using GEANT and checking the input gamma, electrons and positrons and see how the distribution evolves going through the profile stack and the gamma spectrometer.
Based on this decide on a tactic to have a third part (in addition to profile, critical energy measurement) which allows separately to look at just very high energies and see for a selected energy band if the flux changes. This could be adjusted and optimised for certain energies.
If this does not work, point out that there are alternative designs based on pair spectrometers including magnets which might be a better option.



\section{Conclusion}

LWFA dual laser setup to produce BW pairs. Some indication that we saw pairs consistent with the experiment conditions at the optimum.
\vspace{\baselineskip}

Stability and robustness of gamma/X-ray source to a certain degree.
\vspace{\baselineskip}

Limits of gamma spectrometer design and potential improvements.
An improved detector design presented in this chapter.
