\chapter{Characterisation of a LWFA Bremsstrahlung Source for a Measurement of the Linear Breit-Wheeler Process}

\section{Motivation}

The Breit-Wheeler (BW) process is the inverse process of the more commonly observed pair annihilation where electrons and positrons decay into two photons. In BW the two (linear BW) or more (non-linear BW) photons combine to produce an electron-positron pair, in other words matter is created from light. The linear BW process is represented by a simple tree-level Feynman diagram. However, despite its simplicity it is the last tree-level diagram of QED that has not been definitively observed in its pure form, i.e. without the presence of other particles or external fields.

The non-linear process, on the other hand, has been measured and confirmed previously at an experiment at SLAC in the 90s where an ultrarelativistic electron beam of 46.6 GeV energy was scattered with an intense laser pulse XX a0, resulting in highly energetic gamma photons from inverse Compton scattering. These gamma photons in turn interacted with n=XX NUMBER REF E144 laser photons to produce electron-positron pairs. Over the course of XX NUMBER shots XX pairs were measured.
\vspace{\baselineskip}

Whilst the pair annihilation process is easily measured in radioactive decays as its cross section is very high at low energies and an abundance of electrons in ordinary matter provides an abundance of partners to annihilate with. Pair creation process from radiation, on the other hand, requires high energy photons or a very high photon density to overcome the production threshold of twice the electron rest mass energy $2m_e c^2 \approx 1\,\mathrm{MeV}$ in a suitable frame and more energy to provide some momentum to the produced particles. However, finding two bright photon sources at suitable energies at the same place is very challenging and hence mainly mediated pair production processes (Trident, Bethe-Heitler) have been measured.
\vspace{\baselineskip}

\begin{figure}
\includegraphics[width=.9\columnwidth]{2018QED_Chicane.JPG}
\caption{In pursuit of creating matter from light. View upstream on the gamma-ray axis through the magnetic chicane that is designed to transport electron-positron pairs from BW to the single particle detectors.}
\end{figure}

High energy radiation in the keV range can be produced from highly relativistic electrons, for instance at XFELs and synchrotron sources using bending magnets and insertion devices. An alternative thermal X-ray source can be provided by sufficiently heating solid targets, for instance hohlraums or foils.

Directed energetic gamma radiation can be generated by passing electrons through solid material to produce bremsstrahlung or relying on inverse Compton scattering, which is for instance the method used in the previously E144 experiment.
\vspace{\baselineskip}

In the experiment presented in this work, two lasers are used to provide simultaneously an X-ray source and a gamma ray source to overcome the mass threshold and to produce electron-positron pairs from the linear Breit-Wheeler process. The conceptual idea is based on REF PIKE WORK. The asymmetric distribution of the photon energy means that produced pairs will be strongly directed and easier to distinguish from noise.

The X-rays are produced by heating a solid germanium target with a spectrum peaked at $1.5\,\mathrm{keV}$. The gamma ray source is produced by bremsstrahlung from LWFA electrons.
\vspace{\baselineskip}

The following sections will outline the experiment setup aimed at measuring the linear Breit-Wheeler process, and take the reader through some of the data taken during the experimental campaign and the considerations relevant for its analysis.

The description of the experiment setup will convince the reader immediately of the complexity of this project and will also make it evident that a large number of researchers is required to tackle this challenge, with responsibilities for different parts of the design, execution and analysis of various diagnostics being distributed amongst its contributors. 

The author will focus on the characterisation of the electron beam and the gamma ray source, their impact on the BW cross section and variability on different shot days. In particular, the choice of converter targets for the bremsstrahlung production and the sensitivity of the spectral measurement of the gamma rays are discussed.

In some parts the author will refer to X-ray spectra, diagnostics for single particle detection, beam transport for the produced pairs, and signal-to-noise ratios, but will not cover these parts in extensive detail as it is pursued and will be presented in some form by other colleagues, in other words it does not reflect the primary work of the author. Contributions of fellow colleagues will be highlighted as such if they are used in the course of this work. 

Nonetheless, in a brief section towards the end, the author will attempt to place this work in the context of the full analysis of this campaign, summarise its findings and relevance for the project, and direct the reader to relevant publications.

\section{Experiment Setup}

\subsubsection{General Aim and Location}

The experiment was performed at the Gemini dual 300 TW Ti:Sa laser facility at the Central Laser Facility, UK, in early 2018.
The aim of this experimental setup is to produce and measure electron-positron pairs from the linear Breit-Wheeler process. For this purpose high energy gamma rays are collided with a thermal X-ray source. In this scenario the South arm of the Gemini laser provides the gamma source, the North beam the X-rays.
\vspace{\baselineskip}

\subsubsection{North beam and X-ray source}
In this experiment, the North beam is not fully temporally compressed to its usual 40-45 fs duration but is stretched to a pulse duration of 40 ps FWHM in order to effectively heat the solid target. The beam is focused down onto a solid target using an f/2 off-axis parabola (OAP) with a phase plate in the collimated beam smoothing and increasing the size of the focal spot to about $75\,\mathrm{\mu m} \times 210\,\mathrm{\mu m}$. The target is a $25\,\mathrm{\mu m}$ kapton tape with etched holes reducing the thickness to $5\,\mathrm{\mu m}$ and coated with $100\,\mathrm{nm}$ thick germanium (Ge) dots, that when heated produce a plasma plume emitting a thermal bath of X-rays at the interaction point. The motion of the tape drive is motorised and a fresh Ge target is hence provided on each shot. The tape drive was designed by Brendan Kettle (Imperial College), and the target and fabrication techniques were developed by the Target Fabrication Division at the Central Laser Facility, more specifically Sam Astbury and Chris Spindloe. IS THERE A PUBLICATION/PATENT ON THIS? THERE WERE SOME PLANS.

The X-ray source is diagnosed by an X-ray pinhole camera measuring the flux and the source size, and by a camera attached to a crystal measuring the spectrum of the X-rays in a XXXX NUMBER EV window around the line WHICH LINE emission of Ge (around 1.5 keV).
The X-ray source is the first component for the two-photon interaction. 
\vspace{\baselineskip}

\subsubsection{South beam, electrons and Gamma source}
The South laser is focused down by a $6\,\mathrm{m}$ focal length f/40 OAP onto the edge of a 17.5 mm length gas cell target\footnote{The gas cell has a variable length from few to 20 XX mm length. The target was designed by Nelson Lopes, formerly Imperial College, now IST.} filled with helium and 2 percent nitrogen at an average electron density of XX NUMBER. The typical average spot size was around $44\,\mathrm{\mu m} \times 53\,\mathrm{\mu m}$ at $5.51\pm 0.64\,\mathrm{J}$ energy on target at a pulse duration of XX NUMBER FS THIS WAS NOT MEASURED? IF SO $45 \pm 5\,\mathrm{fs}$, peak intensity $I_0 = 2.75 \times 10^{18} \,\mathrm{W/cm^3}$ corresponding to a normalised vector potential $a_0 = 1.13$ (this is based on focal spots from the 28th March and Matt's analysis MAYBE REDO ANALYSIS?). The energy on target was limited by the damage threshold of the mirrors in the focusing beam.
The laser drives a wakefield and accelerates electrons, injected predominantly via ionisation injection REF HERE, to relativistic energies. The remaining laser exiting the cell is disposed of by a replenishable kapton tape that acts as a plasma mirror (REF).
\vspace{\baselineskip}

The electrons are dispersed vertically by a permanent dipole magnet of integrated field strength $\int B dx = 0.4\,\mathrm{Tm}$ onto a scintillating LANEX screen measuring their energy and charge. The screen is imaged by an Andor Neo camera equipped with a TV lens and a bandpass filter around $546 \pm 10\,\mathrm{nm}$ XX IS THIS CORRECT?. A motorised stage with high-Z foils of various thicknesses can be driven into the path of the electrons and act as bremsstrahlung converter foils. This interaction produces copious amounts of directed bremsstrahlung in propagation direction and stops most of the electrons in the process. Since divergent gamma rays are likely to collide with components and apertures producing particles near axis with comparable properties as the BW pairs, a tungsten collimator of length $100 mm$, inner diameter $2\,\mathrm{mm}$ and outer diameter $20 mm$ is employed to reduce the emitted gamma-ray burst to its central part at low divergence. The exit of the collimator is placed at 280 mm distance from the converter target and hence apertures the beam down to a field of view of $2\,\mathrm{mm}/0.28m = 7.14\,\mathrm{mrad}$. In addition, a thick block of tungsten obstructs the direct line of sight from the converter target to the Ge target drive. Only a collimated bright burst of gamma-rays from bremsstrahlung is incident on the interaction point, providing the second component for the two-photon interaction.
\vspace{\baselineskip}

\begin{figure}
\includegraphics[width=.9\columnwidth]{BW2018_render_V3_annotated3.png}
%\includegraphics[width=.9\columnwidth]{BW2018_render_minV1.png}
\caption{Sketch of the experiment setup.}
\end{figure}

The gamma-ray burst propagates almost unperturbed by the interaction to the central one out of three $125\,\mathrm{\mu m}$ thick kapton windows at the end of the chamber and through it into air. This was a three aperture flange with one central 80 mm diameter circular aperture, then two slits 150 mm x 35 mm on both sides. 

At air a stack of 400 caesium-iodide (CsI) crystals of dimensions $2\,\mathrm{mm}\,\times\,2\mathrm{mm}\,\times20\,\mathrm{mm}$, each individually wrapped in aluminium foil ($\sim 15-20\,\mathrm{\mu m}$, in a $1\,\mathrm{cm}$ thick aluminium casing measures the transverse profile and yield of the gamma-ray signal. The total transverse area is about $40\,\mathrm{mm}\times 40\,\mathrm{mm}$ which corresponds to an acceptance angle of $11.8\,\mathrm{mrad}$ based on a distance of $3.39\,\mathrm{m}$ from the converter targets to the profile screen. 
Another larger stack of $XX \times XX$ CsI crystals measures the decay of the signal in propagation direction to deduce the spectrum. 
Both CsI arrays are imaged by sensitive Andor iXon cameras equipped with suitable objectives and bandpass filters.
\vspace{\baselineskip}

\subsubsection{Chicane and Single Particle Detectors}

Potential electron-positron pairs from the photon-photon interaction are emitted preferably in the propagation direction of the gamma-ray burst. The pairs enter the field of a large aperture permanent dipole magnet of XX FIELD STRENGTH NUMBER that disperses the electrons and positrons horizontally in opposite directions, leaving the vacuum through the the other two kapton apertures at the end of the target chamber. The aperture of the magnet produces a low energy cutoff at about 220 MeV CHECK NUMBER ESPECIALLY BECAUSE OF WINDOW which is a slightly larger acceptance range than the vacuum window would provide. The dispersed electrons and positrons are then caught by an oppositely polarised permanent magnet of FIELD STRENGTH XX NUMBER on each side that bends the electrons or positrons respectively onto a narrow aperture in a lead-shielded enclosure and onto a set of single-particle detectors. The single-particle detectors are CsI arrays attached to sensitive CCD cameras. In addition, there are two TimePix silicon detectors in front of the CsI array on the positron side.

\section{Electron Spectra Measurement}

\subsubsection{Application of Electrons in this context and need for characterisation}
In this experiment the energy of relativistic electrons is converted into directed and highly energetic bremsstrahlung.
Whenever bremsstrahlung is produced from a converter target, the spectrum of the electrons can not be measured simultaneously as they are almost entirely stopped by the high-Z material in order to produce an as high as possible flux of gamma rays. The electron spectra, their charge and stability hence have to be characterised without the converter target and not during BW data runs. The gamma signal, however, is measured during the BW data runs and both are linked together.
\vspace{\baselineskip}

The electron beam is dispersed by the a permanent dipole magnet onto a scintillating lanex screen. Electrons with energy below XX NUMBERS hit the yoke of the magnet and can hence not be measured on the lanex screen. The screen is imaged with an Andor Neo camera equipped with a TV lens and suitable filters to reduce ambient laser light.
\vspace{\baselineskip}

\begin{figure}
\includegraphics[width=.9\columnwidth]{2018QED_ElecSpecs_montage.jpg}
\caption{Montage of electron spectra (screen raw).}
\end{figure}

\subsubsection{Raw images and general note on processing data}
Raw images of typical electron signals are seen in FIGURE XX, lineouts of electron spectra on an energy axis can be seen in FIGURE XX.
The images are treated for background and are projective transformed to account for the image distortion from the viewing angle. Space to energy calibration based on a field map. See Methods for more details.
\vspace{\baselineskip}

\subsubsection{General overview and shape of beams}
The performance of the accelerator varied from day to day and is characterised separately for each day, in some cases the performance even degraded over the course of the day.
The measured quantities are charge, energy, divergence and pointing of the beams along with the stability (standard deviation) of these quantities from shot to shot.
Due to the variations from day to day only shot days and electron data from days where BW data was taken is being considered.
The relevant shot days were the 6th, 9th and 10th but will for simplicity be referred to as days 1, 2 and 3 in this context. 
\vspace{\baselineskip}

An overview of the quantities is given in TABLE XX NUMBER and plotted in a comparative way in figure XX NUMBER.

General overview and shape:
The electrons produced with ionisation injection in this context have a large energy spread from continuous injection reaching maximum energies of around 800 MeV in best conditions. The performance of the accelerator varies from day to day but typically either an almost flat spectrum up to 800 MeV or a distinct peak at 400 MEV was produced (see figure). Depending on the day the average charge was 10's of pC which is relatively low for the Gemini laser but might be linked to the limited energy on-target that could be delivered. The response and the imaging system was cross-calibrated with an image plate providing an absolute calibration of the images.

\subsubsection{Investigation of main quantities and variation on the days}
Comparison for different days along with stability.

Conclusion about the days.

Seeing in the figures it appears that the electrons on the 6th were in particular better in various dimensions:
The charge is close to doubled with respect to other days and relatively stable. The maximum energy is increased by 100-200 MeV than on other days.
At the same time a decent divergence at a stable level is paired with a good pointing stability of smaller than 1 mrad, which is not the optimum but a very good value in the subset of data taken.

The shape of the spectra (see average spectra) shows a distinct peak at few hundred MeV and a tail in some cases extending to 500-550 MeV on most days and in case of the 6th even up to 800 MeV.


\begin{figure}
\includegraphics[width=.5\columnwidth]{2018QED_ElecSpecs.png}\includegraphics[width=.5\columnwidth]{2018QED_ElecSpecs_average}
\caption{Left: 2D example of spectrum. Right:  Lineouts for electron spectra (maybe waterfall?). Maybe average spectra for day with shaded errorbars.}
\end{figure}

\begin{figure}
\includegraphics[width=.9\columnwidth]{2018QED_Espec_Variations.png}
\caption{Electron charge and energy over shot days. Could leave out the 5th as it is not a shot day.}
\end{figure}



\begin{table}
\centering
\begin{tabular}{l|l|l|l|l|l|l}
Date & Run & Charge & Max Energy & Divergence & Pointing & Nshots\\ \hline \hline
5 & 2 & $20 \pm 10.8$ & $599 \pm 57$ & $3.14 \pm 0.75$ & 1.25 & 32\\ \hline
6 & 1 & $26.26 \pm 3.8$ & $709 \pm 46$ & $2.26 \pm 0.29$ & 0.62 & 10\\ \hline
9 & 1 & $14.73 \pm 5.5$ & $565 \pm 43$ & $2.5 \pm 0.72$ & 0.93 & 18\\ 
9 & 4 & $7.7 \pm 4.4$ & $551 \pm 16$ & $1.7\pm 0.31$ & 0.81 & 12\\ \hline
10 & 1 & $11.55 \pm 2.7$ & $511 \pm 19$ & $2.84 \pm 0.93$ & 0.9 & 22\\ 
10 & 2 & $15.24 \pm 5.1$ & $512 \pm 26$ & $2.3 \pm 0.52$ & 0.5 & 6\\ 
10 & 3 & $9.56 \pm 3.5$ & $535 \pm 21$ & $2.82 \pm 0.91$ & 1.55 & 21
\end{tabular}
\caption{Details on runs. Electron characteristica. Could leave out the 5th as it is not a shot day.}
\end{table}

\section{Gamma Profile Diagnostic}

\subsubsection{General Description of diagnostic and how to extract data.}

\subsubsection{Field of view and calibration.}

\section{Bremsstrahlung Targets and Detector Response}


\subsubsection{General bits on the diagnostic, signal extraction and characterisation}
\EliasComm{Add some raw data and examples from analysis.}

Electrons characterised in the previous section are converted to a burst of gamma rays by inserting a high-Z converter foil in the beam path, in this context mainly 1 mm of bismuth. This is to balance the need for a decent amount of gamma rays and a low divergence of the resulting beam. Thick targets result in the highest yield but the beam is more divergent and more signal will be cut out by the collimator etc. and more noise is to be expected.
\vspace{\baselineskip}


\EliasComm{Add a plot showing the raw data from GammaSpec for different converters.}

\EliasComm{Add a plot for correction factor.}

The tungsten collimator only lets radiation through an aperture of XX NUMBER. In addition the thick tungsten block aims to half the signal to block any gamma rays from hitting the Ge target drive. The alignment of these components will change from day to day based on human error.
\vspace{\baselineskip}

The detector used to measure the spectrum of this gamma flux is a stack of scintillating crystals as described in Methods imaged by an Andor iXon camera.
RAL stack with spectral measurement and plastic, gamma profile in the path. 
\vspace{\baselineskip}

Signal extraction. Show an example.
Selecting the crystals and get a pixelated response.
\vspace{\baselineskip}

\subsubsection{GEANT simulation and correction factors using bremsstrahlung}

Simulating response of detectors using GEANT.
Monoenergetic photons in a different range and present the response curves.
\vspace{\baselineskip}

Correcting using brems.
Get a brems run using a thick converter and simulate the response based on that.
Correct out the crystal efficiencies and viewing angle. Reduce this section maybe based on Methods.
\vspace{\baselineskip}

Correlating Gamma profile and spec counts. No change in response for those detectors.
Seeing that this is very linear the results from gamma profile assessment are still or again valid.
This is not useful to distinguish energies.
\vspace{\baselineskip}

\subsubsection{Choice of converter material, overview on materials, yield vs. divergence consideration}

The options were 2 mm and 4 mm tungsten, 0.5 and 1 mm bismuth (see 23.3.18 run 2) on 0.5 mm PTFE (!?).

From the lab book the summary was:
2 mm tungsten: divergent and high yield.
4 mm tungsten: more divergent and more yield (depending on shot)
0.5 mm bismuth: less divergent than the tungsten, fewer counts, sometimes similar peak counts but only on axis
1 mm bismuth: more divergent than 0.5 mm bismuth but less than tungsten. High yield on axis, sometimes even more concentrated than 2 mm tungsten.


\begin{figure}
\centering
\includegraphics[width=.5\columnwidth]{2018QED_ConverterGSpec_Yield.png}\includegraphics[width=.5\columnwidth]{2018QED_TimePixNoise.png}
\caption{Left: Gamma Spec yield. Right: Noise on TimePix.}
\end{figure}

\begin{figure}
\centering
\includegraphics[width=.9\columnwidth]{2018QED_Converter_YieldVDiv.png}
\caption{Gamma Profile Yield and divergence. Use mrad instead of mm for FWHM and present a combined plot for x and y axis. The distance to gamma profile from the converter target is about 3.4 m. So 20/3.4 for a full angle, so less than 3 mrad half angle for 0.5 mm bismuth and so on.}
\end{figure}

\begin{figure}
\centering
\includegraphics[width=.5\columnwidth]{2018QED_Converter_YieldvNoise.png}\includegraphics[width=.5\columnwidth]{2018QED_Converter_DivvNoise.png}
\caption{Left: Gamma Spec yield. Right: Noise on TimePix.}
\end{figure}


\begin{figure}
\centering
\includegraphics[width=.5\columnwidth]{2018QED_SignalNoise_Transmission.png}\includegraphics[width=.5\columnwidth]{2018QED_SignalNoise_AbsTransmission.png}
\caption{Left: Gamma Spec yield. Right: Noise on TimePix.}
\end{figure}

From the analysis we can see that the divergence and yield is for each material larger when the solid target is thicker, i.e. the measured gamma yield is higher for 1 mm of bismuth than for 0.5 mm, and higher for 4 mm than for 2 mm of tungsten, and similarly for the divergence.
Comparing the bismuth targets and the tungsten we also observe that the yield does not increase much more from bismuth 1 mm to any of the tungsten targets whilst preserving a better divergence. This indicates that most of the electrons already transferred all or a large fraction of their energy into gamma radiation and more material mainly leads to scattering and showers of radiation.
Give numbers: only few percent increase in yield for a divergence increase of 10-15 percent.


\subsubsection{Discussion if the choice was right also considering collimator/Wblock view}

What is the size of the X-ray source and how large is the gamma beam at that point? Already 6 mm or so.
The North beam acting as a heater focuses down to a spot of $75\,\mathrm{\mu m} \times 210\,\mathrm{\mu m}$, plus minus defocusing terms.
This leads to the formation of a plasma plume that then heats things. Do we have any pinhole measurements that indicate the source size?
The distance from the converter to the interaction point is XX NUMBER. Also consider pointing etc., so better base this on noise and not source overlap. The difference is only 10 percent at 3.4 mm. The gamma beam is bigger than the X-ray bath.
However, the collimator is positioned 180 mm away from the converter target. It is 100 mm long and has an inner diameter of 2 mm. This permits a free field of view of 1 mm / 0.28 m = 3.57 mrad half angle. The effective field of view is slightly larger with some radiation penetrating the last mm's of the collimator walls. This is just about or a bit smaller than the average divergence using the collimator. The tungsten targets have a larger divergence and will produce noise which will outweigh the additional yield. This is all without taking into consideration pointing.

We want a low divergence bright gamma flash (high yield) to stimulate reactions but also to reduce the noise. In particular dangerous noise are particles produced close to the axis. A too divergent gamma ray beam could produce these pairs that resemble a real signal.

The TimePix results show the level of noise measured for the different converters. It shows that, despite having a similar yield as 1 mm bismuth, the noise levels are much higher.

Hence, 1 mm bismuth appears to be the best choice in this case combining low divergence, high yield and low noise.


\subsubsection{The electron regime (plasma density) was also right in terms of yield and divergence.}
We tested the converter targets for two different electron regimes (70 mbar and 90 mbar backing pressure), where the higher densities were more divergent.
We operated at the lower regime, later on at an intermediate backing pressure but related to an increase in leak.


\subsubsection{Detector response in experiment, shot to shot and day by day}

Seeing not really much change in experimental response.
Give a number on critical energy. Any idea of brightness (absolute number?).
The curves are fairly similar. Might see an overall change in the spectrum but everything within a crystal. Check if there is a difference between the days.
Absolute calibrations for gamma profile is possible but might be harder for gamma spectrometer due to the components in between.
\vspace{\baselineskip}

The response is not sensitive to such highly energetic radiation or the number of photons is much lower and hence the response is based on that.
This means the detector is not suitable to discriminate a spectrum to such detail and also is not suitable to deduce a gamma spectrum independent of assuming its shape. Luckily bremsstrahlung is well understood. It is essential to have many reference electron shots. Also a lesson for the future to take data in between.
\vspace{\baselineskip}

\subsubsection{General critical energy fitting}

\subsubsection{Comparison yield per day with and without blocks}

\subsubsection{Simulated spectra from Electrons and variation in resulting gamma spectra}

\subsubsection{Characterisation of fluctuations and lack of change in response}

The produced spectrum can be simulated using the electron spectra characterised earlier. GEANT. Typical gamma spectra produced are shown in FIGURE XX. 
Show that the responses for all electron spectra is fairly similar in simulations and the resulting spectra vary mostly at a low photon level and high energies.
The response of the detector only results in small variations.
\vspace{\baselineskip}

The experimental results also show only small variations which is consistent with simulations. The systematic offset is including the viewing angle, efficiency of crystals and so on.
Does the flux vary as much as the electron charge? (for the shots with W block, collimator and converter in the correlation is washed out and there is no clear indication that these things are linked). Add some plots for that as well.
Stability of gamma flux?


\begin{figure}
\centering
\includegraphics[width=.5\columnwidth]{2018QED_ElecSpecs_examples.png}\includegraphics[width=.5\columnwidth]{2018QED_GammaSpec_simspec.png}
\caption{Left: Representative extreme examples of electron spectra from one day. Right: Resulting gamma spectrum based on GEANT simulations. Maybe show an average with shaded error bars instead.}
\end{figure}

Add the differences from day to day somewhere as well as comparison.

\begin{figure}
\centering
\includegraphics[width=.5\columnwidth]{2018QED_ElecSpecs.png}\includegraphics[width=.5\columnwidth]{2018QED_GammaSpec_simresp.png}

\includegraphics[width=.5\columnwidth]{2018QED_GammaSpec_expresp.png}\includegraphics[width=.5\columnwidth]{2018QED_GammaSpec_Average_expsim.png}

\caption{Left: Lineouts for most electron spectra. Right: Simulated detector responses. BLeft: Experimental responses. BRight: Comparison average responses.}
\end{figure}

What is the statistical variation of the signal on each day. Use the electron spectra to put a number onto this.
Use plot of relative and absolute number of photons above a threshold. Sum them up and also sum up number times energy as indicator for high energy photons.

Provide a plot with numbers on fluctuation of number of photons to expect in total and in particular above a threshold and how that varies shot-to-shot and day by day.

Provide some plots from the GEANT simulation (visualisation).

\begin{figure}
\centering
\includegraphics[width=.5\columnwidth]{2018QED_GSpec_Variations.png}
\caption{Variations of gamma signal (based on GEANT simulations).}
\end{figure}

\section{Gamma Profile Assessment}


\subsubsection{Brief look at gamma profile and correlation of counts. Show change in alignment/pointing.}
Look at gamma profile and see that from day to day the alignment changes.

Show that the 6th had the ideal alignment based on this.

Show the pointing varies and the total alignment.

Show the gamma yield is varying from day to day but not as much as the electron charge which indicates that the alignment is washing out any correlation.

\EliasComm{Might just be incorporated as side note in the next section.}

\section{Performance and Stability of the Bremsstrahlung Source}

This part is about the spectra on the BW data days.
\vspace{\baselineskip}



\section{Impact of Spectral Fluctuations on the total BW cross section}

\subsubsection{Simple calculation on impact on the total cross section based on the GEANT spectra}

Cross section using number of photons. How many photons above a threshold and how is this affected?

Here some estimates from Robbie whether this change in flux is affecting the number of BW pairs or not.
If not, this is somewhat a description that the detector is not sensitive enough but it does not matter.
If it is somewhat important, it becomes crucial to develop a spectrometer that is sensitive to these photons.
This is the part about the shot-to-shot fluctuations.

Using the new spectra with photons only up to a certain energy. 

Put a plot average fraction of photons above an energy on a plot.
\vspace{\baselineskip}

\begin{figure}
\centering
\includegraphics[width=.5\columnwidth]{BWxsection.png}\includegraphics[width=.5\columnwidth]{BWxsection_2D.png}
\includegraphics[width=.9\columnwidth]{BW_Xsec_NxNg.png}
\caption{BW cross section at a fixed and variable X-ray photon energy.}
\end{figure}


\begin{figure}
\centering
\includegraphics[width=.5\columnwidth]{2018QED_Xsec_VariationAvDay.png}\includegraphics[width=.5\columnwidth]{2018QED_Xsec_VariationMorton.png}
\caption{Day to day fluctuation of relative total cross section: Left using the measured X-ray spectra (average per day) with limited energy window. Right: using Morton simulations over a larger energy window.}
\end{figure}

The part about the additional photons on the 6th should be discussed here as well.
By how much, assuming an X-ray spectrum, do we expect the total cross section to rise with a change in gamma spectrum as estimated from day to day?
How much more likely is seeing electron-positron pairs based on this?

Make a simple equation basing this on one Xray photon energy (1.5 keV) and put a threshold on the photons we need from the gamma source.
Then see how much the cross section increases and check if it increases by more than the general exponential spectrum.
Be clear that a proper calculation needs to convolve the X-ray spectra, gamma spectra, efficiencies and so on.

Looking at the cross section plot the cross section decreases towards higher energies at a constant X-ray energy.
The spectrum and the number of photons also become less important at the higher energy end of photons.
The key question is how the X-ray spectrum evolves around the right energies and if there is a large abundance of lower energy photons that can interact at high cross section with the tail of photons.

Multiply the curve for 1.5 keV with the gamma spectra to get the total cross section. Then multiply this with the respective average charge to get an estimate how much more likely it would be to see a positron.

This approach is just a simple calculation to demonstrate the difference we expect.
Of course one also has to fold in the X-ray spectrum over a wider energy range along with other factors like higher noise levels due to the increased charge and so on.

What angle are we assuming? There will be a range of photon angles but what is the design main angle?
According to an Evernote note (X-ray diagnostics survey by BK I assume) the angle was roughly 40 degrees, so main axis is 140 degrees but with a large spread of relative angles I would assume. If the beam has good emittance (or if there is a different term for gamma beams) then the divergence will determine the range of angles. The collimator lets through 3.57 mrad half angle which is 0.2 degrees, so the main range of angles will come from the X-ray source.

The closer to angle is to head-on collision to more the peak of the cross-section shifts towards lower gamma energies. 
If the X-ray spectrum is mainly centred around 1.5 keV the additional 800 MeV does not have an impact.
If the full X-ray spectrum (for instance Morton simulations) are included then a large pool of X-rays are available at lower energies to be paired up with the high energy photons of 600-800 MeV.

\section{Future Outlook: A gamma spectrometer to discriminate high energy photons}

\subsubsection{Explanation why detector is not sensitive to the changes seen in the simulated spectra}
Here some ideas and simulations on how to improve the gamma spectrometer design to determine more accurately the spectrum.

Ideas: Insert different materials?
Check cross sections.
Pair spectrometer works at which energies?
Is there a compact way to measure this?
Would this also be of advantage for radiation reaction? 
\vspace{\baselineskip}

The elongation of the signal for the dual axis spectrometer was to measure the decay at higher resolution. However, most of the decay happens early in the crystal and the length of the detector only produces uncertainties and problems to consider when extracting the signal.
A cut off at reasonable crystal length using heavier materials could help.
Test in GEANT for range of gamma photons what the output is after a certain number of photons. A thicker lead piece could cancel out lower energies and ensure the higher energy tail is considered properly.

The problem is that in some scenarios the number of photons at higher energies is much lower and the signal is swamped. Having two parts, one to determine the overall critical energy, and a second one only looking at the high energy behind some shielding, would be an option.
\vspace{\baselineskip}

Show that the problem is the following:
above 10's of MeV the cross section for pair production increases and is relatively flat over long time. What happens is that gammas convert into electrons which convert into gammas in showers and so on. In most cases a high number of low few MeV gamma rays is present and the main indicator for the initial energy is how many times these showers continue to proceed and where they deposit energy.

\subsubsection{Based on problem present ideas to make the detector more sensitive along with GEANT simulations.}

The approach to check how we can distinguish things beyond the peak of the main number of photons is using GEANT and checking the input gamma, electrons and positrons and see how the distribution evolves going through the profile stack and the gamma spectrometer.
Based on this decide on a tactic to have a third part (in addition to profile, critical energy measurement) which allows separately to look at just very high energies and see for a selected energy band if the flux changes. This could be adjusted and optimised for certain energies.
If this does not work, point out that there are alternative designs based on pair spectrometers including magnets which might be a better option.


\section{Results of the Breit-Wheeler Analysis}


\subsubsection{Discussion of how this work fits in the big picture of the complete BW analysis}
\EliasComm{Here a brief discussion of the findings of the complete BW analysis and if there was any sign of BW, reference publications that came out of this and so on.}

How much to talk about different parts?
Single particle detectors, tungsten blob, TimePix?

There are some positron particles, noise etc..
Positrons can be created from Bethe-Heitler, wall apertures, windows and so on.

Decision on whether it is significant or not.



\section{Conclusion}

LWFA dual laser setup to produce BW pairs. Some indication that we saw pairs consistent with the experiment conditions at the optimum.
\vspace{\baselineskip}

Stability and robustness of gamma/X-ray source to a certain degree.
\vspace{\baselineskip}

Limits of gamma spectrometer design and potential improvements.
An improved detector design presented in this chapter.
