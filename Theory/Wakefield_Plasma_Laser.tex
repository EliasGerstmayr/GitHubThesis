\chapter{Laser Propagation and Wakefield Acceleration}
\label{Chap:Theory:LWFA}

This Chapter addresses the propagation of a laser pulse in an underdense plasma and its evolution at high-intensities, which give rise to interesting phenomena such as self-focusing and self-guiding, pulse compression and photon acceleration.
In contrast to the previous Chapter that was concerned with the motion of individual charged particles and consequences thereof, e.g. radiation production and radiation reaction, here the interaction of an intense laser pulse with a plasma, a medium exhibiting collective behaviour, is discussed.
This leads over to the topic of laser wakefield acceleration (LWFA): here an intense laser pulse drives a plasma wave through the ponderomotive force and the resulting charge separation in its wake generates strong electric fields that are suitable to capture and accelerate electrons to relativistic energies.
In the context of this discussion key quantities like the maximum energy gain and its fundamental limiting factors, dephasing and depletion, are introduced, along with a brief note on different injection mechanisms relevant to this work.

\section{Laser Propagation in an Underdense Medium}

\subsection{Plasma Properties}

A plasma is an on average charge-neutral medium that consists of charged particles, in the context of wakefield acceleration ions and free electrons. This property is called quasi-neutrality and can mathematically be expressed as \cite{Kruer}
\begin{equation}
-e n_e + Z e n_i \approx 0,
\end{equation}
where $e$ is the fundamental charge, $n_e$ the electron number density, $n_i$ the ion number density and $Z$ the charge number of the ions.
Whilst this balance might locally not be true at every moment in time, the mobility of the charges lead to shielding at length scales larger than the Debye length, $\lambda_D$ \cite{Hutchinson2002_PlasmaDiag}:
\begin{equation}
\lambda_D = \left(\frac{\epsilon_0 k_B T_e}{n_e e^2}\right)^{1/2},
\end{equation}
where $\epsilon_0$ is the dielectric constant, $k_B$ the Boltzmann constant and $T_e$ the electron temperature. 
The formation of Debye-spheres is an example of a collective response of the plasma to a small perturbation. In a plasma particles can act together on macroscopic scales through long range electromagnetic forces to form instabilities and waves. Even small imbalances can stimulate a significant response of the medium. This means that even though plasmas can be modelled as fluids and simulated using hydrodynamic codes, introducing electromagnetic forces will give rise to more complex and exotic phenomena than in fluids dominated by gravitational forces \cite{Kruer}.


\subsection{Plasma Formation}
\label{Theory:Sec:PlasmaFormation}
In the context of this work, the process of forming a plasma is achieved by ionising a neutral gas in the electric field of an intense laser pulse. The resulting collection of unbound electrons and ions is a plasma. After a certain time the electrons will return to the ionic cores of the atoms and back again to a bound energy state emitting any excess energy in form of radiation in the process. This process is called \textit{recombination} and the emitted radiation \textit{recombination light}.

There are different mechanisms through which an oscillating laser field ionises a gas. In single and multi-photon ionisation \cite{Keldysh1965_ION} one electron is excited by absorbing one or multiple photons, respectively, before decaying to its original ground state. Another mechanism relies on the collision of electrons and atoms \cite{Latypov1964_ION}. The electric field of the laser can also distort the Coulomb potential of the atom, resulting in a finite probability for electrons to tunnel into the continuum \cite{Perelomov1966_ION}.

The dominant ionisation mechanism is indicated by the Keldysh parameter, $\gamma_K$, considering the strength of the electric field $E$ relative to the binding potential $V$ and the frequency of the oscillating field $\omega$:
\begin{equation}
\gamma_K = \omega \frac{\sqrt{2 m_e V}}{eE}.
\end{equation}
At $\gamma_K \gg 1$ \textit{multiphoton ionisation} dominates, whereas for small $\gamma_K < 0.5$ \textit{tunnel ionisation} is expected to be important. If the electric field is larger than the binding Coulomb potential, the atom is immediately ionised: this is called \textit{barrier-suppression ionisation} \cite{Reiss1970_ION}.
In the simple example of a hydrogen-like atom in an external electric field $E$, the potential $V$ at a distance $r$ from the core is given by
\begin{equation}
V(r) = -\frac{Ze^2}{4\pi\epsilon_0 r} - eEr,
\end{equation}
where the position of the maximum potential is given by $r_{max} = (Ze/4\pi\epsilon E)^{1/2}$. 
Hence the electric field required for complete barrier suppression of an ionisation potential $V_e$ amounts to
\begin{equation}
E = \frac{\pi \epsilon_0 V^2_e}{Ze^3} \approx 173.6 \times \left(\frac{V_e}{eV}\right)^2 \frac{1}{Z} \, \mathrm{MV/m},
\end{equation} 
or in terms of intensity using $I = \frac{c\epsilon_0}{2}|E|^2$ in (near) vacuum:
\begin{equation}
I = \frac{c\epsilon_0^3 \pi^2 V^4_e}{2e^6 Z^2} \approx 4 \times 10^{9} \left(\frac{V_e}{eV}\right)^4\frac{1}{Z^2} \,\mathrm{W/cm^2}.
\end{equation}

For hydrogen with $V_e = 13.6\,\mathrm{eV}$ the lower bound intensity required for barrier-suppression ionisation amounts to $1.37 \times 10^{14} \,\mathrm{W/cm^2}$, for helium with $Z=2$ and $V_e = 54.4\,\mathrm{eV}$ the equation gives $I = 8.76\times10^{15}\,\mathrm{W/cm^2}$.
The laser intensities in LWFA experiments routinely reach peak intensities above $10^{18}\,\mathrm{W/cm^2}$. Hence, one can safely assume that ionisation effects are negligible for low-$Z$ gases. Higher-$Z$ gases like nitrogen are not as easily fully ionised: in hydrogen-like nitrogen the last electron is bound by a potential of $V_e = 667\,\mathrm{eV}$ at $Z = 7$, which translates in intensities required in excess of $10^{19}\,\mathrm{W/cm^2}$. Even very intense laser pulses might struggle to fully ionise a nitrogen gas at all and one can not assume any longer that a fully ionised plasma is in place when the main part of the laser pulse arrives. The gas might then reach higher degrees of ionisation only at peak fields of the laser either through full barrier suppression or by triggering an increased tunnelling rate: this is taken advantage of in an injection scheme for laser wakefield acceleration called `ionisation injection' \cite{McGuffey2010_ION,Pak2010_ION}. 


\subsection{Plasma Frequency}
\label{Chap:Theory:Sec:PlasmaFreq}

The plasma frequency, $\omega_p$, is the natural frequency at which electrons will oscillate in a plasma when being displaced \cite{Kruer}. It constitutes the typical time scale and the corresponding plasma wavelength, $\lambda_p$, the typical length scale of a plasma response.
\vspace{\baselineskip}

Assume a uniform and neutral plasma, i.e. $n_e = Z n_i = n_{e0}$, where $n_e$ is the electron number density, $Z$ the charge number of the ions, $n_i$ the ion number density, and the initial electric field be $\mathbf{E} = \mathbf{0}$.
Now consider displacing a slab of charge by $X$ in an arbitrary axis $x$.
Starting with Gauss' Law we obtain:
\begin{equation}
\nabla \cdot \mathbf{E} = \frac{\rho}{\epsilon_0} = \frac{\partial E_x}{\partial x},
\end{equation}
where we used that the displacement is only along $x$, simplifying the Nabla operator to a spatial derivative in $x$.
We assume a uniform plasma so that the charge density is a constant $\rho = e n_{e0}$. Integrating the previous equation then gives a constant factor times the displacement $X$:
\begin{align}
E_x = \frac{1}{\epsilon_0} \int^x_0 \rho (x') dx' + E_x(0) &= \frac{e n_{e0}}{\epsilon_0} X,\nonumber\\
E &= \frac{e n_{e0}}{\epsilon_0} X.
\end{align}

If we now consider a force based on this field, we immediately see that this is equivalent to the equation of motion of a harmonic oscillator: 

\begin{equation}
m_e \frac{\mathrm{d}^2 X}{\mathrm{d}t^2} = - \frac{e^2 n_{e0}}{\epsilon_0} X = - m_e \omega^2_p X,
\end{equation}

with the (plasma) frequency, $\omega_p$:
\begin{equation}
\boxed{\omega_p = \sqrt{\frac{n_{e0} e^2}{\epsilon_0 m_e}},}
\end{equation}

and the corresponding wavelength $\lambda_p = 2 \pi v_{ph}/\omega_p$, where $v_{ph}$ is the phase velocity of the wave.
\begin{figure}
\centering
\includegraphics[width=.5\columnwidth]{omegap_lambdap_density.pdf}
\caption[Plasma frequency and wavelength as function of plasma density.]{Plasma frequency (blue) in Hz and wavelength (red) in micrometres as function of plasma density assuming $v_{ph} = c$.}
\label{Theory:Figs:PlasmaFreqLambdap}
\end{figure}
A typical wakefield accelerator in the context of this work operates at an electron density of around $10^{18}-10^{19}\,\mathrm{cm^{-3}}$. This corresponds to a plasma frequency of $(5.6 - 17.7) \times 10^{13} \,\mathrm{Hz}$ or a plasma wavelength of $\lambda_p \approx 33-10\,\mathrm{\mu m}$ for $v_{ph} = c$ (see Figure \ref{Theory:Figs:PlasmaFreqLambdap}).

\subsection{Laser Propagation in Vacuum}
\label{Chap:Theory:Sec:LaserInVac}

An electromagnetic wave of arbitrary shape $u$ has to satisfy Maxwell's equations and hence also the wave equation \cite{Jackson}
\begin{equation}
\left(\nabla^2 - \frac{1}{v^2}\frac{\partial^2}{\partial t^2} \right) u= 0,
\end{equation}
where $v = c/\sqrt{\mu \epsilon}$ and in vacuum $v = c$.
In the case of waves with an electric field $\mathbf{E} = \mathbf{E}(\mathbf{r},t) = \mathbf{u}(\mathbf{r}) v(t)$, where the spatial and temporal dependencies can be separated, the components solve two separate differential equations 
\begin{align}
\left(\nabla^2 + k^2\right) \mathbf{u} &= 0,\\
\left(\frac{\partial^2}{\partial t^2} + k^2\right) v &= 0,
\end{align}
where $k = \sqrt{\mu \epsilon} \omega/c$. The spatial component is called the Helmholtz equation.
For an an electric field $\mathbf{E} = \mathbf{E_0}(r,z) e^{i(kz-\omega t)}$ with cylindrical symmetry in the paraxial approximation $|k \mathbf{E}_0| \gg |\partial_z \mathbf{E}_0|$ the Helmholtz equation takes the form
\begin{equation}
\left(\frac{\partial^2}{\partial r^2} + \frac{1}{r}\frac{\partial}{\partial r}- 2ik \frac{\partial}{\partial z}\right) \mathbf{E}_0 = 0.
\end{equation}

For an in $x$-direction linearly-polarised electric field moving in $z$ the solution is
\begin{equation}
\mathbf{E}(r,z,t) = E_0 \frac{w_0}{w(z)} \exp\left(-\frac{r^2}{w(z)^2}\right) \exp\left[-i(kz-\omega t)-i k\frac{r^2}{2 R(z)} + i \zeta (z)\right]\mathbf{\hat{x}}.
\end{equation}
This family of solutions is referred to as Gaussian modes and the physical properties of the beam are captured in the different parameters of the wave function. The function $\mathbf{E}$ describes a beam converging from $z < 0$ at a size $w(z)$ onto a minimum at $z = 0$ achieving the highest field strengths, before diverging again for $z > 0$.
\vspace{\baselineskip}

$w(z)$ is the \textit{waist size} of the Gaussian beam at position $z$ and indicates the transverse size of the beam when the intensity drops to $1/e^2$ of its maximum:
\begin{equation}
\boxed{w(z) = w_0 \sqrt{1+ \left(\frac{z}{z_R}\right)^2},}
\end{equation}
where $w_0$ is the waist size at focus, $z=0$, and $z_R$ is the \textit{Rayleigh length} defined as 
\begin{equation}
\boxed{z_R = \frac{\pi w^2_0}{\lambda}.}
\end{equation}
$z_R$ is the distance over which a Gaussian beam expands from $w_0$ to $\sqrt{2}w_0$ and is hence an indicator for how quickly a beam diverges or in other words over which distance it maintains an intensity close to its maximum.
Assuming the beam is a Gaussian in both time and space, the intensity at the focal plane $z = 0$ is given by
\begin{equation}
I(r,t) = I_0 \exp\left( -\frac{2 r^2}{w^2_0}\right) \exp\left(-\frac{2 t^2}{\tau^2}\right).
\end{equation}
The intensity is commonly expressed in terms of the normalised vector potential $a_0$
\begin{equation}
\boxed{a_0 = \frac{e E_0}{m_e c \omega},}
\end{equation}
where $e$ is the electron charge, $m_e$ the electron mass, $c$ the speed of light, $\omega$ the frequency of the laser and $E_0$ the amplitude of the electric field. The intensity of the wave is given by $I = \epsilon_0 c E^2_0/2$, so that we can express $a_0$ in useful units for this work:
\begin{equation}
\boxed{a_0 = 0.856 \times \sqrt{I[10^{18} \mathrm{W/cm}^2]} \lambda[\mathrm{\mu m}].}
\end{equation}

$R(z)$ is the radius of curvature of the wavefront
\begin{equation}
R(z) = z \left[1+\left(\frac{z_R}{z}\right)^2\right].
\end{equation}
The wavefront is strongly curved around the focus $z < z_R$ and flattens out for $z \gg z_R$. The difference in radii of curvature for beams with different Rayleigh length can be used in spatial interferometry applied in the context of synchronising two laser pulses in time.
$\zeta$ is called the Guoy phase and is an additional phase term that changes signs once the Gaussian mode passes through its focus.

When focusing a beam with an optic, for instance an off-axis parabola or spherical mirror, the ratio of the focal length, $f$, and the beam diameter, $d$, is a useful quantity that indicates how tightly a beam is focussed. It is called the f-number, $f_\# = f/d$, and is in shorthand typically written as $f/2$ for $f_\# = 2$, $f/40$ for $f_\# = 40$ and so on.
Expressing some of the previous quantities in terms of the f-number gives
\begin{equation}
w_0 = \frac{2 \sqrt{2}}{\pi} \lambda f_\# \approx 0.9 \lambda f_\#,
\end{equation}
and the distance over which a laser pulse remains intense
\begin{equation}
z_R = \frac{\pi w^2_0}{\lambda} \approx 2.5 \lambda f^2_\#.
\end{equation}

However, real laser beams in experiments are typically not Gaussian but mostly flat-top beams. In addition, changes in the wave front, curvature profile, spatio-temporal couplings and so on consist further deviations from an ideal beam. These are commonly expressed in terms of the parameter $M^2$, with the minimum spot size being the ideal diffraction-limited spot size times $M^2$ and $M^2 > 1$ \cite{PoderThesis}.


\subsection{Laser Propagation in Plasma}
\label{Theory:Sec:LaserPropagationPlasma}
Consider a quasi-neutral plasma (charge density $\rho \approx 0$) with a current density $\mathbf{j} = -e n_e \mathbf{v}$, depending on the velocity $\mathbf{v}$ of the electrons in the plasma and the electron density $n_e$. The wave equation for the electric field $\mathbf{E}$ becomes

\begin{equation}
\nabla^2 \mathbf{E} = - \mu e n_e \frac{\partial \mathbf{v}}{\partial t} + \frac{1}{c^2} \frac{\partial^2 \mathbf{E}}{\partial t^2}.
\end{equation}

The acceleration of the electrons is due to the electric field of the wave, i.e. $\partial \mathbf{v}/\partial t = -e\mathbf{E}/m_e$. Now using the solution for a plane wave with $\mathbf{E} = \mathbf{E_0} \exp \left[i (\mathbf{k}\cdot\mathbf{r} - \omega t) \right]$ the dispersion relation is given by $\omega = c k \left[1-\left(\frac{\omega_p}{\omega}\right)^2\right]^{-1/2}$, and hence the group velocity $v_g$ and the phase velocity $v_p$ are given by:
\begin{align}
v_p &= \frac{\omega}{k} = \frac{c}{\sqrt{1-\left(\frac{\omega_p}{\omega}\right)^2}},\\
v_g &= \frac{\partial\omega}{\partial k} = c \sqrt{1-\left(\frac{\omega_p}{\omega}\right)^2},
\end{align}
where $\omega$ is the frequency of the electromagnetic wave.
At $\omega \ll \omega_p$ this the expression can be approximated by a Taylor series:
\begin{align}
v_p &= \frac{\omega}{k} \approx c \left(1+ \frac{1}{2}\frac{\omega^2_p}{\omega^2}\right),\\
v_g &= \frac{\partial\omega}{\partial k} \approx c \left(1 - \frac{1}{2}\frac{\omega^2_p}{\omega^2}\right).
\label{Theory:Eqs:Vg_Vp_medium}
\end{align}


\begin{figure}
\centering
\includegraphics[width= 0.5\columnwidth ]{GroupVelocity_density.pdf}\includegraphics[width=0.5\columnwidth]{critical_density.pdf}
\caption[Group velocity at different densities and critical densities for different laser wavelengths.]{Left: Group velocity at different densities for photons of wavelength $\lambda =$ 0.8 $\mu m$ (blue) and 10 $\mu m$ (orange) in units of the vacuum speed of light (indicated in green). The red dotted line shows the density at which the group velocity reaches zero, called the critical density. Right: Critical density for wavelengths up to 2 $\mu m$ (blue). The critical density corresponding to a central wavelength of 0.8 $\mu m$, as for the commonly used titanium-doped sapphire (Ti:Sa) lasers, is marked in red on the graph. The region above the critical density (blue line) is called overdense, below underdense.}
\label{Theory:Figs:Ncrit}
\end{figure}

The group velocity reaches zero for $\omega_p = \omega$, where $\omega_p$ is the plasma frequency, derived in the previous section and given by $\omega_p = (n_e e^2/\epsilon_0 m_e)^{1/2}$. Solving this for the density, we obtain a relation for the \textit{critical density}, $n_c$ (see also Figure \ref{Theory:Figs:Ncrit}):

\begin{equation}
\boxed{n_c = \frac{m_e \epsilon_0 \omega^2}{e^2},}
\end{equation}
which only depends on the frequency of the incident electromagnetic wave, and so we can write the refractive index $\eta = c/v_p$ in terms of the electron density $n_e$ and the critical density $n_c$:
\begin{equation}
\eta = \sqrt{1-\left(\frac{\omega_p}{\omega}\right)^2} = \sqrt{1-\frac{n_e}{n_c}}.
\end{equation}

For $n_e > n_c$, an `overdense plasma', the refractive index is imaginary and the wave becomes evanescent in the medium. This means a wave can only enter the material up to a skin depth before being reflected again.
A plasma with a density $n_e < n_c$ is transparent and `underdense'. LWFA requires the laser pulse to propagate through the plasma to drive the wave and hence targets used are typically operating at densities of only a fraction of $n_c$, whilst many ion acceleration mechanisms are based on overdense media.

\subsection{Nonlinear Refractive Index}

As seen previously, electrons interacting with a highly intense laser perform a relativistic motion (see Section \ref{Theory:Sec:SingleParticle:FigOfEight}). This results into a relativistic mass increase and in turn to a modification of the plasma frequency and the refractive index \cite{Mori1997_NONLINEARPLASMA}:
\begin{equation}
\eta = \frac{c}{v_p} = \left(1-\frac{\omega^2_p}{\gamma_\perp \omega^2}\right),
\end{equation}
where $\gamma_\perp = 1 + \frac{1}{2}a^2_0$.
Taking into consideration local modifications of the plasma density, laser frequency and laser intensity
\begin{align}
n &= n_0 + \frac{\delta n}{n_0} n_0,\\
\omega_L &= \omega_0 + \frac{\delta \omega_L}{\omega_0} \omega_0,\\
\left\langle \gamma \right\rangle &= 1+ \frac{a^2_0}{4},
\end{align}

the nonlinear refractive index then becomes
\begin{equation}
\boxed{\eta = 1 - \frac{1}{2}\frac{\omega^2_p}{\omega^2_0}\left(1+ \frac{\delta n}{n_0} - \frac{2 \delta \omega_L}{\omega_0} - \frac{a^2_0}{4}\right),}
\end{equation}
assuming that the perturbations are small and higher order terms vanish.


\subsection{Relativistic Self-focusing and Self-guiding}
\label{Theory:Sec:SelffocusingSelfguiding}

The diffraction of a Gaussian mode near focus ($z < z_R$) can be calculated by differentiating the waist size, $w$ \cite{ColeThesis}:
\begin{equation}
\frac{\partial w}{\partial\tau} = \frac{\partial z}{\partial \tau} \frac{\partial w}{\partial z},
\end{equation}
\begin{equation}
\frac{\partial^2 w}{\partial \tau^2} = \frac{4 c^4}{\omega^2_0 w^3_0},
\label{Theory:Eq:Diffract}
\end{equation}
which takes a positive value resulting in an increase of the waist size.
On the other hand, using the expression of the nonlinear refractive index and focusing on the term involving the vector potential of the laser we find
\begin{equation}
\frac{\partial^2 w}{\partial \tau^2} = \frac{c^2}{\eta}\frac{\partial \eta}{\partial r} = -\frac{c^2}{8} \frac{\omega^2_p}{\omega^2_0} \frac{\partial}{\partial r} a^2 (r,z) \approx - \frac{1}{8}\frac{\omega^2_p}{\omega^2_0}\frac{a^2_0}{w_0}c^2,
\end{equation}
which has a negative sign indicating focusing and counteracts the diffraction term in Equation \eqref{Theory:Eq:Diffract}.
To find the conditions when diffraction and focusing balance each other we set both terms equal
\begin{align}
w^2_0 a^2_0 &= 32 \frac{c^2}{\omega^2_p},\nonumber\\
w_0 &= \frac{\sqrt{32}}{a_0} \frac{c}{\omega_p},
\end{align}
and expressed in terms of laser power \cite{Sprangle1987_SELFFOCUS}:
\begin{equation}
\boxed{P_c = \frac{8\pi \epsilon_0 m^2_e c^5}{e^2} \frac{\omega^2}{\omega^2_p} \approx 17 \frac{\omega^2_0}{\omega^2_p} \,\mathrm{GW} = 17 \frac{n_c}{n_e} \, \mathrm{GW},}
\end{equation}
where $P_c$ is the \textit{critical power} or minimum power required to outweigh diffraction by relativistic self-focusing. This is referred to as \textit{self-guiding} as it enables guiding without external guiding structures \cite{Wagner1997_SELFGUIDED,Thomas2007_SELFGUIDED}. 
When balancing both terms the laser pulse remains focused over several Rayleigh lengths while oscillating in size \cite{Esarey1993_SPOTOSCILLATION}. The waist evolution is described by
\begin{equation}
\frac{\dif^2 w}{\dif z^2} = \frac{w^2_0}{z^2_R w^3}\left(1 - \frac{P}{P_c}\right).
\end{equation}
In the `bubble' regime \cite{Pukhov2002_BUBBLESIM} a highly intense laser pulse completely clears the laser axis from electrons forming a cavity of radius $r_b =  2 \sqrt{a_0} \frac{c}{\omega_p}$, the bubble radius. This cavity is void of electrons $n_e = 0$ and no focusing occurs within it. The guided spot size is matched to $r_b$ as smaller spots diffract and larger spots are focused again.

\subsection{Pulse compression}

In a wakefield accelerator an intense laser pulse propagates through a plasma and expels the electrons in its way. As a result, the front and the back of the laser pulse witness different electron densities. This in turn corresponds to a variation in group velocities [see Equation \eqref{Theory:Eqs:Vg_Vp_medium}] leading to a change in the length of the laser pulse over time \cite{Mori1997_NONLINEARPLASMA,ColeThesis}.

Consider two points in the laser pulse at $z_1$ and $z_2$, separated by a distance $L = z_2 - z_1$, and travelling at a group velocity of $v_{g1}$ and $v_{g2}$, respectively. If the laser pulse propagates for a time interval, $\Delta t$, the difference in the group velocities, $\Delta v_g$, will translate into a change in the separation of the two points by $\Delta L$
\begin{equation}
\Delta L = (v_{g2} - v_{g1}) \Delta t = \Delta v_g \Delta t.
\end{equation}

If we write the difference in group velocities as rate of change in the group velocity over the pulse length, $L$,
\begin{equation}
\Delta v_g \approx \frac{\partial v_g}{\partial z} L,
\end{equation}
such that
\begin{equation}
\Delta L \approx L  \frac{\partial v_g}{\partial z} \Delta t.
\end{equation}

If we only consider the variation of the refractive index, $\eta$, due to a change in the electron density, $n_e$, we obtain
\begin{equation}
 \frac{\partial v_g}{\partial z} =  \frac{\partial}{\partial z} (c \eta) \approx  \frac{\partial}{\partial z} c \left( 1 - \frac{n_e}{2n_c}\right) = - \frac{c}{2n_c}\frac{\partial n_e}{\partial z},
\end{equation}
where $n_c$ is the critical density.
In the bubble regime the leading edge of the laser pulse sees a density $n_e = n_0$, whereas the remaining pulse propagates approximately in vacuum with $n_e \approx 0$. If we assume that the density transition occurs linearly over a distance $l$ then $\partial_z n_e = n_0/l$, then
\begin{equation}
\frac{\partial L}{\partial t} \approx L \frac{\partial v_g}{\partial z} \approx -L\frac{c}{2n_c} \frac{\partial n_e}{\partial z} = - L \frac{n_0 c}{2 n_c l},
\end{equation}
which can be solved by the exponential function
\begin{equation}
L(t) = L_0 e^{- cn_0 t/(2n_cl)},
\end{equation}
where $L_0$ is the initial separation.
This indicates that sharper density transitions (small $l$) lead to faster pulse compression.

\subsection{Photon acceleration}

In the previous section we motivated that a laser pulse travelling through a density gradient can experience pulse compression.
In the context of wakefield acceleration the density gradient is produced by the intense laser pulse itself. A compression of the pulse duration as described has to be accompanied by an increase of the bandwidth of the laser pulse.

Similarly as before, consider two points in the laser pulse at $z_1$ and $z_2$ with phase velocities $v_{p1}$ and $v_{p2}$, respectively. If we impose that the wavefronts at these points separated by a phase of $2\pi$, then the spatial separation corresponds then to the wavelength $\lambda_0 = z_2 - z_1$. After a time interval $\Delta t$, the spatial separation of the wavefronts is now different by $\Delta \lambda$
\begin{equation}
\lambda_0 + \Delta \lambda = \lambda_0 + (v_{p2} - v_{p1}) \Delta t.
\end{equation}
Substituting similarly as in the previous section 
\begin{equation}
(v_{p2} - v_{p1}) \approx \frac{\partial v_p}{\partial z} \lambda_0,
\end{equation}
so that 
\begin{equation}
\frac{\partial \lambda}{\partial t} \approx \lambda_0 \frac{\partial v_p}{\partial z},
\end{equation}
or in terms of the wave frame variables $\xi$ and $\tau$, and the refractive index $\eta$:
\begin{align}
\frac{\partial \lambda}{\partial \tau} &= -\lambda_0 \frac{c}{\eta^2} \frac{\partial \eta}{\partial \xi},\\
\frac{1}{\omega_L} \frac{\partial \omega}{\partial \tau} &= \frac{c}{\eta^2} \frac{\partial \eta}{\partial \xi}.
\end{align}
This equation now describes how the wavelength or frequency shifts as a function of the density gradient. A positive density gradient results in an increase of the frequency, $\omega$, i.e. blue-shifting or `photon acceleration'. A negative gradient, on the other hand, would lead to a decrease in the frequency, so red-shifting or `photon deceleration'. In the previous section we considered pulse compression as a result of a positive density gradient, so we now see that pulse compression is accompanied by photon acceleration. This phenomenon has been more rigourously investigated theoretically in \cite{Mendoncca1994_PHOTONACC,Mendoncca2000_PHOTONACC} and has been confirmed in experiment \cite{Murphy2006_PHOTONACC}.
\clearpage

\section{Laser Wakefield Acceleration}
\label{Theory:Sec:LWFA}

In laser wakefield acceleration (LWFA) a short (10`s fs), intense and focused laser pulse ($a_0 \geq 1$) travels through an underdense plasma, driving a plasma wave by pushing the electrons out of its way by means of the ponderomotive force (see Section \ref{Theory:Sec:PonderomotiveForce}). In beam-driven plasma wakefield acceleration (PWFA) the ponderomotive force is replaced by the Coulomb force. On the time scale of the laser pulse the ions remain static and the Coulomb force pulls the electrons back onto the laser axis, setting up a density wave. A highly intense pulse ($a_0 \geq 2$) almost completely evacuates the region in its wake, producing a cavity with strong longitudinal and transverse electromagnetic fields, which are suited to focus and accelerate electrons to relativistic energies. This is commonly referred to as the non-linear or `bubble' regime, which enables the production of quasi-monoenergetic electron beams, first reported in 2004 \cite{Mangles2004_MONO,Faure2004_MONO,Geddes2004_MONO}. By now maximum electron energies of beams produced in LWFA have reached the multi-GeV level in one \cite{Leemans2006_GEV,Leemans2014_GEV,Gonsalves2019_GEV,PoderThesis} or two coupled acceleration stages \cite{Steinke2016_STAGING,Kim2013_GEV}.

A comprehensive overview on this topic explaining the fundamental physics is given in \cite{Esarey2009_LPA_Review}. A more recent overview on the progress and status of experiments in the laser wakefield community can be found in \cite{Mangles2016_LPA_Review}.

\subsection{Wavebreaking}

At large electric fields neighbouring electron sheets in the plasma start to cross and the plasma wave loses coherence. This is referred to as \textit{wavebreaking} and marks the breakdown of the fluid approximation of the plasma.
The maximum electric field that can be sustained by a plasma wave is
\begin{equation}
E_0 = \frac{m_e c \omega_p}{e},
\end{equation}
which is also called the \textit{cold non-relativistic wavebreaking limit} \cite{Esarey2009_LPA_Review,Dawson1959_COLDWAVEBREAKING}.
In useful units we can write:
\begin{equation}
E_0 [\mathrm{GV/m}] \approx 96 \sqrt{n_0 [10^{18} \mathrm{cm^{-3}}]},
\end{equation}
which indicates that the maximum field sustained by a wave in a plasma of density $10^{18}\mathrm{cm^{-3}}$ is roughly $100\,\mathrm{GV/m}$.
Nonlinear plasma waves can exceed this value and the wavebreaking limit for a 1D relativistic, non-linear periodic plasma wave is, for instance, given by \cite{Akhiezcr1956_Waves}
\begin{equation}
E_{WB} = \sqrt{2(\gamma_p -1)} E_0,
\end{equation}
where in the 1D low-intensity limit $\gamma_p \approx \omega/\omega_p$, and $E_0$ is again the cold non-relativistic wave breaking limit given above.

\subsection{Bubble regime}

A short and relativistic laser pulse can fully expel the electrons in its way. The pure ion cavity left behind is an approximately spherical `bubble' of radius $r_b$ \cite{Pukhov2002_BUBBLESIM}:
\begin{equation}
\boxed{r_b = 2 \sqrt{a_0} \frac{c}{\omega_p},}
\end{equation}
which can be derived by balancing the ponderomotive force with the Lorentz force acting on the electrons (see e.g. \cite{WoodThesis}).
This is a highly non-linear regime leading to wavebreaking such that wakefields are only sustained for few periods.
The electromagnetic fields associated to the cavity are given by \cite{Kostyukov2004_BUBBLEFIELDS,ColeThesis}
\begin{align}
E_z &= \frac{n_0 e}{\epsilon_0} \frac{\xi}{4},\\
E_r &=\frac{n_0 e}{\epsilon_0} \frac{r}{4},\\
B_\theta &= n_0 e \mu_0 \frac{r}{4},
\end{align}
where $n_0$ is the background ion density in the bubble and $\xi = z - v_{ph} t$ is the waveframe coordinate.
Here the region in which electrons are accelerated and focused is half of the cavity $r_b \approx \lambda_p/2$.
The effective focusing field on an relativistic electron beam is given in useful units by \cite{Esarey2009_LPA_Review}
\begin{equation}
E_r [\mathrm{GV/m}] = 9.06 \times n_e [\mathrm{10^{18}\,cm^{-3}}] \sigma_r [\mathrm{\mu m}],
\end{equation}
where $\sigma_r$ is the size of the electron beam. This focusing field forces off-axis electrons onto the axis, as a result of which the electron beam performs betatron oscillations.

\subsection{Trapping and self-injection threshold}
\label{Theory:Sec:TrappingSelfInjection}

In order to trap electrons from the ambient plasma in the wakefield cavity, the following condition has to be satisfied \cite{Thomas2010_WAKESCALINGS}:
\begin{equation}
\boxed{r_b \geq \frac{2c \sqrt{\ln (2\gamma^2_{ph}-1)}}{\omega_p },}
\end{equation}
where $\gamma_{ph} = \omega/\omega_p \approx \sqrt{n_c/(3n_e)}$ and $r_b$ is the radius of the plasma bubble. This process is referred to as \textit{self-trapping} and triggering the injection of electrons into the cavity solely through the evolution of the laser pulse is called \textit{self-injection}.

Based on the trapping condition above the minimum power required to reach self-injection, the self-injection threshold, is for the depletion-limited case estimated by \cite{Mangles2012_SELF}
\begin{equation}
\boxed{\frac{\alpha P}{P_c} > \frac{1}{16} \left[\ln\left(\frac{2n_c}{3n_e}\right) - 1 \right]^3,}
\end{equation}
with $\alpha$ the fraction of energy contained in the \textsc{fwhm} of the laser focal spot, and $P$, the laser power. $P_c$ is the laser power at which relativistic self-focusing and diffraction are balanced, with $P_c \approx 17 n_c/n_e\,\mathrm{GW}$ (see Section \ref{Theory:Sec:SelffocusingSelfguiding}).

\subsection{Depletion}

As the laser pulse propagates through the plasma it slowly loses energy through diffraction and driving the wake. The energy loss is concentrated at the front of the laser pulse that is interacting with the medium and thus slowly etches away at a velocity $v_{etch}$ \cite{Decker1996_FV}

\begin{equation}
v_{etch} = \frac{\omega^2_p}{\omega^2_0} c.
\end{equation}

The wakefield phase velocity or front velocity, $v_f$, is then given by the group velocity reduced by the etch velocity, $v_{etch}$,
\begin{equation}
\boxed{v_f = v_g - v_{etch} \approx c \left( 1-\frac{3}{2}\frac{\omega^2_p}{\omega^2}\right).}
\label{Theory:Eqs:FrontVelocity}
\end{equation}

After a certain distance, the pump-depletion length $L_{pd}$, the pulse is not intense enough any more to drive a stable wave with sufficient field gradients - the cavity collapses. The depletion length in the 3D bubble regime is given by \cite{Lu2007_3DWAKE}
\begin{equation}
\boxed{L_{pd} = \frac{\omega^2}{\omega^2_p}\frac{\omega_p}{k_p}\tau \propto \frac{1}{n_e}.}
\end{equation}

Higher plasma densities lead to faster etching of the pulse and shortens the depletion length, whilst reducing the density allows sustaining the accelerating structure over longer distances.

\subsection{Dephasing}

In the high field gradients of the wake electrons are relatively quickly accelerated to relativistic energies, reaching velocities close to the speed of light. The driving laser pulse on the other hand propagates through the plasma at a group velocity smaller than $c$. As a result, the electrons slowly catch up with the driver and leave the accelerating phase of the cavity, hence called \textit{dephasing}.
The relativistic electrons are moving close to the speed of light ($\beta_e \rightarrow 1$) whilst the plasma wave moves at a velocity $\beta_p$ (in the 1D linear case):
\begin{equation}
\beta_p = \frac{v_g}{c} = \left(1-\frac{n_e}{n_c}\right)^{\frac{1}{2}},
\end{equation}
where $n_e$ is the electron and $n_c$ the critical density.
Using this one can estimate the time it takes the electrons to move out of the acceleration phase of the wave sizing half the length of the plasma wave:
\begin{equation}
t_d = \frac{\lambda_p}{2c\left(\beta_e-\beta_p\right)}\approx \frac{\lambda_p}{c} \frac{n_c}{n_e}.
\end{equation}

The distance over which electrons leave the acceleration phase, the dephasing length $L_{dp}$, is then for the 1D linear regime
\begin{equation}
L_{dp} = \lambda_p \frac{n_c}{n_e} \propto n^{-\frac{3}{2}}_e.
\end{equation}

The dephasing length for the 3D bubble regime is given by \cite{Lu2007_3DWAKE}
\begin{equation}
\boxed{L_{dp} = \frac{2}{3}\frac{\omega^2}{\omega^2_p} r_b = \frac{4}{3}\frac{\omega^2}{\omega^2_p}\frac{\sqrt{a_0}}{k_p} \propto \sqrt{a_0} n^{-\frac{3}{2}}_e,}
\end{equation}
which is similar to the 1D linear case but using the bubble radius and considering the due to the etching reduced front velocity, $v_f$ [see Equation \eqref{Theory:Eqs:FrontVelocity}].
A higher plasma density reduces the front velocity of the laser pulse and the dephasing length.

\subsection{Maximum Energy Gain}

Wakefield cavities support several orders of magnitude higher electric fields than conventional radio-frequency cavities. For a 3D non-linear wake the maximum electric field that can be supported by the plasma wave is given by \cite{Lu2007_3DWAKE}
\begin{equation}
\boxed{E_{max} = \sqrt{a_0} m_e c \omega_p/e \propto \sqrt{a_0 n_e}.}
\end{equation}
A higher plasma density results in a larger amount of separated charge and hence a stronger electric field which is desirable for our accelerator.

On the other hand, both limiting factors introduced in the previous sections, depletion and dephasing, are inversely proportional to the plasma density. Whilst the depletion length can be extended by using a more intense laser pulse, the dephasing length at a constant plasma density remains a fixed limitation. Solutions to overcome this limit include, for instance, increasing the density either gradually or abruptly to rephase the electrons \cite{Guillaume_REPHASING} and/or employ multiple lasers \cite{Debus2019_BEYONDDEPHASING} and acceleration stages injecting the electron bunch into the accelerating phase of the new wakefield \cite{Steinke2016_STAGING}.

The maximum energy gain is the optimised product of the electric field and the acceleration distance. By assuming that ultimately the dephasing length is the limiting factor, the maximum energy gain $W_{max}$ for a 3D non-linear wake at matched conditions is given by \cite{LuThesis,Lu2007_3DWAKE}
\begin{equation}
\boxed{W_{max} = \frac{2}{3}a_0\left(\frac{\omega_L}{\omega_p}\right)^2 m_e c^2 \propto \frac{a_0}{n_e}.}
\end{equation}

This gives the counter-intuitive result that one has to choose low plasma densities (corresponding to lower field gradients) to accelerate particles to higher energies. 
A variety of injection mechanisms have been proposed to optimise the properties of the electron beam (internal \cite{Faure2006_STABLEJET,Geddes2008_STABLE,Pak2010_ION,Buck2013_SHOCK} and external \cite{Deng2019_TROJAN,Steinke2016_STAGING}).
%In addition, this means that one has to balance between high beam charge (requiring high $n_e$) and high peak energy. 
%To solve this dilemma and optimise charge and energy, a variety of injection mechanisms (internal \cite{Faure2006_STABLEJET,Geddes2008_STABLE,Pak2010_ION,Buck2013_SHOCK} and external \cite{Deng2019_TROJAN,Steinke2016_STAGING}) have been proposed. 

\subsection{Injection Mechanisms and Accelerator Tailoring}
\label{Theory:Sec:InjectionMechanismsAccTailor}

In Section \ref{Theory:Sec:TrappingSelfInjection} we introduced the self-trapping condition and the threshold for self-injection. In 1D this threshold is equivalent to the wavebreaking limit, whereas in 3D the crossing of electron sheets that trigger the injection can also occur without catastrophic wavebreaking. 
Other injection mechanisms have been developed to produce electron beams with specific properties, for instance high charge or low energy spread. Out of the plethora of different techniques we will only introduce the mechanisms used in the context of this thesis, which are, besides self-injection, ionisation and shock injection.

\subsubsection{Ionisation Injection}

Electrons can be released directly into the bubble cavity through delayed ionisation of the atoms or molecules close to the peak intensity of the laser pulse. 
Higher-Z gases bind electrons strongly and require higher field strengths to release them. When the intense laser pulse passes through the plasma it triggers an increased rate of tunnel ionisation near the centre of the pulse (see Section \ref{Theory:Sec:PlasmaFormation} on plasma formation). The released electrons can then be trapped and subsequently accelerated. This is called \textit{ionisation injection}.

Typically, a gas with low ionisation threshold like helium or hydrogen is used as base, which is then doped with a few percent of high-Z gas, for instance nitrogen, to provide the injection events. The interplay of the laser, its evolution and intensity, and the properties of the dopant determine the amount of charge released into the cavity and the duration of the injection event. The duration of the injection translates into energy spread (Liouville's theorem, see e.g. \cite{ColeThesis}) and hence provides a way to tune the properties of the accelerated electron beam.

\subsubsection{Shock Injection}

The accelerating structure in a wakefield depends on the complex interplay of the driving laser pulse and its evolution in interaction with the surrounding medium. An evident handle onto the shape and behaviour of the cavity is the plasma density as $r_b \sim 1/\sqrt{n_e}$ and the changes in the group velocity $\Delta v_g \sim - \Delta n_e$.

An abrupt fall-off in plasma density leads to a temporary elongation of the bubble in the longitudinal direction as the effective velocity of the back of the bubble is reduced for a short time. Electrons can now be trapped inside the cavity as part of a very localised injection event. Since such a sharp density transition can in experiments be achieved by inducing a shock into the gas, this injection mechanism is called \textit{shock injection}.
Shock injection has been demonstrated as powerful tool to produce high-quality electron beams of high charge, stability and tunable energy \cite{Schmid2010_SHOCK,Buck2013_SHOCK}.
Similarly, but less localised, an extended density gradient can also trigger the injection of electrons by gradually extending the bubble size, then called \textit{downramp injection} \cite{Bulanov1998_Downramp}.
In contrast, a positive density gradient will lead to a decrease in the size of the bubble. A decrease in bubble size is not suitable to capture electrons but it moves the centre of the cavity closer to the laser pulse shifting the accelerating phase. This can be used to keep electrons for a longer distance in an accelerating phase to delay dephasing, and is hence referred to as \textit{rephasing} \cite{Guillaume_REPHASING}.


