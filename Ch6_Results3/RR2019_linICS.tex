\chapter{Linear Inverse Compton Scattering at Astra Gemini}

\section{Motivation}

\section{Experimental Setup}

This experiment was performed at the Astra Gemini facility in early 2019 using both laser arms of the dual laser beam facility.

The first laser pulse is focused with an f/40 off-axis parabola onto the edge of a 15 mm conical supersonic helium gas jet. The measured pulse duration was around 42 femtoseconds with an average energy on-target of about XX NUMBER. The size of the focal spot was around XX XX UM, reaching an a0 of around XX NUMBER.

The second laser pulse is focused down tightly by an f/2 off-axis parabola at the other end of the gas jet in a head-on collision. The f/2 OAP has a hole in the centre to allow the electrons and radiation from ICS to propagate through. A round piece of plastic is also fitted around the hole to protect the optic from scattered light from the other laser. This reduces the energy by around 7.84 percent which brings the laser energy on-target to around XX NUMBER on average at a pulse duration of 42 femtoseconds. The minimum spot size reached was about XX NUMBER, which translates into a normalised vector potential of a0 ~ NUMBER. The f/2 OAP is in addition protected from debris by a thin pellicle. The laser is linearly polarised in the vertical plane.

The radiation and the electrons propagate through the hole into a large aperture permanent magnet.
The electrons are dispersed in the horizontal plane and propagate through a kevlar-kapton vacuum window.
A scintillating lanex screen measures the spectrum just after the window. Another lanex screen captures the spectrum about 1 m further downstream.
The electron spectrum is cut off at around 200 MeV as lower energetic electrons hit the yoke of the magnet and do not make out of the magnet.

The radiation propagates straight through the magnet and is damped by a 10-12 layer aluminium foil beam block and the kevlar-kapton window. Hard and bright radiation is also captured by the first LANEX screen as it extends beyond the axis. A scintillating stack of caesium-iodide crystals is measuring the profile of the radiation, another long stack of crystals can be used to retrieve the spectrum of the radiation.

\section{Electron Spectra}

\section{Timing procedure}

\section{Raster Scan}

\section{GEANT simulations}

\section{Radiation signal}

\section{Conclusion}