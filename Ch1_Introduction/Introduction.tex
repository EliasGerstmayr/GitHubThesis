

\chapter{Introduction}
\label{Chap:Introduction}



\cite{Danson2019_PWLASERS}
CN Danson (2019): Petawatt and exawatt class lasers worldwide
ICS in Nature https://physics.aps.org/articles/v12/87


\vspace*{\fill}

\section{Thesis Outline}

This thesis puts its focus onto the versatile capabilities of laser wakefield accelerators and high-intensity lasers to generate different kinds of highly energetic radiation. It provides experimental examples spanning a range of photon energies from few $\mathrm{keV}$ to several hundreds of $\mathrm{MeV}$, and discusses their application in studies of fundamental phenomena of quantum electrodynamics (QED). 

After an introduction of \nameref{Chap:Theory} and \nameref{Chap:Methods} underpinning this work, the experimental results of this thesis will be presented in Chapters \ref{Chap:linICS}, \ref{Chap:RR15} and \ref{Chap:BW}:

\begin{enumerate}
\setcounter{enumi}{3}
\item \textbf{Linear Inverse Compton Scattering and Beam Profile Diagnostic:}\\
Highly relativistic electrons from a wakefield accelerator were collided with a defocused laser pulse at $a_0 \sim 0.3$ generating sharply peaked gamma radiation from linear inverse Compton Scattering in the 10s $\mathrm{MeV}$ range. A novel gamma-spectrometer design that allows measuring the response of the detector from two sides was fielded for the first time. The application of linear ICS as non-invasive electron beam profile diagnostic is explored to measure divergence and pointing.
\item \textbf{Nonlinear Inverse Compton Scattering and Radiation Reaction:}\\
By scattering a tightly focused high-intensity laser pulse from a relativistic electron beam and generating gamma radiation from nonlinear Inverse Compton Scattering, it was possible to measure radiation reaction in an experiment. The energy loss of the electron beam and the spectrum of the radiation were used to pinpoint the interaction conditions and their agreement with theoretical models was tested. In particular, a statistical analysis of the electron beam fluctuations and its impact on the statistical significance of the results is presented.
\item \textbf{Bremsstrahlung source and the Linear Breit-Wheeler Process:}\\
An LWFA electron beam is passed through a solid target to produce bremsstrahlung with photon energies of 100s of MeV. The bremsstrahlung source is commissioned and optimised for yield and noise. The bremsstrahlung source is combined with a $\mathrm{keV}$ X-ray source in an attempt to measure the elusive linear Breit-Wheeler process. 
\end{enumerate}

Finally, the results are discussed and their impact for future research evaluated. As conclusion an experimental setup for a future precision measurement of radiation reaction is presented, taking into considerations the findings and insights of the results presented previously. 
\EliasComm{Review thesis outline}

\vspace*{\fill}


%\item \textbf{Hard Betatron Radiation from dual shock features:}\\
%Electrons travelling in a wakefield transitioning through two density spikes experience an increase of transverse oscillations resulting in an enhancement of measured betatron radiation reaching hundreds of $\mathrm{keV}$. The dual shock structure was induced by introducing a blade into the supersonic gas flow of a helium gas jet and measured by a stack of scintillating crystals.