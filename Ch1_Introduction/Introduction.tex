
\chapter{Introduction}

\section{Motivation}

%Since the beginning of mankind light has been a central and essential concept, from the sun as life-spending source during the day to fire to spend light and warmth and to prepare food. The significance of light and the sun in particular for early and high cultures can be seen in its appearance in a wide and diverse range of myths and legends, Echnaton, Ra,.... and so on DETAILS HERE.

%In addition, light has not only taken the role of nourishing and life-spending source but has been an essential tool and symbol of discovering the new and unravelling the unknown. Its importance for clarifying and contributing to our understanding is visible in our language through a variety of phrases, sayings and metaphors combining light and the gain of knowledge or better understanding: `shedding light on it', `bringing light into the darkness', 'illuminating something' or `seeing something in a new light', whilst darkness or the absence of light often means the opposite as for instance in `being in the dark'. Light and brightness often has a positive connotation, think of the words `brilliant' (from French for `shining') or `shady' as a negative word. On the other hand in the religious context, for instance in the new testament when St. Paul in his Damascus experience `sees the light' and changes his ways or even earlier in Greek mythology when the titan Prometheus stole the fire for the humans, one of the great deeds for humankind punished by the gods.

%Fire and light have been a central motif for humanity since the very beginnings and in fact life itself would not have been possible without us orbiting in this very distance around the great white light source we call sun, providing us with an average of $1 kW/cm^{-2}$ on zenith of power every day from clean fusion energy, subject to geographical, seasonal and daily variations.
%\vspace{\baselineskip}

%ADD SOME IMAGES HERE FOR THE HISTORICAL REFERENCE
%\vspace{\baselineskip}

%From banishing the night, cooking food, to burning the ships of enemies or trapping single atoms; the applications for light and the demands on its properties have changed drastically over the centuries and even the last years. In science, light or more generally electromagnetic waves in its very different appearances have been used to understand the property of materials, processes and reactions, at smaller spatial and faster time scales, creating the demand for sources at smaller wavelengths and at shorter pulse durations along with high brightness and repetition rates.
%\vspace{\baselineskip}

Since the systematic characterisation of X-rays by Wilhelm Roentgen in Wuerzburg in 1895 \cite{Rontgen1895}, honoured with the first Novel Prize in Physics in 1901, light sources have revolutionised imaging in all fields of science from physics to material science, from biology to chemistry and medicine, making them an indispensable tool in research and daily life. Whilst the discovery of the X-rays lead to the ability to penetrate flesh to see bones and metals inside the body depending on the absorption in the material, the following generations of light sources have further pushed the boundary of what we can observe. Bright and short pulsed light sources enable us, for instance, to understand chemical reactions and to probe the structure of proteins.
\vspace{\baselineskip}

\begin{figure}
\centering
\includegraphics[height=0.6\columnwidth]{medX_Roentgen.jpg}
\includegraphics[height=0.6\columnwidth]{Radiograph_hand.jpg}
\caption{Comparison of two radiographs of adult hands. The image on the left is the first medical X-ray in history, titled `Hand mit Ringen' (`Hand with rings') and taken by Wilhelm Roentgen in 1895. The second image is a modern radiograph\protect\footnotemark showing the individual bones very clearly as well as the silhouette of the hand itself.}
\end{figure}

It has become a custom to categorise the different light sources in generations. By now there are four distinct generations: 
The first generation of light sources are the particle accelerators built for other purposes than harvesting the light itself, for instance particle and nuclear physics. Synchrotron radiation generated through the deflection of charged particles was a by-product and used parasitically.

The beamlines of the second generation were adjusted to specifically generate synchrotron radiation, using dipole bending magnets.

The third generation optimised the gain of radiation and its properties by adding so-called insertion devices, periodic arrays of oppositely polarised magnets called wigglers or undulators, and components to select and improve the spectrum of the radiation depending on the application, for instance monochromators (several crystals where the right frequency will satisfy the Bragg condition).

The latest generation is represented by free electron lasers (FELs), using state-of-the-art technology to achieve coherent radiation with high spectral brightness and short pulse durations to image ultra-fast reactions. The energy of the radiation produced in FELs even reaches the so-called water window between the carbon and oxygen K-absorption edges at $0.28$ and $0.53\,\mathrm{keV}$ respectively where water is almost transparent. This gives the opportunity to image objects in aqueous solutions which is of particular interest for some samples in biology. 
\vspace{\baselineskip}

\footnotetext{Image taken from WikiRadiography: \url{http://www.wikiradiography.net/page/Hand+Radiographic+Anatomy}}

Whilst Roentgen's achievement at the time was revolutionising medicine by laying foundation for radiography, it also made us understand that high-energy light sources are based on the acceleration and deceleration of charged particles.
In most cases, sources of X-rays are particle accelerators, be it the deceleration following an acceleration in a X-ray tube or the oscillation of accelerated electrons in a plasma cavity, all this phenomena are linked to particle acceleration and deceleration.
Exceptions are, for instance, radioactive decays or some astrophysical phenomena.
\vspace{\baselineskip}

Particle accelerators have improved greatly the maximum energy as well as the beam quality in terms of emittance, luminosity and charge. However, acceleration gradients in RF cavities, used in all conventional accelerators, are limited to tens of $\mathrm{MV/m}$ or around $100\,\mathrm{MV/m}$ in superconducting cavities designed for the Compact Linear Collider (CLIC) \cite{Aicheler2014}. Beyond these gradients the cavities start to break down and gas out, deteriorating the beam quality and posing a safety hazard at the same time.
In order to generate higher energy and shorter pulses at high brightness, or in order to reach higher particle energies in general, accelerator structures have to be scaled up to meet the requirements of the science community.
This results in large and expensive projects, reaching the limits of what might be financially and in some cases even geographically feasible.
An alternative to RF technology is the use plasma-based structures that can support arbitrarily high electric fields as the medium plasma is `broken down' already by definition. 
\vspace{\baselineskip}

\begin{figure}
\centering
\includegraphics[height=0.3\columnwidth]{lhc_aerial.jpg}
\includegraphics[height=0.3\columnwidth]{harwell_edit.png}
\caption{Aerial view on the Large Hadron Collider (LHC) at CERN (left) with a 27 km circumference, accelerating protons to $7\,\mathrm{TeV}$. On the right, the Diamond Light source with a circumference of more than 500 m, generating synchrotron radiation using electrons travelling at around $3\,\mathrm{GeV}$, and encircled the Astra Gemini laser system, also a $\mathrm{GeV}$-scale particle accelerator.}
\end{figure}

In this context laser wakefield acceleration (LWFA) has been shown to be a feasible and promising alternative or, in some implementations, complementary route, exhibiting favourable properties for radiation such as short pulse duration, decent brightness and micrometer source size and accelerating particles to few $\mathrm{GeV}$ in centimetres. In LWFA a short and intense laser pulse drives a plasma wave using the ponderomotive force, leaving a positively charged cavity tailing in its wake that can be used to accelerate trapped particles. Since 2004 quasi-monoenergetic electron beams at $100s\,\mathrm{MeV}$ \cite{Mangles2004,Geddes2004,Faure2004}, then at one to few GeV with an energy spread of a few percent and a relatively low divergence of a few mrad have been produced routinely in experiment \cite{Leemans2006,Wang2013,Leemans2014}. 
In addition, these accelerators can provide bright and extremely short pulsed X-rays at short wavelengths and with micrometer-scale source size. Electrons injected into the positive space charge in the wake of the driver start to oscillate around the laser axis (betatron oscillation) due to focusing fields in transversal and longitudinal direction, resulting in the emission of X-rays in a forward-facing cone, also called `betatron radiation'. Betatron radiation has been demonstrated to have favourable properties for phase contrast imaging \cite{Cole2015} and has the potential to track ultrafast reactions.
\vspace{\baselineskip}

Recently, the development of beam-driven, i.e. particle-driven, wakefield accelerators has been engaged in more ambitiously and with great success \cite{Blumenfeld2007,Caldwell2009}. Similarly to laser pulses, bunches of charged particles, for instance electrons, positrons or protons, are able to drive plasma waves using the electric field of their space charge \cite{Chen1985}. Although the process has many parallels some limitations and the control over the injection might differ in relation to LWFA.
\vspace{\baselineskip}

In this context, DESY in Hamburg is setting up a new project called FLASHforward. It is planned to drive a wakefield accelerator using the high quality electron beam of the FLASH accelerator and to generate an electron beam with significantly higher energy than the FLASH beam \cite{Aschikhin2016}. The resulting high energy electrons can then be used subsequently in a free electron laser to access keV X-rays with exceptional brightness. The betatron radiation will constitute an important diagnostic tool for the properties of the electron bunch inside the plasma accelerator or could even act as a useful X-ray source itself.
\vspace{\baselineskip}

Although these novel type of accelerators still depend on relatively large laser systems or conventional particle accelerators as drivers at the moment, it strongly indicates the potential for a new generation of compact particle accelerators and X-ray sources which opens up a wide field of possible applications in particle physics, medicine \cite{Cole2015},  industrial production or defence technology -- just to mention a few.

\section{Particle Accelerators}

\subsection{Conventional Accelerators}

From the electrostatic accelerators of the early 20th century to the soon to be built state-of-the-art accelerators involving high-temperature (HT) superconductors, the design, performance and finally the energy gain has developed impressively, so as the impact of research related to them. Evaluating the progress in accelerator science over the almost a century Livingston noticed that a factor of 10 in energy increase was achieved every 6 years (see figure \ref{Intro:Figs:Livingston}). Furthermore, stability, emittance and luminosity were improved leading to more and more discoveries in particle physics, including the latest achievement of the CERN facility, the discovery of the Higgs boson, the last missing piece of the Standard Model (SM) \cite{AtlasCollaboration2012}.

However, conventional accelerators have not only served the purpose of pushing frontiers in particle physics but have provided the means for research in other disciplines and a wide range of application, e.g. cancer therapy in medicine, imaging and many more.
\vspace{\baselineskip}

Even though the design, energy retrieval and other related technology have progressed immensely, the basic concept has remained somewhat similar over the last decades:


\begin{figure}
\centering
\includegraphics[width=0.6\columnwidth]{Livingston_COM.png}
\caption{Livingston plot\protect\footnotemark: years of first physics versus center-of-mass (CoM) energy per constituent for ee- and hadron-colliders. The energy increased exponentially over the last decades and doubles every 2 to 3 years.}
\label{Intro:Figs:Livingston}
\end{figure}


metal cavities are used as alternating RF resonators for electromagnetic waves to push the particles step by step to be accelerated forwards, giving them a fixed energy amount per cavity and, in case of circular accelerators, per turn. The frequency and sizes are matched with the beam motion which means to accelerate the particles even further, either the acceleration distance or the gradient have to be increased. Scaling up can turn out to be quite costly while a higher gradient poses a problem as high electric fields can lead to a gasing out of the material being used. Conventional accelerators have and continue to provide the highest energies, luminosities and lowest emittances ever seen, but at the same time are pushing towards a saturation.

%\subsubsection{Electrostatic Accelerators}

%Cockcroft-Walton generator

%Van-de-Graff accelerator

%Maximum energy?

%\subsubsection{Cyclotrons}

%Early accelerator type: permanent magnet and RF field to accelerate particles whilst spiralling outwards.
%Cyclotrons are still in use today in nuclear medicine to produce radionuclides.
%The largest cyclotrons can accelerate protons up to energies of 500-700 MeV.

%Invented by Ernest Lawrence in 1934, awarded with the Nobel Prize 1939

%synchrocyclotrons go up to GeV level but can only work in pulsed mode due to change in frequency
%or isochronous cyclotron with a change in B field (suitable for CW operation)


\subsubsection{Linear Accelerators}
\footnotetext{Image taken from `The Evolution of Particle Accelerators and Colliders' by Wolfgang Panofsky. Available at \url{http://www.slac.stanford.edu/pubs/beamline/27/1/27-1-panofsky.pdf}}
The first linear accelerator (LINAC) was built by Wideroe in 1924 and accelerated up to an energy of $50\,\mathrm{keV}$ in a single drift tube with two gaps. Since then the scale of the accelerators has increased significantly, stretching several miles and reaching a million times higher energies than Wideroe's tube.
\vspace{\baselineskip}

One design of a LINAC consists of an array of drift tubes without an electric field alternating with gaps over which the particles are accelerated using RF fields. The length of the drift and acceleration distances have to be matched with the RF frequency. 
If the particles are accelerated from non-relativistic energies, the tubes to accelerate the particles change size to match the change of velocity $v$ until they reach relativistic energies and velocities $v \approx c$. The length of the drift tubes $l$ matches the radio-frequency $f_{rf}$ in use with $l = v/2f_{rf}$.

A different design consists of coupled cavities designed to resonate with the specific RF frequency.
\vspace{\baselineskip}

\begin{figure}
\centering
\includegraphics[height=4.5cm]{DT_LINAC.png}
\includegraphics[height=4.5cm]{CC_LINAC.png}
\caption{Two different LINAC designs: drift tube (DTL, top) and coupled cavities LINAC (CLC, bottom)\protect\footnotemark.}
\end{figure}

An advantage of LINACs is that particles are only accelerated in a straight line and do not pass through bends that would lead to the energy loss through radiation. The absence of bending magnets and a slight bend in the trajectory also removes complications in the alignment. Compact linear accelerators relying on superconducting components supporting high field gradients such as CliC have been put forward as counter-proposals to the incredibly large scale facilities like the LHC at CERN.

\footnotetext{Images by Ciprian Plostinar (RAL), taken from the JAI Accelerator Physics Course in 2015/2016 (Suzie Sheehy).}
On the other hand, the structures can only be used once and particles can not be stored and re-used after the interaction but new particles have to be accelerated again. If the particles are to be collided with other particles to reach a high center-of-mass energy and probe interesting phenomena, the total acceleration distance has to be divided between both beam lines. In some cases the LINAC serves a purpose were a collision is not necessary and this does not pose a problem, e.g. solid-target experiments or free-electron lasers (FEL).
\vspace{\baselineskip}

LINACs are often also used as injectors for larger circular accelerators as the ability to synchronise the magnetic fields with the particle energy is limited to a certain energy range and response time. 

\subsubsection{Circular Accelerators}

Instead of using cavities with high field gradients, one can use components with lower field strengths and guide particles several times through them by arranging them in a circle. Circular accelerators are composed of bending, focusing and acceleration segments. Whilst the bending and focusing segments consist of magnets (dipole for bending and quadrupole or higher orders for focusing), the acceleration is achieved in radio-frequency (RF) cavities just as in LINACs.
\vspace{\baselineskip}

As the energy of the particles increases the magnetic fields have to be adjusted to keep them on a circular trajectory with a constant radius of curvature. Energy gain and field strengths have to be synchronised, hence the name synchrotron. 
However, the range over which the fields can be adapted is limited, just as the response time of the components. That is why larger structures and acceleration regimes, e.g. LHC at CERN from $0$ to $7\,\mathrm{TeV}$, requires several structures to be combined, particles are accelerated step by step and then injected into the next structure as the requirements on cavities and magnets change drastically.
\vspace{\baselineskip}

Circular accelerators enable re-using components several times to accelerate the same particles as well as storing particles over a certain period of time and even re-using particles that have not interacted in a collision.

At the same time, the circular motion leads to dissipation of energy through synchrotron radiation. Especially light particles like electrons are strongly affected and lose constantly energy: the energy gain has to at least compensate the energy loss. This imposes a limit on the maximum energy one can reach and introduces additional technological considerations like cooling and radiation safety.

The energy loss per turn per electron $U_0$ is given by
\begin{equation}
U_0 (keV) = \frac{e^2 \gamma^4}{3 \epsilon_0 \rho} = 88.46 \frac{E(GeV)^4}{\rho(m)},
\end{equation}
where $\rho$ is the bending radius.

The energy loss per turn scales with the energy of the particle to the fourth power which means the energy dissipation increases significantly when the particles reach higher energies.

One approach to reach high center-of-mass (CoM) energies and losing less energy to radiation is accelerating heavier particles as they radiate less ($1836^4$ times less!). A disadvantage of these hadron colliders is that these particles are not fundamental and the CoM energy is shared amongst their constituents, the partons. The resulting collision products are more complex and the calculations related are very involved.

\subsubsection{Future Outlook}

As new generations of accelerators enter the global stage, it appears that the field strengths achievable in conventional accelerators (using RF cavities) are slowly reaching their saturation limit. Smaller gains are achieved by using superconducting components, which become more feasible as high temperature superconductors mature in their applicability.

However, these seem to be only smaller adjustments to an inevitable destiny all conventional accelerators face if new and interesting science is to be probed: scaling up by building bigger facilities which is linked to a variety of technological and logistical challenges. Famous hypothetical and hyperbolic examples are the Fermitron and the Planckatron demonstrating the unrealistically large scales researches would require to investigate exciting fundamental physics based on conventional technology used over the last decades. Considering, for instance, the LHC at CERN in Geneva, an expansion with conventional technology is soon to reach its limits -- of feasibility but also geographically.
\vspace{\baselineskip}

It could be the time to consider new technologies and approaches alongside or included into existing conventional accelerators, a by now very matured technology. However, one should be careful to realise that every new technology -- as promising it might be -- will initially be significantly inferior in terms of stability, energy, luminosity and emittance. Nevertheless, it is necessary in the long run as the limits of conventional accelerators become evident.

\subsection{Plasma based accelerators}

Since its proposal by Dawson and Tajima \cite{Tajima1979}, plasma based accelerators have been able to make promising progress over the last decades. After the breakthroughs in 2004 \cite{Mangles2004,Faure2004,Geddes2004} (quasi-monoenergetic beams at hundreds of MeV), 2006 \cite{Leemans2006} (electrons accelerated to GeV scale energies) and 2007 \cite{Blumenfeld2007} (energy doubling at SLAC) major facilities have started investigating and investing in laser or plasma wakefield acceleration (LWFA/PWFA) to complement their setup and explore the promising possibilities \cite{Gschwendtner2016,Aschikhin2016,Seryi2015}.

The compelling potential of plasma based accelerators are the high electric field gradients ($\sim 100\,\mathrm{GV/m}$) supported by the medium, enabling the acceleration of charged particles to relativistic energies on a much smaller scale than conventional accelerators.

\subsubsection{Laser wakefield acceleration (LWFA)}

In a laser-driven wakefield accelerator a high intensity short-pulse laser ($I\geq 10^{19} W\,cm^{-1}$) propagates through a plasma, where it displaces electrons from regions of high fields of the wave front via the ponderomotive force creating a positive space charge in its wake. The electrons, feeling the restoring force of the field, return to their initial position and overshoot: the laser pulse drives a relativistic plasma wave that supports electric fields of up to $100\,\mathrm{GV/m}$ -- 3 orders of magnitude beyond the break-down limit of materials in conventional accelerators using radio-frequency technology \cite{Aicheler2014} and hence, exhibiting the potential to build more compact particle accelerators.
\vspace{\baselineskip}

However, the maximum energy reached by the electrons, either injected via relativistic self-injection \cite{Modena,Mangles2012} or more sophisticated external injection mechanisms \cite{Clayton1993,Everett1994,Li2013a,MartinezdelaOssa2013a,Xi2013}, is restricted by the acceleration lengths achievable:
the laser pulse depletes as it deposits energy to drive the wave, and injected electrons quickly catch up with the driver as they become relativistic (dephasing). The acceleration length is maximized for lower plasma densities at cost of electrons available for self-injection and field strengths occurring. This may pose a problem for many imaging applications requiring high particle fluxes.
\vspace{\baselineskip}

Whilst the scale is impressive, the energy spread, emittance and charge is below those of comparable conventional accelerators and much research has to be done.


\subsubsection{Plasma wakefield acceleration (PWFA)}

Plasma wakefield acceleration describes a similar process as LWFA but using a particle beam to drive the wave instead of a laser pulse. Examples of feasible particle drivers are protons \cite{Caldwell2009}, electrons \cite{Chen1985} or positrons \cite{Lee2001}.

The research in this direction is a bit more recent as the access to high quality particle beams is constrained to large particle accelerator facilities. However, after the promising results in 2007 at the Stanford Linear Accelerator (SLAC) \cite{Blumenfeld2007} other major facilities have intensified their efforts to harness the synergy of conventional and plasma technology.

PWFA promises to overcome the issue of dephasing and accelerate particles over longer distance at higher stability, propagating much closer to the speed of light and avoiding non-linear effects ubiquitous in LWFA.

\section{Light Sources}

\subsection{Terminology}

Light sources are emitter of light where `light' is a very vague and general term, often referring to electromagnetic waves in the visible range.

The author will focus on sources that emit radiation in the X-ray regime instead and call these in the following `light sources' as it is also common in the science community: examples are for the Diamond Light Source at the Rutherford Appleton Laboratories (RAL,UK) or the Advanced Light Source (ALS) at the Lawrence Berkeley National Laboratories (LBNL) in California. The field of X-ray sources in itself is already very diverse as the author tries to outline in the following sections.
\vspace{\baselineskip}

As there is a wide range of X-ray sources it is important to develop a terminology to characterise and categorise the sources regarding the radiation process and the properties of the radiation emitted. Some terms might appear quite intuitive and their meaning obvious but to avoid confusion some central expressions will be discussed here.
\vspace{\baselineskip}

The \textbf{source size} is spot size from where the radiation is emitted, usually given as diameter in units of length. In many imaging techniques a small source size is preferred as it improves the resolution of the final image. This can, for instance, be achieved by focusing down the electron beam used to generate the X-rays.

The X-ray sources discussed in the following emit radiation usually in a preferred direction, i.e. are highly directional. From the source the majority of the radiation is emitted within a cone with an opening angle $\theta$, called the \textbf{divergence}. A beam with low divergence is also referred to as \textbf{collimated}.

A source with a small source size and divergence also has a low \textbf{emittance} which is the product of the both.
\vspace{\baselineskip}

The \textbf{bandwidth} (BW) is the range of frequencies part of the spectrum emitted.

The \textbf{tunability} on the other hand describes how easily different energies can be accessed with the source. Either in terms of how variable the source is itself or by having a decent flux over a large bandwidth and a range of monochromators to select different energies.

One of the key terms is the \textbf{brightness} or sometimes also called \textbf{brilliance} of a source, given in units of number of photons in $0.1\%$ of the bandwidth around the the central wavelength or frequency per unit time, angular divergence squared, per unit cross section:

\begin{equation}
\mathrm{brilliance} = \frac{\mathrm{photons}}{\mathrm{second} \cdot \mathrm{mrad}^2 \cdot \mathrm{mm}^2 \cdot 0.1 \% \mathrm{BW}}
\end{equation}

This means this takes in account several factors like how broadband is the source, how short pulsed, how divergent and so on, but all of this at the same time expressed in just one number. Whilst the brightness is a good indicator when comparing light sources, one has to be careful to jump conclusions, especially when comparing two very different light sources. One source might be called `very bright' due to its high flux of photons whilst the other source might be equally `bright' but mainly due to a very short pulse duration. This is, for instance, the case for 3rd generation synchrotrons versus betatron radiation from laser wakefield acceleration (LWFA).

This should not be confused with \textbf{intensity} which in turn is the power per unit area, usually given in terms of $\mathrm{W}\,\mathrm{cm}^{-2}$.
\vspace{\baselineskip}

Finally, talking about X-rays sometimes one might encounter the term \textbf{gamma rays} instead. This is a somewhat loosely defined term as its meaning varies depending on the field. In nuclear physics gamma rays are related to nuclear processes, i.e. the radiation is emitted by interactions within the nucleus, whereas in this definition X-rays are related to mechanisms involving the shell, i.e. the electrons. Traditionally, the origin of the radiation was immediately also linked to a typical distinct energy regime with radionuclides being the sources of the most energetic and penetrating electromagnetic radiation available. Whilst some fields are insisting on a distinction of the terms X-ray and gamma ray in terms, for instance nuclear physics, other fields use the term to relate to an energy scale where gamma rays basically are equivalent to `very hard X-rays', e.g. in astrophysics. The author will use the second definition and use the term gamma ray to emphasise that the X-rays are particularly hard.

\subsection{Light Sources in Nature}

\subsubsection{Radionuclides}

A source of hard X-rays or gamma rays in nature is the decay of radioactive nuclides, which also also gave gamma rays their name as third type of emission after alpha and beta particles (or sometimes traditionally referred to radiation). Gamma radiation is emitted through de-excitation of the nucleus, often following an alpha or beta decay that leaves the nucleus in an unstable and excited state. In some cases the gamma radiation interacts with electrons in the shell of the atom.

The variety in the composition of radioactive isotopes leads to a wide range of energy states and energies the radiation is emitted at. Despite their wide range, in nuclear physics all radiation from the nucleus is referred to gamma radiation.

The decay of radionuclides is a stochastic process with characteristic half-lives and activity depending on the composition of the material. If these properties are well known, radioactive isotopes are frequently used to calibrate diagnostics, also in the context of X-ray diagnostics in LWFA.

Gamma radiation was first noticed by Paul Villard whilst investigating the radionuclide radium but did not recognise it as a distinct type of radiation from alpha and beta particles, which Ernest Rutherford only demonstrated more than a decade later.

\subsubsection{Cosmic Radiation}

Whilst most readers might think of our sun immediately when prompted with light sources in nature and cosmic radiation, the author will divert the reader's attention to sources of X-rays and gamma radiation coming from the depths of space.
Since despite its major importance to provide and sustain the conditions for life on Earth, the sun mainly radiates in the infrared, visible and ultraviolet wavelength regime, making it not particularly interesting when considering cosmic X-ray sources.
\vspace{\baselineskip}

Open space is busy with different types of particles and radiation from various sources overlaying each other, some just recently emitted, some having travelled for light-years and some are remnants from the very first moments of our universe after the Big Bang.
\vspace{\baselineskip}

\begin{figure}
\centering
\includegraphics[width=0.5\columnwidth]{AGN.png}
\caption{Sketch of an active galactic nucleus (AGN)\protect\footnotemark: the supermassive black hole in the centre is surrounded by an accretion disk and ejects relativistic jets of matter perpendicular to the disk.}
\end{figure}
\footnotetext{Image taken from the Wikipedia article on AGNs: \url{https://en.wikipedia.org/wiki/Active_galactic_nucleus}}

An example for the last case is the Cosmic Microwave Background (CMB). After the phase of recombination photons decoupled from the opaque plasma made up of quarks, electrons and in some cases hadrons in the very first millionth of a seconds after the Big Bang. These photons, slowly redshifting as the universe expands, are what we call the CMB today. As the name implies the radiation is in the microwave range and hence, again, should not bother us much further.
However, just as there is background radiation in the microwave regime, space is filled with a background of X-ray radiation which is not definitely accounted for yet. One candidate for a majority of the isotropic component of this background is indeed related to the CMB. Jets of highly relativistic particles forming dense shock fronts can interact with the CMB and shift the microwave photons to the X-ray and gamma regime via inverse Compton scattering (ICS) \cite{Loeb2000}.
\vspace{\baselineskip}

One example where this could happen are so-called blazars which are astrophysical objects and part of the larger group of active galactic nuclei (AGN). All AGNs are compact regions in the centre of galaxies based on supermassive black holes forming accretion disks. These compact regions exhibit a very high luminosity and emit large amounts of electromagnetic radiation, including the X-ray and gamma spectrum. Blazars are AGNs that eject hot relativistic jets of matter perpendicular to the accretion disk and are believed to be responsible for a good fraction of the cosmic gamma background\cite{Inoue2014}.
\vspace{\baselineskip}

The remaining parts are divided onto radiation from other stars, galaxies or `normal' black holes and a fraction is speculated to be related to interactions with dark matter \cite{Inoue2014}.


\subsection{Conventional Light Sources}

\subsubsection{X-ray tubes}

The first tunable source to produce to X-rays in a controlled laboratory environment is the X-ray tube. An X-ray tube is a evacuated glass tube with an electron gun to accelerate electrons and a metal anode to decelerate them and to generate radiation.

Wilhelm Roentgen used a vacuum tube (Lenard and Crookes tubes) with an electron gun to investigate `cathode rays', as the beam of electrons emerging from the electron gun was called back then, when he noticed faint light on photo plates across the room. After testing the properties of the light, its absorption, dependence on the cathode beam, brightness, the ability to deflect the beam and many more, Roentgen concluded to have investigated a `new type of rays', which is also the name of publications published between 1895 to 1897 (German original title: `Ueber eine neue Art von Strahlen' \cite{Rontgen1895}). His work on and systematic study of the X-rays, as he called them to emphasise that they were of unknown kind, just like in the mathematical terminology, resulted in the award of the first Nobel Prize in physics in 1901. In some countries, including his native country Germany, X-rays are called Roentgen-rays to honour Wilhelm Roentgen's achievements. It might be worth mentioning that other scientists had noticed the production of X-rays previously as they were imminent in discharge tubes invented 20 years earlier already. Roentgen, however, was the first to systematically study the phenomena and investigate the properties of X-rays, building the foundation for the imaging method of radiography.
\vspace{\baselineskip}

\begin{figure}
\centering
\includegraphics[width=0.8\columnwidth]{WaterCooledXrayTube.pdf}
\caption{Sketch of a watercooled X-ray tube\protect\footnotemark: A coiled wire (C) and a slightly angled anode (A) are placed into an evacuated glass tube. A voltage $U_h$ is applied to the wire and a potential difference $U_a$ is set between the wire and the anode. The wire is heats up and starts emitting electrons, where the number of electrons increases with higher voltages $U_h$. The electrons are then accelerated towards the anode over the potential $U_a$ and collide with the anode, decelerating rapidly and emitting X-rays with a maximum energy $eU_a$ per photon. During the process the anode heats up and can be water-cooled if necessary.}
\label{Introduction:Figs:XrayTube}
\end{figure}

In an X-ray tube electrons are emitted from a hot wire and accelerated towards a metal anode (see figure \ref{Introduction:Figs:XrayTube}), e.g. copper, tungsten or cobalt depending on the application and requirements to the source. When hitting the anode the electrons rapidly decelerate as they are deflected by the atomic nuclei, leading to the emission of photons.
\footnotetext{Image taken from the Wikipedia page on X-ray tubes: \url{https://en.wikipedia.org/wiki/X-ray_tube}}
The spectrum of this bremsstrahlung (German for `radiation from deceleration') is composed of two parts: firstly a broadband X-ray spectrum, called the continuous spectrum, depending on the initial energy of the electrons prior to interacting with the cathode material. Secondly, characteristic peaks that depend on the material used as cathode.

The continuous spectrum in terms of distribution of photons $I$ per unit wavelength $\mathrm{d}\lambda$ is described by Kramer's law:

\begin{equation}
I(\lambda) \mathrm{d}\lambda = K \left(\frac{\lambda}{\lambda_{min}} - 1 \right) \frac{1}{\lambda^2} \mathrm{d}\lambda,
\end{equation}
where $K$ is a constant proportional to the atomic number of the anode material, $\lambda$ denotes the wavelength and $\lambda_{min}$ the cut-off wavelength when an electron emits its entire kinetic energy via one photon.

\subsubsection{Synchrotrons}

What first seemed to be an annoying limitation on the acceleration of particles in cyclotrons or circular accelerators turned out to become an incredibly useful tool to illuminate and probe materials across a wide range of research fields. In synchrotrons charged particles are accelerated on a circular storage ring. The generated radiation is used in several work stations around the ring at the same time. New generations have additional insertion devices and monochromators to tailor the properties of the radiation to the needs of the researchers. As the maximum energies up to which particles can be accelerated increases constantly, the energy, brightness and collimation of synchrotron radiation improves as well.
\vspace{\baselineskip}

Today synchrotrons are essential tools for many applications due to their favourable properties. This includes high brilliance, high collimation, low emittance as well as tunability in energy by monochromatisation. Other properties are a high level of polarisation and the ability to pulse the beam to relatively short durations as well as high repetition rates.
\vspace{\baselineskip}

\begin{figure}
\centering
\includegraphics[width=0.8\columnwidth]{diamond_aerial.jpg}
\caption{Aerial view on the Diamond synchrotron light source in the UK\protect\footnotemark.}
\end{figure}

\footnotetext{Photo courtesy Diamond Ltd: \url{http://www.diamond.ac.uk}}
Typical quantities to characterise the synchrotron spectrum are the critical energy $\epsilon_c$, the associated frequency $\omega_c$ and the critical angle $\theta_c$.

When integrating the emission spectrum of a synchrotron spectrum, half of the total radiated energy is contained in the spectrum up to $\omega_c$ and half beyond. The peak of the distribution occurs approximately at around $0.3\omega_c$.
The critical frequency is given by 
\begin{equation}
\omega_c = \frac{3}{2} \frac{c}{\rho} \gamma^3,
\end{equation}
where $c$ is the speed of light, $\rho$ the bending radius and $\gamma$ the relativistic Lorentz factor. The critical energy $\epsilon_c$ is the energy of a photon emitted at $\omega_c$ and hence simply $\epsilon_c=\hbar \omega_c$ or in more practical units:

\begin{equation}
\epsilon_c = 2.218 \frac{E(GeV)^3}{\rho(m)} = 0.665 \cdot E(GeV)^2 \cdot B(T).
\end{equation}

The critical angle is calculated using

\begin{equation}
\theta_c = \frac{1}{\gamma} \left(\frac{\omega_c}{\omega}\right)^{\frac{1}{3}},
\end{equation}
and indicates the angular divergence of the emission cone containing most of the radiation. Outside of this cone the energy radiated is negligible. Harder X-rays are emitted at a smaller angle.

\subsubsection{Wiggler and Undulator}

As alternative to circular accelerators, the emission of radiation can be stimulated by making charged particles oscillate around a fixed axis using an array of alternating magnets, so-called insertion devices. Devices related to large oscillations are usually referred to as wigglers whereas fast and small oscillations are happening within an undulator. The radiation from undulators is more collimated and coherent as the light emissions interfere constructively with each other. 
This process is characterised using the dimensionless parameter K, called Wiggler strength parameter, giving indication of the relationship of period of length and radius of the bend:

\begin{equation}
K = \frac{\gamma \lambda_u}{\pi \rho} = \frac{e B_0 \lambda_u}{2 \pi m_e c}.
\end{equation}

The wavelength of the emitted radiation $\lambda_r$ on-axis is given by:

\begin{equation}
\lambda_r = \frac{\lambda_u}{2 \gamma^2} \left(1 + K^2 \right),
\end{equation}
where $\lambda_u/2$ is the distance between two magnets and $K$ the wiggler parameter.


\begin{figure}
\centering
\includegraphics[width=0.5\columnwidth]{synchrotron_wiggler_undulator.png}
\caption{Typical methods to produce synchrotron radiation in conventional light sources\protect\footnotemark: bending magnets in synchrotrons (top) lead to pulses of light emitted tangentially to the particle trajectory. An array of oppositely polarised dipole magnets (insertion device) can either result in a wide cone of incoherent radiation when the radius of the bends is relatively large (wiggler, middle) or X-rays with low divergence and coherent interference (undulator, bottom). }
\end{figure}


\subsubsection{Free-Electron Laser (FEL)}

Free-electron lasers (FELs) provide collimated and coherent radiation like ordinary lasers but use free electrons as a gain medium which are accelerated to relativistic energies and stimulated with long undulator insertion devices reaching wavelengths in the soft X-ray regime.
\vspace{\baselineskip}

After oscillating over a long distance in the insertion device the electrons start interacting with the radiation emitted and the bunch structure begins to be modulated, finally resulting in microbunching and the radiation of resonant short pulsed laser-like radiation in the X-ray regime. This process requires seeding which is provided through radiation emitted in the first part of the undulator: this is called Self Amplified Spontaneous Emission (SASE).
\footnotetext{Image taken from R. Bartolini's lecture on synchrotron radiation in the context of the JAI Accelerator Physics Course in 2015/2016.}
\vspace{\baselineskip}

FELs incorporate the finest accelerator technology available to provide high quality electron beams with small emittance at high energy to become the brightest light sources in the world, billion times brighter than conventional X-ray sources revolutionising research in many fields of science by probing protein structures, chemical reactions or atomic processes -- just to name a few examples -- at spatial and temporal scales never seen before.
\vspace{\baselineskip}

However, FELs require large structures to accelerate the electrons to sufficient energies and are hence quite expensive whilst the number of accessible beamlines that can be used at the same time is more limited than at synchrotrons.

This is a motivation to look into new technologies to build more affordable and compact, and hence more accessible FELs. Some researchers try to improve the quality and stability of electron beams from plasma-based accelerators to build FELs solely based on wakefield acceleration \cite{Schlenvoigt2008,Fuchs2009}, but there are also ideas of hybrid approaches combining conventional and plasma-based technology  \cite{Aschikhin2016,Seryi2015}.

%One example of an FEL is the free-electron laser in Hamburg (FLASH), now overtaken by the recently opened European XFEL. Whilst it still sustains user access and will continue to do so for a while as the demand for FELs is large, a PWFA project is being implemented as well.

%What are the properties of the FLASH beamline?

%The high quality electron beam of the free electron laser in Hamburg (FLASH) will be used to drive a wakefield accelerator to double the energy of the witness bunch, further increasing the maximum energy of the X-rays. The proposed idea uses plasma based acceleration to boost conventional methods. This could be used at other major facilities as well and is a affordable method to enhance existing facilities. This could also be employed at the European XFEL that is being built at DESY as well.

\subsection{Plasma based light sources}

Plasma based technology can not only be used to accelerate particles but also just as other accelerators applied to produce radiation with interesting properties. The author will mention two types of radiation, betatron radiation and radiation from inverse Compton scattering (ICS), as they are predominantly of interest for his research.

A good overview on applications and the feasibility of plasma sources for them can be found in \cite{Albert2016}.

\subsubsection{Betatron Radiation}

Betatron radiation is a bright, collimated and ultrafast X-ray source produced through betatron oscillations of electrons in the cavity of a wakefield. 

The radiation is emitted by relativistic particles in a narrow forward cone with an angular divergence proportional to $\gamma^{-1}$. The spectrum can be approximated by the equation for on-axis synchrotron radiation and behaves very similar to a spectrum expected in an insertion device.

The high brightness of the source is related mainly due to the short pulse duration which is on the scale of femtoseconds but has yet to be measured down to this accuracy in experiment. So far the upper limit has been set in the regime of picoseconds \cite{TaPhuoc2007}. The ultrafast character of the betatron radiation exhibits the potential to measure ultrafast reactions, for instance in chemistry. 

The small source size ($\sim \mathrm{\mu m}$) on the other hand makes betatron radiation suitable for phase contrast imaging as demonstrated in \cite{Cole2015}, which does not rely on the absorption, i.e. deposition of dose, into the material and hence could find application in medical imaging.

\subsubsection{Compton Sources}

Compton sources are based on the scattering of photons from electrons leading to a frequency shift of the outgoing photon. Whilst in the process of Compton scattering the photon loses energy to a quasi-free electron, in inverse Compton scattering (ICS) a photon interacts with a relativistic electron and gains energy due to the relativistic Doppler shift. The emitted photon is then enhanced by a factor $(2\gamma)^2$ resulting from boosting into the rest frame of the electron and back again. Using a high intensity laser pulse, the high number of photons can also lead to multi-photon or non-linear ICS increasing the energy of the emitted photon by another factor of the normalised vector potential $a_0$:

\begin{equation}
\hbar \omega' \approx 4 \hbar \omega a_0 \gamma^2,
\end{equation}
where $\hbar$ is the reduced Planck constant, $\omega$ the frequency of the photon before and $\omega'$ after the interaction.

This enables the production of hard X-rays up to the $\mathrm{MeV}$ range as demonstrated at the Astra Gemini laser system in the UK using LWFA \cite{Sarri2014}.

