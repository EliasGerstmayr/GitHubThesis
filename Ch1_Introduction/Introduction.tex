
\chapter{Introduction}

\section{Particle Accelerators}

The advent of particle accelerators unveiled fundamental physics, new particles and processes. More energy to probe smaller and more fundamental particles and interactions.

\subsection{Conventional Accelerators}

\subsection{Plasma-based Accelerators}


\EliasComm{Plots to show energy versus year, energy spread and type of injection and whether it holds to the scaling laws.}

\section{Light Sources}

\subsection{Conventional Light Sources (Gen)}

Particle accelerators are traditionally also the source of bright radiation.

Laser wakefield accelerators are the source of a variable and tunable radiation source covering the few keV to 100 MeV range.
Wakefields are producing hard X-ray radiation through betatron radiation.
Through hosing or amplification of oscillations the radiation can reach the MeV range.
A second beam or backreflected can be used to weakly produce linear inverse Compton scattering in the tens to twenty MeV range.
In the nonlinear highly intense regime radiation to 40 or more MeV is possible.
They also are able to produce bremsstrahlung when dumped.

Source size, time duration are unique properties.

\EliasComm{How about a plot showing the capability of energies on a plot versus synchrotron sources.}

High energy radiation sources are rare on earth. Typically found in astrophysical phenomena, where these kinds of processes are more dominant as well. Inverse Compton scattering is a common sight there. Hard radiation in wakefields can be produced from several sources. Betatron radiation through amplified oscillation, bremsstrahlung, inverse Compton scattering, nonlinear inverse Compton scattering. Also several applications from imaging hard materials, nuclear processes, QED processes will require a diagnostics applicable over a whole bandwidth. New QED experiments reaching hundreds of MeV gamma radiation will have to rely on a shape-independent diagnostics and we have shown that this is necessary to distinguish models. Higher energy accelerators facilities will generate harder radiation as well and need to diagnose it.


Light sources are synchrotrons, large scale particle accelerators. KeV X-rays can be produced from relatively compact X-ray tubes.

Diagnostics:
Absorption X-rays
X-rays via filter packs, K-edges, crystals
converter targets and spectral measurement
differentially damped spectrum

\section{High field interactions and QED}

Radiation Reaction, Breit-Wheeler


\section{Thesis Outline}

Following this introduction the author will introduce some theoretical concepts and experimental methods relevant to this work.

2. theory
3. Experimental Methods 
4. some simulation stuff used in most of the results chapters


Results Chapters: High energy radiation from wakefield accelerators and their application

\begin{enumerate}
\setcounter{enumi}{4}
\item \textbf{Hard Betatron Radiation from dual shock structure} Electrons travelling in a wakefield transitioning through two density spikes experienced an increase of transverse oscillations resulting in an enhancement of measured betatron radiation. The dual shock structure was induced by introducing a blade into the supersonic gas flow of a helium gas jet. This radiation source could be used to image high-Z materials that can not be penetrated by standard betatron radiation. At the same time this opens up the possibility to explore this regime further and increase the brightness of the source even further.
\item \textbf{linear Inverse Compton Scattering} Electrons from a wakefield accelerator were collided with a strongly defocused laser pulse generating gamma radiation from linear inverse Compton Scattering. (and development of dual-axis gamma spectrometer)
\item \textbf{Nonlinear Inverse Compton Scattering and Radiation Reaction} By scattering a tightly focused high intensity laser pulse from a relativistic electron beam and generating gamma radiation from nonlinear Inverse Compton Scattering, it was possible to measure radiation reaction in an experiment. (as gateway to measuring and diagnosing radiation reaction)
\item \textbf{Linear Breit-Wheeler Process and Bremsstrahlung source} A dual laser setup is used to produce a thermal X-ray source and a gamma beam from bremsstrahlung in an attempt to measure the elusive linear Breit-Wheeler process. (or using a wakefield-based Bremsstrahlung source and a laser-solid interaction X-ray source to measure linear BW)
\end{enumerate}

\EliasComm{Put a plot of the photon energy ranges illuminated in the different result sections.}