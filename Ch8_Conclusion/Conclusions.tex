
\chapter{Conclusion}

\section{Current Research and Future Directions}

This report details key theoretical and experimental concepts in the context of wakefield acceleration encountered by the author within the early stage of his PhD at Imperial College.

As the relative proportion of the different chapters implies the author has been extensively involved in experiments, mainly at the Central Laser Facility (CLF, UK), and gaining a firm knowledge of experimental methods applied in the field, whilst investing comparatively limited amount of time in theoretical work, analysis and so far barely in simulations.
\vspace{\baselineskip}

After the involvement in several experiments at the Astra Gemini laser system (target areas 2 and 3) at the Central Laser facility (Rutherford Appleton Laboratory), particular an experiment in December 2015 aiming to measure radiation reaction, the direction of the author's PhD research has become more defined and will focus on high-energy light sources in the context of wakefield acceleration, their applications and relevance in measuring high-field QED effects. In addition, the author will be able to contribute to experiments at FLASHForward (DESY) diagnosing the witness bunch in a plasma wakefield accelerator (PWFA) by analysing its betatron spectrum. Responsibilities on-site so far included setting up various types of optical and particle diagnostics, for instance interferometry, shadowgraphy or an electron spectrometer, but started to evolve with increasing experience and to expand to also incorporate more complex alignment procedures without supervision and even guiding other students in the laboratory. In the future the author plans to shift the perspective from individual tasks and diagnostics to the bigger picture to understand the challenges of planning and performing an experiment in its entirety.
\vspace{\baselineskip}

The analysis of the data acquired in the radiation reaction experiment was a key part of the research over the last months, also with the aim to improve the setup of the experiment for another potential attempt to make precision measurements of radiation reaction after having demonstrated the feasibility of this setup. After a large part of sighting and processing the data was performed by the author as presented in the analysis section of this report, the analysis has been continued on a more advanced level by experienced PhD students and postdocs, potentially leading to a publication in the near future.

Following the completion of analysing the data set from this perspective the author plans to investigate the background radiation measured in the same Gemini experiment and that was also mentioned in the analysis section. The analysis of the scintillator signals implies that the X-ray spectrum extends to much higher energies than usual for betatron radiation in this context. The electron spectra indicate strong oscillations possibly related to shocks in the density profile of the target. The author aims to investigate if these factors are indeed related. This will include a characterisation of the gas nozzle used in experiment to confirm the appearance of shocks possibly due to manufacturing or design errors, and subsequently using the density profile acquired in simulations to see if radiation at this energies could be generated under experiment conditions.
\vspace{\baselineskip}

In the process of further research and analysis, the author will have to become familiar with a variety of tools to simulate wakefield accelerators and the radiation generated in this context. This will in particular involve working with the full particle-in-cell (PIC) code EPOCH \cite{Arber} to simulate laser-plasma interactions. In particular, it is planned to make use of a modified version of the EPOCH QED package provided by a recently graduated student of the group in order to calculate the betatron radiation generated during the acceleration. Potentially, this work might be paired and compared with a post-processing code developed recently by master students as their final year project.

In the future, the set of tools might be extended by the quasi-static PIC-code HiPACE \cite{Mehrling2014}, which is being developed by the FLASHForward group at DESY and is suitable to model beam-driven wakes.
\vspace{\baselineskip}

Finally, taking part in training weeks at RAL, and attending summer schools at ELI in Prague (laser plasma interactions) and Tsinghua university in Beijing (research skills and communication) as well as other workshops and meetings have taught the author a diverse set of skills and overview over the field, extending systematically the hands-on knowledge already acquired during the time at RAL.

\newpage
\section{Acknowledgments}

On this occasion I would like to express my gratitude to a number of people that have supported and guided me throughout this first year of my PhD here at Imperial College. Coping with the challenges of research and a PhD in general is a team effort relying on academic and private support from various sides.
\vspace{\baselineskip}

Firstly, thank you very much to my supervisor Dr. Stuart Mangles. He always kept the door and an ear open for my concerns and was, and continues to be, a source of inspiration and guidance on this path. You always had a fresh idea to contribute when my perspective had become too rigid.
\vspace{\baselineskip}

Secondly, a special and honest thank you very much to the higher PhD students or by now postdocs in the group. You have been a huge support for all of us newbies and we have learned so much from you over the past year. I especially would like to take the opportunity and thank Dr. Jason Cole, who patiently provided tech support and consisted an indispensable source of never-ending knowledge, answering questions from how to align a Mach-Zehnder interferometer to how to generate beautiful plots on Matlab. 
\vspace{\baselineskip}

Thirdly, thanks to my fellow PhD students who have become a nice little supportive group, lightening up daily life with an occasional joke or two.
\vspace{\baselineskip}

Another thanks to the CLF and its team. Links, laser guys, engineers.

Lastly, a warm thank you to my friends and family who have given me a counterbalance to all the work and have kept me to the ground, standing by my side when I was struggling and grumpy. Thank you for your understanding and being there for me.
\vspace{\baselineskip}

I would also like to gratefully acknowledge the joint funding of the John Adams Institute and the Virtual Helmholtz Institute as well as the support of the German National Academic Foundation (Studienstiftung des deutschen Volkes).