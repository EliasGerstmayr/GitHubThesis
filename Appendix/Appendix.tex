
\chapter{Symbols and constants}

\section{Frequently used symbols}

\begin{table}[h]

\begin{minipage}{.5\linewidth}
\centering
{\rowcolors{2}{white}{lightgray!50}
\begin{tabular}{|l|l|}
\hline 
\textbf{Symbol} & \textbf{Name}  \\ \hline \hline
$a_0$ & Normalised vector potential\\
$B$ & Magnetic field\\
$\beta$ & Relativistic velocity\\
$E$ & Electric field strength\\
$\epsilon$ & Energy, e.g. $\epsilon_\gamma$ gamma energy\\
$\eta$ & Quantum nonlinearity param.\\
$\gamma$ & Relativistic Lorentz factor\\
$\lambda$ & Wavelength\\
$N$ & Number of particles/photons\\
$n_c$ & Critical density\\
$n_e$ & Electron density\\
$\omega$ & Frequency\\
\hline
\end{tabular}
}
\end{minipage}
\begin{minipage}{.5\linewidth}
\centering
{\rowcolors{2}{white}{lightgray!50}
\begin{tabular}{|l|l|}
\hline 
\textbf{Symbol} & \textbf{Name}  \\ \hline \hline
$\omega_p$ & Plasma frequency\\
$Q$ & Charge\\
$r$ & Radius\\
$r_\beta$ & Betatron radius\\
$\rho$ & (Material) density\\
$R_c$ & Classical radiation reaction param.\\
$s$ & Centre-of-mass energy\\
$\sigma$ & Standard deviation\\
$\sigma_{X}$ & Cross section for process $X$ \\
$\theta$ & Divergence or angle\\
$w$ & Waist size\\
$X_0$ & Radiation length\\
\hline
\end{tabular}
}
\end{minipage}
\caption[Frequently used symbols.]{Frequently used symbols.}
\end{table}

\clearpage

\section{Abbreviatons}


\begin{table}[h]
\centering
{\rowcolors{2}{white}{lightgray!50}
\begin{tabular}{|l|l|}
\hline
\textbf{Abbreviation} & \textbf{Meaning}  \\ \hline \hline
BW & Breit-Wheeler (process)\\
CCD & Charge-coupled device\\
CLF & Central Laser Facility\\
FWHM & Full width at half maximum\\
ICS & Inverse Compton scattering\\
LWFA & Laser wakefield acceleration\\
LAD & Lorentz-Abraham-Dirac (equation)\\
LL & Landau-Lifschitz (equation)\\
OAP & Off-axis parabola\\
QED & Quantum electrodynamics\\
RAL & Rutherford Appleton Laboratory\\
RR & Radiation reaction\\
\hline
\end{tabular}
}
\caption[Frequently used abbreviations.]{Frequently used abbreviations.}
\end{table}


\section{Notation}

\begin{table}[h]
\centering
{\rowcolors{2}{white}{lightgray!50}
\begin{tabular}{|l|l|}
\hline
\textbf{Notation} & \textbf{Meaning} \\ \hline \hline
\textbf{Bold} & Vector\\
$\langle ...\rangle$ & Average\\
$\hbar$ & Reduced quantity, $h/2\pi$\\
$\dot{x}$ & Time derivative of $x$\\
\hline
\end{tabular}
}
\caption[Notations.]{Notations.}
\end{table}

\clearpage

\section{Fundamental constants}

\begin{table}[h]
\centering
{\rowcolors{2}{white}{lightgray!50}
\begin{tabular}{|l|l|l|l|}
\hline
\textbf{Symbol} & \textbf{Name} & \textbf{Value (SI)} & \textbf{Unit (SI)} \\ \hline \hline
$\alpha$ & Fine-structure constant & $7.3 \times 10^{-3}$ & $1$\\
$c$ & Vacuum speed of light & $3 \times 10^8$ & $\mathrm{m/s}$ \\
$e$ & Electron charge & $1.6 \times 10^{-19}$ & $\mathrm{C}$\\
$E_S$ & Schwinger field & $1.32 \times 10^{18}$ & $\mathrm{V/m}$\\
$\epsilon_0$ & Vacuum permittivity & $8.9 \times 10^{-12}$ & $\mathrm{F/m}$ \\
$h$ & Planck constant & $6.6 \times 10^{-34}$ & $\mathrm{J}\cdot\mathrm{s}$\\
$\lambda_C$ & Compton wavelength & $2.4 \times 10^{-12}$ & $\mathrm{m}$\\
$m_e$ & Electron mass & $9.1 \times 10^{-31}$ & $\mathrm{kg}$\\
$\mu_0$ & Vacuum permeability & $1.3 \times 10^{-6}$ & $\mathrm{N}\cdot\mathrm{A}$ \\
$r_e$ & Classical electron radius & $2.8 \times 10^{-15}$ & $\mathrm{m}$\\
$\sigma_T$ & Thomson cross section & $6.7 \times 10^{-29}$ & $\mathrm{m^2}$\\
\hline
\end{tabular}
}
\caption[Fundamental constants.]{Fundamental constants.}
\end{table}

Most of the thesis uses SI units. In a few exception natural units ($\hbar = c = \epsilon_0 = 1$) are used, which is then explicitly indicated.

\vspace*{\fill}


\iffalse
\chapter{Notations and Vector Identities}


\subsection{Curl of a gradient}
\label{Appendix:VectorIdentities:CurlGradZero}


For a scalar function $f(x,y,z)$, the gradient in Cartesian coordinates is
\begin{equation}
\nabla f(x,y,z) = \left( \frac{\partial f}{\partial x}, \frac{\partial f}{\partial y}, \frac{\partial f}{\partial z}\right)^t,
\end{equation}
where $\nabla f$ is now a vector field, $\mathbf{F} = \left( F_1, F_2, F_3\right)^t$. The curl of $\mathbf{F}$, $\nabla \times \mathbf{F}$ is then
\begin{equation}
\nabla \times \mathbf{F} = \left(\frac{\partial F_3}{\partial y} - \frac{\partial F_2}{\partial z}, \frac{\partial F_1}{\partial z} - \frac{\partial F_3}{\partial x}, \frac{\partial F_2}{\partial x} - \frac{\partial F_1}{\partial y}\right),
\end{equation}
or again in terms of $f(x,y,z)$:
\begin{equation}
\nabla \times \nabla f = \left(\frac{\partial^2 f}{\partial y \partial z} - \frac{\partial^2 f}{\partial z \partial y}, \frac{\partial^2 f}{\partial z \partial x} - \frac{\partial^2 f}{\partial x \partial z}, \frac{\partial^2 f}{\partial x \partial y} - \frac{\partial^2 f}{\partial y \partial x}\right).
\end{equation}
If $f$ is twice continuously differentiable, the order of the derivatives is interchangeable and $\nabla \times \nabla f = \mathbf{0}$.
The curl of a gradient is always zero.
\fi

\chapter{Single particle motion}
\label{Appendix:Sec:SingleParticleMotion}

\section{Electron in a cylindrically symmetric electric field}

Here we discuss the motion of an electron in a cylindrically symmetric electric field which resembles the case of an electron in a plasma channel.
This derivation is based on \cite{WoodThesis}.

Assume a relativistic electron moving in the longitudinal direction and an electric field that is cylindrically symmetric around the propagation axis.
We ignore any vector potential contributions and only use a scalar potential.
The Lagrangian of this scenario is described by
\begin{equation}
L(\mathbf{r},\mathbf{v},t) = - m_e c^2 \gamma^{-1} - q \Phi,
\end{equation}
where $\Phi$ is the scalar potential describing the electrostatic field of the channel.
The scalar potential is the solution matching the radial Poisson equation in cylindrical symmetry

\begin{equation}
\frac{1}{r} \frac{\partial}{\partial r} \left(r \frac{\partial \phi}{\partial r}\right) = \frac{- e (n_0 - n_e)}{\epsilon_0},
\end{equation}

with the solution 
\begin{equation}
\phi = -\frac{(n_0 - n_e) e r^2}{4 \epsilon_0},
\end{equation}
where $n_0$ is the ion charge density and $n_e$ the electron density in the channel. 
In the limit of a completely evacuated channel, $n_e \rightarrow 0$, $\phi = -n_0 er^2/(4\epsilon_0)$.
As before $\mathbf{E} = - \nabla \phi$ and in this case $E_r = - \partial_r \phi$.
The equation of motion for the radial component is then
\begin{equation}
\frac{d}{dt} \mathbf{p} = - \frac{e^2 n_0}{2 \epsilon_0} r,
\end{equation}
with $\mathbf{p} = \gamma m_e \mathbf{\dot{r}}$.
Assume there is no axial acceleration field, the radial component of the velocity is small compared to $c$ and the longitudinal component is close to $c$: $\dot{\gamma}$ is then close to zero. 
The equation then takes the shape of a harmonic oscillator
\begin{equation}
\ddot{r} = - \frac{e^2 n_0}{2\epsilon_0 m_e \gamma} r = - \omega_\beta^2 r,
\end{equation}

with a frequency we will call the betatron frequency, $\omega_\beta$, which is related to the plasma frequency, $\omega_p$, by $\omega_\beta = \omega_p/\sqrt{2\gamma}$. The plasma frequency is derived explicitly in Section \ref{Chap:Theory:Sec:PlasmaFreq}.
As this is the standard harmonic oscillator equation, it follows a solution of shape
\begin{equation}
r(t) = A \cos (\omega_\beta t + \varphi),
\end{equation}
and with initial conditions $r(t=0) = r_\beta$ and $\dot{r}(t=0) = 0$, we obtain
\begin{equation}
r(t) = r_\beta \cos(\omega_\beta t),
\end{equation}
where $r_\beta$ is the amplitude of the oscillation, also referred to as the betatron radius.
\vspace{\baselineskip}

If we continue with the assumption $\dot{\gamma} = 0$, this requires $\beta^2 = \beta^2_r + \beta^2_z = constant$. This gives $\beta_z = \sqrt{\beta^2 - \beta^2_r} \approx \beta (1-\beta^2/2)$, where we used a Taylor expansion and $\beta \approx 1$. If we use the solution for $r(t)$ derived above, $\beta_r = \dot{r}/c$ and $\sin^2 x = (1-\cos 2\theta)/2$, we obtain:
\begin{equation}
\beta_z \approx \beta \left(1-\frac{r^2_\beta \omega^2_\beta}{4 c^2}\right) + \beta \frac{r^2_\beta \omega^2_\beta}{4c^2} \cos (2 \omega_\beta t).
\end{equation}

Integrating this with the initial condition $z(t=0) = z_0$

\begin{equation}
\frac{z}{c} \approx \frac{z_0}{c} + \beta \left(1-\frac{r^2_\beta \omega^2_\beta}{4 c^2}\right) t + \beta \frac{r^2_\beta \omega_\beta}{8c^2}\sin(2\omega_\beta t),
\end{equation}

where we identify the drift velocity $\beta c \left(1-\frac{r^2_\beta \omega^2_\beta}{4 c^2}\right)$ and in the moving frame the familiar figure-of-eight motion.
\clearpage

\section{Electron in a cylindrically symmetric electric field with axial field}

Now we add a constant axial field in the longitudinal (direction of electron propagation) to the previous case. This comes closer to the reality of a laser wakefield accelerator where electrons experience radial focusing and axial accelerating fields at the same time. Significant parts of this derivation are based on \cite{WoodThesis,Glinec2008_BETATRON}.

The electric field component shall now be $E_z$, with $z$ being the longitudinal axis.
This gives us the extra term $\frac{\partial \phi}{\partial z} = - E_z$.
The Euler-Lagrangian now has two non-trivial equations of motions.
First, as before
\begin{equation}
\frac{d}{dt} (\gamma m_e \dot{r}) = \frac{\partial }{\partial r} L = -\frac{e^2 n_0}{2\epsilon_0} r,
\label{Theory:Eq:EoM_Rad_Ax_R}
\end{equation}
and, second, the $z$-component
\begin{equation}
\frac{d}{dt} (\gamma m_e \dot{z}) = \frac{\partial }{\partial z} L = e E_z = const.
\label{Theory:Eq:EoM_Rad_Ax_Z}
\end{equation}

The change in momentum is now not negligible any more, $\dot{\gamma}\neq 0$, and we have to expand the brackets on the left-hand side of Equations \eqref{Theory:Eq:EoM_Rad_Ax_R} and \eqref{Theory:Eq:EoM_Rad_Ax_Z}.
The radial component is then
\begin{equation}
\dot{\gamma}m_e \dot{r} + \gamma m_e \ddot{r} = - \frac{e^2 n_0}{2 \epsilon_0} r.
\end{equation}

The axial component is equal to a constant, so we can simply integrate
\begin{equation}
\gamma m_e \dot{z} = - e E_z t + const.
\end{equation}

If we assume that the electron is highly relativistic $\dot{z} \rightarrow c$, so $\dot{\gamma} = eE_z/m_e c$ and

\begin{equation}
\gamma \beta_z \approx \gamma = - \frac{e E_z t}{m_e c} + \gamma_0 \beta_0,
\label{Theory:Eq:EoM_Rad_Ax_Gamma}
\end{equation}

where $\gamma_0 \beta_0$ are the initial conditions.
\begin{figure}
\centering
\includegraphics[width=.6\columnwidth]{betatron_traj.pdf}
\caption[Particle trajectory of an electron in a cylindrically symmetric electric field with a constant axial accelerating field.]{Particle trajectory of an electron in a cylindrically symmetric electric field with a constant axial accelerating field as described in Equation \eqref{Theory:Eqs:EoM_Betatron} for a plasma of density $n_e = 2 \times 10^{18} cm^{-3}$ as a function of propagation distance.}
\label{Theory:Figs:BetatronMotion}
\end{figure}
Inserting the result from Equation \eqref{Theory:Eq:EoM_Rad_Ax_Gamma} into Equation \eqref{Theory:Eq:EoM_Rad_Ax_R}, we obtain the differential equation

\begin{equation}
eE_z \dot{r} + (eE_z t + \gamma_0 \beta_0 m_e c) \ddot{r} = - \frac{e^2 n_0 c}{2 \epsilon_0} r,
\end{equation}
which we can write as 
\begin{equation}
(At + B) \ddot{r} + A \dot{r} + C r = 0,
\end{equation}
by using the abbreviations $A = eE_z$, $B = \gamma_0 \beta_0 m_e c$ and $C = e^2 n_0 c/2\epsilon_0$.
The analytic solution for this differential equation is \cite{WoodThesis}:
\begin{subequations}
\begin{empheq}[box=\widefbox]{align}
r(t) = \frac{\pi \sqrt{CB}}{A} r_{\beta 0} \Biggl[ &J_1 \left( 2 \sqrt{CB}/A \right) Y_0 \left(2 \sqrt{C (B+At)}/A\right) \\  \nonumber
&- Y_1 \left(2\sqrt{CB}/A\right)J_0\left(2\sqrt{C(B+At)}/A\right)\Biggr],
\end{empheq}
\label{Theory:Eqs:EoM_Betatron}
\end{subequations}
with the initial conditions $r(t=0) = r_{\beta 0}$ and $\dot{r}(t=0) = 0$. $J_1, J_0$ are Bessel functions of first kind of order $1$ and $0$, respectively. $Y_1, Y_0$ are Bessel function of the second kind of order $1$ and $0$.
This motion is shown in Figure \ref{Theory:Figs:BetatronMotion}. We see that the oscillations quickly lose amplitude and fall to a settled value, whereas the wavelength of the oscillations increases as the electron gains momentum and inertia.


\chapter{Synchrotron radiation}
\label{Appendix:Sec:Synchrotron}

\section{Synchrotron Radiation}

The angularly integrated spectral intensity for synchrotron radiation is given by  \cite{WoodThesis,Jackson}:
\begin{equation}
\frac{dI}{d\omega} = \frac{\sqrt{3}}{4} \frac{e^2}{\pi c \epsilon_0} \gamma \frac{\omega}{\omega_c} \int_{\omega/\omega_c}^{\infty} K_{5/3}(x)dx,
\end{equation}
where $e$ is the electron charge, $\epsilon_0$ the dielectric constant and $c$ is the speed of light. $\gamma$ is the relativistic Lorentz factor of an electron. $K_{5/3}$ is a modified Bessel function of the second kind.
$\omega_c$ is the characteristic or \textit{critical frequency} given by
\begin{equation}
\omega_c = \frac{3}{2} \gamma^3 \frac{c}{\rho},
\end{equation}
where $\rho$ is the bending radius, e.g. the Larmor radius.
The doubly differential spectral intensity reads
\begin{equation}
\frac{d^2 I}{dEd\Omega} = \frac{3e^2}{16\pi^3 \hbar c \epsilon_0} \gamma^2 \frac{E^2}{E_c^2} \left(1+\gamma^2 \theta^2\right)^2 [K_{2/3}^2(\psi) + \frac{\gamma^2 \theta^2}{1+\gamma^2 \theta^2} K_{1/3}^2 (\psi)],
\end{equation}
with
\begin{equation}
\psi = \frac{E}{2E_c}(1+\gamma^2 \theta^2)^{3/2}
\end{equation}


\clearpage
\section{Undulator and Wiggler Radiation}
\label{Theory:Sec:UndulatorWiggler}


To harvest and direct the radiation of an accelerated relativistic electron more efficiently a set of alternating magnets can be used. The electron oscillates and emits radiation preferentially at the turning points as the acceleration is the strongest there in a forwards-pointing cone with a diameter proportional to $K/\gamma$ and $1/\gamma$ \cite{Jackson} in and out of plane of the oscillation, respectively.

The rapid change of direction in the field and hence forced trajectory of the electron in transversal direction results in a change of effective Lorentz factor in the forward direction.

\begin{equation}
\left\langle\gamma\right\rangle \approx \frac{\gamma}{\sqrt{1 + 0.5 K^2}}
\end{equation}

Commonly two regimes are identified based on the \textit{wiggler parameter}, K:
\begin{equation}
K = \frac{eB_0}{m_e ck_u} = \frac{e B_0 \lambda_u}{2\pi m_e c},
\end{equation}
where $B_0$ is the amplitude of the magnetic field, $\lambda_u$ the wavelength of the alternating magnetic field, and $k_u$ is the corresponding wave vector. $e$ is charge of the electron, $m_e$ its mass and $c$ is the speed of light. 
$K \ll 1$ refers to the \textit{undulator regime}, whereas large $K \gg 1$ indicate the \textit{wiggler regime}.
In the wiggler regime the divergence of the cone is large and the radiation has a broad bandwidth.
In the undulator regime the radiation is strongly collimated and interference can occur resulting in single harmonics being emitted.
The constructive interference occurs with the $n$th harmonic of wavelength:
\begin{equation}
\lambda = \frac{\lambda_u}{2 n \gamma^2} \left( 1+ \frac{K^2}{2} + \gamma^2 \theta^2\right),
\end{equation}
where $\gamma$ is the relativistic Lorentz factor of the electron and $\theta$ is the observation angle relative to the beam axis.
The equations introduced for synchrotron radiation work similarly for wiggler radiation. The reference to the bending radius is usually replaced by an expression including the magnet periodicity or the $K$ parameter.
The critical frequency of a wiggler is then, for instance:
\begin{equation}
\omega_c \sim \frac{3}{2} \gamma^2 K k_u c
\end{equation}

\iffalse
Maximum energy 
\begin{equation}
\hbar \omega_{eV} = \frac{0.496[E(GeV)]^2}{(1+ K^2/2) \lambda_0 (m)}
\end{equation}

Number of emitted photons per magnet period (later compare to number of photons emitted in wiggler), is then \cite{Jackson}
\begin{equation}
N_\gamma = \frac{2 \pi}{3} \alpha K^2
\end{equation}
\fi
\clearpage
\section{Betatron Radiation}
\label{Appendix:Sec:Betatron}

Just as the evacuated `bubble' in a wakefield accelerates particles in a similar way as conventional RF cavities would, the generation of X-rays draws another parallel in conventional devices.
\vspace{\baselineskip}

Due to initial transverse momentum on injection, and the focusing and defocusing fields of the cavity, electrons start to oscillate around the central axis. 
The electrons oscillate at the betatron frequency, $\omega_\beta$ \cite{Whittum1992_BETATRON,WoodThesis}
\begin{equation}
\omega_\beta = \omega_p / \sqrt{2 \gamma},
\end{equation}
which depends on the plasma frequency $\omega_p$ and the energy of the electrons, here given by the relativistic Lorentz factor $\gamma$.

Similarly to electrons in an insertion device (undulator or wiggler) these oscillations lead to the emission of radiation in a narrow forward pointing cone due to the relativistic forward momentum of the particles.
Not surprisingly, the equations describing both processes are very similar.

The collimation depends on $K$, the wiggler parameter (for insertion devices) or betatron strength parameter
\begin{equation}
K = \gamma k_\beta r_\beta,
\end{equation}
where $k_\beta = \omega_\beta / c$ is the wavenumber and $r_\beta$ is the betatron radius, the amplitude of the oscillations. If $K \gg 1$ it is called the undulator parameter.

The radiation generated follows the on-axis synchrotron radiation with a critical energy
\begin{equation}
\boxed{E_c = \frac{3}{4} \hbar \gamma^2 \omega^2_p r_\beta / c,}
\end{equation}
where this characteristic energy indicates when approximately half the energy is radiated above and below this value.
\vspace{\baselineskip}

The oscillation of the particles and hence the emission of radiation can be enhanced in wakefield experiments by tailoring the wavefront \cite{Mangles2009_BETATRON,Ferri2016_BETATRON}, density profile \cite{TaPhuoc2008_BETATRON,Kozlova2019_BETATRON} or by using different injection mechanisms \cite{Chen2013_BETATRON}.
In LWFA betatron radiation typically reaches the soft X-ray regime of a few to tens of keV \cite{Kneip2010_BETATRON}.


\chapter{Radiation Length}
\label{Appendix:Sec:RadiationLength}

The radiation length, $X_0$, is a material-specific quantity that indicates how much energy a high-energy particle converts into radiation and secondary particles when passing through a material. The energy of a particle passing through a material of thickness $x$ and radiation length $X_0$ loses energy at a rate \cite{Jackson}:
\begin{equation}
\frac{\dif E}{\dif x} = -\frac{E}{X_0},
\end{equation}
with $E(x) = E_0 e^{-x/X_0}$ and $X_0$ is given by \cite{Tsai1974_PairsBrems,Tanabashi2018_PDGReview}:
\begin{equation}
\boxed{X_0 = \left[ 4 \alpha r^2_e \frac{N_A}{A}\left\lbrace Z^2 \left( L_{rad} - f(Z) \right) + Z L'_{rad} \right\rbrace\right]^{-1},}
\end{equation}
with the atomic weight, $A$, and $4 \alpha r^2_e \frac{N_A}{A} = (716.408\,\mathrm{g}\,\mathrm{cm}^{-2})^{-1}$ for $A = 1~\mathrm{g/mol}$.  For up to uranium $f(Z)$ can be approximated by
\begin{equation}
f(Z) = a^2 \left[(1+a^2)^{-1} + 0.20206 - 0.0369 a^2 + 0.0083a^4 - 0.002 a^6\right],
\end{equation}
with $a = \alpha Z$.
The radiation length in a mixture or compound of $n$ components can be approximated by the sum of radiation lengths of the individual components weighted by their weight
\begin{equation}
\frac{1}{X_0} = \sum^n_{j = 1} \frac{w_j}{X_j}.
\end{equation}

\chapter{Bremsstrahlung}

\section{Bremsstrahlung in the Born approximation}
\label{Appendix:Brems_Born}

The (in energy) differential cross section for electron-nucleus bremsstrahlung in a fixed field in the Born approximation, neglecting the recoil of the nucleus, is given by \cite{Ruffini2010_PAIRSASTRO,PikeThesis}
\begin{align}
\frac{\dif \sigma_{eZ}}{\dif \omega} = \frac{\alpha Z^2 r^2_e}{\omega}\frac{u'}{u}\Bigg\lbrace &\frac{4}{3} - 2 \gamma \gamma' c^2 \frac{u^2 + u'^2}{u^2 u'^2} + c^3 \left( \frac{\epsilon \gamma'}{u^3} + \frac{\epsilon'\gamma}{u'^3}-\frac{\epsilon\epsilon'^3}{u u' c}\right)\\ \nonumber
&+ L\Bigg[\frac{8}{3} \frac{\gamma \gamma' c^2}{u u'} + \frac{(\hbar \omega)^2 c^2}{m^2_e u^3 u'^3} (\gamma^2 \gamma'^2 + u^2 u'^2/c^4)\\ \nonumber
& +\frac{\hbar \omega}{2 m_e u u'}\left( \frac{\gamma \gamma'+ u^2/c^2}{u^3/c^3}- \frac{\gamma \gamma' + u'^2/c^2}{u'^3/c^3}-\frac{2\hbar \omega \gamma \gamma'c^2}{m_e u^2 u'^2}\right)\Bigg]\Bigg\rbrace,
\end{align}
where
\begin{align*}
L &= \ln \left( \frac{\gamma \gamma' + u u'/c^2 -1}{\hbar \omega/m_e c^2}\right),\\
\epsilon &= 2 \ln(\gamma + u/c),\\
\epsilon' &= 2 \ln(\gamma' + u'/c).
\end{align*}
$u$ is the momentum of the electron, $\gamma$ its energy and $m_e$ its mass. $\hbar \omega$ is the energy of the photon and $c$ is the speed of light. Primed quantities denote post-interaction values and $\gamma m_e c^2 = \hbar \omega + \gamma' m_e c^2$, assuming all energy is transferred into radiation.  $r_e$ is the electron radius, $\alpha$ the fine-structure constant and $Z$ the atomic number.


\section{Bremsstrahlung including screening}
\label{Appendix:Brems_with_Screening}

The (in energy) differential cross section for electron-nucleus bremsstrahlung is in the ultra-relativistic limit, including electron contributions and screening, given by \cite{Tsai1974_PairsBrems,Tanabashi2018_PDGReview}
\begin{align}
\frac{\dif \sigma_{eZ}}{\dif \omega} = \frac{\alpha r^2_e}{\omega} \Bigg\lbrace &\left(\frac{4}{3} - \frac{4}{3}y + y^2\right) \times \left[Z^2\left(\phi_1 - \frac{4}{3}\ln Z - 4f\right) + Z \left(\psi_1 - \frac{8}{3}\ln Z\right)\right]\\ \nonumber 
& + \frac{2}{3}(1-y) \left[ Z^2 (\phi_1 - \phi_2) + Z(\psi_1 - \psi_2)\right]\Bigg\rbrace,
\end{align}
with $y = \hbar \omega/\gamma m_e c^2$, the ratio of the emitted photon energy to the initial electron energy. $r_e$ is the electron radius, $\alpha$ the fine-structure constant and $Z$ is the atomic number. The $\phi$ and $\psi$ terms are given by
\begin{align*}
\phi_1 &= 2\left[1+\ln\left(a^2 Z^{2/3} m^2_e\right)\right] - 2\ln(1+b^2) - 4b \tan^{-1} (b^{-1}),\\
\phi_2 &= 2\left[\frac{2}{3}+\ln\left(a^2 Z^{2/3}m^2_e\right)\right] -\ln(1+b^2) + 8b^2 \left[1-b \tan^{-1}(b^{-1}) - \frac{3}{4}\ln(1+b^{-2})\right],\\
\psi_1 &= 2\left[1+\ln\left(a'^2 Z^{4/3}m^2_e\right)\right] -2\ln(1+b'^2) - 4b' \tan^{-1} (b^{-1}),\\
\psi_2 &= 2\left[\frac{2}{3}+\ln \left(a'^2 Z^{4/3} m^2_e\right)\right] - 2 \ln (1+b'^2) + 8 b'^2 \left[1-b \tan^{-1} (b'^{-1}) - \frac{3}{4} \ln (1+b'^{-2})\right], 
\end{align*}
with $a=184.15(2.718)^{-1/2} Z^{-1/3}/m_e$, $a' = 1194(2.718)^{-1/2} Z^{-2/3}/m_e$, $b = a \delta$, $b' = a' \delta$, $\delta = m^2_e \hbar \omega/(2\gamma m_e c^2(\hbar \omega - \gamma m_e c^2))$.
The Coulomb correction term for the atomic electrons, $f$, is given by
\begin{equation}
f(z) \approx 1.202 z - 1.0369z^2 + 1.008z^3/(1+z).
\end{equation}

\section{Bremsstrahlung in terms of radiation length}
\label{Appendix:Sec:Brems_RadiationLength}

The differential cross section for the bremsstrahlung process is in the ultrarelativistic limit approximated by a simplified expression in terms of the radiation length, $X_0$ (see Section \ref{Appendix:Sec:RadiationLength}) \cite{Tsai1974_PairsBrems,Tanabashi2018_PDGReview}
\begin{equation}
\frac{\dif \sigma}{\dif (\hbar \omega)} = \frac{A}{X_0 N_A \hbar \omega} \left(\frac{4}{3} - \frac{4}{3}y + y^2\right),
\end{equation}
where $\hbar \omega$ is the energy of the emitted photon, $Z$ is the atomic number and $A$ the atomic weight and $N_A$ the Avogadro number. $y= \hbar\omega/E$, i.e. the fraction of the emitted photon energy relative to the initial electron energy.


\chapter{Cross sections for fundamental QED processes}
\label{Appendix:QEDProcesses}

\section{Feynman diagrams and convention}

\subsection{Feynman rules for QED}
\EliasComm{Needs thinking what to inclue.}
Photon rules:

New vertex $- ie \gamma^\mu$

Photon propagator $= \frac{-ig_{\mu\nu}}{q^2 + i\epsilon}$

External photon lines $ =\epsilon_\mu (p)$ and $=\epsilon^\ast_\mu (p)$

Fermion rule

Vertex $=-ig$

Propagator $=\frac{i(\cancel(p) + m)}{p^2 - m^2 + i\epsilon}$

External legs fermion $u^s (p)$ and $\bar{u}^s (p)$

External legs antifermion $\bar{v}^s (k)$ and $v^s(k)$

Impose momentum conservation at each vertex

Integrate over each undetermined loop momentum

Figure out overall sign of diagram

\subsection{Mandelstam variables}

\begin{align}
s & = (p + p')^2 &= (k + k')^2; \nonumber\\
t &= (k-p)^2 &= (k' - p')^2; \nonumber\\
u &= (k'-p)^2 &= (k-p')^2.
\end{align}


\subsection{Trace Identities}

From \cite{PeskinSchroeder}:

Trace theorems:
\begin{align*}
tr(\mathbf{1}) &= 4\\
tr(any~odd~number~of~\gamma s) &= 0\\
tr(\gamma^\mu \gamma^\nu) = 4 g^{\mu \nu}\\
tr(\gamma^\mu \gamma^\nu \gamma^\rho \gamma^\sigma) &= 4(g^{\mu \nu} g^{\rho \sigma} - g^{\mu \rho} g^{\nu \sigma} + g^{\mu \sigma} g^{\nu\rho})\\
tr(\gamma^5) &= 0\\
tr(\gamma^\mu \gamma^\nu \gamma^5) &= 0\\
tr(\gamma^\mu \gamma^\nu \gamma^\rho \gamma^\sigma \gamma^5) = -4i \epsilon^{\mu \nu \rho \sigma}.
\end{align*}

Reversal of orders is allowed within a trace:
\begin{equation}
tr(\gamma^\mu \gamma^\nu \gamma^\rho \gamma^\sigma ...) = tr(...\gamma^\sigma \gamma^\rho\gamma^\nu \gamma^\mu).
\end{equation}

Eliminating dotted $\gamma$-matrices:
\begin{align*}
\gamma^\mu \gamma_\mu = g_{\mu \nu}\gamma^\mu \gamma^\nu = \frac{1}{2}g_{\mu \nu}\lbrace\gamma^\mu,\gamma^\nu\rbrace = g_{\mu \nu} g^{\mu \nu} 4.
\end{align*}

Contraction identities:
\begin{align*}
\gamma^\mu \gamma^\nu \gamma_\mu &= - 2\gamma^\nu\\
\gamma^\mu \gamma^\nu \gamma^\rho \gamma_\mu &= 4 g^{\nu \rho}\\
\gamma^\mu \gamma^\nu \gamma^\rho \gamma^\sigma \gamma_\mu = - 2\gamma^\sigma \gamma^\rho \gamma^\nu
\end{align*}
\subsection{Other identities}

\begin{align*}
\sum u^s(p) \bar{u}^s (p) &= \cancel{p} + m \\
\sum v^s (p) \bar{v}^s (p) &= \cancel{p} - m.
\end{align*}

\subsection{Calculating Matrix Elements from Feynman diagrams}


\section{Cross sections for fundamental QED processes}

\subsection{Derivation of the Klein-Nishina equation}
\label{Appendix:QEDDeriv_KleinNishina}

\begin{figure}[h]
\centering
\includegraphics[width=0.4\columnwidth]{Compton_Feyn1.pdf}\includegraphics[width=0.4\columnwidth]{Compton_Feyn2.pdf}
\caption{Feynman diagrams for Compton scattering. Time axis from left to right. s- and u-channel.}
\label{Appendix:Figs:ComptonScatter_Feynman}
\end{figure}

In this section the Klein-Nishina equation is derived explicitly by evaluating the matrix element of the lowest order Feynman diagrams contributing to Compton scattering (see Figure \ref{Appendix:Figs:ComptonScatter_Feynman}). The cross section combines the kinematics of the process with the matrix element which includes the physics of the interaction. The following derivation is based on \cite{PeskinSchroeder} and \cite{Schwartz}.

Based on the two Feynman diagrams in Figure \ref{Appendix:Figs:ComptonScatter_Feynman} and using the Feynman rules introduced in Section\addnum{} we find the following expression for the matrix element:

\begin{align}
iM &= \bar{u}(p')(-ie\gamma^\mu)\epsilon_\mu^*(k') \frac{i(\cancel{p} + \cancel{k} + m)}{(p+k)^2 - m^2}(-ie\gamma^\nu)\epsilon_\nu(k)u(p)\nonumber\\
&+ \bar{u}(p')(-ie\gamma^\nu)\epsilon_\nu(k) \frac{i(\cancel{p} - \cancel{k}' + m)}{(p - k')^2 - m^2}(-ie\gamma^\mu) \epsilon_\mu^*(k')u(p)\\
&= -ie^2 \epsilon_\mu^*(k')\epsilon_\nu(k) \bar{u}(p') \left[\frac{\gamma^\mu (\cancel{p} + \cancel{k} +m)\gamma^\nu}{(p+k)^2 - m^2} + \frac{\gamma^\nu(\cancel{p}-\cancel{k'}+m)\gamma^\mu}{(p-k')^2-m^2}\right] u(p).
\end{align}

Using $p^2 = m^2$ and $k^2 = 0$, we can simplify the denominators
\begin{align}
(p + k)^2 - m^2 = p^2 + k^2 +2p\cdot k-m^2 &= 2p \cdot k\\
(p - k')^2 - m^2 = p^2 + k'^2 - 2 p\cdot k' - m^2 &= -2p \cdot k'
\end{align}

Also use
\begin{align*}
(\cancel{p} + m) \gamma^\nu u(p) & = (2p^\nu - \gamma^\nu \cancel{p} + \gamma^\nu m) u(p)\\
&= 2 p^\nu u(p) - \gamma^\nu (\cancel{p} - m) u(p)\\
&= 2p^\nu u(p).
\end{align*}

Applying both we obtain:
\begin{equation}
iM = -ie^2\epsilon_\mu^*(k')\epsilon_\nu(k)\bar{u}(p')\left[\frac{\gamma^\mu \cancel{k} \gamma^\nu + 2 \gamma^\mu p^\nu}{2p\cdot k} + \frac{-\gamma_\nu \cancel{k'} \gamma^\mu + 2 \gamma^\nu p^\mu}{-2p\cdot k'}\right]u(p).
\end{equation}

We now have to sum or average over the electron and photon polarisation states.

Use
\begin{equation}
\Sigma \epsilon_\mu^* \epsilon_\nu \longrightarrow - g_{\mu \nu}.
\end{equation}


Average squared amplitude over initial electron and photon polarisations. This is the spin-averaged Klein-Nishina equation.


\begin{align*}
\frac{1}{4} \Sigma_{spins} |M|^2 = \frac{e^4}{4} g_{\mu \rho} g_{\nu \sigma} \cdot tr\Bigg\lbrace &(\cancel{p}' + m \left[\frac{\gamma^\mu\cancel{k}\gamma^\nu + 2 \gamma^\mu p^\nu}{2p\cdot k} + \frac{\gamma^\nu \cancel{k}' \gamma^\mu - 2 \gamma^\nu p^\mu}{2p\cdot k'}\right]\\
& \cdot (\cancel{p} + m) \left[ \frac{\gamma^\sigma \cancel{k} \gamma^\rho + 2 \gamma^\rho p^\sigma}{2p \cdot k} + \frac{\gamma^\rho \cancel{k}' \gamma^\sigma - 2 \gamma^\sigma p^\rho}{2p\cdot k'}\right]\Bigg\rbrace
\end{align*}


\begin{equation}
\frac{1}{4} \Sigma_{spins} |M|^2 = 2 e^4 \left[\frac{p \cdot k'}{p \cdot k} + \frac{p \cdot k}{p \cdot k'} + 2 m^2 \left(\frac{1}{p \cdot k}-\frac{1}{p \cdot k'}\right) + m^4 \left( \frac{1}{p \cdot k} - \frac{1}{p \cdot k'}\right)^2   \right].
\end{equation}

Replacing $p \cdot k = m \omega$ and $p \cdot k' = m\omega'$

\begin{equation}
\frac{d \sigma}{d \cos \theta} = \frac{\pi \alpha^2}{m^2} \left(\frac{\omega'}{\omega}\right)^2 \left[ \frac{\omega'}{\omega} + \frac{\omega}{\omega'} - \sin^2 \theta \right].
\end{equation}


\subsubsection{Spin dependent Klein Nishina}

\begin{equation}
\frac{d \sigma}{d \Omega} = \frac{\alpha^2}{m^2} \left(\frac{\omega'}{\omega}\right)^2 \left[ \frac{\omega'}{\omega} + \frac{\omega}{\omega'} - 2\sin^2 \theta \cos^2 \phi \right].
\end{equation}

\subsubsection{Spin helicity}

In more details this can be written as
\begin{equation}
\frac{\mathrm{d}\sigma}{\mathrm{d}y} = \frac{2 \pi \alpha^2}{s((1-y) + m^2/s)^2} \left[(1-y) + \frac{m^2}{s}\right],
\end{equation}
where the first part in the bracket corresponds to a helicity-conserving interaction and the second term to a process that flips helicity. 




\subsection{Nonlinear Compton scattering cross section}
\label{Appendix:Sec:NLCompton}

The emission/scatter rates for the $n$-th harmonic depend on the laser polarisation: the scattering rate (spin-averaged and unpolarised states in/out) for $n$ scattered photons for circular polarisation is given (in natural units) by \cite{Ritus1985_QRR,BlackburnThesis}
\begin{equation}
\frac{\dif W_n}{\dif x} = \frac{\alpha m^2}{4 E}\left[ -4 J^2_n + a^2_0 \left(1-x+ \frac{1}{1-x}\right) (J^2_{n-1} + J^2_{n+1} - 2 J^2_n)\right],
\end{equation}
and for linear polarisation by
\begin{equation}
\frac{\dif W_n}{\dif x} = \frac{2 \alpha m^2}{\pi E}\int^{\pi/2}_0 \left[-A^2_0 + a^2_0 \left(1-x+ \frac{1}{1-x}\right) (A^2_1 - A_0 A_2)\right]\dif \varphi.
\end{equation}
The energy of the incoming photons is $\omega$ and the scattered photon carries an energy $\omega' = xE$, i.e $x$ is the fraction of energy the photon carries away. $a_0$ is the normalised vector potential of the laser pulse.
$J_n$ is shorthand for $J_n(z)$, and $A_i = A_i(n,a,b)$, with
\begin{align}
z &= \frac{2 a_0}{\sqrt{1+a^2_0}}\frac{\sqrt{x[nu - (1+nu)x]}}{u(1-x)},\\
u &= \frac{2k \cdot q}{q^2} = \frac{4 E \omega}{m^2 (1+a^2_0)}.
\end{align}
The function $A_i$ and its arguments are defined as 
\begin{align}
A_i(n,a,b) &= \frac{1}{\pi}\int^\pi_0 \cos^i \phi \cos [(a+2b\cos \phi) \sin \phi - n\phi] \dif \phi,\\
a &= \frac{\sqrt{8}Qn a_0}{1+a^2_0 + Q^2}\cos \varphi,\\
b &= - \frac{(n/2) a^2_0}{1+a^2_0 + Q^2},\\
Q^2 &= (1+a^2_0) \left[nu \left(\frac{1}{x}-1\right) -1 \right].
\end{align}



\chapter{Radiation Reaction}

\section{Landau-Lifschitz Equation}

In terms of 3-vectors this can be written as \cite{Bulanov2011_LADLL}:
\begin{align*}
\mathbf{F} = \frac{2e^3}{3m_e c^3 \sqrt{1-\frac{v^2}{c^2}}} \left\lbrace \left( \frac{\partial}{\partial t} + (\mathbf{v} \cdot \nabla) \right) \mathbf{E} + \frac{1}{c} \left[ \mathbf{v} \times \left( \frac{\partial}{\partial t} + (\mathbf{v} \cdot \nabla) \right) \mathbf{B} \right] \right\rbrace +\\
+ \frac{2e^4}{3m^2_e c^4}\left\lbrace \mathbf{E}\times\mathbf{B} + \frac{1}{c} ( \mathbf{B} \times (\mathbf{B} \times \mathbf{v})) + \frac{1}{c} \mathbf{E} (\mathbf{v} \cdot \mathbf{E}) \right\rbrace - \\
- \frac{2e^4}{3m^2_e c^5 \left( 1-\frac{v^2}{c^2}\right)} \mathbf{v} \left\lbrace\left( \mathbf{E} + \frac{1}{c} \mathbf{v} \times \mathbf{B}\right)^2 - \frac{1}{c^2} (\mathbf{v} \cdot \mathbf{E})^2 \right\rbrace,
\end{align*}

\chapter{Quantum synchrotron function}
\label{Appendix:Sec:QuantumSynchrotron}


The quantum synchrotron function describing the emissivity and spectral shape of the emitted radiation is given by \cite{BlackburnThesis}
\begin{equation}
F(\eta, \chi) = \frac{4 \chi}{3\eta^2}\left[ \left( 1 - \frac{2 \chi}{\eta} + \frac{1}{1- 2\chi/\eta}\right) K_{2/3} (\delta) - \int^\infty_\delta K_{1/3} (t) \dif t\right],
\end{equation}
where $\chi = \gamma b, \eta = \hbar \omega b/2m_e C^2$ are quantum parameters with $b = |\mathbf{E}_\perp + \mathbf{v}\times \mathbf{B}|/E_s$, $K_{2,3}, K_{1/3}$ are Bessel functions of the second kind, and $\delta$ is given by
\begin{equation}
\delta = \frac{4 \chi}{3 \eta^2}\left(1- \frac{2 \chi}{\eta}\right)^{-1}.
\end{equation}
The classical part of the quantum synchrotron function is here
\begin{equation}
F_{cl} (\eta, \chi) = \delta_{cl} \left[2K_{2/3} (\delta_{cl}) - \int^\infty_{\delta_{cl}} K_{1/3} (t) \dif t\right],
\end{equation}
where $\delta_{cl} = 4\chi/3\eta^2$.
The emission power in terms of $\chi$ reads
\begin{equation}
\frac{\dif P}{\dif \chi} = \frac{2 E \chi}{\eta} \frac{\dif W_\gamma}{\dif \chi} = \frac{\sqrt{3\alpha}}{\pi} \frac{m c^2}{\tau_c} F(\eta, \chi).
\end{equation}
The total emission power of the classical synchrotron emission and the quantum description are related by the Gaunt factor, $g(\eta)$:
\begin{equation}
P = \frac{2 \alpha}{3} \frac{mc^2}{\tau_c} \eta^2 g(\eta),
\end{equation}
were $g(\eta)$ is defined as
\begin{equation}
g(\eta) = \frac{3 \sqrt{3}}{2\pi \eta^2}\int^\infty_0 F(\eta, \chi) \dif \chi.
\end{equation}
In two limits of $\eta$ the Gaunt factor can be approximated by
\begin{equation}
g(\eta) \approx 
\begin{cases}
1 - (55 \sqrt{3}/16) \eta + 48 \eta^2 ~ &\mathrm{for }~ \eta \ll 1,\\
0.5564 \eta^{-4/3}  ~ &\mathrm{for }~ \eta \gg 1,
\end{cases}
\end{equation}
or by an analytic approximation that holds over the entire parameter space to within few percent 
\begin{equation}
g(\eta) = [1 + 4.8(1+\eta)\ln(1+1.7\eta) + 2.44\eta^2]^{-2/3}.
\end{equation}
Examining these expressions we see that the rate at which the power emission increases falls at high intensities to $P \propto \eta^{2/3}$ and as $\eta \propto a_0$, $P \propto a^{2/3}_0$.


\begin{equation}
S(\eta) = \frac{55 \alpha_f c}{24 \sqrt{3} \bar{\lambda}_c} m^2_e c^4 \eta^4 g_2(\eta),
\end{equation}
with $g_2(\eta)$
\begin{equation}
g_2(\eta) = \frac{\int_0^{\eta/2} \chi F(\eta, \chi) \mathrm{d}\chi}{\int_0 \chi F_{cl}\left(\frac{4\chi}{3\eta^2}\right)} = \frac{144}{55 \pi \eta^4} \int_0^{\eta/2}\chi F(\eta,\chi) \mathrm{d}\chi. 
\end{equation}

An accurate fit for $g_2$ is parametrised as follows:
\begin{equation}
g_2(\eta) \approx [1+ (1+4.528\eta)\ln(1+12.29\eta)+4.632\eta^2]^{-7/6}
\end{equation}
The limits are $g_2 \approx 1$ for $\eta \ll 1$, and $g_2 \approx 0.167\eta^{-7/3}$ for $\eta \gg 1$.

For Gaussian:

\begin{equation}
\left(\frac{\mathrm{d}\sigma^2}{\mathrm{d}t}\right)_{st} \approx \frac{\alpha_f  b^2}{\bar{\lambda}_c}\left(\frac{55b}{24\sqrt{3}}\left\langle\gamma\right\rangle^4 - \frac{8}{3}\sigma^2 \left\langle\gamma\right\rangle\right).
\end{equation}

\EliasComm{Needs some work, explanation and sub-sections?}

\chapter{Pair Production}

\section{Nonlinear Breit-Wheeler cross section}
\label{Theory:Sec:NonlinearBW}

The probability of pair production by one photon of momentum $l$ in interaction with $n$ laser photons of momentum $k$ per unit volume in unit time (circular polarisation) \cite{Ruffini2010_PAIRSASTRO}:

\begin{equation}
P_{\gamma\gamma} = \frac{e^2 m^2_e}{16 l_0} \sum^\infty_{n>n_0} \int^\nu_1 \frac{\dif \nu}{\nu^{3/2} (1+\nu)^{1/2}}[2 J^2_n(z) + \eta^2 (2\nu -1) (J^2_{n+1} + J^2_{n-1} - 2J^2_n)],
\end{equation}
with $\nu = (kl)^2/4(kq)(kq')$, $\nu_n = n/n_0$, $n_0 = 2m^2_\ast/(kl)$ and Bessel functions $J_n(z)$, where
\begin{equation}
z = 4m^2_e \frac{\eta (1+ \eta^2)^{1/2}}{(kl)}\left[\frac{\nu}{\nu_n}\left(1- \frac{\nu}{\nu_n}\right)\right]^{1/2}.
\end{equation}

Alternatively, in the high-field limit the rates can be related to the (quantum) synchrotron function (Sec. \ref{Appendix:Sec:QuantumSynchrotron}), such that the differential probability rate is given by \cite{BlackburnThesis}
\begin{equation}
\boxed{\frac{\dif W_{\pm}}{\dif \eta_+} = \frac{\alpha}{\tau_C} \frac{mc^2}{\hbar \omega} \chi \frac{\dif T(\chi)}{\dif \eta_+},}
\label{Theory:Eqns:NLBW:diffW}
\end{equation}
where $\tau_C = \hbar/mc^2 = 1.288\times10^{-6}\fs$ is the Compton time, $\chi$ is the quantum parameter and
\begin{equation}
\frac{\dif T(\chi)}{\dif \eta_+} = \frac{1}{2 \sqrt{3} \pi \chi^2} \left[ \left( \frac{2\chi}{\eta_+} - 1 + \frac{1}{2\chi/\eta_+ - 1}\right) K_{2/3} (\delta) - \int^\infty_\delta K_{1/3}(t)\dif t\right],
\end{equation}
with
\begin{equation}
\delta = \frac{4 \chi}{3\eta^2_+} \left( \frac{2 \chi}{\eta_+} - 1\right)^{-1}.
\end{equation}
The total rate of the Breit-Wheeler pair production is obtained by integrating Equation \eqref{Theory:Eqns:NLBW:diffW}:
\begin{equation}
\boxed{W_{\pm} = \frac{\alpha}{\tau_C} \frac{mc^2}{\hbar \omega}\chi T(\chi),}
\end{equation}
where $T(\chi)$ can analytically be approximated by \cite{Erber1966_PAIRS}
\begin{equation}
T(\chi) \approx \frac{0.16}{\chi} K^2_{2/3} \left(\frac{2}{3\chi}\right),
\end{equation}
or in two limits of $\chi$ by
\begin{equation}
T(\chi) = 
\begin{cases}
\frac{3\sqrt{3}}{8 \sqrt{2}} \exp\left(-\frac{4}{3\chi}\right) \,\, & \chi \ll 1,\\
0.6\chi^{-1/3} \,\,&\chi \gg 1.
\end{cases}
\end{equation}

\clearpage
\section{Bethe-Heitler Pair Production}
\label{Appendix:Sec:BetheHeitler}

The differential cross section of the Bethe-Heitler pair production process is given by crossing symmetry from bremsstrahlung (compare with Section \ref{Appendix:Brems_Born}) \cite{Ruffini2010_PAIRSASTRO}:
\begin{align}
\frac{\dif \sigma_{\gamma Z}}{\dif \gamma_+} = \frac{\alpha Z^2 r^2_e u_+ u_-}{\omega^3}\Bigg\lbrace &-\frac{4}{3} - 2 \gamma_+ \gamma_- c^2 \frac{u^2_+ + u^2_-}{u^2_+ u^2_-} + c^3 \left( \frac{\epsilon_- \gamma_+}{u^3_-} + \frac{\epsilon_+\gamma_-}{u^3_+}-\frac{\epsilon_-\epsilon^3_+}{u_- u_+ c}\right)\\ \nonumber
&+ L\Bigg[\frac{8}{3} \frac{\gamma_+ \gamma_- c^2}{u_+ u_-} + \frac{(\hbar \omega)^2 c^2}{m^2_e u^3_+ u^3_-} (\gamma^2_+ \gamma^2_- + u^2_+ u^2_-/c^4)\\ \nonumber
& -\frac{\hbar \omega}{2 m_e u_+ u_-}\left( \frac{\gamma_+ \gamma_- - u^2_+/c^2}{u^3_+/c^3}+ \frac{\gamma_+ \gamma_- - u^2_-/c^2}{u^3_-/c^3}+\frac{2\hbar \omega \gamma_+ \gamma_- c^2}{m_e u^2_+ u^2_-}\right)\Bigg]\Bigg\rbrace,
\end{align}
where
\begin{align*}
L &= \ln \left( \frac{\gamma_+ \gamma_- + u_+ u_-/c^2 -1}{\hbar \omega/m_e c^2}\right),\\
\epsilon_\pm &= 2 \ln(\gamma_\pm + u_\pm /c),
\end{align*}

with $r_e$ denotes the electron radius, $u_\pm$ the momentum of the positron and electron, respectively, and $\alpha$ is the fine-structure constant. $\gamma_\pm$ is the energy of the electron and positron, and $c$ is the speed of light.
\vspace{\baselineskip}

Similarly as for bremsstrahlung (see Section \ref{Appendix:Sec:Brems_RadiationLength}) the differential cross section can in the ultrarelativistic regime be approximated by a simplified equation in terms of the radiation length, $X_0$ (see Section \ref{Appendix:Sec:RadiationLength}) \cite{Tsai1974_PairsBrems,Tanabashi2018_PDGReview}:
\begin{equation}
\frac{\dif \sigma_{\gamma Z}}{\dif x} = \frac{A}{X_0 N_A} \left[ 1 - \frac{4}{3}x(1-x)\right],
\end{equation}
where $x = E/k$, where $k$ is the incident photon energy and $E$ the energy of the produced electron or positron. $A$ is the atomic weight and $N_A$ the Avogadro constant.
In the high-energy limit the total cross section becomes
\begin{equation}
\sigma_{\gamma Z} = \frac{7}{9}\frac{A}{X_0 N_A},
\end{equation}
which is accurate to within a few percent at $1\GeV$ and high-Z materials.




\chapter{Linear correlation coefficient}

The Pearson correlation coefficient, $\rho$, is a measure of the linear dependence of two variables. For two two random variables, $A$ and $B$, and $N$ data points is defined as \cite{Kendall1948_STATISTICS}:
\begin{equation}
\boxed{\rho(A, B) = \frac{1}{N-1} \sum_{i=1}^N \left( \frac{A_i-\mu_A}{\sigma_A}\right)^\ast \left(\frac{B_i-\mu_B}{\sigma_B}\right),}
\end{equation}
or alternatively in terms of the covariance
\begin{equation}
\rho(A, B) = \frac{\mathrm{cov}(A,B)}{\sigma_A \sigma_B},
\end{equation}
where $\mu$ denotes the mean and $\sigma$ the standard deviation of the distribution. The correlation coefficient can take values in the interval $[-1,1]$, where negative values indicate a negative linear correlation and positive value a positive linear correlation. Values close to $1$ or $-1$ imply a strong linear dependence, whereas small values close to zero suggest that there is no strong linear dependence between $A$ and $B$.


