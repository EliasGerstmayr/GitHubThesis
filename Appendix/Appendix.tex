\chapter{Notations and Vector Identities}

\section{Notation and Glossary}

Vectors will be noted in bold letters. 

In equations using variable names $\times$ will be the cross product and $\cdot$ will notate the dot product. 
In occasions using specific numbers as in $3 \times 10^{18}$ the $\times$ will just be a normal multiplication.

\section{Nabla Operator and Vector Identities}

In Cartesian coordinates
\begin{equation}
\nabla = (\partial x, \partial y, \partial z)^t
\end{equation}

In cylindrical coordinates

In spherical coordinates


\iffalse
\subsection{Convective derivative}

\begin{equation}
\frac{\mathrm{d}}{\mathrm{d}t} y = \frac{\partial y}{\partial} \mathbf{u} \cdot \nabla y,
\end{equation}
with $y = y(\mathbf{x},t)$ and velocity $\mathbf{u}(\mathbf{x},t)$



\subsection{IDENTITY}
Derivation of the identity:

\begin{equation}
\mathbf{v} \cdot \nabla \mathbf{p} = \nabla \mathbf{p} \cdot \mathbf{v} - \mathbf{v} \times (\nabla \times \mathbf{p}),
\end{equation}

with the covariant derivative $\nabla \mathbf{p}$.

\begin{equation}
A = B + C
\end{equation}

A:
\begin{equation}
\mathbf{v} \cdot \nabla \mathbf{p} = \left( \frac{\partial p_i}{\partial x_j}\right)_{ij}
\end{equation}

B:
\begin{equation}
\mathbf{v} \cdot \nabla \mathbf{p} = \left(v_i \frac{\partial p_i}{\partial x_j}\right)_j
\end{equation}

C:
\begin{equation}
\mathbf{v} \times (\nabla \times \mathbf{p}) = v_m \epsilon_{lmi} \epsilon_{ijk} \frac{\partial p_k}{\partial x_j},
\end{equation}
using the cyclic properties of the Levi-Civita symbol and the identity
\begin{equation}
\epsilon_{ilm} \epsilon_{ijk} = \delta_{lj} \delta_{mk} - \delta_{lk} \delta_{mj}
\end{equation}
the equation becomes
\begin{equation}
v_m \left( \delta_{lj}\delta_{mk} - \delta_{lk} \delta_{mj}\right) \frac{\partial p_k}{\partial x_j} = v_k \frac{\partial p_k}{\partial x_l} - v_m \frac{\partial p_l}{\partial x_m}
\end{equation}
\fi


\subsection{Curl of a gradient}
\label{Appendix:VectorIdentities:CurlGradZero}


For a scalar function $f(x,y,z)$, the gradient in Cartesian coordinates is
\begin{equation}
\nabla f(x,y,z) = \left( \frac{\partial f}{\partial x}, \frac{\partial f}{\partial y}, \frac{\partial f}{\partial z}\right)^t,
\end{equation}
where $\nabla f$ is now a vector field, $\mathbf{F} = \left( F_1, F_2, F_3\right)^t$. The curl of $\mathbf{F}$, $\nabla \times \mathbf{F}$ is then
\begin{equation}
\nabla \times \mathbf{F} = \left(\frac{\partial F_3}{\partial y} - \frac{\partial F_2}{\partial z}, \frac{\partial F_1}{\partial z} - \frac{\partial F_3}{\partial x}, \frac{\partial F_2}{\partial x} - \frac{\partial F_1}{\partial y}\right),
\end{equation}
or again in terms of $f(x,y,z)$:
\begin{equation}
\nabla \times \nabla f = \left(\frac{\partial^2 f}{\partial y \partial z} - \frac{\partial^2 f}{\partial z \partial y}, \frac{\partial^2 f}{\partial z \partial x} - \frac{\partial^2 f}{\partial x \partial z}, \frac{\partial^2 f}{\partial x \partial y} - \frac{\partial^2 f}{\partial y \partial x}\right).
\end{equation}
If $f$ is twice continuously differentiable, the order of the derivatives is interchangeable and $\nabla \times \nabla f = \mathbf{0}$.
The curl of a gradient is always zero.

\section{Maxwell's equations}

In differential form

\begin{align}
\nabla \cdot \mathbf{E} &= \frac{\rho}{\epsilon_0}\\
\nabla \cdot \mathbf{B} &= 0\\
\nabla \times \mathbf{E} &= - \frac{\partial \mathbf{B}}{\partial t}\\
\nabla \times \mathbf{B} &= \mu_0 \mathbf{J} + \mu_0 \epsilon_0 \frac{\partial \mathbf{E}}{\partial t}
\end{align}

In integral form

Combining these:

Wave equations.

\chapter{Frames of reference}

\subsection{Relativistic Transformations}

Boosts


\subsection{Rest frame}

\subsection{Center-of-mass frame}

Defined by $\mathbf{p_1} = - \mathbf{p_2}$, so that for scattering events $2 \rightarrow 2$ also $\mathbf{p_3} = -\mathbf{p_4}$.

In the special case $m_1 = m_2$, and $m_3 = m_4$, we also have $E^{CM}_1 = E^{CM}_2$ and $E^{CM}_4 = E^{CM}_4$.

The centre of mass energy 
\begin{equation}
s = \frac{E_1 E_2 (1-\cos \theta)}{4m^2_e c^4}
\end{equation}


\subsubsection{Scattering angle}

The scattering angle in the CM frame, $\theta_{CM}$, be defined as 
\begin{equation}
\mathbf{p} \cdot \mathbf{p}' = |\mathbf{p}| \cdot |\mathbf{p}'| \cos \theta_{CM}
\end{equation}

The angle be defined as 
\begin{equation}
\cos \theta_{CM} = \frac{s (t-u) + (m^2_1 - m^2_2)(m^2_3 - m^2_4)}{\sqrt{\lambda(s,m^2_1,m^2_2)}\sqrt{\lambda(s,m^2_3,m^2_4)}},
\end{equation}
where 
\begin{equation}
\lambda(a,b,c) = a^2 - 2a(b+c) + (b-c)^2,
\end{equation}
which is symmetric under $a \leftrightarrow b \leftrightarrow c$ and asymptotic behaviour $a \gg b,c: \lambda(a,b,c) \rightarrow a^2$.
For photons $m = 0$, so that $\lambda(a,b,c) = a^2$

\chapter{Single particle motion}

\section{Electron in a cylindrically symmetric electric field}

The previously discussed case of a particle in a homogeneous magnetic field corresponds to the motion in a magnetic electron spectrometer or the deflection of charged particles in a circular accelerator.
We will now discuss the motion of an electron in a cylindrically symmetric electric field which resembles the case of an electron in a plasma channel.
This derivation is based on \cite{WoodThesis}.

Assume a relativistic electron moving in the longitudinal direction and an electric field that is cylindrically symmetric around the propagation axis.
We ignore any vector potential contributions and only use a scalar potential.
The Lagrangian of this scenario is described by
\begin{equation}
L(\mathbf{r},\mathbf{v},t) = - m_e c^2 \gamma^{-1} - q \Phi,
\end{equation}
where $\Phi$ is the scalar potential describing the electrostatic field of the channel.
The scalar potential is the solution matching the radial Poisson equation in cylindrical symmetry

\begin{equation}
\frac{1}{r} \frac{\partial}{\partial r} \left(r \frac{\partial \phi}{\partial r}\right) = \frac{- e (n_0 - n_e)}{\epsilon_0},
\end{equation}

with the solution 
\begin{equation}
\phi = -\frac{(n_0 - n_e) e r^2}{4 \epsilon_0},
\end{equation}
where $n_0$ is the ion charge density and $n_e$ the electron density in the channel. 
In the limit of a completely evacuated channel, $n_e \rightarrow 0$, $\phi = -n_0 er^2/(4\epsilon_0)$.
As before $\mathbf{E} = - \nabla \phi$ and in this case $E_r = - \partial_r \phi$.
The equation of motion for the radial component is then
\begin{equation}
\frac{d}{dt} \mathbf{p} = - \frac{e^2 n_0}{2 \epsilon_0} r,
\end{equation}
with $\mathbf{p} = \gamma m_e \mathbf{\dot{r}}$.
Assume there is no axial acceleration field, the radial component of the velocity is small compared to $c$ and the longitudinal component is close to $c$: $\dot{\gamma}$ is then close to zero. 
The equation then takes the shape of a harmonic oscillator
\begin{equation}
\ddot{r} = - \frac{e^2 n_0}{2\epsilon_0 m_e \gamma} r = - \omega_\beta^2 r,
\end{equation}

with a frequency we will call the betatron frequency, $\omega_\beta$, which is related to the plasma frequency, $\omega_p$, by $\omega_\beta = \omega_p/\sqrt{2\gamma}$. The plasma frequency will be derived explicitly in Section \ref{Chap:Theory:Sec:PlasmaFreq}.
As this is the standard harmonic oscillator equation, it follows a solution of shape
\begin{equation}
r(t) = A \cos (\omega_\beta t + \varphi),
\end{equation}
and with initial conditions $r(t=0) = r_\beta$ and $\dot{r}(t=0) = 0$, we obtain
\begin{equation}
r(t) = r_\beta \cos(\omega_\beta t),
\end{equation}
where $r_\beta$ is the amplitude of the oscillation, also referred to as the betatron radius.
\vspace{\baselineskip}

If we continue with the assumption $\dot{\gamma} = 0$, this requires $\beta^2 = \beta^2_r + \beta^2_z = constant$. This gives $\beta_z = \sqrt{\beta^2 - \beta^2_r} \approx \beta (1-\beta^2/2)$, where we used a Taylor expansion and $\beta \approx 1$. If we use the solution for $r(t)$ derived above, $\beta_r = \dot{r}/c$ and $\sin^2 x = (1-\cos 2\theta)/2$, we obtain:
\begin{equation}
\beta_z \approx \beta \left(1-\frac{r^2_\beta \omega^2_\beta}{4 c^2}\right) + \beta \frac{r^2_\beta \omega^2_\beta}{4c^2} \cos (2 \omega_\beta t).
\end{equation}

Integrating this with the initial condition $z(t=0) = z_0$

\begin{equation}
\frac{z}{c} \approx \frac{z_0}{c} + \beta \left(1-\frac{r^2_\beta \omega^2_\beta}{4 c^2}\right) t + \beta \frac{r^2_\beta \omega_\beta}{8c^2}\sin(2\omega_\beta t),
\end{equation}

where we identify the drift velocity $\beta c \left(1-\frac{r^2_\beta \omega^2_\beta}{4 c^2}\right)$ and in the moving frame the familiar figure-of-eight motion again.


\section{Electron in a cylindrically symmetric electric field with axial field}

Now we add will add a constant axial field in the longitudinal (direction of electron propagation) to the previous case. This comes closer to the reality of a laser wakefield accelerator where electrons experience radial focusing and axial accelerating fields at the same time. Significant parts of this derivation are based on \cite{WoodThesis,Glinec2008_BETATRON}.

The electric field component shall now be $E_z$, with $z$ being the longitudinal axis.
This gives us the extra term $\frac{\partial \phi}{\partial z} = - E_z$.
The Euler-Lagrangian now has two non-trivial equations of motions.
First, as before
\begin{equation}
\frac{d}{dt} (\gamma m_e \dot{r}) = \frac{\partial }{\partial r} L = -\frac{e^2 n_0}{2\epsilon_0} r,
\label{Theory:Eq:EoM_Rad_Ax_R}
\end{equation}
and, second, the z-component
\begin{equation}
\frac{d}{dt} (\gamma m_e \dot{z}) = \frac{\partial }{\partial z} L = e E_z = const.
\label{Theory:Eq:EoM_Rad_Ax_Z}
\end{equation}

The change in momentum is now not negligible any more, $\dot{\gamma}\neq 0$, and we have to expand the brackets on the left-hand side of Equations \eqref{Theory:Eq:EoM_Rad_Ax_R} and \eqref{Theory:Eq:EoM_Rad_Ax_Z}.
The radial component is then
\begin{equation}
\dot{\gamma}m_e \dot{r} + \gamma m_e \ddot{r} = - \frac{e^2 n_0}{2 \epsilon_0} r.
\end{equation}

The axial component is equal to a constant, so we can simply integrate
\begin{equation}
\gamma m_e \dot{z} = - e E_z t + const.
\end{equation}

If we assume that the electron is highly relativistic $\dot{z} \rightarrow c$, so $\dot{\gamma} = eE_z/m_e c$ and

\begin{equation}
\gamma \beta_z \approx \gamma = - \frac{e E_z t}{m_e c} + \gamma_0 \beta_0,
\label{Theory:Eq:EoM_Rad_Ax_Gamma}
\end{equation}

where $\gamma_0 \beta_0$ are the initial conditions.
\begin{figure}
\centering
\includegraphics[width=.8\columnwidth]{betatron_traj.pdf}
\caption[Particle trajectory of an electron in a cylindrically symmetric electric field with a constant axial accelerating field.]{Particle trajectory of an electron in a cylindrically symmetric electric field with a constant axial accelerating field as described in Equation \eqref{Theory:Eqs:EoM_Betatron} for a plasma of density $n_e = 2 \times 10^{18} cm^{-3}$ as a function of propagation distance.}
\label{Theory:Figs:BetatronMotion}
\end{figure}
Inserting the result from Equation \eqref{Theory:Eq:EoM_Rad_Ax_Gamma} into Equation \eqref{Theory:Eq:EoM_Rad_Ax_R}, we obtain the differential equation

\begin{equation}
eE_z \dot{r} + (eE_z t + \gamma_0 \beta_0 m_e c) \ddot{r} = - \frac{e^2 n_0 c}{2 \epsilon_0} r,
\end{equation}
which we can write as 
\begin{equation}
(At + B) \ddot{r} + A \dot{r} + C r = 0,
\end{equation}
by using the abbreviations $A = eE_z$, $B = \gamma_0 \beta_0 m_e c$ and $C = e^2 n_0 c/2\epsilon_0$.
The analytic solution for this differential equation is \cite{WoodThesis}:
\begin{subequations}
\begin{empheq}[box=\widefbox]{align}
r(t) = \frac{\pi \sqrt{CB}}{A} r_{\beta 0} \Biggl[ &J_1 \left( 2 \sqrt{CB}/A \right) Y_0 \left(2 \sqrt{C (B+At)}/A\right) \\  \nonumber
&- Y_1 \left(2\sqrt{CB}/A\right)J_0\left(2\sqrt{C(B+At)}/A\right)\Biggr],
\end{empheq}
\label{Theory:Eqs:EoM_Betatron}
\end{subequations}
with the initial conditions $r(t=0) = r_{\beta 0}$ and $\dot{r}(t=0) = 0$. $J_1, J_0$ are Bessel functions of first kind of order $1$ and $0$, respectively. $Y_1, Y_0$ are Bessel function of the second kind of order $1$ and $0$.
This motion is shown in Figure \ref{Theory:Figs:BetatronMotion}. We see that the oscillations quickly lose amplitude and fall to a settled value, whereas the wavelength of the oscillations increases as the electron gains momentum and inertia.


\chapter{Synchrotron radiation}


\section{Synchrotron Radiation}

\EliasComm{Draw parallel (see Jackson on undulator equation) to figure of eight motion and hence ICS/Betatron/harmonics.}

based on Jon Wood \cite{WoodThesis} and Jackson \cite{Jackson}:

\EliasComm{energy loss per turn.}
\EliasComm{sketch on emission and cones.}

Spectrum (angularly integrated)

\begin{equation}
\frac{dI}{d\omega} = \frac{\sqrt{3}}{4} \frac{e^2}{\pi c \epsilon_0} \gamma \frac{\omega}{\omega_c} \int_{\omega/\omega_c}^{\inf} K_{5/3}(x)dx
\end{equation}

Equation for on- and off-axis synchrotron

\begin{equation}
\frac{d^2 I}{dEd\Omega} = \frac{3e^2}{16\pi^3 \hbar c \epsilon_0} \gamma^2 \frac{E^2}{E_c^2} (1+\gamma^2 \theta^2)^2 [K_{2/3}^2(\psi) + \frac{\gamma^2 \theta^2}{1+\gamma^2 \theta^2} K_{1/3}^2 (\psi)]
\end{equation}

\begin{equation}
\psi = \frac{E}{2E_c}(1+\gamma^2 \theta^2)^{3/2}
\end{equation}

Critical frequency
\begin{equation}
\omega_c = \frac{3}{2} \gamma^3 \frac{c}{\rho}
\end{equation}

\section{Undulator and Wiggler Radiation}
\label{Theory:Sec:UndulatorWiggler}


\begin{figure}
\centering
\includegraphics[width=.5\columnwidth]{wiggler_offaxis.png}
\caption{Synchrotron radiation.}
\end{figure}

To harvest and direct the radiation of an accelerated relativistic electron more efficiently a set of alternating magnets is used. The electron oscillates and emits radiation preferentially at the turning points as the acceleration is the strongest there and in a forwards-pointing cone with a diameter proportional to $1/\gamma$.\addref

The rapid change of direction in the field and hence forced trajectory of the electron in transversal direction results in a change of effective Lorentz factor in the forward direction.

\begin{equation}
\left\langle\gamma\right\rangle \approx \frac{\gamma}{\sqrt{1 + 0.5 K^2}}
\end{equation}

Commonly two regimes are identified based on the wiggler parameter:\addref

\begin{equation}
K = \frac{eB_0}{m_e ck_u} = \frac{e B_0 \lambda_u}{2\pi m_e c}
\end{equation}

For $K \ll 1$ it is called undulator regime, large $K \gg 1$ indicate the wiggler regime.
In the wiggler regime the divergence of the cone is larger and the radiation has a large bandwidth.
In the undulator regime the radiation is strongly collimated and interference can occur resulting in single harmonics being emitted.

Constructive interference with harmonics of wavelength:

\begin{equation}
\lambda = \frac{\lambda_u}{2 n \gamma^2} \left( 1+ \frac{K^2}{2} + \gamma^2 \theta^2\right)
\end{equation}

If an electron beam of sufficient quality is inserted in a very long undulator the emitted radiation and the electron beam are able to interact and modulate resulting in the emission of coherent radiation. This is observed in free-electron lasers.

The equations introduced for synchrotron radiation work similarly for wiggler radiation. The reference to the bending radius is usually replaced by an expression including the magnet periodicity or the $K$ parameter.


For a wiggler this becomes:
\begin{equation}
\omega_c \sim \frac{3}{2} \gamma^2 K k_u c
\end{equation}

Maximum energy 
\begin{equation}
\hbar \omega_{eV} = \frac{0.496[E(GeV)]^2}{(1+ K^2/2) \lambda_0 (m)}
\end{equation}

Number of emitted photons per magnet period (later compare to number of photons emitted in wiggler), is then \cite{Jackson}
\begin{equation}
N_\gamma = \frac{2 \pi}{3} \alpha K^2
\end{equation}



\section{Betatron Radiation}

Just as the evacuated `bubble' in a wakefield accelerates particles in a similar way as conventional RF cavities would, the generation of X-rays draws another parallel in conventional devices.
\vspace{\baselineskip}

Due to initial transverse momentum on injection, and the focusing and defocusing fields of the cavity, electrons start to oscillate around the central axis\addref. 
The electrons oscillate at the betatron frequency $\omega_\beta$
\begin{equation}
\omega_\beta = \omega_p / \sqrt{2 \gamma},
\end{equation}
which depends on the plasma frequency $\omega_p$ and the energy of the electrons, here given by the relativistic Lorentz factor $\gamma$.

Similarly to electrons in an insertion device (undulator or wiggler) these oscillations lead to the emission of radiation in a narrow forward pointing cone due to the relativistic forward momentum of the particles.
Not surprisingly, the equations describing both processes are very similar.

The collimation depends on $K$, the wiggler parameter (for insertion devices) or betatron strength parameter
\begin{equation}
K = \gamma k_\beta r_\beta,
\end{equation}
where $k_\beta = \omega_\beta / c$ is the wavenumber and $r_\beta$ is the betatron radius, the amplitude of the oscillations. If $K \gg 1$ it is called the undulator parameter.

The radiation generated follows the on-axis synchrotron radiation with a critical energy
\begin{equation}
\boxed{E_c = \frac{3}{4} \hbar \gamma^2 \omega^2_p r_\beta / c,}
\end{equation}
where this energy parameter indicates when approximately half the energy radiated above and below this value\addref.
\EliasComm{Explain this better.}
\vspace{\baselineskip}

The oscillation of the particles and hence the emission of radiation can be enhanced in wakefield experiments by tailoring the wavefront \cite{Mangles2009_BETATRON} or by using different injection mechanisms\addref.
In LWFA betatron radiation typically reaches the soft X-ray regime of a few to tens of keV.
\EliasComm{Maybe show energy increase can be done by increasing energy, plasma frequency or betatron radius.}

Difference here is that we use an quasi-electrostatic field instead of a magnetic field, but radiation from a synchrotron, an insertion device (wiggler) and betatron radiation share a common description rooted in the equation of motion (figure-of-eight motion) even though using different field structures and $K$ parameters.




\iffalse
\chapter{Radiation Integrals}

\subsection{Bessel functions}

Functions solving the differential equations ....

The Bessel function of first order transforms:
...

The generalized Bessel function
\begin{equation}
J_n(x,y) = \Sigma 
\end{equation}

Recursion

First order for radiation integrals (see Synchrotron/Betatron and in generalized form ICS)


\subsection{Lienard Wiechert potentials}
Lienard Wiechart potentials and fields.

Jackson \cite{Jackson}:
\begin{equation}
\mathbf{E} (\mathbf{r}, t) = \frac{e}{4 \pi \epsilon_0} \left[\frac{\mathbf{n}-\mathbf{\beta}}{\gamma^2 R^2 (1-\mathbf{\beta} \cdot \mathbf{n})^3}+\frac{\mathbf{n}\times\left[(\mathbf{n}-\mathbf{\beta})\times\dot{\mathbf{\beta}}\right]}{cR(1-\mathbf{\beta}\cdot\mathbf{n})^3}\right]_{retarded}
\end{equation}

\begin{equation}
\mathbf{B}(\mathbf{r},t) = \frac{\mathbf{n}\times\mathbf{E}}{c}
\end{equation}

Poynting vector
\begin{equation}
\mathbf{S} = \mu_0^{-1} \mathbf{E}\times\mathbf{B} = (c \mu_0)^{-1} \mathbf{E} \times \mathbf{n} \times \mathbf{E}
\end{equation}

\begin{equation}
[\mathbf{S} \cdot \mathbf{n}]_{retarded} = \frac{e^2}{16 \pi c \epsilon_0} \left[ \frac{1}{R^2}|\frac{\mathbf{n} \times [(\mathbf{n}-\mathbf{\beta})\times\dot{\mathbf{\beta}}]}{(1-\mathbf{\beta}\cdot \mathbf{n})^3}|^2\right]_{retarded}
\end{equation}

\begin{equation}
\frac{dP(t)}{d\Omega} = R^2 [\mathbf{S}\cdot\mathbf{n}]_{retarded}
\end{equation}

Total power radiated:
\begin{equation}
P = \frac{e^2 \gamma^6}{6 \pi c \epsilon_0} \left( |\mathbf{\dot{\beta}}|^2 - |\mathbf{\beta} \times \dot{\mathbf{\beta}}|^2\right)
\end{equation}

%\begin{equation}
%\frac{d^2 I}{d\omega d\Omega} = \frac{e^2 \omega^2}{16 \pi^3 c \epsilon_0} |\int_{-\inf}^\inf \mathbf{n} \times (\mathbf{n} \times \mathbf{\beta}) e^{i\omega (t'- \mathbf{n}\cdot\mathbf{r}(t')/c)} dt|^2
%\end{equation}

\fi

\chapter{Cross sections for fundamental QED processes}

\section{Feynman diagrams and convention}

\subsection{Feynman rules for QED}

Photon rules:

New vertex $- ie \gamma^\mu$

Photon propagator $= \frac{-ig_{\mu\nu}}{q^2 + i\epsilon}$

External photon lines $ =\epsilon_\mu (p)$ and $=\epsilon^\ast_\mu (p)$

Fermion rule

Vertex $=-ig$

Propagator $=\frac{i(\cancel(p) + m)}{p^2 - m^2 + i\epsilon}$

External legs fermion $u^s (p)$ and $\bar{u}^s (p)$

External legs antifermion $\bar{v}^s (k)$ and $v^s(k)$

Impose momentum conservation at each vertex

Integrate over each undetermined loop momentum

Figure out overall sign of diagram

\subsection{Mandelstam variables}

\begin{align}
s & = (p + p')^2 &= (k + k')^2; \nonumber\\
t &= (k-p)^2 &= (k' - p')^2; \nonumber\\
u &= (k'-p)^2 &= (k-p')^2.
\end{align}


\subsection{Trace Identities}

From \cite{PeskinSchroeder}:

Trace theorems:
\begin{align*}
tr(\mathbf{1}) &= 4\\
tr(any~odd~number~of~\gamma s) &= 0\\
tr(\gamma^\mu \gamma^\nu) = 4 g^{\mu \nu}\\
tr(\gamma^\mu \gamma^\nu \gamma^\rho \gamma^\sigma) &= 4(g^{\mu \nu} g^{\rho \sigma} - g^{\mu \rho} g^{\nu \sigma} + g^{\mu \sigma} g^{\nu\rho})\\
tr(\gamma^5) &= 0\\
tr(\gamma^\mu \gamma^\nu \gamma^5) &= 0\\
tr(\gamma^\mu \gamma^\nu \gamma^\rho \gamma^\sigma \gamma^5) = -4i \epsilon^{\mu \nu \rho \sigma}.
\end{align*}

Reversal of orders is allowed within a trace:
\begin{equation}
tr(\gamma^\mu \gamma^\nu \gamma^\rho \gamma^\sigma ...) = tr(...\gamma^\sigma \gamma^\rho\gamma^\nu \gamma^\mu).
\end{equation}

Eliminating dotted $\gamma$-matrices:
\begin{align*}
\gamma^\mu \gamma_\mu = g_{\mu \nu}\gamma^\mu \gamma^\nu = \frac{1}{2}g_{\mu \nu}\lbrace\gamma^\mu,\gamma^\nu\rbrace = g_{\mu \nu} g^{\mu \nu} 4.
\end{align*}

Contraction identities:
\begin{align*}
\gamma^\mu \gamma^\nu \gamma_\mu &= - 2\gamma^\nu\\
\gamma^\mu \gamma^\nu \gamma^\rho \gamma_\mu &= 4 g^{\nu \rho}\\
\gamma^\mu \gamma^\nu \gamma^\rho \gamma^\sigma \gamma_\mu = - 2\gamma^\sigma \gamma^\rho \gamma^\nu
\end{align*}
\subsection{Other identities}

\begin{align*}
\sum u^s(p) \bar{u}^s (p) &= \cancel{p} + m \\
\sum v^s (p) \bar{v}^s (p) &= \cancel{p} - m.
\end{align*}

\subsection{Calculating Matrix Elements from Feynman diagrams}


\section{Cross sections for fundamental QED processes}

\subsection{Derivation of the Klein-Nishina equation}
\label{Appendix:QEDDeriv_KleinNishina}

\begin{figure}[h]
\centering
\includegraphics[width=0.4\columnwidth]{Compton_Feyn1.pdf}\includegraphics[width=0.4\columnwidth]{Compton_Feyn2.pdf}
\caption{Feynman diagrams for Compton scattering. Time axis from left to right. s- and u-channel.}
\label{Appendix:Figs:ComptonScatter_Feynman}
\end{figure}

In this section the Klein-Nishina equation is derived explicitly by evaluating the matrix element of the lowest order Feynman diagrams contributing to Compton scattering (see Figure \ref{Appendix:Figs:ComptonScatter_Feynman}). The cross section combines the kinematics of the process with the matrix element which includes the physics of the interaction. The following derivation is based on \cite{PeskinSchroeder} and \cite{Schwartz}.

Based on the two Feynman diagrams in Figure \ref{Appendix:Figs:ComptonScatter_Feynman} and using the Feynman rules introduced in Section\addnum{} we find the following expression for the matrix element:

\begin{align}
iM &= \bar{u}(p')(-ie\gamma^\mu)\epsilon_\mu^*(k') \frac{i(\cancel{p} + \cancel{k} + m)}{(p+k)^2 - m^2}(-ie\gamma^\nu)\epsilon_\nu(k)u(p)\nonumber\\
&+ \bar{u}(p')(-ie\gamma^\nu)\epsilon_\nu(k) \frac{i(\cancel{p} - \cancel{k}' + m)}{(p - k')^2 - m^2}(-ie\gamma^\mu) \epsilon_\mu^*(k')u(p)\\
&= -ie^2 \epsilon_\mu^*(k')\epsilon_\nu(k) \bar{u}(p') \left[\frac{\gamma^\mu (\cancel{p} + \cancel{k} +m)\gamma^\nu}{(p+k)^2 - m^2} + \frac{\gamma^\nu(\cancel{p}-\cancel{k'}+m)\gamma^\mu}{(p-k')^2-m^2}\right] u(p).
\end{align}

Using $p^2 = m^2$ and $k^2 = 0$, we can simplify the denominators
\begin{align}
(p + k)^2 - m^2 = p^2 + k^2 +2p\cdot k-m^2 &= 2p \cdot k\\
(p - k')^2 - m^2 = p^2 + k'^2 - 2 p\cdot k' - m^2 &= -2p \cdot k'
\end{align}

Also use
\begin{align*}
(\cancel{p} + m) \gamma^\nu u(p) & = (2p^\nu - \gamma^\nu \cancel{p} + \gamma^\nu m) u(p)\\
&= 2 p^\nu u(p) - \gamma^\nu (\cancel{p} - m) u(p)\\
&= 2p^\nu u(p).
\end{align*}

Applying both we obtain:
\begin{equation}
iM = -ie^2\epsilon_\mu^*(k')\epsilon_\nu(k)\bar{u}(p')\left[\frac{\gamma^\mu \cancel{k} \gamma^\nu + 2 \gamma^\mu p^\nu}{2p\cdot k} + \frac{-\gamma_\nu \cancel{k'} \gamma^\mu + 2 \gamma^\nu p^\mu}{-2p\cdot k'}\right]u(p).
\end{equation}

We now have to sum or average over the electron and photon polarisation states.

Use
\begin{equation}
\Sigma \epsilon_\mu^* \epsilon_\nu \longrightarrow - g_{\mu \nu}.
\end{equation}


Average squared amplitude over initial electron and photon polarisations. This is the spin-averaged Klein-Nishina equation.


\begin{align*}
\frac{1}{4} \Sigma_{spins} |M|^2 = \frac{e^4}{4} g_{\mu \rho} g_{\nu \sigma} \cdot tr\Bigg\lbrace &(\cancel{p}' + m \left[\frac{\gamma^\mu\cancel{k}\gamma^\nu + 2 \gamma^\mu p^\nu}{2p\cdot k} + \frac{\gamma^\nu \cancel{k}' \gamma^\mu - 2 \gamma^\nu p^\mu}{2p\cdot k'}\right]\\
& \cdot (\cancel{p} + m) \left[ \frac{\gamma^\sigma \cancel{k} \gamma^\rho + 2 \gamma^\rho p^\sigma}{2p \cdot k} + \frac{\gamma^\rho \cancel{k}' \gamma^\sigma - 2 \gamma^\sigma p^\rho}{2p\cdot k'}\right]\Bigg\rbrace
\end{align*}


\begin{equation}
\frac{1}{4} \Sigma_{spins} |M|^2 = 2 e^4 \left[\frac{p \cdot k'}{p \cdot k} + \frac{p \cdot k}{p \cdot k'} + 2 m^2 \left(\frac{1}{p \cdot k}-\frac{1}{p \cdot k'}\right) + m^4 \left( \frac{1}{p \cdot k} - \frac{1}{p \cdot k'}\right)^2   \right].
\end{equation}

Replacing $p \cdot k = m \omega$ and $p \cdot k' = m\omega'$

\begin{equation}
\frac{d \sigma}{d \cos \theta} = \frac{\pi \alpha^2}{m^2} \left(\frac{\omega'}{\omega}\right)^2 \left[ \frac{\omega'}{\omega} + \frac{\omega}{\omega'} - \sin^2 \theta \right].
\end{equation}


\subsubsection{Spin dependent Klein Nishina}

\begin{equation}
\frac{d \sigma}{d \Omega} = \frac{\alpha^2}{m^2} \left(\frac{\omega'}{\omega}\right)^2 \left[ \frac{\omega'}{\omega} + \frac{\omega}{\omega'} - 2\sin^2 \theta \cos^2 \phi \right].
\end{equation}

\subsubsection{Spin helicity}

In more details this can be written as
\begin{equation}
\frac{\mathrm{d}\sigma}{\mathrm{d}y} = \frac{2 \pi \alpha^2}{s((1-y) + m^2/s)^2} \left[(1-y) + \frac{m^2}{s}\right],
\end{equation}
where the first part in the bracket corresponds to a helicity-conserving interaction and the second term to a process that flips helicity. 


\subsection{Derivation of the Dirac annihilation cross section}

We can re-use the result from Compton scattering to determine the cross section of the pair annihilation process $e^- e^+ \rightarrow \gamma \gamma$ using cross symmetry.

By replacing:
\begin{align*}
p &\rightarrow p_1\\
p' &\rightarrow -p_2\\
k &\rightarrow - k_1\\
k' &\rightarrow k_2
\end{align*}


\subsection{Derivation of the linear Breit-Wheeler cross section}

The cross section for the Breit-Wheeler process is simply given by XX.

From PRL Yu 2019 122\addref{}:

The energy of the electron and the positron in the production process is equal as it is in the CM frame.
\begin{equation}
\epsilon_{ec} = \epsilon_{pc} = \frac{1}{2}\sqrt{2\epsilon_{\gamma1} \epsilon_{\gamma2} (1-\cos \theta_c)}
\end{equation}
The momenta follow
\begin{equation}
|p_{ec}| = |p_{pc}| = \sqrt{\frac{1}{4}\left[\left(\frac{\epsilon_{\gamma1}+\epsilon_{\gamma2}}{c}\right)^2 - (\mathbf{p}_{\gamma1} + \mathbf{p}_{\gamma2})^2\right] - (m_ec^2)^2}
\end{equation}
The gamma momenta are in the laboratory frame. The direction of the electron and positron momentum is isotropic in the CM frame but has to satisfy $\mathbf{p}_{ec} = - \mathbf{p}_{pc}$. In the laboratory frame
\begin{equation}
\epsilon_{e,p} = \gamma_c (\epsilon_{ec,pc} + \mathbf{v}_c \cdot \mathbf{p}_{e,p}),
\end{equation}
with the momentum in lab frame
\begin{equation}
\mathbf{p}_{e,p} = \mathbf{p}_{ec,pc} + \frac{\gamma_c - 1}{v^2_c} (\mathbf{v_c} \cdot \mathbf{p}_{ec,pc}) \cdot \mathbf{v}_c + \frac{\gamma_c \mathbf{v}_c \epsilon_{ec,pc}}{c^2},
\end{equation}
with $\gamma_c = 1/\sqrt{1-(v_c/c)^2}$ and $\mathbf{v}_c$, the velocity of the centre of mass frame
\begin{equation}
\mathbf{v}_c = \frac{(\mathbf{p}_{\gamma1} + \mathbf{p}_{\gamma2})^2}{\epsilon_{\gamma1} + \epsilon_{\gamma2}}
\end{equation}


\subsection{Nonlinear Breit-Wheeler cross section}
\label{Theory:Sec:NonlinearBW}

The probability of pair production by one photon of momentum $l$ in interaction with $n$ laser photons of momentum $k$ per unit volume in unit time (circular polarisation) \cite{Ruffini2010_PAIRSASTRO}:

\begin{equation}
P_{\gamma\gamma} = \frac{e^2 m^2_e}{16 l_0} \sum^\infty_{n>n_0} \int^\nu_1 \frac{\dif \nu}{\nu^{3/2} (1+\nu)^{1/2}}[2 J^2_n(z) + \eta^2 (2\nu -1) (J^2_{n+1} + J^2_{n-1} - 2J^2_n)],
\end{equation}
with $\nu = (kl)^2/4(kq)(kq')$, $\nu_n = n/n_0$, $n_0 = 2m^2_\ast/(kl)$ and Bessel functions $J_n(z)$.

\begin{equation}
z = 4m^2_e \frac{\eta (1+ \eta^2)^{1/2}}{(kl)}\left[\frac{\nu}{\nu_n}\left(1- \frac{\nu}{\nu_n}\right)\right]^{1/2}.
\end{equation}

Alternatively, using results from \cite{BlackburnThesis}:

The differential probability rate:

\begin{equation}
\frac{\dif W_{\pm}}{\dif \eta_+} = \frac{\alpha}{\tau_C} \frac{mc^2}{\hbar \omega} \chi \frac{\dif T(\chi)}{\dif \eta_+}.
\end{equation}

\begin{equation}
\frac{\dif T(\chi)}{\dif \eta_+} = \frac{1}{2 \sqrt{3} \pi \chi^2} \left[ \left( \frac{2\chi}{\eta_+} - 1 + \frac{1}{2\chi/\eta_+ - 1}\right) K_{2/3} (\delta) - \int^\infty_\delta K_{1/3}(t)\dif t\right],
\end{equation}
where
\begin{equation}
\delta = \frac{4 \chi}{3\eta^2_+} \left( \frac{2 \chi}{\eta_+} - 1\right)^{-1}.
\end{equation}
Conservation law $2\chi = \eta_+ + \eta_-$.


The total Breit-Wheeler pair production 
\begin{equation}
W_{\pm} = \frac{\alpha}{\tau_C} \frac{mc^2}{\hbar \omega}\chi T(\chi),
\end{equation}
where
\begin{equation}
T(\chi) = 
\begin{cases}
\frac{3\sqrt{3}}{8 \sqrt{2}} \exp\left(-\frac{4}{3\chi}\right) \,\, & \chi \ll 1,\\
0.6\chi^{-1/3} \,\,&\chi \gg 1.
\end{cases}
\end{equation}

Approximation (ERBER\addref{})
\begin{equation}
T(\chi) \approx \frac{0.16}{\chi} K^2_{2/3} \left(\frac{2}{3\chi}\right).
\end{equation}

\subsection{Nonlinear Compton scattering cross section}


The emission/scatter rates for the $n$-th harmonic depend on the laser polarisation \cite{Ritus1985_QRR,BlackburnThesis} and are given by (averaged spin, in and out, unpolarised states) for $n$ scattered photons for circular polarisation
\begin{equation}
\frac{\dif W_n}{\dif x} = \frac{\alpha m^2}{4 E}\left[ -4 J^2_n + a^2_0 \left(1-x+ \frac{1}{1-x}\right) (J^2_{n-1} + J^2_{n+1} - 2 J^2_n)\right],
\end{equation}
and for linear polarisation
\begin{equation}
\frac{\dif W_n}{\dif x} = \frac{2 \alpha m^2}{\pi E}\int^{\pi/2}_0 \left[-A^2_0 + a^2_0 \left(1-x+ \frac{1}{1-x}\right) (A^2_1 - A_0 A_2)\right]\dif \varphi.
\end{equation}
The frequency of the incoming photons is $\omega$, the scattered photon carries an energy $\omega' = xE$, i.e $x$ is the fraction of energy the photon carries away.
$J_n$ is shorthand for $J_n(z)$, and $A_i = A_i(n,a,b)$, with
\begin{align}
z &= \frac{2 a_0}{\sqrt{1+a^2_0}}\frac{\sqrt{x(nu - (1+nu)x)}}{u(1-x)},\\
u &= \frac{2k \cdot q}{q^2} = \frac{4 E \omega}{m^2 (1+a^2_0)}.
\end{align}
The function $A_i$ and its arguments are defined as 
\begin{align}
A_i(n,a,b) &= \frac{1}{\pi}\int^\pi_0 \cos^i \phi \cos [(a+2b\cos \phi) \sin \phi - n\phi] \dif \phi,\\
a &= \frac{\sqrt{8}Qn a_0}{1+a^2_0 + Q^2}\cos \varphi,\\
b &= - \frac{(n/2) a^2_0}{1+a^2_0 + Q^2},\\
Q^2 &= (1+a^2_0) \left[nu \left(\frac{1}{x}-1\right) -1 \right].
\end{align}

\subsection{Radiation Length}


A useful quantity is the radiation length, $X_0$ \cite{Jackson}:

\begin{equation}
\frac{\dif E}{\dif x} = -\frac{E}{X_0},
\end{equation}
with $E(x) = E_0 e^{-x/X_0}$ and
\begin{equation}
X_0 = \left[ 4N \frac{Z(Z+1)e^2}{\hbar c} \left(\frac{z^2 e^2}{Mc^2}\right)^2 \ln \left(\frac{233 M}{Z^{1/3} m}\right)\right]^{-1}
\end{equation}
or (PDG\addref{}, Tsai REF\addref):
\begin{equation}
X_0 = \left[ 4 \alpha r^2_e \frac{N_A}{A}\left\lbrace Z^2 \left( L_{rad} - f(Z) \right) + Z L'_{rad} \right\rbrace\right]^{-1},
\end{equation}
with $A = 1~ \mathrm{g}\,\mathrm{mol}^{-1}$.  $f(Z)$ for up to uranium can be approximated by (REF\addref)
\begin{equation}
f(Z) = a^2 \left[(1+a^2)^{-1} + 0.20206 - 0.0369 a^2 + 0.0083a^4 - 0.002 a^6\right],
\end{equation}
with $a = \alpha Z$.
The radiation length in a mixture or compound of $n$ components can be approximated by the sum of radiation lengths of the individual components weighted by their weight
\begin{equation}
\frac{1}{X_0} = \sum^n_{j = 1} \frac{w_j}{X_j}.
\end{equation}


\subsection{Bremsstrahlung in Born approximation}
\label{Appendix:Brems_Born}

The (in energy) differential cross section for electron-nucleus bremsstrahlung in a fixed field in Born approximation, neglecting the recoil of the nucleus, is given by\addref
\begin{align}
\frac{\dif \sigma_{eZ}}{\dif \omega} = \frac{\alpha Z^2 r^2_e}{\omega}\frac{u'}{u}\Bigg\lbrace &\frac{4}{3} - 2 \gamma \gamma' c^2 \frac{u^2 + u'^2}{u^2 u'^2} + c^3 \left( \frac{\epsilon \gamma'}{u^3} + \frac{\epsilon'\gamma}{u'^3}-\frac{\epsilon\epsilon^3}{u u' c}\right)\\ \nonumber
&+ L\Bigg[\frac{8}{3} \frac{\gamma \gamma' c^2}{u u'} + \frac{(\hbar \omega)^2 c^2}{m^2_e u^3 u'^3} (\gamma^2 \gamma'^2 + u^2 u'^2/c^4)\\ \nonumber
& +\frac{\hbar \omega}{2 m_e u u'}\left( \frac{\gamma \gamma'+ u^2/c^2}{u^3/c^3}- \frac{\gamma \gamma' + u^2/c^2}{u'^3/c^3}-\frac{2\hbar \omega \gamma \gamma'c^2}{m_e u^2 u'^2}\right)\Bigg]\Bigg\rbrace,
\end{align}
where
\begin{align*}
L &= \ln \left( \frac{\gamma \gamma' + u u'/c^2 -1}{\hbar \omega/m_e c^2}\right),\\
\epsilon &= 2 \ln(\gamma + u/c),\\
\epsilon' &= 2 \ln(\gamma' + u'/c).
\end{align*}
Primed quantities denote post-interaction values and $\gamma m_e c^2 = \hbar \omega + \gamma' m_e c^2$, assuming all energy is transferred into radiation. 
The differential equation in the ultra-relativistic limit including screening and electron contributions is given in the Appendix in Section \ref{Appendix:Brems_with_Screening}.


\subsection{Bremsstrahlung with screening}
\label{Appendix:Brems_with_Screening}

The (in energy) differential cross section for electron-nucleus bremsstrahlung is in the ultra-relativistic limit, including electron contributions and screening, given by\addref
\begin{align}
\frac{\dif \sigma_{eZ}}{\dif \omega} = \frac{\alpha r^2_e}{\omega} \Bigg\lbrace &\left(\frac{4}{3} - \frac{4}{3}y + y^2\right) \times \left[Z^2\left(\phi_1 - \frac{4}{3}\ln Z - 4f\right) + Z \left(\psi_1 - \frac{8}{3}\ln Z\right)\right]\\ \nonumber 
& + \frac{2}{3}(1-y) \left[ Z^2 (\phi_1 - \phi_2) + Z(\psi_1 - \psi_2)\right]\Bigg\rbrace,
\end{align}
with $y = \hbar \omega/\gamma m_e c^2$, the ratio of the emitted photon energy to the initial electron energy. $Z$ is the atomic number of the nucleus.

The Coulomb correction term for the atomic electrons is given by
\begin{equation}
f(z) \approx 1.202 z - 1.0369z^2 + 1.008z^3/(1+z).
\end{equation}

\begin{align*}
\phi_1 &= 2\left[1+\ln\left(a^2 Z^{2/3} m^2_e\right)\right] - 2\ln(1+b^2) - 4b \tan^{-1} (b^{-1}),\\
\phi_2 &= 2\left[\frac{2}{3}+\ln\left(a^2 Z^{2/3}m^2_e\right)\right] -\ln(1+b^2) + 8b^2 \left[1-b \tan^{-1}(b^{-1}) - \frac{3}{4}\ln(1+b^{-2})\right],\\
\psi_1 &= 2\left[1+\ln\left(a'^2 Z^{4/3}m^2_e\right)\right] -2\ln(1+b'^2) - 4b' \tan^{-1} (b^{-1}),\\
\psi_2 &= 2\left[\frac{2}{3}+\ln \left(a'^2 Z^{4/3} m^2_e\right)\right] - 2 \ln (1+b'^2) + 8 b'^2 \left[1-b \tan^{-1} (b'^{-1}) - \frac{3}{4} \ln (1+b'^{-2})\right], 
\end{align*}
with $a=184.15(2.718)^{-1/2} Z^{-1/3}/m_e$, $a' = 1194(2.718)^{-1/2} Z^{-2/3}/m_e$, $b = a \delta$, $b' = a' \delta$, $\delta = m^2_e \hbar \omega/(2\gamma m_e c^2(\hbar \omega - \gamma m_e c^2))$.

\subsection{Bremsstrahlung in terms of radiation length}

We can then write in simplified terms with $X_0$ for the differential cross section (REF\addref)
\begin{equation}
\frac{\dif \sigma}{\dif (\hbar \omega)} = \frac{A}{X_0 N_A \hbar \omega} \left(\frac{4}{3} - \frac{4}{3}y + y^2\right),
\end{equation}
where $y= \hbar\omega/E$.


\chapter{Statistics}


\section{Linear correlation coefficient}

The Pearson correlation coefficient for two random variables and $N$ data points is defined as:
\begin{equation}
\rho(A, B) = \frac{1}{N-1} \sum_{i=1}^N \left( \frac{A_i-\mu_A}{\sigma_A}\right)^\ast \left(\frac{B_i-\mu_B}{\sigma_B}\right)
\end{equation}
or 
\begin{equation}
\rho(A, B) = \frac{\mathrm{cov}(A,B)}{\sigma_A \sigma_B}
\end{equation}

The correlation coefficient is a measure of the linear dependence of the two variables.

REFS: MATLAB and FISHER STATS 1958, KENDALL ADVANCED 1979, PRESS NUMERICAL 1992


\section{Normal Distribution}

Gaussian probability density function
\begin{equation}
f(x) = \frac{1}{\sigma \sqrt{2\pi}} e^{\frac{-(x-\mu)^2}{2\sigma^2}}
\end{equation}

With cumulative distribution 

\begin{equation}
p = F(x) = \frac{1}{\sigma\sqrt{2\pi}}\int_{-\inf}^x e^{\frac{-(t-\mu)^2}{2\sigma^2}}\mathrm{d}t
\end{equation}

Standard normal cumulative distribution function

\begin{equation}
\Phi(x) = \frac{1}{2} \left[ 1 - \mathrm{erf}\left(-\frac{x}{\sqrt{2}}\right)\right]
\end{equation}

\section{Kernel Density Estimate}

MATLAB:

\begin{equation}
\hat{f}_h(x) = \frac{1}{nh}\sum_{i=1}^n K\left(\frac{x-x_i}{h}\right)
\end{equation}



