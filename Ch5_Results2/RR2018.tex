\chapter{Precision Measurements of Radiation Reaction at Astra Gemini}

\section{Motivation}

In the previous chapter the author discussed the first measurement of radiation reaction using a wakefield accelerator. Both experiments were able to see a clear signature of radiation reaction and first indications of deviations from a classical description of the process.
In both cases, however, more data, stability and a careful look at the observables are required to make more reliable statements beyond the `discovery' level.

Following the recommendations from the last experiment, another experimental campaign was planned to conduct a precision measurement of radiation reaction to see further distinction between the models.

At even higher laser intensities and electron energies it is expected that the highly energetic gamma-ray bursts from inverse Compton scattering can generate electron-positron pairs in interaction with parts of the laser pulse. This would mark the beginning of a QED cascade.

\section{Experimental Setup}

The experimental setup combines considerations based on the first radiation reaction campaign in 2015 and an experimental campaign performed in 2018 aimed at measuring the Breit-Wheeler process. The results of this experiment will be discussed in the next chapter.
\vspace{\baselineskip}

The first laser arm of the Gemini laser (South beam), used to drive the LWFA, is focused down by an f/40 off-axis parabola from 150 mm diameter beam to a NUMBER 40 micron spot onto the edge of the gas target. The gas target in the first part of the experiment is a supersonic helium gas jet with a blade attached to induce a shock front.
The polarisation of the South beam is linear in the horizontal plane at the interaction point and is fixed to avoid damages to the optics in the last set of turning mirrors.
Since the long focal distance of the f/40 parabola is prone to pointing jitter at the interaction point, a setup to stabilise the beam pointing was installed.

The second laser arm (North beam) used to scatter off the electron beam is focused down tightly to a few micron spot using an f/2 off-axis parabola. The polarisation of the North beam is linear and set to vertical at the interaction point (perpendicular to the South beam). However, the North arm is equipped with a motorised half-wave plate that allows rotating the linear polarisation from vertical to horizontal. There is also a quarter-wave plate that allows switching from linear to circular/elliptical polarisation.

Both laser arms are equipped with wavefront sensors (HASO) and adaptive optics to flatten the wavefront and optimise the focal spots.

Timing of both arms is performed similarly as described in the previous chapter: prism at TCC.

Since the spatio-temporal overlap of the laser beams (or the electron beam and the laser beam respectively) is crucial and turned out to be challenging last time, diagnostics were implemented to measure the shot-to-shot jitter of the beams and the relative timing of both beams. Spectral interferometry.

A 1 tesla magnet with large aperture (design by Dominik Hollatz) is used as electron spectrometer dispersing the electron beam from LWFA onto two scintillating screens which are in turn imaged by Andor Neo cameras.

Potential positrons from electron-positron pairs are dispersed by the same magnet but into the opposite direction, where they are then bend onto an array of single-particle detectors in a shielded area.
These are silicon based TimePix detectors developed and provided by CERN and a CsI stack developed at Jena.

The gamma-rays from inverse Compton scattering (ICS) and bremsstrahlung are ....
CsI stacks, one to measure the profile and another one for the spectrum.

As another diagnostic to distinguish models an X-ray camera was installed to look at high-divergence X-rays. Crystal in magnet and camera looking at it.


\section{Focal Spot Analysis}

\section{1 Beam: Electron Spectra - shock injection }

\section{1 Beam: Statistical Analysis of Electron Spectra (stability, energy spread etc.)}

\section{1 Beam: Gamma Profile Measurements}

Correlation for Bremsstrahlung

Pointing?

\section{1 Beam: Gamma Spectrum Measurements}

\section{2 Beams: Timing}

\section{2 Beams: Gamma Profile Measurements}

\section{2 Beams: Gamma Spectrum Measurements}

\section{2 Beams: Electron energy measurements}

\section{2 Beams: Electron energy spread measurements}


\section{Conclusion}