\chapter{Measurement of the Linear Breit-Wheeler Process using LWFA}

\section{Motivation}

\section{Experiment Setup}

Maybe also a Blender diagram of the setup.

The experiment layout follows the similar principle as the previous Astra Gemini campaigns described in the previous chapters: the first (South) laser arm is focused down by an f/40 OAP and is used to drive a laser wakefield accelerator. The second (North) laser arm is used to provide a scattering source.


The North beam in this experiment is not compressed to its usual 40-45 fs duration but is stretched to around 60 ps. A phase plate also smoothens and enlargens the focal spot to several millimetres size. This beam is then incident onto a tape drive with Germanium dots, producing a plasma emitting a thermal bath of X-rays at a second interaction point, called secondary target chamber centre (SCC). This is the first component for the two-photon interaction. The X-ray source is diagnosed by an X-ray pinhole camera measuring the flux and the source size, and by a camera attached to a crystal measuring the spectrum of the X-rays in a certain spectral range.

The South beam focal spot reaches a spot size and intensity of NUMBERS.
The gas target is a variable length aluminium gas cell filled with helium or helium with nitrogen as a dopant.
The gas cell has been developed by Nelson Lopes.

The remaining laser beam exiting the cell is disposed of by a tape drive that acts as a plasma mirror.
The electrons from LWFA propagate through the tape and are then incident onto a converter foil (high-Z material). This interaction produces copious amounts of directed bremsstrahlung in propagation direction and stops most of the electrons in the process. A tungsten collimator reduces the emitted gamma-ray burst to its central part which is close to collimated. A thick block of tungsten also blocked the direct line of sight to the Ge target drive. A dipole magnet sweeps any remaining electrons that made it through the converter foil and the collimator out of the way. This way only a collimated bright burst of gamma-rays from bremsstrahlung are incident on a second interaction point, providing the second component for the two-photon interaction. In the absence of the converter target this magnet can be used as electron spectrometer. A scintillating lanex screen is positioned in the path of the electrons. 

Potential electron-positron pairs from this interaction propagate preferably in the propagation direction of the gamma-ray burst. They enter the field of a large aperture magnet that disperses the electrons and positrons horizontally in opposite directions, leaving the vacuum through a wide kapton-kevlar window, where they are then caught by one magnet on each side that bend the electrons or positrons respectively onto single-particle detectors.

On-axis is a stack of CsI crystals to measure the profile of the gamma-ray signal and another large stack to measure the decay of the signal in propagation direction to deduce the spectrum.

\section{Electron Spectra Measurement}

Typical Spectra. Characterised before and after data shots. Electrons themselves are not measurable on data shots as they are being converted and stopped by high-Z material.

\section{Bremsstrahlung Characterisation - Spectral Measurement Gamma Rays}

RAL stack with spectral measurement. GEANT.
Gamma Profile.







\section{Conclusion}