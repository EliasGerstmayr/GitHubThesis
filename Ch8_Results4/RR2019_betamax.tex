\chapter{Hard betatron radiation from dual shock features at Astra Gemini}

\section{Motivation}

\section{Experimental Setup}

The following experiment was performed at the Astra Gemini laser facility in early 2019. This dataset only used one of the two laser arms available at Gemini.

The laser is focused with an f/40 off-axis parabola onto the edge of a 15 mm conical supersonic helium gas jet. The laser is linearly polarised in the horizontal plane. The pulse duration of the laser was measured to be XX NUMBER with an energy on target on average of XX NUMBER onto a focal spot of size XX. This results in an a0 of around XX NUMBER. A razor blade with motorised translation and rotation motion can be introduced into the gas flow to induce a shock. The gas flow is characterised by a transverse optical probe split off from the main beam ending in a shadowgraphy and a Mach-Zehnder interferometry. A DSLR camera is imaging the recombination light of the plasma channel from the side and another camera, also used for aligning the gas target and the razor blade, is imaging the channel from the top. For this dataset the tip of the blade was driven in from the side the laser was propagating into the plasma up to around XX MM into the gas jet.

Relativistic electrons accelerated through LWFA and radiation produced travel further downstream through a XX NUMBER sized hole in another off-axis parabola at 300 mm distance that is used in another dataset as part of a colliding pusle experiment. The laser and electron beam are centred onto the hole and do not interact with the optics. It presents, hwoever, an aperture to high divergence radiation beyond XX NUMBER DIVERGENCE.

The electrons are then being dispersed by a large aperture magnet with field strength XX TESLA in the horizontal plane. Electrons of energies up to 200 MeV hit the inside of the yoke. The fan of electrons leaves the vacuum chamber through a wide kevlar-kapton window and passes through a scintillating Biomax LANEX screen that covers the window from the outside up including the laser axis. Radiation signal on axis can then be used to obtain an on-shot axis reference. The electrons then propagate further in air and pass through another standard LANEX screen after about XX METER distance and are then being dumped into a combined plastic-lead wall.
The LANEX screens are imaged by Andor Neo cameras and the electron energies can be deduced by measuring the field strength and the distances of the magnet and screens carefully.

The radiation propagates approximately on laser axis and unperturbed through the magnet in the chamber. On laser-axis a beam block of 10-12 layers of standard kitchen aluminium foil is attached to the kevlar-kapton window. Hard X-rays continue to propagate through the aluminium beam block and the vacuum window, the LANEX screen that also covers the axis and then hit a 45x45 array of 1 mm x 1 mm x 10 mm scintillating caesium-iodide crystals that record the beam profile of the radiation burst. Harder radiation propagates another XX DISTANCE before hitting another array of caesium-idodide crystals (see Methods for dual axis spectrometer). The emission of caesium-iode crystals is imaged by Andor iXon cameras.

\section{Density profile}

\section{Electron Spectra}

\section{GEANT simulations}

\section{Background treatment}

\section{Estimating the spectrum}

\section{Conclusion}