

\chapter{Introduction}
\label{Chap:Introduction}

Developing a better understanding of the fundamental principles of nature lies at the very heart of science. 
Ultrarelativistic particle accelerators represent powerful `microscopes', which provide invaluable insights into the fundamental building blocks of nature.
The recent discovery of the Higgs boson \cite{Higgs1964_HIGGS,Englert1964_HIGGS,Guralnik1964_HIGGS,ATLAS2012_expHIGGS,CMS2012_expHIGGS} completes the \textit{Standard Model} of particle physics \cite{Shears2012_StandardModel}, which classifies the zoo of known elementary particles and describes the interaction of the fundamental forces (except gravitation).
Moreover, they have enabled many interdisciplinary applications, including ultra-bright FEL-based photon sources \cite{Deacon1977_FEL,Emma2010_FEL}.

Since the invention of the Nobel-prize awarded \textit{chirped-pulse amplification} (CPA) technique \cite{Strickland1985_CPA} peak intensities of optical lasers have increased exponentially, while at the same time the number of high-intensity laser systems has surged \cite{Danson2019_PWLASERS}. This development has revolutionised our understanding of light-matter interactions and lead to disruptive transformations \cite{Mourou2019_Nobel}. The advent of CPA created entire novel scientific fields, such as strong-field atom and molecular science \cite{Protopapas1997_HIAtomic,Schafer1993_AboveThreshold,Corkum1993_multiPhoton,Freeman1987_AboveThreshold}, attosecond physics \cite{Calegari2014_ATTOSECONDS}, relativistic and nonlinear optics \cite{Mori1997_NONLINEARPLASMA}, and laboratory astrophysics \cite{Albertazzi2014_LABASTRO}. It also enabled the realisation of the plasma-based \textit{laser wakefield acceleration} (LWFA) \cite{Mangles2004_MONO,Faure2004_MONO,Geddes2004_MONO}, where a laser pulse of relativistic intensities forms an accelerating cavity with high field gradients \cite{Esarey2009_LPA_Review}. 

The marriage of both technologies -- ultra-relativistic particle accelerators and ultra-intense lasers -- is bound to boost our understanding of nature once again, by providing access to electromagnetic fields which reach and potentially exceed those encountered in extreme astrophysical environments, e.g., close to the surface of a neutron star \cite{Denisov2017_VacuumPulsarMagnetar,Mignani2017_VBAstro}. 
The fundamental scale, at which qualitatively new physics is expected, is the so-called \textit{Schwinger field}, \textit{Sauter field} or also \textit{critical field of QED} \cite{Sauter1931_ECRIT,Heisenberg1936_ECRIT,Schwinger1951_VACPAIRS,Brezin1970_VACPAIRS,Popov1971_VACPAIRS}. 
This scale defines the ``strong-field frontier''\addref, which is complementary to the ``energy frontier''\addref, that is traditionally explored with accelerators. 
This regime bears ample research opportunities that will be explored at conventional accelerator facilities \cite{Burkart2019_LUXE,Abramowicz2019_LUXE,SFQEDOverview2019,Hogan2016_FACETII} and the next generation of multi-PW laser facilities \cite{Gales2018_ELINP,Weber2017_ELIBeamlines,Zou2015_Apollon,Li2017_SULF,Shen2018_SULF,Cartlidge2018_SEL,Toth2017_BELLA,Bashinov2014_XCELS,Kessel2018_PWMPQ,EPAC_Website,Maksimchuk2019_ZEUS,Yanovsky2008_HERCULES}  in laser-matter, light-light and even laser-vacuum interactions \cite{DiPiazza2012_ICS}.
%This work and the work presented here is then consequently located at the intersection of accelerators, high-energy light sources and high-intensity lasers.


\section{Particle Accelerators}
\label{Introduction:Sec:PartAccelerators}

\begin{figure}
\centering
\includegraphics[height=0.5\columnwidth]{CERN_Aerial.jpg}\includegraphics[height=0.5\columnwidth]{Livingston_Plot_1.png}
\caption[Aerial view of the LHC and Livingston Plot.]{Left: Aerial view of the Large Hadron Collider (LHC) at CERN located near (Lake) Geneva \cite{CERN_Aerial}. Right: Livingston plot showing the particle energies achieved in particle accelerators as a function of their year of commissioning \cite{Panofsky1997_Livingston}. The dashed line shows that the energy increases by a factor of 10 every six years (by now old).}
\label{Introduction:Figs:CERN_Livingston}
\end{figure}

In an ordinary microscope the smallest feature that can ultimately be resolved is determined by the wavelength of the optical light that is used to backlight the sample. 
More generally we can postulate that the characteristic scale of the probe has to be smaller than the object it tries to resolve. 
The wavelength associated with a particle of momentum $p$ is given by the \textit{de Broglie wavelength} \cite{deBroglie1925_dB}, $\lambda_{dB} = h/p$, where $h$ is the Planck constant. 
As a consequence, this intuitively motivates that higher particle energies are required to resolve smaller scales. The equivalence of energy, $\epsilon$, and mass, given by $\epsilon = mc^2$, on the other hand, indicates that higher energies enable the production of heavier particles in an interaction. 
The key quantity is the \textit{centre-of-mass energy}, $\sqrt{s}$. For two ultrarelativistic, i.e. $\epsilon \gg mc^2$, particles of the same species that collide with each other at an angle $\theta$ it is given by:
\begin{equation}
\boxed{s = \frac{(1-\cos \theta) \epsilon_1 \epsilon_2}{2 m^2 c^4}}
\end{equation}
Modern particle accelerators rely on radio-frequency (RF) cavities that support field strengths of up to $\sim 100\,\mathrm{MV/m}$, before reaching their breakdown limit \cite{Geng2005_RFGradient}.

One way to reduce the footprint of the accelerator is the use of circular accelerators where same accelerator segments are use more than once, where magnets are used to keep the particles in the loop. However, due to the bending radius the particles constantly are accelerated and radiate as a consequence. The associated emission power, and consequently the rate at which the particle loses energy, is the \textit{Larmor emission power} \cite{Jackson}
\begin{equation}
\boxed{P_L = \frac{1}{6 \pi \epsilon_0 c^5} \left( \frac{q}{m}\right)^4 \epsilon^2 B^2,}
\end{equation}
where $\epsilon_0$ is the dielectric constant, $c$ the speed of light, $q$ the charge of the particle, $m$ its mass, $\epsilon$ its energy and $B$ is the magnetic field.
The emission power increases quadratically with the energy of the particle and the applied magnetic field, which implies that higher energies require larger accelerators (stronger fields result in smaller bending radii). We also see that it scales with the charge to mass ratio by $(q/m)^4$, on the other hand, which indicates that this geometry is, for instance, more suited for protons than for electrons. Consequently, circular accelerators are typically only used to accelerate protons or heavy ions. Figure \ref{Introduction:Figs:CERN_Livingston} (left) shows an aerial view of the Large Hadron Collider (LHC) at CERN with a circumference of 27 km. Electron or other lepton accelerators are, on the other hand, typically linear accelerators, for instance the Stanford Linear Accelerator (SLAC) reaching up to $46.6\GeV$ within 3 km. Future facilities like CLIC \cite{CLIC2012}, ILC \cite{ILC2013} or the FCC \cite{Benedikt2016_FCC} will be of even larger scales\addnum{}.

A complementary technology that could help reducing the size of accelerators is the use of plasmas \cite{Lee2002_Afterburner}, which can in principle sustain orders of magnitude higher field gradients \cite{Dawson1959_COLDWAVEBREAKING}, e.g. $\sim 100\,\mathrm{GV/m}$ for a plasma of density $n_e = 10^{18}\,\mathrm{cm}^{-3}$. 

An important distinction to note between hadron and lepton colliders is that hadron interactions are much more difficult due to the complex nature of quantum chromodynamics (QCD), whereas lepton interactions are dominated by QED such that these signals are much `cleaner' and elementary lepton interactions are highly desirable. In hadron interactions the energy is also distributed amongst the quarks such that the effective energy per parton is reduced.
\vspace{\baselineskip}

Despite the cost and the challenges associated with large particle accelerators the centre-of-mass energy has continuously increased over the years, doubling every X\addnum{} years. This is visualised in a so-called \textit{Livingston plot} \cite{LivingstonPlot} shown in Figure \ref{Introduction:Figs:CERN_Livingston} (right) \cite{Panofsky1997_Livingston}. The increasing energies enabled access to a wide range of particles and phenomena, so that the story of accelerators is a success story, with the standard model being well documented and predictive, outlining a zoo of elementary particles and how they are governed by the fundamental forces. The Higgs boson is the final particle to complete the set \cite{Higgs1964_HIGGS,Guralnik1964_HIGGS,Englert1964_HIGGS,ATLAS2012_expHIGGS,CMS2012_expHIGGS}. However, there is still more on the horizon, including the need for a unified theory including gravitation and physics beyond the standard model including dark matter and energy candidates\addref.
\vspace{\baselineskip}

On the other hand, the radiation that is a limiting factor for the maximum energy in circular accelerators has itself also proven to be immensely useful. Synchrotron radiation has been widely used to understand material structures and chemical processes\addref. The state-of-the-art light sources are nowadays free-electron lasers (FELs)\addref. Entire facilities to produce this radiation and opened up fields, useful for atom, molecular, chemical sciences and so on\addref.
\vspace{\baselineskip}

Until the commissioning of the next generation of new facilities, existing facilities are now upgrading their systems to increase the luminosity. The high-luminosity upgrade of the LHC (HL-LHC) \cite{Apollinari2015_HLLHC}, for instance, will produce more data and facilitate precision studies of rare processes \cite{Atlas2012_HLLHCPhysics} like light-light scattering in ultra-peripheral scattering \cite{Enterria2013_PhPh,ATLAS2017_GammaGamma}, also for studies of physics beyond the standard model \cite{Beresford2019_PhPh_SLEPTONS,Knapen2017_PhPh_AXIONLIKE}.

\section{High-intensity lasers}
\label{Introduction:Sec:HILasers}

Whilst in particle physics the centre-of-mass energy was the driving factor, here the \textit{peak intensity} of the laser field determines access to regimes of qualitatively different phenomena.
The intensity is the unit energy delivered per unit area and time, commonly given in units $\mathrm{W/cm^2}$. The peak intensity, $I_0$, for a laser pulse which is Gaussian in time and space is given by
\begin{equation}
\boxed{I_0 = 0.415 \frac{\epsilon_J}{d^2 \tau}},
\end{equation}
where $\epsilon_J$ is the energy in the laser pulse, $d$ its \textsc{fwhm} spatial size and $\tau$ its \textsc{fwhm} pulse duration.
The peak intensity in the focus of a laser spot is shown in Figure \ref{Introduction:Figs:IntensityPhysics} as a function of the year. We see that the invention of CPA resulted in an exponential increase in laser intensities from $< 10^{15}\,\mathrm{W/cm^2}$ to an excess of $10^{22}\,\mathrm{W/cm^2}$. At the same time the number of high-intensity lasers has increased rapidly as the applications and field grows \cite{Danson2019_PWLASERS}. The rapid access to higher intensities have revolutionised our understanding of laser-matter interactions which is indicated in the Figure \ref{Introduction:Figs:IntensityPhysics}.
\EliasComm{Replace this plot by own or so as low quality.}
\vspace{\baselineskip}

\begin{figure}
\centering
\includegraphics[width=0.8\columnwidth]{LaserIntensities.png}
\caption[Evolution of focused laser intensities over time.]{Evolution of focused laser intensity in $W/cm^2$ over the past decades along with relevant intensity regimes. Printed with permission, from \cite{Mourou2019_Nobel}.}
\label{Introduction:Figs:IntensityPhysics}
\end{figure}

For instance, approaching the atomic intensity of $\sim\!10^{16}\,\mathrm{W/cm^2}$ nonperturbative laser-atom interactions give rise to a wealth of strong-field phenomena, including tunnel ionization and recollision-induced high-harmonic generation (HHG) \cite{Protopapas1997_HIAtomic,Schafer1993_AboveThreshold,Freeman1987_AboveThreshold,Corkum1993_multiPhoton}. 

\iffalse
\begin{figure}
\centering
\includegraphics[width=0.9\columnwidth]{ICUIL3Maps.pdf}
\caption[Ultrahigh intensity laser facilities worldwide.]{Ultrahigh intensity laser facilities worldwide according to ICUIL \cite{ICUIL2019_Map}. To qualify peak laser intensity of $10$ TW.}
\end{figure}
\fi

Reaching intensities of $10^{18}\,\mathrm{W/cm^2}$, on the other hand, the electron motion in the electromagnetic field becomes relativistic: we enter the  \textit{relativistic regime}. In this context, the intensity is commonly expressed in terms of the \textit{normalised vector potential}, $a_0$\addref{}:
\begin{equation}
\boxed{a_0 = \frac{eE_0}{m_e c \omega_L} = 0.856 \sqrt{\frac{I \lambda^2}{10^{18}\,\mathrm{W/cm^2 \upmu m^2}}},}
\end{equation}
where an $a_0$ of unity indicates that the electric field is of comparable strength as the electron rest mass. 
Laser-matter interactions at relativistic intensities give rise to relativistic and nonlinear optics \cite{Mori1997_NONLINEARPLASMA}, i.e. nonlinear effects in media as relativistic focusing \cite{Sprangle1987_SELFFOCUS}, relativistic plasma waves \cite{Akhiezcr1956_Waves} and plasma mirror \cite{Kapteyn1991_PM}, and is also an indicator of nonlinear photon interactions, for instance in nonlinear Compton scattering. See figure of eight motion (relativistic motion gives rise to higher harmonics). These intensities are also important for studies of high energy density physics.
One example relevant in this context of relativistic motions is the \textit{ponderomotive force} described by\addref{}
\begin{equation}
\boxed{\mathbf{F}_p = - m_e c^2 \nabla \frac{\langle a^2 \rangle}{2},}
\end{equation}
which pushes away electrons from regions of high-intensities.
\vspace{\baselineskip}

Relativistic intensities and nonlinear optics also enabled the development of plasma-based accelerator techniques, in particular laser wakefield acceleration, where an intense laser pulse travels through a plasma and expels electrons in its way through the ponderomotive force \cite{Esarey2009_LPA_Review}. The charge separation it forms in its wake, at high intensities called `bubble' \cite{Pukhov2002_BUBBLESIM}, supports strong focusing and accelerating field gradients \cite{Kostyukov2004_BUBBLEFIELDS}. Electrons that are injected in the right phase of this cavity can be accelerated to relativistic energies in short distances. Since its breakthrough in 2004 yielding quasi-monoenergetic particle beams of hundred MeV \cite{Mangles2004_MONO,Faure2004_MONO,Geddes2004_MONO}, beam quality has improved \cite{Osterhoff2008_CELL} and maximum electron energies reached up to $\sim 8\GeV$ \cite{Gonsalves2019_GEV}. Due to the structure of the accelerating cavity and the short laser pulses driving the accelerator the electron bunches are intrinsically also of short pulse durations \cite{Lundh2011_BUNCH} and of micron source size \cite{Weingartner2012_BUNCH}. Whilst the overall beam quality is to date still inferior to conventional accelerators in terms of emittance and reproducibility, for instance, it has a large potential and in the meantime hybrid schemes can boost the performance of existing facilities. It also provides a future avenue to provide cost-efficient access to accelerators on a smaller scale. 

Electrons that are injected off-axis start to oscillate due to the strong field gradients. These are called betatron oscillations, which result in the emission of betatron radiation in the X-ray regime \cite{Whittum1992_BETATRON}. As a result, X-ray radiation is emitted which due to its broadband radiation, short source size and pulse duration, is also suitable for phase-contrast imaging \cite{Cole2015_tomography} and time-resolved measurements \cite{Kettle2019_XANES}.
\EliasComm{Add plot: Wakefield plot (maybe Jason's)}
\vspace{\baselineskip}


\begin{figure}
\centering
\includegraphics[height=0.35\columnwidth]{ral_aerial_photo.jpg}\includegraphics[height=0.35\columnwidth]{gemini_laser_area_02.jpg}
\caption[Aerial view of RAL and photo of the Gemini laser area.]{Left: Aerial view of the Rutherford Appleton Laboratories including the Central Laser Facility. Right: Astra-Gemini laser area. From CLF website.}
\end{figure}

At the ``critical intensity'' $I_{cr}\!\sim\!10^{29}\,\mathrm{W/cm^2}$, far beyond the relativistic regime, the interaction with the vacuum itself becomes sizeable \cite{Sauter1931_ECRIT,Heisenberg1936_ECRIT}, opening a qualitatively new regime of light-matter interactions \cite{Schwinger1951_VACPAIRS,Brezin1970_VACPAIRS,Popov1971_VACPAIRS,DiPiazza2012_ICS}. The associated electric field strength is given by
\begin{equation}
\boxed{E_S = \frac{m^2_e c^3}{e \hbar}= 1.33 \times 10^{18}\,\mathrm{V/m},}
\end{equation}
which is the electric field that performs work equal to the rest mass of the electron within a (reduced) Compton length. 

Whilst the next generation of lasers are now commissioned across \cite{Gales2018_ELINP,Weber2017_ELIBeamlines}, Asia \cite{Li2017_SULF,Shen2018_SULF} and the US \cite{Maksimchuk2019_ZEUS} aiming to reach unprecedented peak powers of several to hundred PW, with peak intensities of XX\addnum{} \cite{Danson2019_PWLASERS}, these lasers are still far out of reach of the critical intensity. 

Reaching intensities beyond the now planned or commissioned laser systems is, however, challenging and requires novel amplification schemes beyond CPA and OPCPA \cite{Danson2019_PWLASERS}. Ideas are to combine multiple beams coherently and incoherently \cite{Shen2018_SULF}. 
Other challenges include improving efficiency and repetition rates (similar as luminosity for particle accelerators) to increase high average power \cite{Danson2019_PWLASERS,Mourou2019_Nobel}.

\section{Particle accelerators and high-intensity lasers}
\label{Introduction:Sec:HILasers_and_PartAcc}

(High-intensity) lasers and (ultrarelativistic) accelerators have separately revolutionised our understanding of nature. By combining both we are able to access a new regime of physics to explore qualitatively new physics\addref.
Lasers and accelerators are currently already used in pump-probe experiments crucial for material, atom, molecular science and HEDP experiments at XFELs for instance. 
They have been proposed to enhance acceleration via hybrid schemes, e.g. Trojan horse.

One in this context particularly interesting configuration is the collision of an intense laser pulse with a relativistic electron. The electric field in the rest frame of the relativistic electron is greatly enhanced, such that optical photons are backscattered to X-ray or gamma-ray energies. This process is referred to as \textit{Compton backscattering}, \textit{relativistic Thomson scattering} or \textit{inverse Compton scattering}. This has been widely used as diagnostic for particle accelerators\addref{} and the energetic radiation is also used in nuclear studies\addref.
\vspace{\baselineskip}

\begin{figure}
\centering
\includegraphics[width=0.5\columnwidth]{RROverview_Blackburn.png}
\caption[Overview of radiation reaction regimes and experiments as a function of $\eta$ and $a_0$.]{Overview of radiation reaction regimes and experiments as a function of the quantum nonlinearity parameter, $\eta$ (here $\chi$), and the normalised vector potential, $a_0$. The red lines indicate when quantum effects become significant (dashed) or dominate (solid). More details are provided in Section \ref{Theory:Sec:RR:RegimesOfRR}. Figure reprinted from \cite{Blackburn2019_RRReview}.}
\end{figure}

\EliasComm{Also somehow mention E-144.}
In this geometry, we realise that the electric field of the laser is enhanced in the rest frame of the relativistic electron, such that we can probe interesting phenomena. This is also the geometry we take advantage of in this work. By combining high-intensity lasers with ultra-relativistic electron beams, e.g., at SLAC's FACET-II \cite{Hogan2016_FACETII}, such extreme intensities become accessible in the electron rest frame due to the Lorentz boost of the intensity ($I' \sim 4\gamma^2 I$). 
The parameter of interest is here $\eta$\addref:
\begin{equation}
\boxed{\eta = \frac{E_{RF}}{E_S} = \frac{(1-\cos\theta) \gamma E_L}{E_S} \approx 0.1 \left(\frac{\epsilon}{500\MeV}\right)\left(\frac{I_0}{10^{21}\,\mathrm{W/cm^2}}\right)^{1/2},}
\end{equation}
with the Schwinger field $E_S = 1.38 \times 10^{18}\,\mathrm{V/m}$ or $I_S \approx 10^{29}\,\mathrm{W/cm^2}$. 
We see that we can reach high values by increasing $\gamma$ and the field strength, intensity $a_0$.
Critical field  \cite{Sauter1931_ECRIT,Heisenberg1936_ECRIT}. We see that we can reach these values by either reaching high $a_0$ or high $\gamma$, such that this is a field that is approached from two sides, conventional accelerators and multi-petawatt laser facilities.
\vspace{\baselineskip}

\begin{figure}
\centering
\includegraphics[width=0.7\columnwidth]{BlackburnRR.png}
\caption[Visualisation of classical and quantum radiation reaction.]{Trajectory (red) of an electron in a circularly polarised laser field. The radiation the particle emits is indicated by the yellow arrows. Left: The particle undergoes classical radiation reaction which acts as a slow continuous force. Right: The particle experiences quantum radiation reaction. The stochastic, incoherent emission of multiple high-energy photons leads to a significant alteration of the trajectory. Figure courtesy T. Blackburn, from \cite{Blackburn2019_RRReview}.}
\end{figure}


Such extreme conditions can be encountered in violent astrophysical phenomena like gamma ray bursts \cite{Harding1991_GRB} and supernova explosions \cite{Uzdensky2014_PlasmaAstro}, in the vicinity of pulsars and magnetars \cite{Denisov2017_VacuumPulsarMagnetar,Mignani2017_VBAstro}, close to magnetized black holes \cite{Blandford1977_KerrBlackHole}, and in the early universe \cite{Ruffini2010_PAIRSASTRO, Nikishov1962_PAIRSASTRO,Kandus2011_Primordial,Grasso2001_earlyUniverse}. 
Due to stringent luminosity requirements a future lepton collider (e.g., CLIC \cite{CLIC2012} or ILC \cite{ILC2013}) will reach the critical field strength at the interaction point \cite{Yokoya1992_SFQED_BeamBeam,Esberg2014_SFQED_BeamBeam}. 
Moreover, it might be beneficial to exceed this scale by orders of magnitude \cite{Yakimenko2019_NONPERTURB}, necessitating a better understanding of light and matter in extreme electromagnetic fields. 
At multi-petawatt laser facilities, which are currently being constructed in Europe \cite{Gales2018_ELINP,Weber2017_ELIBeamlines}, Asia \cite{Li2017_SULF,Shen2018_SULF} and the US \cite{Maksimchuk2019_ZEUS}, laser-plasma interactions will probe an interplay between strong-field quantum and collective effects. 
These facilities will open an exciting new research field within high-energy-density physics (HEDP) \cite{Grismayer2017_SeededQEDCascades,Ridgers2012_DENSE,Nerush2011_LaserAbs_PAIRS}. 
Reports from Brightest Light \cite{National2018_BrightestLight} and Frontiers of Plasma Science \cite{DoE2015_FES}.

In this so-called \textit{nonperturbative strong-field regime of QED} ($I'\!\gtrsim\!I_{cr}$) the recoil of emitted photons significantly perturbs electron/positron trajectories (strong-field radiation reaction) and prolific electron-positron pair production becomes substantial (vacuum breakdown).
\vspace{\baselineskip}


\begin{figure}
\centering
\includegraphics[width=0.6\columnwidth]{ParticleShower.png}
\caption[Visualisation of an electromagnetic particle shower.]{Ultrarelativistic electrons (blue) collide with an intense laser pulse (green, purple), and emit gamma-rays (yellow). Some of the gamma-rays interact with the laser field to produce an electron-positron pair (blue and red) via the nonlinear Breit-Wheeler process. The secondary particles can in turn interact with the laser field to emit more radiation. Figure courtesy T. Blackburn, adapted from \cite{Blackburn2017_pairs}.}
\end{figure}


Wide range of phenomena  \cite{DiPiazza2012_ICS}
Compton scattering: radiation production meaning compton scattering from classical \cite{Yan2017_ICS} and the perturbative quantum \cite{Bula1996_RR} to the non-perturbative strong-field regime \cite{Bula1996_RR,TaPhuoc2012_ICS,Chen2013_ICS,Powers2014_ICS,Sarri2014_ICS,Khrennikov2015_ICS,Mackenroth2013_nlCompton}. The high field strengths also give rise to radiation reaction.
\vspace{\baselineskip}

\EliasComm{Explanation of what radiation reaction is.}
Radiation reaction is the knock the knock-back force a charged particle experiences when radiating. At high energies per single emission it becomes quantum radiation reaction or strong field radiation reaction \cite{Harvey2017_QUENCHING,Blackburn2019_RRReview,Blackburn2014_QRR,Ritus1985_QRR,Ridgers2017_QRR,Shen1972_STRAGGLING,Thomas2012_LL,Vranic2014_RR,Dinu2016_QRR}
Very fundamental process but no practicable solution for this.

Intense interactions could lead to spin polarisation \cite{DelSorbo2017_SPIN,Seipt2018_SPIN,Li2019_singleshot_beampol}.
LCFA \cite{DiPiazza2018_LCFA,DiPiazza2019_LCFANUM}
For future facilities it would be important to mitigate these effects.
\vspace{\baselineskip}

There are different geometries to probe radiation reaction. Colliding relativistic particles with laser pulses enhances the field the strongest.
It can also be probed in crystals \cite{DiPiazza2017_CrystalRR,Artru1994_CHANNELING,Katkov1998_CHANNELING,Wistisen2018_RR,Wistisen2019_RR} and in laser-solid interactions and in standing waves.
\vspace{\baselineskip}

These interactions also enable the study of astrophysical equivalent of cascades and showers cascades and showers, multi-step  \cite{Bulanov2013_Cascade,Grismayer2017_SeededQEDCascades,Sokolov2010_QEDPAIRS}.
In extreme conditions we are approaching high energy density physics in QED plasmas/producing dense electron-positron pair plasmas \cite{Ridgers2012_DENSE,Nerush2011_LaserAbs_PAIRS,Grismayer2017_SeededQEDCascades}
\vspace{\baselineskip}

Overall a wide range of light-light interaction (AGAINST SUPERPOSITION PRINCIPLE):
From nonlinear BW \cite{DiPiazza2016_nonlinearBW}, to light-light BW pair production (linear), light-light scattering, vacuum pair production (BREAKING THE VACUUM \cite{Cartlidge2018_SEL}) and MATTER FROM LIGHT.
This also can be used to probe physics beyond the standard model in photon-photon \cite{Beresford2019_PhPh_SLEPTONS,Knapen2017_PhPh_AXIONLIKE,Baldenegro2018_PhPh_AXIONLIKE} and vacuum recollision physics \cite{Meuren2015_HERecollision,Kuchiev2007_Recollision}.
Also note that compare with LHC etc. this includes light-light interactions which are at particle accelerators only virtual/quasi-free possible

We can realise photon-photon colliders, and see an analogue to nonlinear physics and optics. Now the vacuum acts as a medium. We investigate vacuum birefringence \cite{King2016_VB,Nakamiya2017_VB} and vacuum dichroism \cite{Bragin2017_VBVD}.

At these supercritical fields we reach the field of nonperturbative physics \cite{Yakimenko2019_NONPERTURB,Blackburn2019_SUPER,Baumann2019_NONPERTURB}. 
\vspace{\baselineskip}

Here some reasons why to do this kind of experiments with wakefield accelerators: High-intensity laser pulses and accelerators are available at the same location, which is rare. The laser pulse is synchronised with the electron beam which enables easier overlap. The small beam size of the electron beam means all of the electron beam can be overlapped with a tightly focused laser pulse such that the electron is a real probe of the laser pulse (probe has to be smaller than what it probes).
\vspace{\baselineskip}

Even deeper into the supercritical regime we find another threshold: 
At extreme fields $\chi \alpha \approx 1$ the radiative corrections become sizable, such that QED enters the strong-coupling regime. Here our understanding and descriptions break down. This is called the Narozhny conjecture \cite{Ritus1972_NAROZHNY,Narozhny1980_NAROZHNY,Fedotov2017_NAROZHNY,Ilderton2019_QEDPerturbBreakdown}. This is very hard to reach.

\vspace*{\fill}

\section{Thesis Outline}

This work focuses on the versatile capabilities of laser wakefield accelerators and high-intensity lasers to generate highly energetic radiation of varied spectral shape and across a wide energy range. It provides three experimental examples where laser wakefield accelerators were used to produce gamma radiation with photon energies from few to several hundreds of $\mathrm{MeV}$, and highlights their application in studies of fundamental phenomena of quantum electrodynamics (QED). 

The thesis starts by discussing single particle motions in electromagnetic fields, radiation mechanisms, radiation reaction and pair production in \textbf{Chapter \ref{Chap:Theory:SingleParticle}}. \textbf{Chapter \ref{Chap:Theory:LWFA}} introduces principles regarding the collective motion in plasmas and their interaction with an intense laser pulse, which is relevant for wakefield acceleration. This is followed by a discussion of experimental methods underpinning this work in \textbf{Chapter \ref{Chap:Methods}}. The experimental results that are at the core of this thesis are presented in \textbf{Chapters \ref{Chap:linICS}}, \textbf{\ref{Chap:RR15}} and \textbf{\ref{Chap:BW}}:
\begin{enumerate}[label=\textbf{\arabic*})]
\setcounter{enumi}{4}
\item \textbf{Linear Inverse Compton Scattering and Beam Profile Diagnostic:}\\
Relativistic electrons from a wakefield accelerator were collided with a defocused laser pulse at $a_0 \sim 0.2 - 1$ generating gamma radiation from linear inverse Compton Scattering in the 10's of $\mathrm{MeV}$ photon energy range. The radiation is used to diagnose the properties of the laser pulse at the interaction and its application as single-shot electron beam diagnostic is explored.
\item \textbf{Nonlinear Inverse Compton Scattering and Radiation Reaction:}\\
By scattering a tightly focused high-intensity laser pulse from a relativistic electron beam and generating gamma radiation from nonlinear Inverse Compton Scattering, it was possible to measure radiation reaction in an experiment. The energy loss of the electron beam and the spectrum of the radiation were used to pinpoint the interaction conditions and their agreement with theoretical models was tested. 
%A statistical analysis of the electron beam fluctuations and its impact on the statistical significance of the results is presented.
\item \textbf{Bremsstrahlung source and the Linear Breit-Wheeler Process:}\\
An electron beam from a wakefield accelerator was passed through a solid target to produce bremsstrahlung with photon energies of 100s of MeV. The source was optimised with regards to its yield of photons and energetic secondary particles. The high energy photons were then collided with a $\sim$ keV X-ray field generated by a second laser in an attempt to measure the elusive linear Breit-Wheeler process. 
\end{enumerate}
Finally, \textbf{Chapter \ref{Chap:Conclusion_and_Outlook}} summarises the results and discusses future research opportunities. Taking into considerations the findings and insights of the presented results, an experimental layout for future radiation reaction studies is proposed, outlining beam geometries and relevant diagnostics.
%their impact for future research are evaluated
%\vspace*{\fill}


%\item \textbf{Hard Betatron Radiation from dual shock features:}\\
%Electrons travelling in a wakefield transitioning through two density spikes experience an increase of transverse oscillations resulting in an enhancement of measured betatron radiation reaching hundreds of $\mathrm{keV}$. The dual shock structure was induced by introducing a blade into the supersonic gas flow of a helium gas jet and measured by a stack of scintillating crystals.