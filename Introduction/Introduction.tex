

\chapter{Introduction}
\label{Chap:Introduction}

Developing a better understanding of the fundamental principles of nature lies at the very heart of science. 
Ultrarelativistic particle accelerators represent powerful `microscopes', which provide invaluable insights into the fundamental building blocks of nature.
Moreover, they have enabled many interdisciplinary applications, including ultra-bright FEL-based photon sources.
\EliasComm{Mention Standard Model and Higgs}

Since the invention of the Nobel-prize awarded chirped-pulse amplification technique peak intensities of optical lasers have increased exponentially. This development has revolutionised our understanding of light-matter interactions and lead to disruptive transformations. They created entire novel scientific fields, such as strong-field atom and molecular science, attosecond physics, relativistic and nonlinear optics, and high energy density physics (HEDP).
\EliasComm{Mention high number of facilities.}

The marriage of both technologies -- ultra-relativistic particle accelerators and ultra-intense lasers -- is bound to boost our understanding of nature once again, by providing access to electromagnetic fields which reach and potentially exceed those encountered in extreme astrophysical environments, e.g., close to the surface of a neutron star. The fundamental scale, at which qualitatively new physics is expected, is the so-called QED critical field. This scale defines the ``strong-field frontier'', which is complementary to the ``energy frontier'', that is traditionally explored with accelerators. 
\EliasComm{Mention future research and that multi-PW facilities will in laser-plasma interactions probe an interplay between strong-field quantum and collective effects.}
\EliasComm{Mention that this work presents results at the intersection of accelerators, high-energy light sources and lasers.}
\clearpage

\section{Particle Accelerators}

\EliasComm{Add plot: Nobel prizes accelerators?}



\begin{figure}
\centering
\includegraphics[height=0.4\columnwidth]{CERN_Aerial.jpg}\includegraphics[height=0.4\columnwidth]{Livingston_Plot_1.png}
\caption{From \cite{CERN_Aerial}. Second be Livingston plot.  \cite{LivingstonPlot}}
\end{figure}
\EliasComm{Add Plot: Livingston changes, own plot.}

\textbf{Driving force for understanding is higher energy (smaller wavelength), secondary now luminosity}

Centre-of-mass energy (this is for two identical particles)
\begin{equation}
s = \frac{(1-\cos \theta) \epsilon_1 \epsilon_2}{2 m c^2}
\end{equation}


Charged particles are accelerated by electric fields. In the first accelerators static strong high voltage fields were used for this purpose. Helium and neon atoms were ionised and the atom was split. Maintaining large electric field gradients over long distances is, however, difficult. In modern particle accelerators the metal cavities are set under high fields at a radio-frequency, in phase with particles passing through them. Newer generations are based on superconducting coils reaching field strengths of up to $100 MV/m$, where materials start to break down at higher field strengths. Some accelerators are hence circular to use the same accelerating segments over and over again. However, particles radiate at $P$\addnum{} such that electron accelerators would have to be very big. Proton accelerator CERN is massive. At the same time proton or more complex particles are difficult and messy when interacting due to the complex nature of QCD. The particle energy determines the physics that can be resolved, see de Broglie wavelength, such that accelerators grow further and further in size (and expense) in order to access new physics. Figure XX\addnum{} Livingston plot shows the increasing centre-of-mass energy. Figure XX\addnum{} shows an aerial view of the LHC at CERN and its massive outline. Future facilities like CLIC, ILC or the FCC aim to make use of improved cavities and extend their size 
One complementary technology to extend the energy range is the use of plasmas which can sustain thousands of times higher field gradients than solid materials.
Reaching a slow down in energy existing facilities are upgrading their systems to instead increase the luminosity to produce more data and to facilitate precision studies of rare processes like light-light scattering in ultra-peripheral scattering and other studies beyond the standard model including exotic projects like aligned crystals, laser through wall, and so on.

The story of accelerators is a success story, with the standard model being well documented and predictive. The Higgs boson is the final particle to complete the set.
The radiation (limiting factor) of accelerators has also proven to be immensely useful to image and understand material structures and chemical processes. Free-electron lasers. Entire facilities to produce this radiation and opened up fields.

\section{High-intensity lasers}

\textbf{Driving force is intensity, secondary is short durations (for time resolution).}


\EliasComm{Add plot: Intensity plot showing interesting science \cite{Mourou2019_Nobel}}


\begin{figure}
\centering
\includegraphics[width=0.9\columnwidth]{ICUIL3Maps.pdf}
\caption[Ultrahigh intensity laser facilities worldwide.]{Ultrahigh intensity laser facilities worldwide according to ICUIL \cite{ICUIL2019_Map}. To qualify peak laser intensity of $10$ TW.}
\end{figure}

\EliasComm{Add plot laser intensities. Like in Colin's paper but higher res.}

\begin{figure}
\centering
\includegraphics[width=0.8\columnwidth]{LaserIntensities.png}
\caption{Laser intensities. Printed with permission \cite{Mourou2019_Nobel}. Alternative could be from 2002.}
\end{figure}

\EliasComm{Add plot: Nobel prizes?}
\EliasComm{Add plot: Wakefield plot (maybe Jason's)}

Critical field  \cite{Sauter1931_ECRIT,Heisenberg1936_ECRIT}


\begin{figure}
\centering
\includegraphics[height=0.35\columnwidth]{ral_aerial_photo.jpg}\includegraphics[height=0.35\columnwidth]{gemini_laser_area_02.jpg}
\caption{From CLF website.}
\end{figure}


\begin{itemize}
\item lasers have revolutionised understanding of matter light interactions
\item ionisation processes, single, multi, tunnel
\item strong field atom and molecular sciences
\item attosecond physics
\item relativistic and nonlinear optics
\item high energy density physics (HEDP)
\item intensity has gradually increased and this opened up new physics (see Mourou plot, similar Livingston)
\item number of intense lasers has increased as field opens up and more and more applications
\item one more recent is the application to develop accelerators (LWFA) in the interplay with plasmas
\item also as light source
\item LWFA progress also motivated conventional developments to return and use PWFA as afterburner
\item future also involves dielectric accelerator technologies
\item driving factor and limitation is here the laser intensity (and repetition rate, see analogue in energy and luminosity for accelerator)
\item mention next gen: ELI, ZEUS etc., highest intensity so far?
\end{itemize}
\cite{Danson2019_PWLASERS}

CN Danson (2019): Petawatt and exawatt class lasers worldwide
ICS in Nature https://physics.aps.org/articles/v12/87
\section{Particle accelerators and high-intensity lasers}

\textbf{Driving force will be energy and intensity from both sides}

\begin{itemize}
\item lasers and accelerators have by themselves revolutionised our understanding
\item by combining both technologies we can open up a new regime to probe interesting and qualitatively new physics
\item this includes light-light interactions which are at particle accelerators only virtual/quasi-free possible
\item break the vacuum and vacuum as nonlinear medium, nonlinear vacuum physics
\item radiation reaction and cascades for astro
\item QED plasmas
\item eventually reaching end of SFQED description
\item topic of high-intensity facilities next gen: ELI, ZEUS, Munich
\item topic of conventional facilities at EuXFEL, SLAC \cite{Hogan2016_FACETII},...
\item also mention hybrid schemes, e.g. Trojan horse and PWFA afterburner \cite{Lee2002_Afterburner}
\item laser wires etc.
\item laser as widely used diagnostics also in particle accelerators (linear ICS)
\end{itemize}

\EliasComm{Visualisations of QED effects or so on?}

Here some reasons why to do this kind of experiments with wakefield accelerators:
\begin{itemize}
\item Synchronisation of the laser pulse and the electron beam
\item high-intensity lasers are available
\item small beam size of the electron beam means all of the electron beam can be overlapped with a tightly focused laser pulse such that the electron is a real probe of the laser pulse (probe has to be smaller than what it probes).
\end{itemize}


\begin{figure}
\centering
\includegraphics[width=0.7\columnwidth]{RROverview_Blackburn.png}
\caption{Overview of radiation reaction regimes and experiments. From \cite{Blackburn2019_RRReview}.}
\end{figure}




Approaching the atomic intensity of $\sim\!10^{16}\,\mathrm{W/cm^2}$ nonperturbative laser-atom interactions give rise to a wealth of strong-field phenomena, including tunnel ionization and recollision-induced high-harmonic generation (HHG) \cite{Protopapas1997_HIAtomic,Schafer1993_AboveThreshold,Freeman1987_AboveThreshold,Corkum1993_multiPhoton}. 
At the ``critical intensity'' $I_{cr}\!\sim\!10^{29}\,\mathrm{W/cm^2}$ the interaction with the vacuum itself becomes sizeable \cite{Sauter1931_ECRIT,Heisenberg1936_ECRIT}, opening a qualitatively new regime of light-matter interactions \cite{Schwinger1951_VACPAIRS,Brezin1970_VACPAIRS,Popov1971_VACPAIRS,DiPiazza2012_ICS}. 
By combining high-intensity lasers with ultra-relativistic electron beams, e.g., at SLAC's FACET-II \cite{Hogan2016_FACETII}, such extreme intensities become accessible in the electron rest frame due to the Lorentz boost of the intensity ($I' \sim 4\gamma^2 I$). 
In this so-called \textit{nonperturbative strong-field regime of QED} ($I'\!\gtrsim\!I_{cr}$) the recoil of emitted photons significantly perturbs electron/positron trajectories (strong-field radiation reaction) and prolific electron-positron pair production becomes substantial (vacuum breakdown). 

Such extreme conditions can be encountered in violent astrophysical phenomena like gamma ray bursts \cite{Harding1991_GRB} and supernova explosions \cite{Uzdensky2014_PlasmaAstro}, in the vicinity of pulsars and magnetars \cite{Denisov2017_VacuumPulsarMagnetar,Mignani2017_VBAstro}, close to magnetized black holes \cite{Blandford1977_KerrBlackHole}, and in the early universe \cite{Ruffini2010_PAIRSASTRO, Nikishov1962_PAIRSASTRO,Kandus2011_Primordial,Grasso2001_earlyUniverse}. 
Due to stringent luminosity requirements a future lepton collider (e.g., CLIC \cite{CLIC2012} or ILC \cite{ILC2013}) will reach the critical field strength at the interaction point \cite{Yokoya1992_SFQED_BeamBeam,Esberg2014_SFQED_BeamBeam}. 
Moreover, it might be beneficial to exceed this scale by orders of magnitude \cite{Yakimenko2019_NONPERTURB}, necessitating a better understanding of light and matter in extreme electromagnetic fields. 
At multi-petawatt laser facilities, which are currently being constructed in Europe \cite{Gales2018_ELINP,Weber2017_ELIBeamlines}, Asia \cite{Li2017_SULF,Shen2018_SULF} and the US \cite{Maksimchuk2019_ZEUS}, laser-plasma interactions will probe an interplay between strong-field quantum and collective effects. 
These facilities will open an exciting new research field within high-energy-density physics (HEDP) \cite{Grismayer2017_SeededQEDCascades,Ridgers2012_DENSE,Nerush2011_LaserAbs_PAIRS}. 


Interesting phenomena
\begin{itemize}
\item (strong field) radiation reaction \cite{Harvey2017_QUENCHING,Blackburn2019_RRReview,Blackburn2014_QRR,Ritus1985_QRR,Ridgers2017_QRR,Shen1972_STRAGGLING,Thomas2012_LL,Vranic2014_RR,Dinu2016_QRR}
\item RR in crystals \cite{DiPiazza2017_CrystalRR,Artru1994_CHANNELING,Katkov1998_CHANNELING,Wistisen2018_RR,Wistisen2019_RR}
\item radiation production meaning compton scattering from classical \cite{Yan2017_ICS} and the perturbative quantum \cite{Bula1996_RR} to the non-perturbative strong-field regime \cite{Bula1996_RR,TaPhuoc2012_ICS,Chen2013_ICS,Powers2014_ICS,Sarri2014_ICS,Khrennikov2015_ICS,Mackenroth2013_nlCompton}
\item mitigation of RR effects
\item cascades and showers, multi-step  \cite{Bulanov2013_Cascade,Grismayer2017_SeededQEDCascades}
\item pair production \cite{Sokolov2010_QEDPAIRS}
\item nonlinear BW \cite{DiPiazza2016_nonlinearBW}
\item vacuum pair production
\item light-light scattering
\item light-light BW pair production
\item vacuum birefringence \cite{King2016_VB,Nakamiya2017_VB}
\item vacuum dichroism \cite{Bragin2017_VBVD}
\item vacuum recollision physics \cite{Meuren2015_HERecollision,Kuchiev2007_Recollision}
\item LCFA \cite{DiPiazza2018_LCFA,DiPiazza2019_LCFANUM}
\item spin polarisation \cite{DelSorbo2017_SPIN,Seipt2018_SPIN,Li2019_singleshot_beampol}
\item nonperturbative physics \cite{Yakimenko2019_NONPERTURB,Blackburn2019_SUPER,Baumann2019_NONPERTURB}
\item Narozhny conjecture \cite{Ritus1972_NAROZHNY,Narozhny1980_NAROZHNY,Fedotov2017_NAROZHNY,Ilderton2019_QEDPerturbBreakdown}
\item high energy density physics in QED plasmas/producing dense electron-positron pair plasmas \cite{Ridgers2012_DENSE,Nerush2011_LaserAbs_PAIRS,Grismayer2017_SeededQEDCascades}
\item physics beyond the standard model in photon-photon \cite{Beresford2019_PhPh_SLEPTONS,Knapen2017_PhPh_AXIONLIKE,Baldenegro2018_PhPh_AXIONLIKE}
\end{itemize}


Some headlines
\begin{itemize}
\item pair production: matter from light
\item astrophysics in the lab
\item pair production: breaking the vacuum, matter from thin air (see Light fantastic SCIENCE \cite{Cartlidge2018_SEL})
\item vacuum as a nonlinear medium
\item analogue laser science to vacuum SFQED
\item the unexplored frontier of nonperturbative physics
\item radiation reaction: stopping particles with light
\end{itemize}

Review \cite{DiPiazza2012_ICS}

\vspace*{\fill}

\section{Thesis Outline}

This work focuses onto the versatile capabilities of laser wakefield accelerators and high-intensity lasers to generate highly energetic radiation of varied spectral shape and across a wide energy range. It provides three experimental examples where laser wakefield accelerators were used to produce gamma radiation with photon energies from few to several hundreds of $\mathrm{MeV}$, and discusses their application in studies of fundamental phenomena of quantum electrodynamics (QED). 

The thesis starts by discussing single particle motions in electromagnetic fields, radiation mechanisms, radiation reaction and pair production in \textbf{Chapter \ref{Chap:Theory:SingleParticle}}. \textbf{Chapter \ref{Chap:Theory:LWFA}} introduces principles regarding the collective motion in plasmas and their interaction with an intense laser pulse, which is relevant for wakefield acceleration. This is followed by a discussion of experimental methods underpinning this work in \textbf{Chapter \ref{Chap:Methods}}. The experimental results that are at the core of this thesis are presented in \textbf{Chapters \ref{Chap:linICS}}, \textbf{\ref{Chap:RR15}} and \textbf{\ref{Chap:BW}}:
\begin{enumerate}
\setcounter{enumi}{4}
\item \textbf{Linear Inverse Compton Scattering and Beam Profile Diagnostic:}\\
Relativistic electrons from a wakefield accelerator were collided with a defocused laser pulse at $a_0 \sim 0.2 - 1$ generating gamma radiation from linear inverse Compton Scattering in the 10's of $\mathrm{MeV}$ photon energy range. The radiation is used to diagnose the properties of the laser pulse at the interaction and the application of as single-shot electron beam diagnostic is explored.
\item \textbf{Nonlinear Inverse Compton Scattering and Radiation Reaction:}\\
By scattering a tightly focused high-intensity laser pulse from a relativistic electron beam and generating gamma radiation from nonlinear Inverse Compton Scattering, it was possible to measure radiation reaction in an experiment. The energy loss of the electron beam and the spectrum of the radiation were used to pinpoint the interaction conditions and their agreement with theoretical models was tested. 
%A statistical analysis of the electron beam fluctuations and its impact on the statistical significance of the results is presented.
\item \textbf{Bremsstrahlung source and the Linear Breit-Wheeler Process:}\\
An electron beam from a wakefield accelerator was passed through a solid target to produce bremsstrahlung with photon energies of 100s of MeV. The source was optimised with regards to its yield of photons and energetic secondary particles. The high energy photons were then collided with a $\sim$ keV X-ray field generated by a second laser in an attempt to measure the elusive linear Breit-Wheeler process. 
\end{enumerate}
Finally, the results are summarised and future research opportunities are discussed. Taking into considerations the findings and insights of the presented results, an experimental layout for future radiation reaction studies is proposed, outlining beam geometries and relevant diagnostics.
%their impact for future research are evaluated
%\vspace*{\fill}


%\item \textbf{Hard Betatron Radiation from dual shock features:}\\
%Electrons travelling in a wakefield transitioning through two density spikes experience an increase of transverse oscillations resulting in an enhancement of measured betatron radiation reaching hundreds of $\mathrm{keV}$. The dual shock structure was induced by introducing a blade into the supersonic gas flow of a helium gas jet and measured by a stack of scintillating crystals.